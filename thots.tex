\chapter{Thoughts on PSolving, a note to thyself}
\thispagestyle{empty}
\section{Be DUMB, Keep it SIMPLE} 

Remember what Paul Zeitz said? Think wishfully, make dumb wishes. When first approaching the problem, you can do whatever you want. You can loosen some constraints, you can add some new. This works exceptionally well when you need to build an object from one given object, you can do whatever you want. Putting additional constraints decreases the number of test cases. Sometimes loosening some constraints help to give better observation of the problem. \\


Like in ISL 2016 N5, after deciding that we are going to build a pair $ (x_2, y_2) $ from the previous pair $ (x_1, y_1) $, we should look for the most innocent looking relation between these four variable. Now it's time to play around, try dumb things. Rewriting the equation, we want to use the fact that $ x_1, x_2 $ have to be on different sides of $ \sqrt{a} $. How can we insert this constraint in our equation in the most simple and natural way? This is where we need to be dumb, and amature.\\


In \autoref{problem:simurgh_2019_p3} the trick is to keep things simple. Making the most natural assumptions. In construction problems, think of how the result can be achieved in the most natural way. Can we make some extra assumptions that might result in the immediate proof the result's existence? 
