\chapter{IMO Shortlist Algebra Problems A1-3 from year 1998-2017}




\newpage\section{A1}


\prob{}{}{}{Let $a_{1},a_{2},\ldots ,a_{n}$ be positive real numbers such that $a_{1}+a_{2}+\cdots +a_{n}<1$. Prove that

\[ \frac{a_{1} a_{2} \cdots a_{n} \left[ 1 - (a_{1} + a_{2} + \cdots + a_{n}) \right] }{(a_{1} + a_{2} + \cdots + a_{n})( 1 - a_{1})(1 - a_{2}) \cdots (1 - a_{n})} \leq \frac{1}{ n^{n+1}}. \]}




\prob{}{}{}{Let $n \geq 2$ be a fixed integer. Find the least constant $C$ such the inequality

\[\sum_{i<j} x_{i}x_{j} \left(x^{2}_{i}+x^{2}_{j} \right) \leq C \left(\sum_{i}x_{i} \right)^4\]

holds for any $x_{1}, \ldots ,x_{n} \geq 0$ (the sum on the left consists of $\binom{n}{2}$ summands). For this constant $C$, characterize the instances of equality.}





\prob{}{}{}{Let $ a, b, c$ be positive real numbers so that $ abc = 1$. Prove that
\[ \left( a - 1 + \frac 1b \right) \left( b - 1 + \frac 1c \right) \left( c - 1 + \frac 1a \right) \leq 1. \]}


\prob{}{}{}{Let $ T$ denote the set of all ordered triples $ (p,q,r)$ of nonnegative integers. Find all functions $ f: T \rightarrow \mathbb{R}$ satisfying
\[ f(p,q,r) = \begin{cases} 0 & \text{if} \; pqr = 0, \\ 1 + \frac{1}{6}\{f(p + 1,q - 1,r) + f(p - 1,q + 1,r) & \\ + f(p - 1,q,r + 1) + f(p + 1,q,r - 1) & \\ + f(p,q + 1,r - 1) + f(p,q - 1,r + 1)\} & \text{otherwise} \end{cases} \]
for all nonnegative integers $ p$, $ q$, $ r$.}




\prob{}{}{}{Find all functions $f$ from the reals to the reals such that

\[f\left(f(x)+y\right)=2x+f\left(f(y)-x\right)\]

for all real $x,y$.}




\prob{}{}{}{Let $a_{ij}$ $i=1,2,3$; $j=1,2,3$ be real numbers such that $a_{ij}$ is positive for $i=j$ and negative for $i\neq j$.

Prove the existence of positive real numbers $c_{1}$, $c_{2}$, $c_{3}$ such that the numbers \[a_{11}c_{1}+a_{12}c_{2}+a_{13}c_{3},\qquad a_{21}c_{1}+a_{22}c_{2}+a_{23}c_{3},\qquad a_{31}c_{1}+a_{32}c_{2}+a_{33}c_{3}\]are either all negative, all positive, or all zero.}



\prob{}{}{}{Let $n \geq 3$ be an integer. Let $t_1$, $t_2$, ..., $t_n$ be positive real numbers such that \[n^2 + 1 > \left( t_1 + t_2 + \cdots + t_n \right) \left( \frac{1}{t_1} + \frac{1}{t_2} + \cdots + \frac{1}{t_n} \right).\] Show that $t_i$, $t_j$, $t_k$ are side lengths of a triangle for all $i$, $j$, $k$ with $1 \leq i < j < k \leq n$.}



\prob{}{}{}{Find all pairs of integers $a,b$ for which there exists a polynomial $P(x) \in \mathbb{Z}[X]$ such that product $(x^2+ax+b)\cdot P(x)$ is a polynomial of a form \[ x^n+c_{n-1}x^{n-1}+\cdots+c_1x+c_0 \]}



\prob{}{}{}{A sequence of real numbers $ a_{0},\ a_{1},\ a_{2},\dots$ is defined by the formula
\[ a_{i + 1} = \left\lfloor a_{i}\right\rfloor\cdot \left\langle a_{i}\right\rangle\qquad\text{for}\quad i\geq 0; \]here $a_0$ is an arbitrary real number, $\lfloor a_i\rfloor$ denotes the greatest integer not exceeding $a_i$, and $\left\langle a_i\right\rangle=a_i-\lfloor a_i\rfloor$. Prove that $a_i=a_{i+2}$ for $i$ sufficiently large.}




\prob{}{}{}{Real numbers $ a_{1}$, $ a_{2}$, $ \ldots$, $ a_{n}$ are given. For each $ i$, $ (1 \leq i \leq n )$, define
\[ d_{i} = \max \{ a_{j}\mid 1 \leq j \leq i \} - \min \{ a_{j}\mid i \leq j \leq n \} \]
and let $ d = \max \{d_{i}\mid 1 \leq i \leq n \}$.

(a) Prove that, for any real numbers $ x_{1}\leq x_{2}\leq \cdots \leq x_{n}$,
\[ \max \{ |x_{i} - a_{i}| \mid 1 \leq i \leq n \}\geq \frac {d}{2}. \quad \quad (*) \]
(b) Show that there are real numbers $ x_{1}\leq x_{2}\leq \cdots \leq x_{n}$ such that the equality holds in (*).}




\prob{}{}{}{Find all functions $ f: (0, \infty) \mapsto (0, \infty)$ (so $ f$ is a function from the positive real numbers) such that
\[ \frac {\left( f(w) \right)^2 + \left( f(x) \right)^2}{f(y^2) + f(z^2) } = \frac {w^2 + x^2}{y^2 + z^2} \]
for all positive real numbers $ w,x,y,z,$ satisfying $ wx = yz.$}




\prob{}{}{}{Find the largest possible integer $k$, such that the following statement is true:
Let $2009$ arbitrary non-degenerated triangles be given. In every triangle the three sides are coloured, such that one is blue, one is red and one is white. Now, for every colour separately, let us sort the lengths of the sides. We obtain
\[ \left. \begin{array}{rcl} & b_1 \leq b_2\leq\ldots\leq b_{2009} & \textrm{the lengths of the blue sides }\\ & r_1 \leq r_2\leq\ldots\leq r_{2009} & \textrm{the lengths of the red sides }\\ \textrm{and } & w_1 \leq w_2\leq\ldots\leq w_{2009} & \textrm{the lengths of the white sides }\\ \end{array}\right.\]
Then there exist $k$ indices $j$ such that we can form a non-degenerated triangle with side lengths $b_j$, $r_j$, $w_j$.}



\prob{}{}{}{Find all function $f:\mathbb{R}\rightarrow\mathbb{R}$ such that for all $x,y\in\mathbb{R}$ the following equality holds \[ f(\left\lfloor x\right\rfloor y)=f(x)\left\lfloor f(y)\right\rfloor \] where $\left\lfloor a\right\rfloor $ is greatest integer not greater than $a.$}



\prob{}{}{}{Given any set $A = \{a_1, a_2, a_3, a_4\}$ of four distinct positive integers, we denote the sum $a_1 +a_2 +a_3 +a_4$ by $s_A$. Let $n_A$ denote the number of pairs $(i, j)$ with $1 \leq i < j \leq 4$ for which $a_i +a_j$ divides $s_A$. Find all sets $A$ of four distinct positive integers which achieve the largest possible value of $n_A$.}




\prob{}{}{}{Find all functions $f:\mathbb Z\rightarrow \mathbb Z$ such that, for all integers $a,b,c$ that satisfy $a+b+c=0$, the following equality holds:
\[f(a)^2+f(b)^2+f(c)^2=2f(a)f(b)+2f(b)f(c)+2f(c)f(a).\]
(Here $\mathbb{Z}$ denotes the set of integers.)}



\prob{}{}{}{Let $n$ be a positive integer and let $a_1, \ldots, a_{n-1} $ be arbitrary real numbers. Define the sequences $u_0, \ldots, u_n $ and $v_0, \ldots, v_n $ inductively by $u_0 = u_1 = v_0 = v_1 = 1$, and $u_{k+1} = u_k + a_k u_{k-1}$, $v_{k+1} = v_k + a_{n-k} v_{k-1}$ for $k=1, \ldots, n-1.$

Prove that $u_n = v_n.$}



\prob{}{}{}{Let $a_0 < a_1 < a_2 \ldots$ be an infinite sequence of positive integers. Prove that there exists a unique integer $n\geq 1$ such that
\[a_n < \frac{a_0+a_1+a_2+\cdots+a_n}{n} \leq a_{n+1}.\]}




\prob{}{}{}{Suppose that a sequence $a_1,a_2,\ldots$ of positive real numbers satisfies \[a_{k+1}\geq\frac{ka_k}{a_k^2+(k-1)}\]for every positive integer $k$. Prove that $a_1+a_2+\ldots+a_n\geq n$ for every $n\geq2$.}



\prob{}{}{}{Let $a$, $b$, $c$ be positive real numbers such that $\min(ab,bc,ca) \ge 1$. Prove that $$\sqrt[3]{(a^2+1)(b^2+1)(c^2+1)} \le \left(\frac{a+b+c}{3}\right)^2 + 1.$$}



\prob{}{}{}{Let $a_1,a_2,\ldots a_n,k$, and $M$ be positive integers such that
$$\frac{1}{a_1}+\frac{1}{a_2}+\cdots+\frac{1}{a_n}=k\quad\text{and}\quad a_1a_2\cdots a_n=M.$$If $M>1$, prove that the polynomial
$$P(x)=M(x+1)^k-(x+a_1)(x+a_2)\cdots (x+a_n)$$has no positive roots.}






\newpage\section{A2}


\prob{}{}{}{Let $r_{1},r_{2},\ldots ,r_{n}$ be real numbers greater than or equal to 1. Prove that

\[ \frac{1}{r_{1} + 1} + \frac{1}{r_{2} + 1} + \cdots +\frac{1}{r_{n}+1} \geq \frac{n}{ \sqrt[n]{r_{1}r_{2} \cdots r_{n}}+1}. \]
}
\prob{}{}{}{The numbers from 1 to $n^2$ are randomly arranged in the cells of a $n \times n$ square ($n \geq 2$). For any pair of numbers situated on the same row or on the same column the ratio of the greater number to the smaller number is calculated. Let us call the characteristic of the arrangement the smallest of these $n^2\left(n-1\right)$ fractions. What is the highest possible value of the characteristic ?
}

\prob{}{}{}{Let $ a, b, c$ be positive integers satisfying the conditions $ b > 2a$ and $ c > 2b.$ Show that there exists a real number $ \lambda$ with the property that all the three numbers $ \lambda a, \lambda b, \lambda c$ have their fractional parts lying in the interval $ \left(\frac {1}{3}, \frac {2}{3} \right].$

}
\prob{}{}{}{Let $a_0, a_1, a_2, \ldots$ be an arbitrary infinite sequence of positive numbers. Show that the inequality $1 + a_n > a_{n-1} \sqrt[n]{2}$ holds for infinitely many positive integers $n$.
}

\prob{}{}{}{Let $a_1,a_2,\ldots$ be an infinite sequence of real numbers, for which there exists a real number $c$ with $0\leq a_i\leq c$ for all $i$, such that \[\left\lvert a_i-a_j \right\rvert\geq \frac{1}{i+j} \quad \text{for all }i,\ j \text{ with } i \neq j. \] Prove that $c\geq1$.
}

\prob{}{}{}{Find all nondecreasing functions $f: \mathbb{R}\rightarrow\mathbb{R}$ such that
(i) $f(0) = 0, f(1) = 1;$
(ii) $f(a) + f(b) = f(a)f(b) + f(a + b - ab)$ for all real numbers $a, b$ such that $a < 1 < b$.
}


\prob{}{}{}{Let $a_0$, $a_1$, $a_2$, ... be an infinite sequence of real numbers satisfying the equation $a_n=\left|a_{n+1}-a_{n+2}\right|$ for all $n\geq 0$, where $a_0$ and $a_1$ are two different positive reals.

Can this sequence $a_0$, $a_1$, $a_2$, ... be bounded?
}

\prob{}{}{}{We denote by $\mathbb{R}^+$ the set of all positive real numbers.

Find all functions $f: \mathbb R^ + \rightarrow\mathbb R^ +$ which have the property:
\[f(x)f(y)=2f(x+yf(x))\]
for all positive real numbers $x$ and $y$.

}

\prob{}{}{}{The sequence of real numbers $a_0,a_1,a_2,\ldots$ is defined recursively by \[a_0=-1,\qquad\sum_{k=0}^n\dfrac{a_{n-k}}{k+1}=0\quad\text{for}\quad n\geq 1.\]Show that $ a_{n} > 0$ for all $ n\geq 1$.
}


\prob{}{}{}{Consider those functions $ f: \mathbb{N} \mapsto \mathbb{N}$ which satisfy the condition
\[ f(m + n) \geq f(m) + f(f(n)) - 1 \]
for all $ m,n \in \mathbb{N}.$ Find all possible values of $ f(2007).$
}


\prob{}{}{}{(a) Prove that
\[\frac {x^{2}}{\left(x - 1\right)^{2}} + \frac {y^{2}}{\left(y - 1\right)^{2}} + \frac {z^{2}}{\left(z - 1\right)^{2}} \geq 1\] for all real numbers $x$, $y$, $z$, each different from $1$, and satisfying $xyz=1$.

(b) Prove that equality holds above for infinitely many triples of rational numbers $x$, $y$, $z$, each different from $1$, and satisfying $xyz=1$.
}

\prob{}{}{}{Let $a$, $b$, $c$ be positive real numbers such that $\dfrac{1}{a} + \dfrac{1}{b} + \dfrac{1}{c} = a+b+c$. Prove that:
\[\frac{1}{(2a+b+c)^2}+\frac{1}{(a+2b+c)^2}+\frac{1}{(a+b+2c)^2}\leq \frac{3}{16}.\]
}

\prob{}{}{}{Let the real numbers $a,b,c,d$ satisfy the relations $a+b+c+d=6$ and $a^2+b^2+c^2+d^2=12.$ Prove that
\[36 \leq 4 \left(a^3+b^3+c^3+d^3\right) - \left(a^4+b^4+c^4+d^4 \right) \leq 48.\]
}


\prob{}{}{}{Determine all sequences $(x_1,x_2,\ldots,x_{2011})$ of positive integers, such that for every positive integer $n$ there exists an integer $a$ with \[\sum^{2011}_{j=1} j x^n_j = a^{n+1} + 1\]
}


\prob{}{}{}{Let $\mathbb{Z}$ and $\mathbb{Q}$ be the sets of integers and rationals respectively.
a) Does there exist a partition of $\mathbb{Z}$ into three non-empty subsets $A,B,C$ such that the sets $A+B, B+C, C+A$ are disjoint?
b) Does there exist a partition of $\mathbb{Q}$ into three non-empty subsets $A,B,C$ such that the sets $A+B, B+C, C+A$ are disjoint?

Here $X+Y$ denotes the set $\{ x+y : x \in X, y \in Y \}$, for $X,Y \subseteq \mathbb{Z}$ and for $X,Y \subseteq \mathbb{Q}$.
}

\prob{}{}{}{Prove that in any set of $2000$ distinct real numbers there exist two pairs $a>b$ and $c>d$ with $a \neq c$ or $b \neq d $, such that \[ \left| \frac{a-b}{c-d} - 1 \right|< \frac{1}{100000}. \]
}


\prob{}{}{}{Define the function $f:(0,1)\to (0,1)$ by \[\displaystyle f(x) = \left\{ \begin{array}{lr} x+\frac 12 & \text{if}\ \ x < \frac 12\\ x^2 & \text{if}\ \ x \ge \frac 12 \end{array} \right.\] Let $a$ and $b$ be two real numbers such that $0 < a < b < 1$. We define the sequences $a_n$ and $b_n$ by $a_0 = a, b_0 = b$, and $a_n = f( a_{n -1})$, $b_n = f (b_{n -1} )$ for $n > 0$. Show that there exists a positive integer $n$ such that \[(a_n - a_{n-1})(b_n-b_{n-1})<0.\]
}

\prob{}{}{}{Determine all functions $f:\mathbb{Z}\rightarrow\mathbb{Z}$ with the property that \[f(x-f(y))=f(f(x))-f(y)-1\]holds for all $x,y\in\mathbb{Z}$.
}

\prob{}{}{}{Find the smallest constant $C > 0$ for which the following statement holds: among any five positive real numbers $a_1,a_2,a_3,a_4,a_5$ (not necessarily distinct), one can always choose distinct subscripts $i,j,k,l$ such that
\[ \left| \frac{a_i}{a_j} - \frac {a_k}{a_l} \right| \le C. \]
}

\prob{}{}{}{Let $q$ be a real number. Gugu has a napkin with ten distinct real numbers written on it, and he writes the following three lines of real numbers on the blackboard:

    In the first line, Gugu writes down every number of the form $a-b$, where $a$ and $b$ are two (not necessarily distinct) numbers on his napkin.
    In the second line, Gugu writes down every number of the form $qab$, where $a$ and $b$ are
    two (not necessarily distinct) numbers from the first line.
    In the third line, Gugu writes down every number of the form $a^2+b^2-c^2-d^2$, where $a, b, c, d$ are four (not necessarily distinct) numbers from the first line.

Determine all values of $q$ such that, regardless of the numbers on Gugu's napkin, every number in the second line is also a number in the third line.
}




\newpage\section{A3}


\prob{}{}{}{Let $x,y$ and $z$ be positive real numbers such that $xyz=1$. Prove that


\[ \frac{x^{3}}{(1 + y)(1 + z)}+\frac{y^{3}}{(1 + z)(1 + x)}+\frac{z^{3}}{(1 + x)(1 + y)} \geq \frac{3}{4}. \]
}




\prob{}{}{}{A game is played by $n$ girls ($n \geq 2$), everybody having a ball. Each of the $\binom{n}{2}$ pairs of players, is an arbitrary order, exchange the balls they have at the moment. The game is called nice nice if at the end nobody has her own ball and it is called tiresome if at the end everybody has her initial ball. Determine the values of $n$ for which there exists a nice game and those for which there exists a tiresome game.}




\prob{}{}{}{Find all pairs of functions $ f : \mathbb R \to \mathbb R$, $g : \mathbb R \to \mathbb R$ such that \[f \left( x + g(y) \right) = xf(y) - y f(x) + g(x) \quad\text{for all } x, y\in\mathbb{R}.\]
}



\prob{}{}{}{Let $x_1,x_2,\ldots,x_n$ be arbitrary real numbers. Prove the inequality

\[ \frac{x_1}{1+x_1^2} + \frac{x_2}{1+x_1^2 + x_2^2} + \cdots + \frac{x_n}{1 + x_1^2 + \cdots + x_n^2} < \sqrt{n}. \]
}

\prob{}{}{}{Let $P$ be a cubic polynomial given by $P(x)=ax^3+bx^2+cx+d$, where $a,b,c,d$ are integers and $a\ne0$. Suppose that $xP(x)=yP(y)$ for infinitely many pairs $x,y$ of integers with $x\ne y$. Prove that the equation $P(x)=0$ has an integer root.
}

\prob{}{}{}{Consider pairs of the sequences of positive real numbers \[a_1\geq a_2\geq a_3\geq\cdots,\qquad b_1\geq b_2\geq b_3\geq\cdots\]and the sums \[A_n = a_1 + \cdots + a_n,\quad B_n = b_1 + \cdots + b_n;\qquad n = 1,2,\ldots.\]For any pair define $c_n = \min\{a_i,b_i\}$ and $C_n = c_1 + \cdots + c_n$, $n=1,2,\ldots$.


(1) Does there exist a pair $(a_i)_{i\geq 1}$, $(b_i)_{i\geq 1}$ such that the sequences $(A_n)_{n\geq 1}$ and $(B_n)_{n\geq 1}$ are unbounded while the sequence $(C_n)_{n\geq 1}$ is bounded?

(2) Does the answer to question (1) change by assuming additionally that $b_i = 1/i$, $i=1,2,\ldots$?

Justify your answer.
}


\prob{}{}{}{Does there exist a function $s\colon \mathbb{Q} \rightarrow \{-1,1\}$ such that if $x$ and $y$ are distinct rational numbers satisfying ${xy=1}$ or ${x+y\in \{0,1\}}$, then ${s(x)s(y)=-1}$? Justify your answer.
}



\prob{}{}{}{Four real numbers $ p$, $ q$, $ r$, $ s$ satisfy $ p+q+r+s = 9$ and $ p^{2}+q^{2}+r^{2}+s^{2}= 21$. Prove that there exists a permutation $ \left(a,b,c,d\right)$ of $ \left(p,q,r,s\right)$ such that $ ab-cd \geq 2$.}


\prob{}{}{}{The sequence $c_{0}, c_{1}, . . . , c_{n}, . . .$ is defined by $c_{0}= 1, c_{1}= 0$, and $c_{n+2}= c_{n+1}+c_{n}$ for $n \geq 0$. Consider the set $S$ of ordered pairs $(x, y)$ for which there is a finite set $J$ of positive integers such that $x=\textstyle\sum_{j \in J}{c_{j}}$, $y=\textstyle\sum_{j \in J}{c_{j-1}}$. Prove that there exist real numbers $\alpha$, $\beta$, and $M$ with the following property: An ordered pair of nonnegative integers $(x, y)$ satisfies the inequality \[m < \alpha x+\beta y < M\] if and only if $(x, y) \in S$.
}


\prob{}{}{}{Let $ n$ be a positive integer, and let $ x$ and $ y$ be a positive real number such that $ x^n + y^n = 1.$ Prove that
\[ \left(\sum^n_{k = 1} \frac {1 + x^{2k}}{1 + x^{4k}} \right) \cdot \left( \sum^n_{k = 1} \frac {1 + y^{2k}}{1 + y^{4k}} \right) < \frac {1}{(1 - x) \cdot (1 - y)}. \]
}



\prob{}{}{}{Let $ S\subseteq\mathbb{R}$ be a set of real numbers. We say that a pair $ (f, g)$ of functions from $ S$ into $ S$ is a Spanish Couple on $ S$, if they satisfy the following conditions:
(i) Both functions are strictly increasing, i.e. $ f(x) < f(y)$ and $ g(x) < g(y)$ for all $ x$, $ y\in S$ with $ x < y$;

(ii) The inequality $ f\left(g\left(g\left(x\right)\right)\right) < g\left(f\left(x\right)\right)$ holds for all $ x\in S$.

Decide whether there exists a Spanish Couple

    on the set $ S = \mathbb{N}$ of positive integers;
    on the set $ S = \{a - \frac {1}{b}: a, b\in\mathbb{N}\}$
}



\prob{}{}{}{Determine all functions $ f$ from the set of positive integers to the set of positive integers such that, for all positive integers $ a$ and $ b$, there exists a non-degenerate triangle with sides of lengths
\[ a, f(b) \text{ and } f(b + f(a) - 1).\]
(A triangle is non-degenerate if its vertices are not collinear.)
}


\prob{}{}{}{Let $x_1, \ldots , x_{100}$ be nonnegative real numbers such that $x_i + x_{i+1} + x_{i+2} \leq 1$ for all $i = 1, \ldots , 100$ (we put $x_{101 } = x_1, x_{102} = x_2).$ Find the maximal possible value of the sum $S = \sum^{100}_{i=1} x_i x_{i+2}.$
}


\prob{}{}{}{Determine all pairs $(f,g)$ of functions from the set of real numbers to itself that satisfy \[g(f(x+y)) = f(x) + (2x + y)g(y)\] for all real numbers $x$ and $y$.
}



\prob{}{}{}{Let $n\ge 3$ be an integer, and let $a_2,a_3,\ldots ,a_n$ be positive real numbers such that $a_{2}a_{3}\cdots a_{n}=1$. Prove that
\[(1 + a_2)^2 (1 + a_3)^3 \dotsm (1 + a_n)^n > n^n.\]
}



\prob{}{}{}{Let $\mathbb Q_{>0}$ be the set of all positive rational numbers. Let $f:\mathbb Q_{>0}\to\mathbb R$ be a function satisfying the following three conditions:

(i) for all $x,y\in\mathbb Q_{>0}$, we have $f(x)f(y)\geq f(xy)$;
(ii) for all $x,y\in\mathbb Q_{>0}$, we have $f(x+y)\geq f(x)+f(y)$;
(iii) there exists a rational number $a>1$ such that $f(a)=a$.

Prove that $f(x)=x$ for all $x\in\mathbb Q_{>0}$.
}



\prob{}{}{}{For a sequence $x_1,x_2,\ldots,x_n$ of real numbers, we define its $\textit{price}$ as \[\max_{1\le i\le n}|x_1+\cdots +x_i|.\] Given $n$ real numbers, Dave and George want to arrange them into a sequence with a low price. Diligent Dave checks all possible ways and finds the minimum possible price $D$. Greedy George, on the other hand, chooses $x_1$ such that $|x_1 |$ is as small as possible; among the remaining numbers, he chooses $x_2$ such that $|x_1 + x_2 |$ is as small as possible, and so on. Thus, in the $i$-th step he chooses $x_i$ among the remaining numbers so as to minimise the value of $|x_1 + x_2 + \cdots x_i |$. In each step, if several numbers provide the same value, George chooses one at random. Finally he gets a sequence with price $G$.

Find the least possible constant $c$ such that for every positive integer $n$, for every collection of $n$ real numbers, and for every possible sequence that George might obtain, the resulting values satisfy the inequality $G\le cD$.
}


\prob{}{}{}{Let $n$ be a fixed positive integer. Find the maximum possible value of \[ \sum_{1 \le r < s \le 2n} (s-r-n)x_rx_s, \]where $-1 \le x_i \le 1$ for all $i = 1, \cdots , 2n$.
}



\prob{}{}{}{Find all positive integers $n$ such that the following statement holds: Suppose real numbers $a_1$, $a_2$, $\dots$, $a_n$, $b_1$, $b_2$, $\dots$, $b_n$ satisfy $|a_k|+|b_k|=1$ for all $k=1,\dots,n$. Then there exists $\varepsilon_1$, $\varepsilon_2$, $\dots$, $\varepsilon_n$, each of which is either $-1$ or $1$, such that
\[ \left| \sum_{i=1}^n \varepsilon_i a_i \right| + \left| \sum_{i=1}^n \varepsilon_i b_i \right| \le 1. \]
}


\prob{}{}{}{Let $S$ be a finite set, and let $\mathcal{A}$ be the set of all functions from $S$ to $S$. Let $f$ be an element of $\mathcal{A}$, and let $T=f(S)$ be the image of $S$ under $f$. Suppose that $f\circ g\circ f\ne g\circ f\circ g$ for every $g$ in $\mathcal{A}$ with $g\ne f$. Show that $f(T)=T$.
}
