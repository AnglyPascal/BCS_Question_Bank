\chapter{IMO Shortlist Geometry Problems from year 2000-2017}

	
	\section{G1}
	
		\prob{}{}{}{In the plane we are given two circles intersecting at $ X$ and $ Y$. Prove that there exist four points with the following property:
			
			(P) For every circle touching the two given circles at $ A$ and $ B$, and meeting the line $ XY$ at $ C$ and $ D$, each of the lines $ AC$, $ AD$, $ BC$, $ BD$ passes through one of these points.}
		
		\prob{}{}{}{Let $A_1$ be the center of the square inscribed in acute triangle $ABC$ with two vertices of the square on side $BC$. Thus one of the two remaining vertices of the square is on side $AB$ and the other is on $AC$. Points $B_1,\ C_1$ are defined in a similar way for inscribed squares with two vertices on sides $AC$ and $AB$, respectively. Prove that lines $AA_1,\ BB_1,\ CC_1$ are concurrent.}
		
		\prob{}{}{}{Let $B$ be a point on a circle $S_1$, and let $A$ be a point distinct from $B$ on the tangent at $B$ to $S_1$. Let $C$ be a point not on $S_1$ such that the line segment $AC$ meets $S_1$ at two distinct points. Let $S_2$ be the circle touching $AC$ at $C$ and touching $S_1$ at a point $D$ on the opposite side of $AC$ from $B$. Prove that the circumcentre of triangle $BCD$ lies on the circumcircle of triangle $ABC$.%\figdf{.8}{ISL_GEO/1_1_3}{}
		}
		
		\prob{}{}{}{Let $ABCD$ be a cyclic quadrilateral. Let $P$, $Q$, $R$ be the feet of the perpendiculars from $D$ to the lines $BC$, $CA$, $AB$, respectively. Show that $PQ=QR$ if and only if the bisectors of $\angle ABC$ and $\angle ADC$ are concurrent with $AC$.%\figdf{.8}{ISL_GEO/1_1_4}{}
		}
		
		\prob{}{}{}{Let $ABC$ be an acute-angled triangle with $AB\neq AC$. The circle with diameter $BC$ intersects the sides $AB$ and $AC$ at $M$ and $N$ respectively. Denote by $O$ the midpoint of the side $BC$. The bisectors of the angles $\angle BAC$ and $\angle MON$ intersect at $R$. Prove that the circumcircles of the triangles $BMR$ and $CNR$ have a common point lying on the side $BC$.%\figdf{.8}{ISL_GEO/1_1_5}{}
		}
		
		\prob{}{}{}{Given a triangle $ABC$ satisfying $AC+BC=3\cdot AB$. The incircle of triangle $ABC$ has center $I$ and touches the sides $BC$ and $CA$ at the points $D$ and $E$, respectively. Let $K$ and $L$ be the reflections of the points $D$ and $E$ with respect to $I$. Prove that the points $A$, $B$, $K$, $L$ lie on one circle.%\figdf{.8}{ISL_GEO/1_1_6}{}
		}
		
		\prob{}{}{}{Let $ABC$ be triangle with incenter $I$. A point $P$ in the interior of the triangle satisfies \[\angle PBA+\angle PCA = \angle PBC+\angle PCB.\] Show that $AP \geq AI$, and that equality holds if and only if $P=I$.}
		
		\prob{}{}{}{In triangle $ ABC$ the bisector of angle $ BCA$ intersects the circumcircle again at $ R$, the perpendicular bisector of $ BC$ at $ P$, and the perpendicular bisector of $ AC$ at $ Q$. The midpoint of $ BC$ is $ K$ and the midpoint of $ AC$ is $ L$. Prove that the triangles $ RPK$ and $ RQL$ have the same area.}
		
		\prob{}{}{}{Let $ H$ be the orthocenter of an acute-angled triangle $ ABC$. The circle $ \Gamma_{A}$ centered at the midpoint of $ BC$ and passing through $ H$ intersects the sideline $ BC$ at points $ A_{1}$ and $ A_{2}$. Similarly, define the points $ B_{1}$, $ B_{2}$, $ C_{1}$ and $ C_{2}$.
			
			Prove that the six points $ A_{1}$, $ A_{2}$, $ B_{1}$, $ B_{2}$, $ C_{1}$ and $ C_{2}$ are concyclic.}
		
		\prob{}{}{}{Let $ ABC$ be a triangle with $ AB = AC$ . The angle bisectors of $ \angle C AB$ and $ \angle AB C$ meet the sides $ B C$ and $ C A$ at $ D$ and $ E$ , respectively. Let $ K$ be the incentre of triangle $ ADC$. Suppose that $ \angle B E K = 45^\circ$ . Find all possible values of $ \angle C AB$ .}
		
		\prob{}{}{}{Let $ABC$ be an acute triangle with $D, E, F$ the feet of the altitudes lying on $BC, CA, AB$ respectively. One of the intersection points of the line $EF$ and the circumcircle is $P.$ The lines $BP$ and $DF$ meet at point $Q.$ Prove that $AP = AQ.$}
		
		\prob{}{}{}{Let $ABC$ be an acute triangle. Let $\omega$ be a circle whose centre $L$ lies on the side $BC$. Suppose that $\omega$ is tangent to $AB$ at $B'$ and $AC$ at $C'$. Suppose also that the circumcentre $O$ of triangle $ABC$ lies on the shorter arc $B'C'$ of $\omega$. Prove that the circumcircle of $ABC$ and $\omega$ meet at two points.}
		
		\prob{}{}{}{ Given triangle $ABC$ the point $J$ is the centre of the excircle opposite the vertex $A.$ This excircle is tangent to the side $BC$ at $M$, and to the lines $AB$ and $AC$ at $K$ and $L$, respectively. The lines $LM$ and $BJ$ meet at $F$, and the lines $KM$ and $CJ$ meet at $G.$ Let $S$ be the point of intersection of the lines $AF$ and $BC$, and let $T$ be the point of intersection of the lines $AG$ and $BC.$ Prove that $M$ is the midpoint of $ST.$
			
			(The excircle of $ABC$ opposite the vertex $A$ is the circle that is tangent to the line segment $BC$, to the ray $AB$ beyond $B$, and to the ray $AC$ beyond $C$.)
		}
	
		\prob{}{}{}{Let $ABC$ be an acute triangle with orthocenter $H$, and let $W$ be a point on the side $BC$, lying strictly between $B$ and $C$. The points $M$ and $N$ are the feet of the altitudes from $B$ and $C$, respectively. Denote by $\omega_1$ is the circumcircle of $BWN$, and let $X$ be the point on $\omega_1$ such that $WX$ is a diameter of $\omega_1$. Analogously, denote by $\omega_2$ the circumcircle of triangle $CWM$, and let $Y$ be the point such that $WY$ is a diameter of $\omega_2$. Prove that $X,Y$ and $H$ are collinear.}
		
		\prob{}{}{}{Let $P$ and $Q$ be on segment $BC$ of an acute triangle $ABC$ such that $\angle PAB=\angle BCA$ and $\angle CAQ=\angle ABC$. Let $M$ and $N$ be the points on $AP$ and $AQ$, respectively, such that $P$ is the midpoint of $AM$ and $Q$ is the midpoint of $AN$. Prove that the intersection of $BM$ and $CN$ is on the circumference of triangle $ABC$.}
		
		\prob{}{}{}{Let $ABC$ be an acute triangle with orthocenter $H$. Let $G$ be the point such that the quadrilateral $ABGH$ is a parallelogram. Let $I$ be the point on the line $GH$ such that $AC$ bisects $HI$. Suppose that the line $AC$ intersects the circumcircle of the triangle $GCI$ at $C$ and $J$. Prove that $IJ = AH$.}
		
		\prob{}{}{}{Triangle $BCF$ has a right angle at $B$. Let $A$ be the point on line $CF$ such that $FA=FB$ and $F$ lies between $A$ and $C$. Point $D$ is chosen so that $DA=DC$ and $AC$ is the bisector of $\angle{DAB}$. Point $E$ is chosen so that $EA=ED$ and $AD$ is the bisector of $\angle{EAC}$. Let $M$ be the midpoint of $CF$. Let $X$ be the point such that $AMXE$ is a parallelogram. Prove that $BD,FX$ and $ME$ are concurrent.}
		
		\prob{}{}{}{Let $ABCDE$ be a convex pentagon such that $AB=BC=CD$, $\angle{EAB}=\angle{BCD}$, and $\angle{EDC}=\angle{CBA}$. Prove that the perpendicular line from $E$ to $BC$ and the line segments $AC$ and $BD$ are concurrent.}
		
	
	\newpage\section{G2}
		
		\prob{}{}{}{Two circles $ G_1$ and $ G_2$ intersect at two points $ M$ and $ N$. Let $ AB$ be the line tangent to these circles at $ A$ and $ B$, respectively, so that $ M$ lies closer to $ AB$ than $ N$. Let $ CD$ be the line parallel to $ AB$ and passing through the point $ M$, with $ C$ on $ G_1$ and $ D$ on $ G_2$. Lines $ AC$ and $ BD$ meet at $ E$; lines $ AN$ and $ CD$ meet at $ P$; lines $ BN$ and $ CD$ meet at $ Q$. Show that $ EP = EQ$.}
		
		\prob{}{}{}{Consider an acute-angled triangle $ABC$. Let $P$ be the foot of the altitude of triangle $ABC$ issuing from the vertex $A$, and let $O$ be the circumcenter of triangle $ABC$. Assume that $\angle C \geq \angle B+30^{\circ}$. Prove that $\angle A+\angle COP < 90^{\circ}$.}
		
		\prob{}{}{}{Let $ABC$ be a triangle for which there exists an interior point $F$ such that $\angle AFB=\angle BFC=\angle CFA$. Let the lines $BF$ and $CF$ meet the sides $AC$ and $AB$ at $D$ and $E$ respectively. Prove that \[ AB+AC\geq4DE. \]}
		
		\prob{}{}{}{Three distinct points $A$, $B$, and $C$ are fixed on a line in this order. Let $\Gamma$ be a circle passing through $A$ and $C$ whose center does not lie on the line $AC$. Denote by $P$ the intersection of the tangents to $\Gamma$ at $A$ and $C$. Suppose $\Gamma$ meets the segment $PB$ at $Q$. Prove that the intersection of the bisector of $\angle AQC$ and the line $AC$ does not depend on the choice of $\Gamma$.}
		
		\prob{}{}{}{Let $\Gamma$ be a circle and let $d$ be a line such that $\Gamma$ and $d$ have no common points. Further, let $AB$ be a diameter of the circle $\Gamma$; assume that this diameter $AB$ is perpendicular to the line $d$, and the point $B$ is nearer to the line $d$ than the point $A$. Let $C$ be an arbitrary point on the circle $\Gamma$, different from the points $A$ and $B$. Let $D$ be the point of intersection of the lines $AC$ and $d$. One of the two tangents from the point $D$ to the circle $\Gamma$ touches this circle $\Gamma$ at a point $E$; hereby, we assume that the points $B$ and $E$ lie in the same halfplane with respect to the line $AC$. Denote by $F$ the point of intersection of the lines $BE$ and $d$. Let the line $AF$ intersect the circle $\Gamma$ at a point $G$, different from $A$.
			
			Prove that the reflection of the point $G$ in the line $AB$ lies on the line $CF$.}
		
		\prob{}{}{}{Six points are chosen on the sides of an equilateral triangle $ABC$: $A_1$, $A_2$ on $BC$, $B_1$, $B_2$ on $CA$ and $C_1$, $C_2$ on $AB$, such that they are the vertices of a convex hexagon $A_1A_2B_1B_2C_1C_2$ with equal side lengths.
			
			Prove that the lines $A_1B_2$, $B_1C_2$ and $C_1A_2$ are concurrent.}
		
		\prob{}{}{}{Let $ ABC$ be a trapezoid with parallel sides $ AB > CD$. Points $ K$ and $ L$ lie on the line segments $ AB$ and $ CD$, respectively, so that $AK/KB=DL/LC$. Suppose that there are points $ P$ and $ Q$ on the line segment $ KL$ satisfying \[\angle{APB} = \angle{BCD}\qquad\text{and}\qquad \angle{CQD} = \angle{ABC}.\] Prove that the points $ P$, $ Q$, $ B$ and $ C$ are concyclic.
		}
	
		\prob{}{}{}{Denote by $ M$ midpoint of side $ BC$ in an isosceles triangle $ \triangle ABC$ with $ AC = AB$. Take a point $ X$ on a smaller arc $ MA$ of circumcircle of triangle $ \triangle ABM$. Denote by $ T$ point inside of angle $ BMA$ such that $ \angle TMX = 90$ and $ TX = BX$.
			
			Prove that $ \angle MTB - \angle CTM$ does not depend on choice of $ X$.}
		
		\prob{}{}{}{Given trapezoid $ ABCD$ with parallel sides $ AB$ and $ CD$, assume that there exist points $ E$ on line $ BC$ outside segment $ BC$, and $ F$ inside segment $ AD$ such that $ \angle DAE = \angle CBF$. Denote by $ I$ the point of intersection of $ CD$ and $ EF$, and by $ J$ the point of intersection of $ AB$ and $ EF$. Let $ K$ be the midpoint of segment $ EF$, assume it does not lie on line $ AB$. Prove that $ I$ belongs to the circumcircle of $ ABK$ if and only if $ K$ belongs to the circumcircle of $ CDJ$.}
		
		\prob{}{}{}{Let $ ABC$ be a triangle with circumcentre $ O$. The points $ P$ and $ Q$ are interior points of the sides $ CA$ and $ AB$ respectively. Let $ K,L$ and $ M$ be the midpoints of the segments $ BP,CQ$ and $ PQ$. respectively, and let $ \Gamma$ be the circle passing through $ K,L$ and $ M$. Suppose that the line $ PQ$ is tangent to the circle $ \Gamma$. Prove that $ OP = OQ.$}
		
		\prob{}{}{}{Let $P$ be a point interior to triangle $ABC$ (with $CA \neq CB$). The lines $AP$, $BP$ and $CP$ meet again its circumcircle $\Gamma$ at $K$, $L$, respectively $M$. The tangent line at $C$ to $\Gamma$ meets the line $AB$ at $S$. Show that from $SC = SP$ follows $MK = ML$.}
		
		\prob{}{}{}{Let $A_1A_2A_3A_4$ be a non-cyclic quadrilateral. Let $O_1$ and $r_1$ be the circumcentre and the circumradius of the triangle $A_2A_3A_4$. Define $O_2,O_3,O_4$ and $r_2,r_3,r_4$ in a similar way. Prove that
			\[\frac{1}{O_1A_1^2-r_1^2}+\frac{1}{O_2A_2^2-r_2^2}+\frac{1}{O_3A_3^2-r_3^2}+\frac{1}{O_4A_4^2-r_4^2}=0.\]
			
			Proposed by Alexey Gladkich, Israel}
		
		\prob{}{}{}{Let $ABCD$ be a cyclic quadrilateral whose diagonals $AC$ and $BD$ meet at $E$. The extensions of the sides $AD$ and $BC$ beyond $A$ and $B$ meet at $F$. Let $G$ be the point such that $ECGD$ is a parallelogram, and let $H$ be the image of $E$ under reflection in $AD$. Prove that $D,H,F,G$ are concyclic.}
		
		\prob{}{}{}{Let $\omega$ be the circumcircle of a triangle $ABC$. Denote by $M$ and $N$ the midpoints of the sides $AB$ and $AC$, respectively, and denote by $T$ the midpoint of the arc $BC$ of $\omega$ not containing $A$. The circumcircles of the triangles $AMT$ and $ANT$ intersect the perpendicular bisectors of $AC$ and $AB$ at points $X$ and $Y$, respectively; assume that $X$ and $Y$ lie inside the triangle $ABC$. The lines $MN$ and $XY$ intersect at $K$. Prove that $KA=KT$.}
		
		\prob{}{}{}{Let $ABC$ be a triangle. The points $K, L,$ and $M$ lie on the segments $BC, CA,$ and $AB,$ respectively, such that the lines $AK, BL,$ and $CM$ intersect in a common point. Prove that it is possible to choose two of the triangles $ALM, BMK,$ and $CKL$ whose inradii sum up to at least the inradius of the triangle $ABC$.}
		
		\prob{}{}{}{Triangle $ABC$ has circumcircle $\Omega$ and circumcenter $O$. A circle $\Gamma$ with center $A$ intersects the segment $BC$ at points $D$ and $E$, such that $B$, $D$, $E$, and $C$ are all different and lie on line $BC$ in this order. Let $F$ and $G$ be the points of intersection of $\Gamma$ and $\Omega$, such that $A$, $F$, $B$, $C$, and $G$ lie on $\Omega$ in this order. Let $K$ be the second point of intersection of the circumcircle of triangle $BDF$ and the segment $AB$. Let $L$ be the second point of intersection of the circumcircle of triangle $CGE$ and the segment $CA$.
			
			Suppose that the lines $FK$ and $GL$ are different and intersect at the point $X$. Prove that $X$ lies on the line $AO$.}
		
		\prob{}{}{}{Let $ABC$ be a triangle with circumcircle $\Gamma$ and incenter $I$ and let $M$ be the midpoint of $\overline{BC}$. The points $D$, $E$, $F$ are selected on sides $\overline{BC}$, $\overline{CA}$, $\overline{AB}$ such that $\overline{ID} \perp \overline{BC}$, $\overline{IE}\perp \overline{AI}$, and $\overline{IF}\perp \overline{AI}$. Suppose that the circumcircle of $\triangle AEF$ intersects $\Gamma$ at a point $X$ other than $A$. Prove that lines $XD$ and $AM$ meet on $\Gamma$.}
			
		\prob{}{}{}{Let $R$ and $S$ be different points on a circle $\Omega$ such that $RS$ is not a diameter. Let $\ell$ be the tangent line to $\Omega$ at $R$. Point $T$ is such that $S$ is the midpoint of the line segment $RT$. Point $J$ is chosen on the shorter arc $RS$ of $\Omega$ so that the circumcircle $\Gamma$ of triangle $JST$ intersects $\ell$ at two distinct points. Let $A$ be the common point of $\Gamma$ and $\ell$ that is closer to $R$. Line $AJ$ meets $\Omega$ again at $K$. Prove that the line $KT$ is tangent to $\Gamma$.}
		
		
		
	\newpage\section{G3}
		
		\prob{}{}{}{Let $O$ be the circumcenter and $H$ the orthocenter of an acute triangle $ABC$. Show that there exist points $D$, $E$, and $F$ on sides $BC$, $CA$, and $AB$ respectively such that \[ OD + DH = OE + EH = OF + FH\] and the lines $AD$, $BE$, and $CF$ are concurrent.}
		
		
		
		\prob{}{}{}{Let $ABC$ be a triangle with centroid $G$. Determine, with proof, the position of the point $P$ in the plane of $ABC$ such that $AP{\cdot}AG + BP{\cdot}BG + CP{\cdot}CG$ is a minimum, and express this minimum value in terms of the side lengths of $ABC$.}
		
		
		
		\prob{}{}{}{The circle $S$ has centre $O$, and $BC$ is a diameter of $S$. Let $A$ be a point of $S$ such that $\angle AOB<120{{}^\circ}$. Let $D$ be the midpoint of the arc $AB$ which does not contain $C$. The line through $O$ parallel to $DA$ meets the line $AC$ at $I$. The perpendicular bisector of $OA$ meets $S$ at $E$ and at $F$. Prove that $I$ is the incentre of the triangle $CEF.$}
		
		
		
		\prob{}{}{}{Let $ABC$ be a triangle and let $P$ be a point in its interior. Denote by $D$, $E$, $F$ the feet of the perpendiculars from $P$ to the lines $BC$, $CA$, $AB$, respectively. Suppose that \[AP^2 + PD^2 = BP^2 + PE^2 = CP^2 + PF^2.\]Denote by $I_A$, $I_B$, $I_C$ the excenters of the triangle $ABC$. Prove that $P$ is the circumcenter of the triangle $I_AI_BI_C$.}
		
		
		
		\prob{}{}{}{Let $O$ be the circumcenter of an acute-angled triangle $ABC$ with ${\angle B<\angle C}$. The line $AO$ meets the side $BC$ at $D$. The circumcenters of the triangles $ABD$ and $ACD$ are $E$ and $F$, respectively. Extend the sides $BA$ and $CA$ beyond $A$, and choose on the respective extensions points $G$ and $H$ such that ${AG=AC}$ and ${AH=AB}$. Prove that the quadrilateral $EFGH$ is a rectangle if and only if ${\angle ACB-\angle ABC=60^{\circ }}$.}
		
		
		
		\prob{}{}{}{Let $ABCD$ be a parallelogram. A variable line $g$ through the vertex $A$ intersects the rays $BC$ and $DC$ at the points $X$ and $Y$, respectively. Let $K$ and $L$ be the $A$-excenters of the triangles $ABX$ and $ADY$. Show that the angle $\measuredangle KCL$ is independent of the line $g$.}
		
		
		
		\prob{}{}{}{Let $ ABCDE$ be a convex pentagon such that
			\[ \angle BAC = \angle CAD = \angle DAE\qquad \text{and}\qquad \angle ABC = \angle ACD = \angle ADE. \]The diagonals $BD$ and $CE$ meet at $P$. Prove that the line $AP$ bisects the side $CD$.}
		
		
		
		\prob{}{}{}{The diagonals of a trapezoid $ ABCD$ intersect at point $ P$. Point $ Q$ lies between the parallel lines $ BC$ and $ AD$ such that $ \angle AQD = \angle CQB$, and line $ CD$ separates points $ P$ and $ Q$. Prove that $ \angle BQP = \angle DAQ$.}
		
		
		
		\prob{}{}{}{Let $ ABCD$ be a convex quadrilateral and let $ P$ and $ Q$ be points in $ ABCD$ such that $ PQDA$ and $ QPBC$ are cyclic quadrilaterals. Suppose that there exists a point $ E$ on the line segment $ PQ$ such that $ \angle PAE = \angle QDE$ and $ \angle PBE = \angle QCE$. Show that the quadrilateral $ ABCD$ is cyclic.}
		
		
		
		\prob{}{}{}{Let $ABC$ be a triangle. The incircle of $ABC$ touches the sides $AB$ and $AC$ at the points $Z$ and $Y$, respectively. Let $G$ be the point where the lines $BY$ and $CZ$ meet, and let $R$ and $S$ be points such that the two quadrilaterals $BCYR$ and $BCSZ$ are parallelogram.
			Prove that $GR=GS$.}
		
		
		
		\prob{}{}{}{Let $A_1A_2 \ldots A_n$ be a convex polygon. Point $P$ inside this polygon is chosen so that its projections $P_1, \ldots , P_n$ onto lines $A_1A_2, \ldots , A_nA_1$ respectively lie on the sides of the polygon. Prove that for arbitrary points $X_1, \ldots , X_n$ on sides $A_1A_2, \ldots , A_nA_1$ respectively,
			\[\max \left\{ \frac{X_1X_2}{P_1P_2}, \ldots, \frac{X_nX_1}{P_nP_1} \right\} \geq 1.\]}
		
		
		
		\prob{}{}{}{Let $ABCD$ be a convex quadrilateral whose sides $AD$ and $BC$ are not parallel. Suppose that the circles with diameters $AB$ and $CD$ meet at points $E$ and $F$ inside the quadrilateral. Let $\omega_E$ be the circle through the feet of the perpendiculars from $E$ to the lines $AB,BC$ and $CD$. Let $\omega_F$ be the circle through the feet of the perpendiculars from $F$ to the lines $CD,DA$ and $AB$. Prove that the midpoint of the segment $EF$ lies on the line through the two intersections of $\omega_E$ and $\omega_F$.}
		
		
		
		\prob{}{}{}{In an acute triangle $ABC$ the points $D,E$ and $F$ are the feet of the altitudes through $A,B$ and $C$ respectively. The incenters of the triangles $AEF$ and $BDF$ are $I_1$ and $I_2$ respectively; the circumcenters of the triangles $ACI_1$ and $BCI_2$ are $O_1$ and $O_2$ respectively. Prove that $I_1I_2$ and $O_1O_2$ are parallel.}
		
		
		
		\prob{}{}{}{In a triangle $ABC$, let $D$ and $E$ be the feet of the angle bisectors of angles $A$ and $B$, respectively. A rhombus is inscribed into the quadrilateral $AEDB$ (all vertices of the rhombus lie on different sides of $AEDB$). Let $\varphi$ be the non-obtuse angle of the rhombus. Prove that $\varphi \le \max \{ \angle BAC, \angle ABC \}$.}
		
		
		
		\prob{}{}{}{Let $\Omega$ and $O$ be the circumcircle and the circumcentre of an acute-angled triangle $ABC$ with $AB > BC$. The angle bisector of $\angle ABC$ intersects $\Omega$ at $M \ne B$. Let $\Gamma$ be the circle with diameter $BM$. The angle bisectors of $\angle AOB$ and $\angle BOC$ intersect $\Gamma$ at points $P$ and $Q,$ respectively. The point $R$ is chosen on the line $P Q$ so that $BR = MR$. Prove that $BR\parallel AC$.
			(Here we always assume that an angle bisector is a ray.)}
		
		
		
		\prob{}{}{}{Let $ABC$ be a triangle with $\angle{C} = 90^{\circ}$, and let $H$ be the foot of the altitude from $C$. A point $D$ is chosen inside the triangle $CBH$ so that $CH$ bisects $AD$. Let $P$ be the intersection point of the lines $BD$ and $CH$. Let $\omega$ be the semicircle with diameter $BD$ that meets the segment $CB$ at an interior point. A line through $P$ is tangent to $\omega$ at $Q$. Prove that the lines $CQ$ and $AD$ meet on $\omega$.}
		
		
		
		\prob{}{}{}{Let $B = (-1, 0)$ and $C = (1, 0)$ be fixed points on the coordinate plane. A nonempty, bounded subset $S$ of the plane is said to be nice if
			
			$\text{(i)}$ there is a point $T$ in $S$ such that for every point $Q$ in $S$, the segment $TQ$ lies entirely in $S$; and
			
			$\text{(ii)}$ for any triangle $P_1P_2P_3$, there exists a unique point $A$ in $S$ and a permutation $\sigma$ of the indices $\{1, 2, 3\}$ for which triangles $ABC$ and $P_{\sigma(1)}P_{\sigma(2)}P_{\sigma(3)}$ are similar.
			
			Prove that there exist two distinct nice subsets $S$ and $S'$ of the set $\{(x, y) : x \geq 0, y \geq 0\}$ such that if $A \in S$ and $A' \in S'$ are the unique choices of points in $\text{(ii)}$, then the product $BA \cdot BA'$ is a constant independent of the triangle $P_1P_2P_3$.}
		
		
		
		\prob{}{}{}{Let $O$ be the circumcenter of an acute triangle $ABC$. Line $OA$ intersects the altitudes of $ABC$ through $B$ and $C$ at $P$ and $Q$, respectively. The altitudes meet at $H$. Prove that the circumcenter of triangle $PQH$ lies on a median of triangle $ABC$.}
		
		
	\newpage\section{G4}
	
		\prob{}{}{}{Let $ A_1A_2 \ldots A_n$ be a convex polygon, $ n \geq 4.$ Prove that $ A_1A_2 \ldots A_n$ is cyclic if and only if to each vertex $ A_j$ one can assign a pair $ (b_j, c_j)$ of real numbers, $ j = 1, 2, \ldots, n,$ so that $ A_iA_j = b_jc_i - b_ic_j$ for all $ i, j$ with $ 1 \leq i < j \leq n.$}
		
		
		
		\prob{}{}{}{Let $M$ be a point in the interior of triangle $ABC$. Let $A'$ lie on $BC$ with $MA'$ perpendicular to $BC$. Define $B'$ on $CA$ and $C'$ on $AB$ similarly. Define
			
			\[ p(M) = \frac{MA' \cdot MB' \cdot MC'}{MA \cdot MB \cdot MC}. \]
			
			Determine, with proof, the location of $M$ such that $p(M)$ is maximal. Let $\mu(ABC)$ denote this maximum value. For which triangles $ABC$ is the value of $\mu(ABC)$ maximal?}
		
		
		
		\prob{}{}{}{Circles $S_1$ and $S_2$ intersect at points $P$ and $Q$. Distinct points $A_1$ and $B_1$ (not at $P$ or $Q$) are selected on $S_1$. The lines $A_1P$ and $B_1P$ meet $S_2$ again at $A_2$ and $B_2$ respectively, and the lines $A_1B_1$ and $A_2B_2$ meet at $C$. Prove that, as $A_1$ and $B_1$ vary, the circumcentres of triangles $A_1A_2C$ all lie on one fixed circle.}
		
		
		
		\prob{}{}{}{Let $\Gamma_1$, $\Gamma_2$, $\Gamma_3$, $\Gamma_4$ be distinct circles such that $\Gamma_1$, $\Gamma_3$ are externally tangent at $P$, and $\Gamma_2$, $\Gamma_4$ are externally tangent at the same point $P$. Suppose that $\Gamma_1$ and $\Gamma_2$; $\Gamma_2$ and $\Gamma_3$; $\Gamma_3$ and $\Gamma_4$; $\Gamma_4$ and $\Gamma_1$ meet at $A$, $B$, $C$, $D$, respectively, and that all these points are different from $P$. Prove that
			
			\[ \frac{AB\cdot BC}{AD\cdot DC}=\frac{PB^2}{PD^2}. \]}
		
		
		
		\prob{}{}{}{In a convex quadrilateral $ABCD$, the diagonal $BD$ bisects neither the angle $ABC$ nor the angle $CDA$. The point $P$ lies inside $ABCD$ and satisfies \[\angle PBC=\angle DBA\quad\text{and}\quad \angle PDC=\angle BDA.\] Prove that $ABCD$ is a cyclic quadrilateral if and only if $AP=CP$.}
		
		
		
		\prob{}{}{}{Let $ABCD$ be a fixed convex quadrilateral with $BC=DA$ and $BC$ not parallel with $DA$. Let two variable points $E$ and $F$ lie of the sides $BC$ and $DA$, respectively and satisfy $BE=DF$. The lines $AC$ and $BD$ meet at $P$, the lines $BD$ and $EF$ meet at $Q$, the lines $EF$ and $AC$ meet at $R$.
			
			Prove that the circumcircles of the triangles $PQR$, as $E$ and $F$ vary, have a common point other than $P$.}
		
		
		
		\prob{}{}{}{A point $D$ is chosen on the side $AC$ of a triangle $ABC$ with $\angle C < \angle A < 90^\circ$ in such a way that $BD=BA$. The incircle of $ABC$ is tangent to $AB$ and $AC$ at points $K$ and $L$, respectively. Let $J$ be the incenter of triangle $BCD$. Prove that the line $KL$ intersects the line segment $AJ$ at its midpoint.}
		
		
		
		
		\prob{}{}{}{Consider five points $ A$, $ B$, $ C$, $ D$ and $ E$ such that $ ABCD$ is a parallelogram and $ BCED$ is a cyclic quadrilateral. Let $ \ell$ be a line passing through $ A$. Suppose that $ \ell$ intersects the interior of the segment $ DC$ at $ F$ and intersects line $ BC$ at $ G$. Suppose also that $ EF = EG = EC$. Prove that $ \ell$ is the bisector of angle $ DAB$.}
		
		
		
		\prob{}{}{}{In an acute triangle $ ABC$ segments $ BE$ and $ CF$ are altitudes. Two circles passing through the point $ A$ anf $ F$ and tangent to the line $ BC$ at the points $ P$ and $ Q$ so that $ B$ lies between $ C$ and $ Q$. Prove that lines $ PE$ and $ QF$ intersect on the circumcircle of triangle $ AEF$.}
		
		
		
		
		\prob{}{}{}{Given a cyclic quadrilateral $ABCD$, let the diagonals $AC$ and $BD$ meet at $E$ and the lines $AD$ and $BC$ meet at $F$. The midpoints of $AB$ and $CD$ are $G$ and $H$, respectively. Show that $EF$ is tangent at $E$ to the circle through the points $E$, $G$ and $H$.}
		
		
		
		
		\prob{}{}{}{Given a triangle $ABC$, with $I$ as its incenter and $\Gamma$ as its circumcircle, $AI$ intersects $\Gamma$ again at $D$. Let $E$ be a point on the arc $BDC$, and $F$ a point on the segment $BC$, such that $\angle BAF=\angle CAE < \dfrac12\angle BAC$. If $G$ is the midpoint of $IF$, prove that the meeting point of the lines $EI$ and $DG$ lies on $\Gamma$.}
		
		
		
		
		\prob{}{}{}{Let $ABC$ be an acute triangle with circumcircle $\Omega$. Let $B_0$ be the midpoint of $AC$ and let $C_0$ be the midpoint of $AB$. Let $D$ be the foot of the altitude from $A$ and let $G$ be the centroid of the triangle $ABC$. Let $\omega$ be a circle through $B_0$ and $C_0$ that is tangent to the circle $\Omega$ at a point $X\not= A$. Prove that the points $D,G$ and $X$ are collinear.}
		
		
		
		
		\prob{}{}{}{Let $ABC$ be a triangle with $AB \neq AC$ and circumcenter $O$. The bisector of $\angle BAC$ intersects $BC$ at $D$. Let $E$ be the reflection of $D$ with respect to the midpoint of $BC$. The lines through $D$ and $E$ perpendicular to $BC$ intersect the lines $AO$ and $AD$ at $X$ and $Y$ respectively. Prove that the quadrilateral $BXCY$ is cyclic.}
		
		
		
		
		\prob{}{}{}{Let $ABC$ be a triangle with $\angle B > \angle C$. Let $P$ and $Q$ be two different points on line $AC$ such that $\angle PBA = \angle QBA = \angle ACB $ and $A$ is located between $P$ and $C$. Suppose that there exists an interior point $D$ of segment $BQ$ for which $PD=PB$. Let the ray $AD$ intersect the circle $ABC$ at $R \neq A$. Prove that $QB = QR$.}
		
		
		
		
		\prob{}{}{}{Consider a fixed circle $\Gamma$ with three fixed points $A, B,$ and $C$ on it. Also, let us fix a real number $\lambda \in(0,1)$. For a variable point $P \not\in\{A, B, C\}$ on $\Gamma$, let $M$ be the point on the segment $CP$ such that $CM =\lambda\cdot CP$ . Let $Q$ be the second point of intersection of the circumcircles of the triangles $AMP$ and $BMC$. Prove that as $P$ varies, the point $Q$ lies on a fixed circle.}
		
		
		
		
		\prob{}{}{}{Let $ABC$ be an acute triangle and let $M$ be the midpoint of $AC$. A circle $\omega$ passing through $B$ and $M$ meets the sides $AB$ and $BC$ at points $P$ and $Q$ respectively. Let $T$ be the point such that $BPTQ$ is a parallelogram. Suppose that $T$ lies on the circumcircle of $ABC$. Determine all possible values of $\frac{BT}{BM}$.}
		
		
		
		
		\prob{}{}{}{Let $ABC$ be a triangle with $AB = AC \neq BC$ and let $I$ be its incentre. The line $BI$ meets $AC$ at $D$, and the line through $D$ perpendicular to $AC$ meets $AI$ at $E$. Prove that the reflection of $I$ in $AC$ lies on the circumcircle of triangle $BDE$.}
		
		
		
		
		\prob{}{}{}{In triangle $ABC$, let $\omega$ be the excircle opposite to $A$. Let $D, E$ and $F$ be the points where $\omega$ is tangent to $BC, CA$, and $AB$, respectively. The circle $AEF$ intersects line $BC$ at $P$ and $Q$. Let $M$ be the midpoint of $AD$. Prove that the circle $MPQ$ is tangent to $\omega$.}


	
	\newpage\section{G5}
	
		\prob{}{}{}{The tangents at $B$ and $A$ to the circumcircle of an acute angled triangle $ABC$ meet the tangent at $C$ at $T$ and $U$ respectively. $AT$ meets $BC$ at $P$, and $Q$ is the midpoint of $AP$; $BU$ meets $CA$ at $R$, and $S$ is the midpoint of $BR$. Prove that $\angle ABQ=\angle BAS$. Determine, in terms of ratios of side lengths, the triangles for which this angle is a maximum.}
		
		
		
		\prob{}{}{}{Let $ABC$ be an acute triangle. Let $DAC,EAB$, and $FBC$ be isosceles triangles exterior to $ABC$, with $DA=DC, EA=EB$, and $FB=FC$, such that
			
			\[ \angle ADC = 2\angle BAC, \quad \angle BEA= 2 \angle ABC, \quad \angle CFB = 2 \angle ACB. \]
			
			Let $D'$ be the intersection of lines $DB$ and $EF$, let $E'$ be the intersection of $EC$ and $DF$, and let $F'$ be the intersection of $FA$ and $DE$. Find, with proof, the value of the sum
			
			\[ \frac{DB}{DD'}+\frac{EC}{EE'}+\frac{FA}{FF'}. \]}
		
		
		
		\prob{}{}{}{For any set $S$ of five points in the plane, no three of which are collinear, let $M(S)$ and $m(S)$ denote the greatest and smallest areas, respectively, of triangles determined by three points from $S$. What is the minimum possible value of $M(S)/m(S)$ ?}
		
		
		
		\prob{}{}{}{Let $ABC$ be an isosceles triangle with $AC=BC$, whose incentre is $I$. Let $P$ be a point on the circumcircle of the triangle $AIB$ lying inside the triangle $ABC$. The lines through $P$ parallel to $CA$ and $CB$ meet $AB$ at $D$ and $E$, respectively. The line through $P$ parallel to $AB$ meets $CA$ and $CB$ at $F$ and $G$, respectively. Prove that the lines $DF$ and $EG$ intersect on the circumcircle of the triangle $ABC$.}
		
		
		
		
		\prob{}{}{}{Let $A_1A_2A_3\ldots A_n$ be a regular $n$-gon. Let $B_1$ and $B_n$ be the midpoints of its sides $A_1A_2$ and $A_{n-1}A_n$. Also, for every $i\in\left\{2,3,4,\ldots ,n-1\right\}$. Let $S$ be the point of intersection of the lines $A_1A_{i+1}$ and $A_nA_i$, and let $B_i$ be the point of intersection of the angle bisector bisector of the angle $\measuredangle A_iSA_{i+1}$ with the segment $A_iA_{i+1}$.
			
			Prove that $\sum_{i=1}^{n-1} \measuredangle A_1B_iA_n=180^{\circ}$.}
		
		
		
		\prob{}{}{}{Let $\triangle ABC$ be an acute-angled triangle with $AB \not= AC$. Let $H$ be the orthocenter of triangle $ABC$, and let $M$ be the midpoint of the side $BC$. Let $D$ be a point on the side $AB$ and $E$ a point on the side $AC$ such that $AE=AD$ and the points $D$, $H$, $E$ are on the same line. Prove that the line $HM$ is perpendicular to the common chord of the circumscribed circles of triangle $\triangle ABC$ and triangle $\triangle ADE$.}
		
		
		
		
		\prob{}{}{}{In triangle $ABC$, let $J$ be the center of the excircle tangent to side $BC$ at $A_{1}$ and to the extensions of the sides $AC$ and $AB$ at $B_{1}$ and $C_{1}$ respectively. Suppose that the lines $A_{1}B_{1}$ and $AB$ are perpendicular and intersect at $D$. Let $E$ be the foot of the perpendicular from $C_{1}$ to line $DJ$. Determine the angles $\angle{BEA_{1}}$ and $\angle{AEB_{1}}$.}
		
		
		
		\prob{}{}{}{Let $ ABC$ be a fixed triangle, and let $ A_1$, $ B_1$, $ C_1$ be the midpoints of sides $ BC$, $ CA$, $ AB$, respectively. Let $ P$ be a variable point on the circumcircle. Let lines $ PA_1$, $ PB_1$, $ PC_1$ meet the circumcircle again at $ A'$, $ B'$, $ C'$, respectively. Assume that the points $ A$, $ B$, $ C$, $ A'$, $ B'$, $ C'$ are distinct, and lines $ AA'$, $ BB'$, $ CC'$ form a triangle. Prove that the area of this triangle does not depend on $ P$.}
		
		
		
		
		\prob{}{}{}{Let $ k$ and $ n$ be integers with $ 0\le k\le n - 2$. Consider a set $ L$ of $ n$ lines in the plane such that no two of them are parallel and no three have a common point. Denote by $ I$ the set of intersections of lines in $ L$. Let $ O$ be a point in the plane not lying on any line of $ L$. A point $ X\in I$ is colored red if the open line segment $ OX$ intersects at most $ k$ lines in $ L$. Prove that $ I$ contains at least $ \dfrac{1}{2}(k + 1)(k + 2)$ red points.}
		
		
		
		
		\prob{}{}{}{Let $P$ be a polygon that is convex and symmetric to some point $O$. Prove that for some parallelogram $R$ satisfying $P\subset R$ we have \[\frac{|R|}{|P|}\leq \sqrt 2\]
			where $|R|$ and $|P|$ denote the area of the sets $R$ and $P$, respectively.}
		
		
		
		
		\prob{}{}{}{Let $ABCDE$ be a convex pentagon such that $BC \parallel AE,$ $AB = BC + AE,$ and $\angle ABC = \angle CDE.$ Let $M$ be the midpoint of $CE,$ and let $O$ be the circumcenter of triangle $BCD.$ Given that $\angle DMO = 90^{\circ},$ prove that $2 \angle BDA = \angle CDE.$}
		
		
		
		
		\prob{}{ISL 2011 G5}{}{Let $ABC$ be a triangle with incentre $I$ and circumcircle $\omega$. Let $D$ and $E$ be the second intersection points of $\omega$ with $AI$ and $BI$, respectively. The chord $DE$ meets $AC$ at a point $F$, and $BC$ at a point $G$. Let $P$ be the intersection point of the line through $F$ parallel to $AD$ and the line through $G$ parallel to $BE$. Suppose that the tangents to $\omega$ at $A$ and $B$ meet at a point $K$. Prove that the three lines $AE,BD$ and $KP$ are either parallel or concurrent.}
		
		
		
		
		\prob{}{}{}{Let $ABC$ be a triangle with $\angle BCA=90^{\circ}$, and let $D$ be the foot of the altitude from $C$. Let $X$ be a point in the interior of the segment $CD$. Let $K$ be the point on the segment $AX$ such that $BK=BC$. Similarly, let $L$ be the point on the segment $BX$ such that $AL=AC$. Let $M$ be the point of intersection of $AL$ and $BK$.
			
			Show that $MK=ML$.}
		
		
		
		
		\prob{}{}{}{Let $ABCDEF$ be a convex hexagon with $AB=DE$, $BC=EF$, $CD=FA$, and $\angle A-\angle D = \angle C -\angle F = \angle E -\angle B$. Prove that the diagonals $AD$, $BE$, and $CF$ are concurrent.}
		
		
		
		\prob{}{}{}{Convex quadrilateral $ABCD$ has $\angle ABC = \angle CDA = 90^{\circ}$. Point $H$ is the foot of the perpendicular from $A$ to $BD$. Points $S$ and $T$ lie on sides $AB$ and $AD$, respectively, such that $H$ lies inside triangle $SCT$ and \[ \angle CHS - \angle CSB = 90^{\circ}, \quad \angle THC - \angle DTC = 90^{\circ}. \] Prove that line $BD$ is tangent to the circumcircle of triangle $TSH$.}
		
		
		
		
		\prob{}{}{}{Let $ABC$ be a triangle with $CA \neq CB$. Let $D$, $F$, and $G$ be the midpoints of the sides $AB$, $AC$, and $BC$ respectively. A circle $\Gamma$ passing through $C$ and tangent to $AB$ at $D$ meets the segments $AF$ and $BG$ at $H$ and $I$, respectively. The points $H'$ and $I'$ are symmetric to $H$ and $I$ about $F$ and $G$, respectively. The line $H'I'$ meets $CD$ and $FG$ at $Q$ and $M$, respectively. The line $CM$ meets $\Gamma$ again at $P$. Prove that $CQ = QP$.}
		
		
		
		
		\prob{}{}{}{Let $D$ be the foot of perpendicular from $A$ to the Euler line (the line passing through the circumcentre and the orthocentre) of an acute scalene triangle $ABC$. A circle $\omega$ with centre $S$ passes through $A$ and $D$, and it intersects sides $AB$ and $AC$ at $X$ and $Y$ respectively. Let $P$ be the foot of altitude from $A$ to $BC$, and let $M$ be the midpoint of $BC$. Prove that the circumcentre of triangle $XSY$ is equidistant from $P$ and $M$.}
		
		
		
		
		
		\prob{}{}{}{Let $ABCC_1B_1A_1$ be a convex hexagon such that $AB=BC$, and suppose that the line segments $AA_1, BB_1$, and $CC_1$ have the same perpendicular bisector. Let the diagonals $AC_1$ and $A_1C$ meet at $D$, and denote by $\omega$ the circle $ABC$. Let $\omega$ intersect the circle $A_1BC_1$ again at $E \neq B$. Prove that the lines $BB_1$ and $DE$ intersect on $\omega$.}
		
		
		
		
		
	\newpage\section{G6}
	
	
		\prob{}{}{}{Let $ ABCD$ be a convex quadrilateral. The perpendicular bisectors of its sides $ AB$ and $ CD$ meet at $ Y$. Denote by $ X$ a point inside the quadrilateral $ ABCD$ such that $ \measuredangle ADX = \measuredangle BCX < 90^{\circ}$ and $ \measuredangle DAX = \measuredangle CBX < 90^{\circ}$. Show that $ \measuredangle AYB = 2\cdot\measuredangle ADX$.}
		
		
		
		
		
		\prob{}{}{}{Let $ABC$ be a triangle and $P$ an exterior point in the plane of the triangle. Suppose the lines $AP$, $BP$, $CP$ meet the sides $BC$, $CA$, $AB$ (or extensions thereof) in $D$, $E$, $F$, respectively. Suppose further that the areas of triangles $PBD$, $PCE$, $PAF$ are all equal. Prove that each of these areas is equal to the area of triangle $ABC$ itself.}
		
		
		
		\prob{}{}{}{Let $n\geq3$ be a positive integer. Let $C_1,C_2,C_3,\ldots,C_n$ be unit circles in the plane, with centres $O_1,O_2,O_3,\ldots,O_n$ respectively. If no line meets more than two of the circles, prove that \[ \sum\limits^{}_{1\leq i<j\leq n}{1\over O_iO_j}\leq{(n-1)\pi\over 4}. \]}
		
		
		
		\prob{}{}{}{Each pair of opposite sides of a convex hexagon has the following property: the distance between their midpoints is equal to $\dfrac{\sqrt{3}}{2}$ times the sum of their lengths. Prove that all the angles of the hexagon are equal.}
		
		
		
		\prob{}{}{}{Let $P$ be a convex polygon. Prove that there exists a convex hexagon that is contained in $P$ and whose area is at least $\frac34$ of the area of the polygon $P$.
			
			Alternative version. Let $P$ be a convex polygon with $n\geq 6$ vertices. Prove that there exists a convex hexagon with
			
			a) vertices on the sides of the polygon (or)
			b) vertices among the vertices of the polygon
			
			such that the area of the hexagon is at least $\frac{3}{4}$ of the area of the polygon.}
		
		
		
		
		\prob{}{}{}{Let $ABC$ be a triangle, and $M$ the midpoint of its side $BC$. Let $\gamma$ be the incircle of triangle $ABC$. The median $AM$ of triangle $ABC$ intersects the incircle $\gamma$ at two points $K$ and $L$. Let the lines passing through $K$ and $L$, parallel to $BC$, intersect the incircle $\gamma$ again in two points $X$ and $Y$. Let the lines $AX$ and $AY$ intersect $BC$ again at the points $P$ and $Q$. Prove that $BP = CQ$.}
		
		
		
		
		\prob{}{}{}{Circles $ w_{1}$ and $ w_{2}$ with centres $ O_{1}$ and $ O_{2}$ are externally tangent at point $ D$ and internally tangent to a circle $ w$ at points $ E$ and $ F$ respectively. Line $ t$ is the common tangent of $ w_{1}$ and $ w_{2}$ at $ D$. Let $ AB$ be the diameter of $ w$ perpendicular to $ t$, so that $ A, E, O_{1}$ are on the same side of $ t$. Prove that lines $ AO_{1}$, $ BO_{2}$, $ EF$ and $ t$ are concurrent.}
		
		
		
		\prob{}{}{}{Determine the smallest positive real number $ k$ with the following property. Let $ ABCD$ be a convex quadrilateral, and let points $ A_1$, $ B_1$, $ C_1$, and $ D_1$ lie on sides $ AB$, $ BC$, $ CD$, and $ DA$, respectively. Consider the areas of triangles $ AA_1D_1$, $ BB_1A_1$, $ CC_1B_1$ and $ DD_1C_1$; let $ S$ be the sum of the two smallest ones, and let $ S_1$ be the area of quadrilateral $ A_1B_1C_1D_1$. Then we always have $ kS_1\ge S$.}
		
		
		
		
		\prob{}{}{}{There is given a convex quadrilateral $ ABCD$. Prove that there exists a point $ P$ inside the quadrilateral such that
			
			$\quad \angle PAB + \angle PDC = \angle PBC + \angle PAD = \angle PCD + \angle PBA = \angle PDA + \angle PCB =$ $ 90^{\circ}$
			
			if and only if the diagonals $ AC$ and $ BD$ are perpendicular.}
		
		
		
		
		\prob{}{}{}{Let the sides $AD$ and $BC$ of the quadrilateral $ABCD$ (such that $AB$ is not parallel to $CD$) intersect at point $P$. Points $O_1$ and $O_2$ are circumcenters and points $H_1$ and $H_2$ are orthocenters of triangles $ABP$ and $CDP$, respectively. Denote the midpoints of segments $O_1H_1$ and $O_2H_2$ by $E_1$ and $E_2$, respectively. Prove that the perpendicular from $E_1$ on $CD$, the perpendicular from $E_2$ on $AB$ and the lines $H_1H_2$ are concurrent.}
		
		
		
		
		\prob{}{}{}{The vertices $X, Y , Z$ of an equilateral triangle $XYZ$ lie respectively on the sides $BC, CA, AB$ of an acute-angled triangle $ABC.$ Prove that the incenter of triangle $ABC$ lies inside triangle $XYZ.$}
		
		
		
		
		
		\prob{}{}{}{Let $ABC$ be a triangle with $AB=AC$ and let $D$ be the midpoint of $AC$. The angle bisector of $\angle BAC$ intersects the circle through $D,B$ and $C$ at the point $E$ inside the triangle $ABC$. The line $BD$ intersects the circle through $A,E$ and $B$ in two points $B$ and $F$. The lines $AF$ and $BE$ meet at a point $I$, and the lines $CI$ and $BD$ meet at a point $K$. Show that $I$ is the incentre of triangle $KAB$.
			%\fig{.4}{ISL_GEO/5_6_12}{}
		}
			
			\solu{Draw the circle centered at $ D $ with radius $ DA=DC $}
		
		
		
		\prob{}{}{}{Let $ABC$ be a triangle with circumcenter $O$ and incenter $I$. The points $D,E$ and $F$ on the sides $BC,CA$ and $AB$ respectively are such that $BD+BF=CA$ and $CD+CE=AB$. The circumcircles of the triangles $BFD$ and $CDE$ intersect at $P \neq D$. Prove that $OP=OI$.}
		
		
		
		
		
		\prob{}{}{}{Let the excircle of triangle $ABC$ opposite the vertex $A$ be tangent to the side $BC$ at the point $A_1$. Define the points $B_1$ on $CA$ and $C_1$ on $AB$ analogously, using the excircles opposite $B$ and $C$, respectively. Suppose that the circumcentre of triangle $A_1B_1C_1$ lies on the circumcircle of triangle $ABC$. Prove that triangle $ABC$ is right-angled.}
		
		
		
		
		\prob{}{}{}{Let $ABC$ be a fixed acute-angled triangle. Consider some points $E$ and $F$ lying on the sides $AC$ and $AB$, respectively, and let $M$ be the midpoint of $EF$ . Let the perpendicular bisector of $EF$ intersect the line $BC$ at $K$, and let the perpendicular bisector of $MK$ intersect the lines $AC$ and $AB$ at $S$ and $T$ , respectively. We call the pair $(E, F )$ $\textit{interesting}$, if the quadrilateral $KSAT$ is cyclic.
			Suppose that the pairs $(E_1 , F_1 )$ and $(E_2 , F_2 )$ are interesting. Prove that $\displaystyle\frac{E_1 E_2}{AB}=\frac{F_1 F_2}{AC}$}
		
		
		
		
		\prob{}{}{}{Let $ABC$ be an acute triangle with $AB > AC$. Let $\Gamma $ be its cirumcircle, $H$ its orthocenter, and $F$ the foot of the altitude from $A$. Let $M$ be the midpoint of $BC$. Let $Q$ be the point on $\Gamma$ such that $\angle HQA = 90^{\circ}$ and let $K$ be the point on $\Gamma$ such that $\angle HKQ = 90^{\circ}$. Assume that the points $A$, $B$, $C$, $K$ and $Q$ are all different and lie on $\Gamma$ in this order.
			
			Prove that the circumcircles of triangles $KQH$ and $FKM$ are tangent to each other.}
		
		
		
		
		
		\prob{}{}{}{Let $ABCD$ be a convex quadrilateral with $\angle ABC = \angle ADC < 90^{\circ}$. The internal angle bisectors of $\angle ABC$ and $\angle ADC$ meet $AC$ at $E$ and $F$ respectively, and meet each other at point $P$. Let $M$ be the midpoint of $AC$ and let $\omega$ be the circumcircle of triangle $BPD$. Segments $BM$ and $DM$ intersect $\omega$ again at $X$ and $Y$ respectively. Denote by $Q$ the intersection point of lines $XE$ and $YF$. Prove that $PQ \perp AC$.}
		
		
		
		
		
		\prob{}{}{}{Let $n\ge3$ be an integer. Two regular $n$-gons $\mathcal{A}$ and $\mathcal{B}$ are given in the plane. Prove that the vertices of $\mathcal{A}$ that lie inside $\mathcal{B}$ or on its boundary are consecutive.
			
			(That is, prove that there exists a line separating those vertices of $\mathcal{A}$ that lie inside $\mathcal{B}$ or on its boundary from the other vertices of $\mathcal{A}$.)}
	
		
		
	\newpage\section{G7}
	
		
		\prob{}{}{}{Ten gangsters are standing on a flat surface, and the distances between them are all distinct. At twelve o’clock, when the church bells start chiming, each of them fatally shoots the one among the other nine gangsters who is the nearest. At least how many gangsters will be killed?}
		
		
		\prob{}{}{}{Let $O$ be an interior point of acute triangle $ABC$. Let $A_1$ lie on $BC$ with $OA_1$ perpendicular to $BC$. Define $B_1$ on $CA$ and $C_1$ on $AB$ similarly. Prove that $O$ is the circumcenter of $ABC$ if and only if the perimeter of $A_1B_1C_1$ is not less than any one of the perimeters of $AB_1C_1, BC_1A_1$, and $CA_1B_1$.}
		
		
		
		\prob{}{}{}{The incircle $ \Omega$ of the acute-angled triangle $ ABC$ is tangent to its side $ BC$ at a point $ K$. Let $ AD$ be an altitude of triangle $ ABC$, and let $ M$ be the midpoint of the segment $ AD$. If $ N$ is the common point of the circle $ \Omega$ and the line $ KM$ (distinct from $ K$), then prove that the incircle $ \Omega$ and the circumcircle of triangle $ BCN$ are tangent to each other at the point $ N$.}
		
		
		
		\prob{}{}{}{Let $ABC$ be a triangle with semiperimeter $s$ and inradius $r$. The semicircles with diameters $BC$, $CA$, $AB$ are drawn on the outside of the triangle $ABC$. The circle tangent to all of these three semicircles has radius $t$. Prove that
			\[\frac{s}{2}<t\le\frac{s}{2}+\left(1-\frac{\sqrt{3}}{2}\right)r. \]
			Alternative formulation. In a triangle $ABC$, construct circles with diameters $BC$, $CA$, and $AB$, respectively. Construct a circle $w$ externally tangent to these three circles. Let the radius of this circle $w$ be $t$.
			Prove: $\frac{s}{2}<t\le\frac{s}{2}+\frac12\left(2-\sqrt3\right)r$, where $r$ is the inradius and $s$ is the semiperimeter of triangle $ABC$.}
		
		
		
		
		\prob{}{}{}{For a given triangle $ ABC$, let $ X$ be a variable point on the line $ BC$ such that $ C$ lies between $ B$ and $ X$ and the incircles of the triangles $ ABX$ and $ ACX$ intersect at two distinct points $ P$ and $ Q.$ Prove that the line $ PQ$ passes through a point independent of $ X$.}
		
		
		
		
		\prob{}{}{}{In an acute triangle $ABC$, let $D$, $E$, $F$ be the feet of the perpendiculars from the points $A$, $B$, $C$ to the lines $BC$, $CA$, $AB$, respectively, and let $P$, $Q$, $R$ be the feet of the perpendiculars from the points $A$, $B$, $C$ to the lines $EF$, $FD$, $DE$, respectively.
			
			Prove that $p\left(ABC\right)p\left(PQR\right) \ge \left(p\left(DEF\right)\right)^{2}$, where $p\left(T\right)$ denotes the perimeter of triangle $T$ .}
		
		
		
		
		\prob{}{}{}{In a triangle $ ABC$, let $ M_{a}$, $ M_{b}$, $ M_{c}$ be the midpoints of the sides $ BC$, $ CA$, $ AB$, respectively, and $ T_{a}$, $ T_{b}$, $ T_{c}$ be the midpoints of the arcs $ BC$, $ CA$, $ AB$ of the circumcircle of $ ABC$, not containing the vertices $ A$, $ B$, $ C$, respectively. For $ i \in \left\{a, b, c\right\}$, let $ w_{i}$ be the circle with $ M_{i}T_{i}$ as diameter. Let $ p_{i}$ be the common external common tangent to the circles $ w_{j}$ and $ w_{k}$ (for all $ \left\{i, j, k\right\}= \left\{a, b, c\right\}$) such that $ w_{i}$ lies on the opposite side of $ p_{i}$ than $ w_{j}$ and $ w_{k}$ do.
			Prove that the lines $ p_{a}$, $ p_{b}$, $ p_{c}$ form a triangle similar to $ ABC$ and find the ratio of similitude.}
		
		
		
		
		\prob{}{}{}{Given an acute triangle $ ABC$ with $ \angle B > \angle C$. Point $ I$ is the incenter, and $ R$ the circumradius. Point $ D$ is the foot of the altitude from vertex $ A$. Point $ K$ lies on line $ AD$ such that $ AK = 2R$, and $ D$ separates $ A$ and $ K$. Lines $ DI$ and $ KI$ meet sides $ AC$ and $ BC$ at $ E,F$ respectively. Let $ IE = IF$.
			
			Prove that $ \angle B\leq 3\angle C$.}
		
		
		
		
		\prob{}{}{}{Let $ ABCD$ be a convex quadrilateral with $ BA\neq BC$. Denote the incircles of triangles $ ABC$ and $ ADC$ by $ \omega_{1}$ and $ \omega_{2}$ respectively. Suppose that there exists a circle $ \omega$ tangent to ray $ BA$ beyond $ A$ and to the ray $ BC$ beyond $ C$, which is also tangent to the lines $ AD$ and $ CD$. Prove that the common external tangents to $ \omega_{1}$ and $\omega_{2}$ intersect on $ \omega$.}
		
		
		
		
		\prob{}{}{}{Let $ABC$ be a triangle with incenter $I$ and let $X$, $Y$ and $Z$ be the incenters of the triangles $BIC$, $CIA$ and $AIB$, respectively. Let the triangle $XYZ$ be equilateral. Prove that $ABC$ is equilateral too.}
		
		
		
		
		\prob{}{}{}{Three circular arcs $\gamma_1, \gamma_2,$ and $\gamma_3$ connect the points $A$ and $C.$ These arcs lie in the same half-plane defined by line $AC$ in such a way that arc $\gamma_2$ lies between the arcs $\gamma_1$ and $\gamma_3.$ Point $B$ lies on the segment $AC.$ Let $h_1, h_2$, and $h_3$ be three rays starting at $B,$ lying in the same half-plane, $h_2$ being between $h_1$ and $h_3.$ For $i, j = 1, 2, 3,$ denote by $V_{ij}$ the point of intersection of $h_i$ and $\gamma_j$ (see the Figure below). Denote by $\widehat{V_{ij}V_{kj}}\widehat{V_{kl}V_{il}}$ the curved quadrilateral, whose sides are the segments $V_{ij}V_{il},$ $V_{kj}V_{kl}$ and arcs $V_{ij}V_{kj}$ and $V_{il}V_{kl}.$ We say that this quadrilateral is $circumscribed$ if there exists a circle touching these two segments and two arcs. Prove that if the curved quadrilaterals $\widehat{V_{11}V_{21}}\widehat{V_{22}V_{12}}, \widehat{V_{12}V_{22}}\widehat{V_{23}V_{13}},\widehat{V_{21}V_{31}}\widehat{V_{32}V_{22}}$ are circumscribed, then the curved quadrilateral $\widehat{V_{22}V_{32}}\widehat{V_{33}V_{23}}$ is circumscribed, too.}
		
		
		
		
		\prob{}{}{}{Let $ABCDEF$ be a convex hexagon all of whose sides are tangent to a circle $\omega$ with centre $O$. Suppose that the circumcircle of triangle $ACE$ is concentric with $\omega$. Let $J$ be the foot of the perpendicular from $B$ to $CD$. Suppose that the perpendicular from $B$ to $DF$ intersects the line $EO$ at a point $K$. Let $L$ be the foot of the perpendicular from $K$ to $DE$. Prove that $DJ=DL$.}
		
		
		
		
		\prob{}{}{}{Let $ABCD$ be a convex quadrilateral with non-parallel sides $BC$ and $AD$. Assume that there is a point $E$ on the side $BC$ such that the quadrilaterals $ABED$ and $AECD$ are circumscribed. Prove that there is a point $F$ on the side $AD$ such that the quadrilaterals $ABCF$ and $BCDF$ are circumscribed if and only if $AB$ is parallel to $CD$.}
		
		
		
		
		\prob{}{}{}{Let $ABC$ be a triangle with circumcircle $\Omega$ and incentre $I$. Let the line passing through $I$ and perpendicular to $CI$ intersect the segment $BC$ and the arc $BC$ (not containing $A$) of $\Omega$ at points $U$ and $V$ , respectively. Let the line passing through $U$ and parallel to $AI$ intersect $AV$ at $X$, and let the line passing through $V$ and parallel to $AI$ intersect $AB$ at $Y$ . Let $W$ and $Z$ be the midpoints of $AX$ and $BC$, respectively. Prove that if the points $I, X,$ and $Y$ are collinear, then the points $I, W ,$ and $Z$ are also collinear.}
		
		
		
		
		
		\prob{}{}{}{Let $ABCD$ be a convex quadrilateral, and let $P$, $Q$, $R$, and $S$ be points on the sides $AB$, $BC$, $CD$, and $DA$, respectively. Let the line segment $PR$ and $QS$ meet at $O$. Suppose that each of the quadrilaterals $APOS$, $BQOP$, $CROQ$, and $DSOR$ has an incircle. Prove that the lines $AC$, $PQ$, and $RS$ are either concurrent or parallel to each other.}
		
		
		
		
		
		
		\prob{}{}{}{Let $I$ be the incentre of a non-equilateral triangle $ABC$, $I_A$ be the $A$-excentre, $I'_A$ be the reflection of $I_A$ in $BC$, and $l_A$ be the reflection of line $AI'_A$ in $AI$. Define points $I_B$, $I'_B$ and line $l_B$ analogously. Let $P$ be the intersection point of $l_A$ and $l_B$.
			
			Prove that $P$ lies on line $OI$ where $O$ is the circumcentre of triangle $ABC$.
			Let one of the tangents from $P$ to the incircle of triangle $ABC$ meet the circumcircle at points $X$ and $Y$. Show that $\angle XIY = 120^{\circ}$.}
		
		
		
		
		
		\prob{}{}{}{A convex quadrilateral $ABCD$ has an inscribed circle with center $I$. Let $I_a, I_b, I_c$ and $I_d$ be the incenters of the triangles $DAB, ABC, BCD$ and $CDA$, respectively. Suppose that the common external tangents of the circles $AI_bI_d$ and $CI_bI_d$ meet at $X$, and the common external tangents of the circles $BI_aI_c$ and $DI_aI_c$ meet at $Y$. Prove that $\angle{XIY}=90^{\circ}$.}
		
		
		
		
		
	\newpage\section{G8}
	
		\prob{}{}{}{Let $ AH_1, BH_2, CH_3$ be the altitudes of an acute angled triangle $ ABC$. Its incircle touches the sides $ BC, AC$ and $ AB$ at $ T_1, T_2$ and $ T_3$ respectively. Consider the symmetric images of the lines $ H_1H_2, H_2H_3$ and $ H_3H_1$ with respect to the lines $ T_1T_2, T_2T_3$ and $ T_3T_1$. Prove that these images form a triangle whose vertices lie on the incircle of $ ABC$.}
		
		
		
		\prob{}{}{}{Let $ABC$ be a triangle with $\angle BAC = 60^{\circ}$. Let $AP$ bisect $\angle BAC$ and let $BQ$ bisect $\angle ABC$, with $P$ on $BC$ and $Q$ on $AC$. If $AB + BP = AQ + QB$, what are the angles of the triangle?}
		
		
		
		
		\prob{}{}{}{Let two circles $S_{1}$ and $S_{2}$ meet at the points $A$ and $B$. A line through $A$ meets $S_{1}$ again at $C$ and $S_{2}$ again at $D$. Let $M$, $N$, $K$ be three points on the line segments $CD$, $BC$, $BD$ respectively, with $MN$ parallel to $BD$ and $MK$ parallel to $BC$. Let $E$ and $F$ be points on those arcs $BC$ of $S_{1}$ and $BD$ of $S_{2}$ respectively that do not contain $A$. Given that $EN$ is perpendicular to $BC$ and $FK$ is perpendicular to $BD$ prove that $\angle EMF=90^{\circ}$.}
		
		
		
		
		\prob{}{}{}{Given a cyclic quadrilateral $ABCD$, let $M$ be the midpoint of the side $CD$, and let $N$ be a point on the circumcircle of triangle $ABM$. Assume that the point $N$ is different from the point $M$ and satisfies $\frac{AN}{BN}=\frac{AM}{BM}$. Prove that the points $E$, $F$, $N$ are collinear, where $E=AC\cap BD$ and $F=BC\cap DA$.}
		
		
		
		
		
		\prob{}{}{}{Assign to each side $b$ of a convex polygon $P$ the maximum area of a triangle that has $b$ as a side and is contained in $P$. Show that the sum of the areas assigned to the sides of $P$ is at least twice the area of $P$.}
		
		
		
		
		
		\prob{}{}{}{Points $ A_{1}$, $ B_{1}$, $ C_{1}$ are chosen on the sides $ BC$, $ CA$, $ AB$ of a triangle $ ABC$ respectively. The circumcircles of triangles $ AB_{1}C_{1}$, $ BC_{1}A_{1}$, $ CA_{1}B_{1}$ intersect the circumcircle of triangle $ ABC$ again at points $ A_{2}$, $ B_{2}$, $ C_{2}$ respectively ($ A_{2}\neq A, B_{2}\neq B, C_{2}\neq C$). Points $ A_{3}$, $ B_{3}$, $ C_{3}$ are symmetric to $ A_{1}$, $ B_{1}$, $ C_{1}$ with respect to the midpoints of the sides $ BC$, $ CA$, $ AB$ respectively. Prove that the triangles $ A_{2}B_{2}C_{2}$ and $ A_{3}B_{3}C_{3}$ are similar.}
		
		
		
		
		
		\prob{}{}{}{Let $ABCD$ be a convex quadrilateral. A circle passing through the points $A$ and $D$ and a circle passing through the points $B$ and $C$ are externally tangent at a point $P$ inside the quadrilateral. Suppose that \[\angle{PAB}+\angle{PDC}\leq 90^\circ\qquad\text{and}\qquad\angle{PBA}+\angle{PCD}\leq 90^\circ.\] Prove that $AB+CD \geq BC+AD$.}
		
		
		
		
		\prob{}{}{}{Point $ P$ lies on side $ AB$ of a convex quadrilateral $ ABCD$. Let $ \omega$ be the incircle of triangle $ CPD$, and let $ I$ be its incenter. Suppose that $ \omega$ is tangent to the incircles of triangles $ APD$ and $ BPC$ at points $ K$ and $ L$, respectively. Let lines $ AC$ and $ BD$ meet at $ E$, and let lines $ AK$ and $ BL$ meet at $ F$. Prove that points $ E$, $ I$, and $ F$ are collinear.}
		
		
		
		
		
		\prob{}{}{}{Let $ABCD$ be a circumscribed quadrilateral. Let $g$ be a line through $A$ which meets the segment $BC$ in $M$ and the line $CD$ in $N$. Denote by $I_1$, $I_2$ and $I_3$ the incenters of $\triangle ABM$, $\triangle MNC$ and $\triangle NDA$, respectively. Prove that the orthocenter of $\triangle I_1I_2I_3$ lies on $g$.}
		
		
		
		
		
		\prob{}{}{}{Let $ABC$ be an acute triangle with circumcircle $\Gamma$. Let $\ell$ be a tangent line to $\Gamma$, and let $\ell_a, \ell_b$ and $\ell_c$ be the lines obtained by reflecting $\ell$ in the lines $BC$, $CA$ and $AB$, respectively. Show that the circumcircle of the triangle determined by the lines $\ell_a, \ell_b$ and $\ell_c$ is tangent to the circle $\Gamma$.}
		
		
		
		
		
		\prob{}{}{}{Let $ABC$ be a triangle with circumcircle $\omega$ and $\ell$ a line without common points with $\omega$. Denote by $P$ the foot of the perpendicular from the center of $\omega$ to $\ell$. The side-lines $BC,CA,AB$ intersect $\ell$ at the points $X,Y,Z$ different from $P$. Prove that the circumcircles of the triangles $AXP$, $BYP$ and $CZP$ have a common point different from $P$ or are mutually tangent at $P$.}
		
		
		
		
		
		\prob{}{}{}{A triangulation of a convex polygon $\Pi$ is a partitioning of $\Pi$ into triangles by diagonals having no common points other than the vertices of the polygon. We say that a triangulation is a Thaiangulation if all triangles in it have the same area.
			
			Prove that any two different Thaiangulations of a convex polygon $\Pi$ differ by exactly two triangles. (In other words, prove that it is possible to replace one pair of triangles in the first Thaiangulation with a different pair of triangles so as to obtain the second Thaiangulation.)}
		
		
		
		
		
		
		\prob{}{}{}{Let $A_1, B_1$ and $C_1$ be points on sides $BC$, $CA$ and $AB$ of an acute triangle $ABC$ respectively, such that $AA_1$, $BB_1$ and $CC_1$ are the internal angle bisectors of triangle $ABC$. Let $I$ be the incentre of triangle $ABC$, and $H$ be the orthocentre of triangle $A_1B_1C_1$. Show that $$AH + BH + CH \geq AI + BI + CI.$$}
		
		
		
		
		
		\prob{}{}{}{There are $2017$ mutually external circles drawn on a blackboard, such that no two are tangent and no three share a common tangent. A tangent segment is a line segment that is a common tangent to two circles, starting at one tangent point and ending at the other one. Luciano is drawing tangent segments on the blackboard, one at a time, so that no tangent segment intersects any other circles or previously drawn tangent segments. Luciano keeps drawing tangent segments until no more can be drawn.}