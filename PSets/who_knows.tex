\documentclass[12pt]{article}
\usepackage{amssymb, mathtools, amsmath, enumitem, amsfonts, fancyhdr, float, amsthm, chngcntr, amssymb, enumitem, graphicx, hyperref}

\title{\vspace{-.1\textheight}Nemo Ended?!?! Motei nOy}


\newtheorem{problem}{Problem}


\begin{document}
	
	
	
	
	\maketitle
	
	
	\section*{}
	
		Hey Revived NEMO participant. It would be better if you can give this contest in a contest situation(Timed and without distractions). This NEMO has two sets of questions. One for the choto babus and one for the boro babus. You can give whichever you like. You can give both if you're ambitious. You have four and a half hours. Good luck!!!"
	
	
	
	
	\vspace{.1\textheight}
	
	
	\section{Choto Babuder Problems}
		
	
	\bigskip 
	
	\begin{problem}
		\href{https://artofproblemsolving.com/community/c321h125566p712040}{heh}
		Find the sum of all positive integers $n$ for which $n^2-19n+99$ is a perfect square.
	\end{problem}
	
	
	\bigskip
	
	
	\begin{problem}
		\href{https://artofproblemsolving.com/community/c4h87212p508856}{heh}
		
		One hundred concentric circles with radii $1, 2, 3, \dots, 100$ are drawn in a plane. The interior of the circle of radius 1 is colored red, and each region bounded by consecutive circles is colored either red or green, with no two adjacent regions the same color. The ratio of the total area of the green regions to the area of the circle of radius 100 can be expressed as $m/n$, where $m$ and $n$ are relatively prime positive integers. Find $m + n$.
	\end{problem}
	
	
	\bigskip
	
	
	\begin{problem}
		\href{https://artofproblemsolving.com/wiki/index.php/2018_AIME_II_Problems/Problem_1}{heh}
		
		Points $A$, $B$, and $C$ lie in that order along a straight path where the distance from $A$ to $C$ is $1800$ meters. Ina runs twice as fast as Eve, and Paul runs twice as fast as Ina. The three runners start running at the same time with Ina starting at $A$ and running toward $C$, Paul starting at $B$ and running toward $C$, and Eve starting at $C$ and running toward $A$. When Paul meets Eve, he turns around and runs toward $A$. Paul and Ina both arrive at $B$ at the same time. Find the number of meters from $A$ to $B$.
	\end{problem}
	
	
	\bigskip
	
	
	\begin{problem}
		\href{https://artofproblemsolving.com/wiki/index.php/1998_AIME_Problems/Problem_6}{heh}
		
		Let $ABCD$ be a parallelogram. Extend $\overline{DA}$ through $A$ to a point $P,$ and let $\overline{PC}$ meet $\overline{AB}$ at $Q$ and $\overline{DB}$ at $R.$ Given that $PQ=735$ and $QR=112,$ find $RC.$
	\end{problem}
	
	\bigskip
	
	
	\begin{problem}
		\href{https://artofproblemsolving.com/wiki/index.php/1999_AIME_Problems/Problem_4}{heh}
		
		The two squares shown share the same center $O$ and have sides of length 1. The length of $\overline{AB}$ is $43/99$ and the area of octagon $ABCDEFGH$ is $m/n,$ where $m$ and $n$ are relatively prime positive integers. Find $m+n.$
		
		\begin{figure}[H] 
			\centering 
			\includegraphics[width=.4\textwidth]{p2.png} 
			\caption{Problem 4} 
		\end{figure}
	\end{problem}
	
	
	
	
	
	\newpage\section{Boro Babuder Problems}
	
	
	
	\bigskip
	
	\begin{problem}
		\href{https://artofproblemsolving.com/wiki/index.php/2008_USAMO_Problems/Problem_2}{heh}
		
		Let $ABC$ be acute triangle inscribed circle $(O)$, altitude $AH$, $H$ lies on $BC$. $P$ is a point that lies on bisector $\angle BAC$ and $P$ is inside triangle $ABC$. Circle diameter $AP$ cuts $(O)$ again at $G$. $L$ is projection of $P$ on $AH$. Assume that $GL$ bisects $HP$. Prove that $P$ is incenter of $ABC$.
	\end{problem}
	
	\bigskip
	
	
	\begin{problem}
		\href{https://artofproblemsolving.com/wiki/index.php/2008_USAMO_Problems/Problem_1}{heh}
		
		Prove that for each positive integer $n$, there are pairwise relatively prime integers $k_0, k_1\dots k_n$, all strictly greater than $1$, such that $k_0k_1\dots k_n-1$ is the product of two consecutive integers.
	\end{problem}
	
	
	
	\bigskip
	
	
	
	\begin{problem}
		\href{https://artofproblemsolving.com/community/c6h1417932p7979120}{heh}
		
		Find the maximum number of queens you could put on $2017 \times 2017$ chess table such that each queen attacks at most $1$ other queen.
	\end{problem}
	\bigskip
	
	
	\begin{problem}
		\href{https://artofproblemsolving.com/community/q1h1788095p11814194}{heh}
		
		Prove that for no integer $ n$ is $ n^7 + 7$ a perfect square.
	\end{problem}
	
	
	
	
\end{document}