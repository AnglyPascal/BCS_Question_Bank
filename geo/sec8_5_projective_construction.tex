\graphicspath{{Pics/}}

\newpage\subsection{Projective Constructions}

\begin{minipage}{.6\linewidth}
    \begin{construction}[Second Intersection of Line with Conic]
        Given four points $ A, B, C, D $, no three collinear, and a point $ P
        $ on a line $ l $ passing through at most one of the four points,
        construct the point $ P'\in l $ such that $ A, B, C, D, P, P' $ line
        on the same conic.

        \solu{
            Let $ AP\cap BC = X,\ l\cap CD = Y,\ XY\cap AD=Z $. Then by
            Pascal's Hexagrummum Mysticum Theorem, we have, $\ \  P' = BZ\cap
            l $
        }
    \end{construction}
\end{minipage}\hfill%
\begin{minipage}{.35\linewidth}
    \figdf{}{constructing_second_intersection_with_conic}{}
\end{minipage}

\begin{minipage}{.6\linewidth}
    \begin{construction}[Conic touching conic]
        Given a conic $ \mathcal{C} $, and two points $ A, B $ on it, and $ C
        $ inside of it. Construct the conic $ \mathcal{H} $ that is tangent to
        $ \mathcal{C} $ at $ A, B $ and passes through $ C $.
    \end{construction}
    \solu{
        Draw the two tangest at $ A,B $ which meet at $ X $. Take an arbitrary
        line passing through $ X $ that intersects $ AC, BC $ at $ Y, Z $.
        Take $ D=BY\cap AZ $. Then $ D $ lies on $ \mathcal{H} $ by Pascal.
        Construct another point $ E $ similarly and draw the conic.
    }
\end{minipage}\hfill%
\begin{minipage}{.35\linewidth}
    \figdf{.9}{conic_touching_conic}{}
\end{minipage}

\begin{construction}[Inconic of a quadrilateral]
    Given a convex quadrilateral $ ABCD $. $ P=AC\cap BD $, $ S\in AD, T\in BC
    $ such that $ S, P, T $ are collinear. Construct the conic that touches $
    AB, CD $, and also touches $ AD, BC $ at $ S, T $ respectively.
\end{construction}

\solu{[the\textunderscore Construction] 
    Draw the polar line $ l $ of $ P $ wrt to the quadrilateral. Let $
    Z=BC\cap l $. Let $ ZS\cap AB = U $, $ ZT\cap CD = V $. Then $ SSUUTTVV $
    is our desired conic.
}	

\begin{minipage}{.5\linewidth}
    \proof{
        If $ U, V\in CD, AB $ such that $ UV $ passes through $ P $, and if
        the conic passing through $ U, V $ and tangent to $ AD, BC $ at $ S, T
        $ intersects $ CD $ at $ U' $ again, then $ SV, U'T, DB $ are
        concurrent. So to show our construction works, we just need to prove
        that $ U, V, P $ are collinear.\\

        Since Pascal's theorem works on $ SVBTUD $, we know $ S, V, B, T, U, D $
        lie on a conic $ \mathcal{H} $ and $ l $ is the pole of $ P $ wrt $
        \mathcal{H} $. Now, applying Pascal's theorem on $ TDVUBS $, and quadrilateral
        theorem on $ BTUD $ and $ BVSD $, we have, $ ST\cap UV\in AC $, which is $ P
        $. So we are done.
    }	
\end{minipage}\hfill%
\begin{minipage}{.45\linewidth}
    \figdf{}{conic_touching_quad}{}
\end{minipage}



\begin{minipage}{.5\linewidth}
    \begin{construction}[\href{http://www.geometry.ru/persons/beluhov/selected.pdf}{Sharygin
        Olympiad 2010}]
        A conic $ \mathcal{C} $ passing through the vertices of $ \triangle
        ABC $ is drawn, and three points $ A', B', C' $ on its sides $ BC, CA,
        AB $ are chosen. Then the original triangle is erased. Prove that the
        original triangle can be constructed iff $ AA', BB', CC' $ are
        concurrent.		
    \end{construction}

    \solu{[the\textunderscore Construction] 
        Draw $ B'C' $. It intersects the circle at $ X_1, X_2 $. Draw the
        conic $ \mathcal{H} $ that is tangent to $ \mathcal{C} $ at $ X_1, X_2
        $ and passes through $ A' $. Then $ BC $ is tangent to $ \mathcal{H} $
        at $ A' $.
    }

\end{minipage}\hfill%
\begin{minipage}{.45\linewidth}
    \figdf{1}{reconstruct_ABC}{}
\end{minipage}




\proof{
    The only if part is easy to prove. Becase if $ AA', BB', CC' $ aren't
    concurrent, then we can get multiple triangles $ ABC $. So suppose that
    they are concurrent. \\

    Now we define some intersetion points.
    \begin{center}
        \begin{tabular}{ccccc}
            $W_1$	&= &$BB'$	&$\cap$ &$\mathcal{C}$\\
            $S$	&=	&$X_1X_1$	&$\cap$ &$AW_1$\\
            $T$	&=	&$X_1B$	&$\cap$ &$AX_2$\\
            $U$	&=	&$X_1X_1$	&$\cap$ &$BC$\\
            $V$	&=	&$X_2X_2$	&$\cap$ &$BC$\\
            $R$	&=	&$X_2X_2$	&$\cap$ &$AW_1$\\
            $Y_1$	&=	&$A'B'$	&$\cap$ &$SR$
        \end{tabular}
    \end{center}

    $T, S, B'$ are collinear by Pascal's theorem on $ BX_1X_1X_2AW_1 $. $ T,
    B', V $ are similarly collinear for $ AX_2X_2X_1BC $. And similarly $ R,
    B', U $ are collinear.\\

    We will prove that $ \mathcal{H} $ is an inconic of $ SRVU $ that goes
    through $ A', X_1, X_2 $. \\

    For a point $ X $ on $ UV $, define $ f:UV\to UV $ such that $ f(X) $ is
    the second intersection of the conic $ X_1X_1X_2X_2X $ ($ X_1X_1 = SU,
    X_2X_2=RV  $) with $ UV $. 
    $ f $ is an involution by \autoref{theorem:Three Conic Law}. \\

    Suppose $A_1$ is the intersection with the inconic of $ SRUV $ through $
    X_1, X_2 $ and $ UV $. Let $ A_2 = X_1X_2\cap UV $. Then $ f(A_1) =A_1,
    f(A_2)=A_2, f(B)=C $. \\

    Which means, $A(B, C; A_1, A_2) = -1$. Which means $ A_1=A' $. So, $
    X_1X_2A'X_2X_2 $ is an inconic of $ SRVU $, just as we wanted.	

    \figdf{.8}{reconstruct_ABC_1}{}
}

\newpage
\begin{construction}[Focus and Directrix of a Parabola]
    First draw two parallel segments on the parabola, join their midpoints to
    get the line parallel to the axis. Then draw the main axis and find out
    the tip of the parabola. Then draw $f(x) = \frac{x}{2}$ line through $P$.
    And find the foot of the intersection of it with the parabola. It is the
    focus.
    \figdf{.5}{parabola_foci_construction}{}
\end{construction}
