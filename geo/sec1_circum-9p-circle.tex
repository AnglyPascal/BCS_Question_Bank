\graphicspath{{Pics/}}


\newpage\section{orthocenter--circumcircle--ninepoint circle}


\begin{myitemize}
\item \href{http://yufeizhao.com/olympiad/imo2008/zhao-circles.pdf}{circles - yufei zhao}
\item \href{http://yufeizhao.com/olympiad/cyclic_quad.pdf}{big picture - yufei zhao}
\item \href{http://yufeizhao.com/olympiad/power_of_a_point.pdf}{pop - yufei zhao}
\item \href{http://yufeizhao.com/olympiad/three_geometry_lemmas.pdf}{3 lemmas - yufei zhao}
\end{myitemize}


\subsection{some figures, might be good for something, i dunno}

\figdf{.4}{random_circle_pic_1}{$ h $ lies on the line, circles vary}



\subsection{problems}


\begin{minipage}{.3\textwidth}
    \figdf{1}{chinatst2018t1p3_lemm}{}
\end{minipage}\hfill%
\begin{minipage}{.65\textwidth}
    \lem{collinearity with antipode and center}{
        let $ a' $ be the antipod of $
        a $ in $ \odot abc $. let $ bdec $ be a cyclic quadrilateral with $ d\in
        ab $ and $ e\in ac $. let $ p $ be the center of $ bdec $. also, let $
        x=be\cap cd $. then $ a', p, x $ are collinear.
    }
    \solu{
        using ``the big picture" property to show that if $ q=\odot ade \cap
        \odot abc $, then $ p, x, q $ collinear and $ pq\perp aq $. which
        implies that $ p, a', q $ are collinear.
    }
\end{minipage}








\begin{minipage}{.6\textwidth}
    \prob{https://artofproblemsolving.com/community/c6h1441717p8209926}{balkan
        mo 2017 p3}{s}{
        consider an acute-angled triangle $abc$ with $ab<ac$ and let $\omega$
        be its circumscribed circle. let $t_b$ and $t_c$ be the tangents to
        the circle $\omega$ at points $b$ and $c$, respectively, and let $l$
        be their intersection. the straight line passing through the point $b$
        and parallel to $ac$ intersects $t_c$ in point $d$. the straight line
        passing through the point $c$ and parallel to $ab$ intersects $t_b$ in
        point $e$. the circumcircle of the triangle $bdc$ intersects $ac$ in
        $t$, where $t$ is located between $a$ and $c$. the circumcircle of the
        triangle $bec$ intersects the line $ab$ (or its extension) in $s$,
        where $b$ is located between $s$ and $a$. prove that $st$, $al$, and
        $bc$ are concurrent.
    }

    \solu{
        you could've thought like: symmedian is there \& and $ \triangle abc
        \rightarrow \triangle acs $ \& concurrent $ \implies $ parallel lines
        and median concurrency?

        just need to prove $ bt\parallel cs $.
    }
\end{minipage}\hfill%
\begin{minipage}{.4\textwidth}
    \figdf{.9}{balkan_mo_2017_p3}{}
\end{minipage}



\begin{minipage}{.5\textwidth}
    \prob{https://artofproblemsolving.com/community/c4512_2014_usamo}{usamo
        2014 p5}{e}{
        let $abc$ be a triangle with orthocenter $h$ and let $p$ be the second
        intersection of $\odot ahc$ with the internal bisector of $\angle
        bac$. let $x$ be the circumcenter of triangle $apb$ and $y$ the
        orthocenter of triangle $apc$. prove that the length of segment $xy$
        is equal to the circumradius of triangle $abc$.
    }

    \solu{
        no length conditions given, yet we need to prove that two lengths are
        equal. \hl{parallelogram} !!!\\
        just need to prove that $ y\in \odot abc $ \& $ yd\perp ab $
    }
\end{minipage}\hfill%
\begin{minipage}{.48\textwidth}
    \figdf{.99}{usamo_2014_p5}{}
\end{minipage}	




\begin{minipage}{.5\textwidth}
    \prob{}{bewarish 1}{e}{
        let $ def $ be the orthic triangle, and let $ ef\cap bc = p $. let the
        tangent at $ a $ to $ \odot abc $ meet $ bc $ at $ q $. let $ t $ be the
        reflection of $ q $ over $ p $. let $ k $ be the orthogonal projection of
        $ h $ on $ am $. prove that $ \angle okt = 90 $.
    }

    \solu{
        spiral similarity to get rid of $ q $ and $ t $. then spiral
        similarity again to find a trivial circle.
    }
\end{minipage}\hfill%
\begin{minipage}{.48\textwidth}
    \figdf{.99}{bewarish_1}{}
\end{minipage}








\begin{minipage}{.5\linewidth}
    \prob{https://artofproblemsolving.com/community/c74453h1225408}{buratinogigle's
        proposed problems for arab saudi team 2015}{e}{
        let $ abc $ be a triangle with orthocenter $ h $. $ p $ is a point. $
        (k) $ is the circle with diameter $ ap $. $ (k) $ cuts $ ca, ab $
        again at $ e, f $. $ ph $ cuts $ (k) $ again at $ g $. tangent line at
        $ e, f $ of $ (k) $ intersect at $ t $. $ m $ is midpoint of $ bc $. $
        l $ is the point on $ mg $ such that $ al \parallel mt. $ prove that $
        la \perp lh $.
    }

    \begin{solution}[phantom point]
        take $ l' = mg \cap azyh  $, then use spiral similarity to show that $
        al'\parallel mt $.
    \end{solution}
\end{minipage}\hfill%
\begin{minipage}{.45\linewidth}
    \figdf{}{SATST2015proposed_by_bura/derakynay1134-1}{}
\end{minipage}



\begin{minipage}{.45\linewidth}
    \figdf{}{SATST2015proposed_by_bura/derakynay1134-2}{}
\end{minipage}\hfill%
\begin{minipage}{.5\linewidth}
    \prob{https://artofproblemsolving.com/community/c74453h1225408}{buratinogigle's
        proposed problems for arab saudi team 2015}{e}{
        let $abc$ be a triangle inscribed circle $(o)$. $p$ lies on $(o)$. the
        line passes through $p$ and parallel to $bc$ cuts $ca$ at $e$. $k$ is
        circumcenter of triangle $pce$ and $l$ is nine point center of
        triangle $pbc$. prove that the line passes through $l$ and parallel to
        $pk,$ always passes through a fixed point when $p$ moves.
    }

    \begin{solution}[construction]
        notice that if we reflect $ p $ over $ l $ to get $ p' $, then $ op=ah
        $ and $ op\perp bc $ where $ o $ is the circumcenter of $ \odot abc $.
        which trivially implies that the line throught $ l $ passes throught
        the midpoint of $ p'd $ where $ d $ is the reflection of $ h $ over $
        bc $.
    \end{solution}
\end{minipage}


\begin{minipage}{.5\linewidth}
    \prob{https://artofproblemsolving.com/community/c74453h1225408}{buratinogigle's
        proposed problems for arab saudi team 2015}{e}{ let $abc$ be acute
        triangle inscribed circle $(o)$, altitude $ah$, $h$ lies on $bc$. $p$
        is a point that lies on bisector $\angle bac$ and $p$ is inside
        triangle $abc$. circle diameter $ap$ cuts $(o)$ again at $g$. $l$ is
        projection of $p$ on $ah$. assume that $gl$ bisects $hp$. prove that
        $p$ is incenter of $abc$.
    }

    \begin{solution}[angle chase]
        since $ \angle apl = \angle abd = \angle agd $, $ g, l, m $ are
        collinear. let $ e\in bc $ and $ pe\perp bc $. then $ e $ also lies on
        $ dg $. 

        again we have, $ \triangle dpe\sim \triangle dgp $. which implies $
        dp=db=dc $.
    \end{solution}
\end{minipage}\hfill%
\begin{minipage}{.45\linewidth}
    \figdf{}{SATST2015proposed_by_bura/derakynay1134-4}{}
\end{minipage}


\begin{minipage}{.45\linewidth}
    \figdf{}{SATST2015proposed_by_bura/derakynay1134-5}{}
\end{minipage}\hfill%
\begin{minipage}{.5\linewidth}
    \prob{https://artofproblemsolving.com/community/c74453h1225408}{buratinogigle's
        proposed problems for arab saudi team 2015}{e}{
        let $abc$ be an acute triangle inscribed circle $(o)$. $m$ lies on
        small arc $\overline{bc}$ . $p$ lies on $am$. circle diameter $mp$
        cuts $(o)$ again at $n$. $mo$ cuts circle diameter $mp$ again at $q$.
        $an$ cuts circle diameter $mp$ again at $r$. prove that $\angle pra =
        \angle pqa$.
    }

    \begin{solution}[Angle Chase]
        Let $ MO \cap \odot ABC = D $. Becase $ NP\perp MN $, we have $ N, P,
        D $ collinear, and $ APQD $ cyclic.

        So, $ \triangle APQ\sim \triangle ANM \sim \triangle APR $.
    \end{solution}
\end{minipage}






\prob{https://artofproblemsolving.com/community/c74453h1225408}{buratinogigle's
    proposed problems for Arab Saudi team 2015}{E}{
    Let $ABC$ be right triangle with hypotenuse $BC$, bisector $BE$, $E$ lies on
    $CA$. Assume that circumcircle of triangle $BCE$ cuts segment $AB$ again at
    $F$. $K$ is projection of $A$ on $BC$. $L$ lies on segment $AB$ such that $BL
    = BK$. Prove that $\frac{AL}{AF} = \sqrt{\frac{BK}{BC}}$.
}

\fig{.5}{SATST2015proposed_by_bura/derakynay1134-6}{}%bps_1






\prob{https://artofproblemsolving.com/community/c74453h1225408_some_geometric_problems}{buratinogigle's
    proposed problems for Arab Saudi team 2015}{E}{
    Let $ABC$ be acute triangle inscribed circle $(O)$. $AD$ is diameter of
    $(O)$. $M, N$ lie on $BC$ such that $OM \parallel AB$, $ON \parallel AC$.
    $DM, DN$ cut $(O)$ again at $P, Q$. Prove that $BC = DP = DQ$.
}

\fig{.5}{SATST2015proposed_by_bura/derakynay1134-7}{}





\prob{}{}{E}{
    Let $\triangle ABC$ be a triangle. $F, G$ be arbitrary points on $AB, AC$.
    Take $D, E$ midpoint of $BF, CG$. Show that the center of nine-point circle of
    $\triangle ABC,\ \triangle ADE,\ \triangle AFG$ are collinear.
}\label{eriq_lemma_3}




\prob{http://igo-official.ir/wp-content/uploads/2017/09/4th_IGO_Problems-and-Solutions.pdf}{IGO 2017 Advance P3}{}{
    Let $O$ be the circumcenter of $\triangle ABC$. Line
    $CO$ intersects the altitude through $A$ at point $K$. Let $P, M$ be the
    midpoints of $AK, AC$ respectively. If $PO$ intersects $BC$ at $Y$ , and the
    circumcircle of $\triangle BCM$ meets $AB$ at $X$, prove that $BXOY$ is cyclic
}


\solu{
    There is no easily measurable angles, in this case use projective geometry.
    And since we still don't have any easy angles, we look for the second way of
    concyclicity, POP
}


\prob{https://artofproblemsolving.com/community/c6h1615884p10101617}{Turkey TST 2018 P4}{E}{
    In a non-isosceles acute triangle $ABC$, $D$ is the midpoint of $BC$. The
    points $E$ and $F$ lie on $AC$ and $AB$, respectively, and the
    circumcircles of $CDE$ and $AEF$ intersect in $P$ on $AD$. The angle
    bisector from $P$ in triangle $EFP$ intersects $EF$ in $Q$. Prove that the
    tangent line to the circumcircle of $AQP$ at $A$ is perpendicular to $BC$.
}

\solu{
    Inverting around $ A $.
}



\prob{https://artofproblemsolving.com/community/q1h1970138p13654481}{USA Winter TST 2020 P2}{E}{
    Two circles $\Gamma_1$ and $\Gamma_2$ have common external tangents
    $\ell_1$ and $\ell_2$ meeting at $T$. Suppose $\ell_1$ touches $\Gamma_1$
    at $A$ and $\ell_2$ touches $\Gamma_2$ at $B$. A circle $\Omega$ through
    $A$ and $B$ intersects $\Gamma_1$ again at $C$ and $\Gamma_2$ again at
    $D$, such that quadrilateral $ABCD$ is convex.

    Suppose lines $AC$ and $BD$ meet at point $X$, while lines $AD$ and $BC$
    meet at point $Y$. Show that $T$, $X$, $Y$ are collinear.
}

\begin{solution}[Radical Axis]
    It is easy to see that $ X $ lies on the radical axis of $ \Gamma_1 $ and
    $ \Gamma_2 $. Let $ B' = l_1\cap \Gamma_2 $ and $ A' = l_2\cap \Gamma_1 $.
    Let $ C'=A'X\cap \Gamma_1 $ and $ D'=B'X\cap \Gamma_2 $. Let $ A'C\cap
    AC'=Z $.\\

    We have $ AD'CB' $ and $ A'DC'B $ cyclic. Also $ T, D, C' $ and $ T, D', C
    $ are collinear. Which implies $ A'D'CB $ and $ ADC'B' $ are cyclic too.\\

    Applying pascal on $ AAC'CA'A' $, we have $ T, X, Z $ are collinear.\\ 

    Now, it is easy to see that $ Z, Y, T $ lie on the radical axis of $
    A'D'CB $ and $ ADC'B' $. So we have $ T, X, Y, Z $ collinear.

    \figdf{.8}{USA_Winter_TST_2020_P2}{}
\end{solution}

\begin{solution}[mOvInG pOiNtS, by shawnee03]
    Fix $\Gamma_1$ and $\Gamma_2$ (and hence $\ell, T, A, B$) and animate $X$
    linearly on $\ell$. Then

    \begin{itemize}
        \item $C$ moves projectively on $\Gamma_1$ (it is the image of the
            perspectivity through $A$ from $\ell$ to $\Gamma_1$) and thus has
            degree $2$, and similarly for $D$.
        \item $\overline{AD}$ has degree at most $0+2=2$, and similarly for
            $\overline{BC}$.
        \item $Y=\overline{AD}\cap\overline{BC}$ has degree at most $2+2=4.$
        \item The collinearity of $T,X,Y$ has degree at most $0+1+4=5.$
    \end{itemize}

    Thus it suffices to verify the problem for six different choices of $X$.
    We choose:

    \begin{itemize}
        \item $\ell\cap \ell_1$: here $Y$ approaches $A$ as $X$ approaches
            $\ell\cap \ell_1$.
        \item $\ell\cap\ell_2$: here $Y$ approaches $B$ as $X$ approaches
            $\ell\cap \ell_2$.
        \item $\ell\cap \overline{AB}$: here $Y$ approaches $\ell\cap
            \overline{AB}$ as $X$ approaches $\ell\cap \overline{AB}$.
        \item the point at infinity along $\ell$: here $Y=T$.
        \item the two intersections of $\Gamma_1$ and $\Gamma_2$: here $Y=X$.
    \end{itemize}

    (The final two cases may be chosen because we know that there exists a
    choice of $A,B,C,D$ for which $ABCD$ is convex; this forces $\Gamma_1$ and
    $\Gamma_2$ to intersect.)
\end{solution}


\gene{https://artofproblemsolving.com/community/c6h1970138p13654501}{USA Winter TST 2020 P2}{
    Let $ABCD$ be a cyclic quadrilateral, $X=AC\cap BD$, and $Y=AB\cap CD$. Let
    $T$ be a point on line $XY$, $\Gamma_1$ be the circle through $A$ and $C$
    tangent to $TA$, and $\Gamma_2$ be the circle through $B$ and $D$ tangent to
    $TD$. Then $\Gamma_1$ and $\Gamma_2$ are viewed at equal angles from $T$.
}

\begin{solution}[Length Chase, by a1267ab]

    If the radiuses of $ \Gamma_1 $ and $ \Gamma_2 $ are $ r_1, r_2 $, then we
    have to show,\[\frac{TA}{r_1}=\frac{TD}{r_2}\]

    We have, 
    \[r_1= \frac{AB}{2\sin\angle TAB},\ r_2= \frac{CD}{2\sin\angle TDC}\]

    To get the sine ratios, we compare the areas of $ \triangle TAB $ and $
    \triangle TDC $. We have,
    \[\frac{TA\cdot AB\ \sin\angle TAB}{TD\cdot CD\ \sin\angle TDC} =
    \frac{[TAB]}{[TCD]} = \frac{[XAB]}{[XCD]} = \frac{AB^2}{CD^2}\]
    \[\implies \frac{r_1}{TA}=\frac{r_2}{TD}\]

    \figdf{.6}{USA_Winter_TST_2020_P2_generalization}{}
\end{solution}





\prob{https://artofproblemsolving.com/community/c6h1289432p6815893}{IRAN 3rd Round 2016 P1}{E}{
    Let $ABC$ be an arbitrary triangle, $P$ is the intersection point of the
    altitude from $C$ and the tangent line from $A$ to the circumcircle. The
    bisector of angle $A$ intersects $BC$ at $D$. $PD$ intersects $AB$ at $K$, if
    $H$ is the orthocenter then prove : $HK\perp AD$.
}


\solu{Finding a set of Collinear points.}




\prob{http://igo-official.ir/wp-content/uploads/2017/09/4th_IGO_Problems-and-Solutions.pdf}{IGO 2017 Advance P4}{}{Three circles $W_1 , W_2$ and $W_3$  touches a line $l$ at  $A ,B ,C$  respectively ($B$ lies between $A$ and $C$ ). $W_2$ touches  $W_1$ and $W_3$. Let $l_2$ be the other common external tangent of  $W_1$ and $W_3$. $l_2$ cuts $W_2$ at $X ,Y$. Perpendicular to $l$ at $B$ intersects $W_2$ again at $K$. Prove that $KX$ and $KY$ are tangent to the circle with diameter $AC$.}

\solu{Finding a Orthocenter Figure in these circle simplifies the problem a lot.}



\prob{https://artofproblemsolving.com/community/c6h1513396p8996816}{2017 IGO Advanced P2}{E}{We have six pairwise non-intersecting circles that the radius of each is at least one (no circle lies in the interior of any other circle). Prove that the radius of any circle intersecting all the six circles, is at least one.}



\prob{https://artofproblemsolving.com/community/c6h1632766p10256362}{ARO 2018 P10.2}{E}{Let $\triangle ABC$ be an acute-angled triangle with $AB<AC$. Let $M$ and $N$ be the midpoints of $AB$ and $AC$, respectively; let $AD$ be an altitude in this triangle. A point $K$ is chosen on the segment $MN$ so that $BK=CK$. The ray $KD$ meets the circumcircle $\Omega$ of $ABC$ at $Q$. Prove that $C, N, K, Q$ are concyclic.}




\prob{https://artofproblemsolving.com/community/c6h587992p3480807}{ARO 2014 P9.4}{E}{Let $M$ be the midpoint of the side $AC$ of acute-angled triangle $ABC$ with $AB>BC$. Let $\Omega $ be the circumcircle of $ ABC$. The tangents to $ \Omega $ at the points $A$ and $C$ meet at $P$, and $BP$ and $AC$ intersect at $S$. Let $AD$ be the altitude of the triangle $ABP$ and $\omega$ the circumcircle of the triangle $CSD$. Suppose $ \omega$ and $ \Omega $ intersect at $K\not= C$. Prove that $ \angle CKM=90^\circ $.}



\prob{https://artofproblemsolving.com/community/c6h79788p456609}{APMO 1999 P3}{E}{Let $\Gamma_1$ and $\Gamma_2$ be two circles intersecting at $P$ and $Q$. The common tangent, closer to $P$, of $\Gamma_1$ and $\Gamma_2$ touches $\Gamma_1$ at $A$ and $\Gamma_2$ at $B$. The tangent of $\Gamma_1$ at $P$ meets $\Gamma_2$ at $C$, which is different from $P$, and the extension of $AP$ meets $BC$ at $R$.	Prove that the circumcircle of triangle $PQR$ is tangent to $BP$ and $BR$.}



\prob{}{Simurgh 2019 P2}{E}{Let $ ABC $ be an isosceles triangle, $ AB=AC $. Suppoe that $ Q $ is a point such that $ AQ=AB,\ AQ||BC $. Let $ P $ be the foot of perpendicular line from $ Q $ to $ BC $. Prove that the circle with diameter $ PQ $ is tangent to the circumcircle of $ ABC $.}



\prob{http://emc.mnm.hr/wp-content/uploads/2018/12/EMC_2018_Seniors_ENG_Solutions-2.pdf}{European Mathematics Cup 2018 P2}{E}{Later}


\prob{https://artofproblemsolving.com/community/c6h1789909p11836144}{RMM 2019 P2}{E}{Let $ABCD$ be an isosceles trapezoid with $AB\parallel CD$. Let $E$ be the midpoint of $AC$. Denote by $\omega$ and $\Omega$ the circumcircles of the triangles $ABE$ and $CDE$, respectively. Let $P$ be the crossing point of the tangent to $\omega$ at $A$ with the tangent to $\Omega$ at $D$. Prove that $PE$ is tangent to $\Omega$.}


\prob{https://artofproblemsolving.com/community/c6h1709995p11022258}{IGO 2018 A5}{E}{$ABCD$ is a cyclic quadrilateral. A circle passing through $A,B$ is tangent to segment $CD$ at point $E$. Another circle passing through $C,D$ is tangent to $AB$ at point $F$. Point $G$ is the intersection point of $AE,DF$, and point $H$ is the intersection point of $BE,CF$. Prove that the incenters of triangles $AGF,BHF,CHE,DGE$ lie on a circle.}


\figdf{1}{IGO2018A5}{\autoref{problem:IGO 2018 A5} IGO 2018 A5}





\prob{https://artofproblemsolving.com/community/c6h418983p2365045}{ISL 2011 G8}{EM}{Let $ABC$ be an acute triangle with circumcircle $\Gamma$. Let $\ell$ be a tangent line to $\Gamma$, and let $\ell_a, \ell_b$ and $\ell_c$ be the lines obtained by reflecting $\ell$ in the lines $BC$, $CA$ and $AB$, respectively. Show that the circumcircle of the triangle determined by the lines $\ell_a, \ell_b$ and $\ell_c$ is tangent to the circle $\Gamma$.}

\solu{Find the translated triangle circumscribed in $ \odot ABC $. Once you find the properties of this triangle and the relations between this and the common touch point, the problem becomes obvious.}




\prob{https://artofproblemsolving.com/community/c6h1818716p12141505}{ELMO 2019 P3}{EM}{Let $ABC$ be a triangle such that $\angle CAB > \angle ABC$, and let $I$ be its incentre. Let $D$ be the point on segment $BC$ such that $\angle CAD = \angle ABC$. Let $\omega$ be the circle tangent to $AC$ at $A$ and passing through $I$. Let $X$ be the second point of intersection of $\omega$ and the circumcircle of $ABC$. Prove that the angle bisectors of $\angle DAB$ and $\angle CXB$ intersect at a point on line $BC$.}


\begin{solution}[Angle Chase]
    Suppose the bisector of $ \angle BAD $ meet $ BC $ at $ G' $. Then we have,
    \begin{align*}
        \angle BG'A &= \frac{\angle A-\angle B}{2}\\
        \therefore \angle CG'A &= \angle B + \angle BG'A\\
                               &=\frac{A+B}{2}\\[1em]
                               \implies CG'&=CA\\
                               \therefore \angle G'ID &= \angle B
                           \end{align*}

                           Now, let $ M $ be the midpoint of the minor arc $ BC $. Let $ G=XM\cap BC $. So we have \[\triangle MGI \sim \triangle MIX \implies \angle MIG = \angle MXI\]

                           Let $ XI\cap \odot ABC = N \neq X $. Since $ AC $ is tangent to $ \odot AXI $, $ NC\parallel AM $. Which means $$  \angle MXI = \angle B = \angle MIG  $$
                           Which completes our proof by implying that $ G'\equiv G $. 

                           \figdf{.4}{elmo_2019_P3}{}

                       \end{solution}




                       \prob{https://artofproblemsolving.com/community/c6h596927p3542092}{ISL 2014 G5}{M}{Convex quadrilateral $ABCD$ has $\angle ABC = \angle CDA = 90^{\circ}$. Point $H$ is the foot of the perpendicular from $A$ to $BD$. Points $S$ and $T$ lie on sides $AB$ and $AD$, respectively, such that $H$ lies inside triangle $SCT$ and \[ \angle CHS - \angle CSB = 90^{\circ}, \quad \angle THC - \angle DTC = 90^{\circ}.\] Prove that line $BD$ is tangent to the circumcircle of triangle $TSH$.}

                       \solu{First construct using nice circles, then prove the center is on $ AH $ using angle bisector theorem.
                           \figdf{.6}{ISL2014G5}{Construction}
                       \figdf{.3}{ISL2014G5_lemma1}{Lemma}}



                       \prob{https://artofproblemsolving.com/community/c6h1113194p5083564}{ISL 2014 G7}{M}{Let $ABC$ be a triangle with circumcircle $\Omega$ and incentre $I$. Let the line passing through $I$ and perpendicular to $CI$ intersect the segment $BC$ and the arc $BC$ (not containing $A$) of $\Omega$ at points $U$ and $V$ , respectively. Let the line passing through $U$ and parallel to $AI$ intersect $AV$ at $X$, and let the line passing through $V$ and parallel to $AI$ intersect $AB$ at $Y$ . Let $W$ and $Z$ be the midpoints of $AX$ and $BC$, respectively. Prove that if the points $I, X,$ and $Y$ are collinear, then the points $I, W ,$ and $Z$ are also collinear.}

                       \solu{Draw a nice diagram, and use the parallel property to find circles.
                       \figdf{.7}{ISL2014G7}{ISL 2014 G7}}



                       \prob{https://artofproblemsolving.com/community/c6h1112748p5079655}{ISL 2015 G6}{E}{Let $ABC$ be an acute triangle with $AB > AC$. Let $\Gamma $ be its cirumcircle, $H$ its orthocenter, and $F$ the foot of the altitude from $A$. Let $M$ be the midpoint of $BC$. Let $Q$ be the point on $\Gamma$ such that $\angle HQA = 90^{\circ}$ and let $K$ be the point on $\Gamma$ such that $\angle HKQ = 90^{\circ}$. Assume that the points $A$, $B$, $C$, $K$ and $Q$ are all different and lie on $\Gamma$ in this order.\\

                       Prove that the circumcircles of triangles $KQH$ and $FKM$ are tangent to each other.}

                       \solu{Draw the tangent line, and find angles. 
                       \figdf{.7}{ISL2015G6}{ISL 2015 G6}}



                       \prob{https://artofproblemsolving.com/community/c6h1268908p6622796}{ISL 2015 G5}{M}{Let $ABC$ be a triangle with $CA \neq CB$. Let $D$, $F$, and $G$ be the midpoints of the sides $AB$, $AC$, and $BC$ respectively. A circle $\Gamma$ passing through $C$ and tangent to $AB$ at $D$ meets the segments $AF$ and $BG$ at $H$ and $I$, respectively. The points $H'$ and $I'$ are symmetric to $H$ and $I$ about $F$ and $G$, respectively. The line $H'I'$ meets $CD$ and $FG$ at $Q$ and $M$, respectively. The line $CM$ meets $\Gamma$ again at $P$. Prove that $CQ = QP$.}

                       \solu{Don't depend on the figure too much, find facts using facts, not figure.
                       \figdf{.7}{ISL2015G5}{ISL 2015 G5}}




                       \prob{https://artofproblemsolving.com/community/c6h418635p2361976}{ISL 2010 G5}{E}{Let $ABCDE$ be a convex pentagon such that $BC \parallel AE,$ $AB = BC + AE,$ and $\angle ABC = \angle CDE.$ Let $M$ be the midpoint of $CE,$ and let $O$ be the circumcenter of triangle $BCD.$ Given that $\angle DMO = 90^{\circ},$ prove that $2 \angle BDA = \angle CDE.$}

                       \solu{First try to construct the point. Do this the long way, then find a easier way that includes $ B, C $, not $ B, A $ to do that. Then try to translate what $ 90 $ degree condition into angles, and take midpoints, since we have midpoints involved.
                       \figdf{.7}{ISL2010G5}{ISL 2010 G5}}



                       \prob{https://artofproblemsolving.com/community/c6h1918849p13155415}{IGO 2019 A5}{MH}{Let points $A, B$ and $C$ lie on the parabola $\Delta$ such that the point $H$, orthocenter of triangle $ABC$, coincides with the focus of parabola $\Delta$. Prove that by changing the position of points $A, B$ and $C$ on $\Delta$ so that the orthocenter remain at $H$, inradius of triangle $ABC$ remains unchanged.}	

                       \solu{
                           \figdf{.7}{IGO2019G5}{IGO 2019 A5}
                           I think the idea for inversion should have been pretty natural after finding that the incircle is fixed. 
                       }

                       \prob{https://artofproblemsolving.com/community/c6h1140464p5353018}{Iran 3rd Round 2015 P5}{M}{Let $ABC$ be a triangle with orthocenter $H$ and circumcenter $O$. Let $R$ be the radius of circumcircle of $\triangle ABC$. Let $A',B',C'$ be the points on $\overrightarrow{AH},\overrightarrow{BH},\overrightarrow{CH}$ respectively such that $AH.AA'=R^2,BH.BB'=R^2,CH.CC'=R^2$. Prove that $O$ is incenter of $\triangle A'B'C'$.}

                       \solu{The condition easily leads to a nice construction of the points. It should be trivial to figure that the construction is really important. Also, noticing a similarity among the triangles is really important.}







                       %----------------------------------------------------------------------------------------------------------------------------


                       \newpage\subsection{The line parallel to $ BC $}


                       \den{Let the line parallel to $ BC $ through $ O $ meet $ AB, AC $ at $ D, E $. Let $ K $ be the midpoint of $ AH $, $ M $ be the midpoint of $ BC $. $ F $ be the feet of $ A $-altitude on $ BC $ and let $ H' $ be the reflection of $ H $ on $ F $. Let $ O' $ be the circumcenter of $ KBC $.}


                       \lem{}{$ \angle DKC = \angle EKB = 90^{\circ} $}\label{lemma:linethroughO_lemma1}

                       \fig{1}{linethroughO_1}{}


                       \lem{}{$ CD, BE, OH', AM, KO' $ are concurrent. (by \hrf{isogonality_lemma}{lemma})}\label{lemma:linethroughO_lemma2}

                       \fig{1}{linethroughO_2}{}


                       \prob{https://artofproblemsolving.com/community/c6h1412598_circumcircles_intersect_on_ao}{InfinityDots MO Problem 3}{EM}{Let $\triangle ABC$ be an acute triangle with circumcenter $O$ and orthocenter $H$. The line through $O$ parallel to $BC$ intersect $AB$ at $D$ and $AC$ at $E$. $X$ is the midpoint of $AH$. Prove that the circumcircles of $\triangle BDX$ and $\triangle CEX$ intersect again at a point on line $AO$.}


                       \solu{Just using \hrf{lemma:linethroughO_lemma1}{lemma} to get another pair of circle where we can apply radical axis arguments.}

                       \solu{Noticing that the resulting point is the isogonal conjugate of a well defined point,}



                       \lem{}{Let $ P, Q $ be on $ AB, AC $ resp. such that $ PQ\parallel BC $. And let $ A' $ be such that $ A'\in \odot ABC, AA'\parallel BC $. Let $ CP\cap BQ = X $, and let the perpendicular bisector of $ BC $ meet $ PQ $ at $ Y $. Prove that $ A', X, Y $ are collinear.}\label{lemma:concurrency_in_prallel_lines_with_BC}

                       \solu{No angles... Do Lengths...}

                       \fig{1}{concurrency_in_prallel_lines_with_BC}{}


                       \prob{https://artofproblemsolving.com/community/c6h1634961p10278557}{ARO 2018 P11.4}{M}{$ P \in AB,\ Q \in AC,\ PQ\parallel BC,\ BQ\cap CP = X $. $ A' $ is the reflection of $ A $ wrt $ BC $. $ A'X\cap \odot APQ = Y $. Prove that $ \odot BYC $ is tangent to $ \odot APQ $.}

                       \solu{Of co it can be solved using angle chase, \hrf{lemma:concurrency_in_prallel_lines_with_BC}{lemma} makes it almost trivial.}


                       \prob{https://artofproblemsolving.com/community/c6h520197}{buratinogigle}{EM}{Let $(O)$ be a circle and $E,F$ are two points inside $(O)$. $(K),(L)$ are two circles passing though $E,F$ and tangent internally to $(O)$ at $A,D$, respectively. $AE,AF$ cut $(O)$ again at $B,C$, respectively. $BF$ cuts $CE$ at $G$. Prove that reflection of $A$ though $EF$ lies on line $DG$.\\

                       Rephrasing the problem as such: In the setup of \hrf{lemma:concurrency_in_prallel_lines_with_BC}{this lemma}, let $ A'X\cap \odot ABC = Z $, then $ \odot PQZ $ is tangent to $ \odot ABC $.}


                       \solu{Simple angle chase. }

                       \solu{Another solution to this is by taking $ D $ as a phantom point.}

                       \solu{Another solution is with cross ratios}







                       \newpage\subsection{Simson Line and Stuffs}


                       \lem{Simson Line Parallel}{Let $ P $ be a point on the circumcircle, let $ P' $ be the reflection of $ P $ on $ BC $ and let $ PP' \cap \Omega = D $, and let $ l_p $ be the Simson line of $ P $. Prove that $ l_p\parallel AD\parallel HP' $.}

                       \fig{1}{SimsonLineLemma1}{The dotted lines are parallel}



                       \lem{Simson Line Angle}{Given triangle $ ABC $ and its circumcircle $ (O) $. Let $ E,F $ be two arbitrary points on $ (O) $. Then the angle between the Simson lines of two points $ E $ and $ F $ is half the measure of the arc $ EF $.}

                       \fig{1}{SimsonLineLemma2}{}




                       \newpage\subsection{Euler Line}



                       \theo{}{Perspectivity Line with Orthic triangle is perpendicular to Euler line}{Let $ DEF $ be the orthic triangle. Then $ BC\cap EF, CA\cap FD, AB\cap ED $ are collinear, and the line is perpendicular to the Euler line. In fact this line is the radical axis of the Circumcircle and the NinePoint circle}




                       \lem{}{$ DEF $ is orthic triangle of $ ABC $, $ XYZ $ is the orthic triangle of $ DEF $. Prove that the perspective point of $ ABC $ and $ XYZ $ lies on the Euler line of $ ABC $}

                       \solu{Thinking the stuff wrt to the incircle and using cross ratio.}




