\graphicspath{{Pics/}}

\newpage
\section{Orthocenter--Circumcircle--NinePoint Circle}

Notes everyone need to memorize by heart top to bottom:
\begin{enumerate} [itemsep=0pt]
    \item \href{http://yufeizhao.com/olympiad/imo2008/zhao-circles.pdf}{Circles -
        Yufei Zhao}
    \item \href{http://yufeizhao.com/olympiad/cyclic_quad.pdf}{Big Picture - Yufei Zhao} 
    \item \href{http://yufeizhao.com/olympiad/power_of_a_point.pdf}{POP - Yufei Zhao}
    \item \href{http://yufeizhao.com/olympiad/three_geometry_lemmas.pdf}{3 Lemmas
        - Yufei Zhao} 
\end{enumerate}

\den{The usual notations}{
    Unless stated otherwise, we assume that $\triangle ABC$ is an arbitrary
    triangle, with circumcenter $O$, orthocenter $H$, $\omega$ is the
    circumcircle. Usually, $DEF$ is the orthic triangle in this chapter. $MNP$
    is the median triangle.
}


\begin{minipage}{.65\textwidth} 
    \lem{Collinearity with antipode and center}{
        Let $A'$ be the antipode of $ A $ in $ \odot ABC $. Let
        $BDEC$ be a cyclic quadrilateral with $ D\in AB $ and $ E\in AC $. Let $ P $
        be the center of $ BDEC $. Also, let $ X=BE\cap CD $. Then $ A', P, X $ are
        collinear.  
    } 

    \solu{ 
        Using ``The Big Picture" property to show that if $
        Q=\odot ADE \cap \odot ABC $, then $ P, X, Q $ collinear and $ PQ\perp AQ $.
        Which implies that $ P, A', Q $ are collinear.  
    } 

\end{minipage}\hfill% 
\begin{minipage}{.33\textwidth} 
    \figdf{}{ChinaTST2018T1P3_lemm}{}
\end{minipage}

\begin{minipage}{.65\linewidth}
    \lem{Weird point Y}{
        Let $Y$ be a point on $AC$ such that $\triangle CBY \sim CAB$. Let $E$
        bet the foot of $B$ on $AC$, let $N$ be the midpoint of $AB$. Then
        $NE, CO, BE$ are concurrent.
    }
    \begin{solution}
        Draw te circles, $BNKO$ and $BKYC$.
    \end{solution}
\end{minipage}\hfill%
\begin{minipage}{.33\linewidth}
    \figdf{}{CHKMO_2014_p4_lem_1}{}
\end{minipage}

\newpage
\subsection{Problems}

\begin{minipage}{.55\textwidth}
    \prob{https://artofproblemsolving.com/community/c6h1441717p8209926}{Balkan MO
        2017 P3}{S}{
        Let $t_B$ and $t_C$ be the tangents $\omega$ at $B$ and $C$, they meet
        at $L$. The straight lines passing through $B, C$ and
        parallel to $AC, AB$ intersects $t_C, t_B$ at points $D, E$
        respectively. $T = AC \cap \odot BDC, S = AB\cap \odot CBE$. Prove
        that $ST$, $AL$, and $BC$ are concurrent.  
    }

    \begin{solution}
        We have $\triangle ABT \sim \triangle ACB \sim ASC$, which leads to
        $BT\parallel CS$, and $AL$ becomes the median of $\triangle ABT$.
    \end{solution}

    \vspace{1em}

    \prob{https://artofproblemsolving.com/community/c4512_2014_usamo}{USAMO
        2014 P5}{E}{
        Let $P$ be the second intersection of $\odot AHC$ with the internal
        bisector of $\angle BAC$. Let $X$ be the circumcenter of triangle
        $APB$ and $Y$ the orthocenter of triangle $APC$. Prove that the length
        of segment $XY$ is equal to the circumradius of triangle $ABC$.
    }

    \solu{
        No length conditions given, yet we need to prove that two lengths
        are equal. \emph{Parallelogram} !!! Just need to prove that $ Y\in
        \odot ABC \text{ and } YD\perp AB $
    } 

    \vspace{1em}
    \prob{https://artofproblemsolving.com/community/c74453h1225408_some_geometric_problems}
    {Saudi Arab 2015}{E}{
        $P$ is a point. $(K)$ is the circle with diameter $AP$. $(K)$ cuts $CA, AB$
        again at $ E, F $. $ PH $ cuts $ (K) $ again at $ G $. Tangent line at
        $E, F $ of $ (K) $ intersect at $ T $. $ M $ is midpoint of $ BC $. 
        $L$ is the point on $ MG $ such that $ AL \parallel MT.  $ Prove that
        $LA \perp LH $.
    }

    \begin{solution}[Phantom Point] 
        Take $ L' = MG \cap AZYH  $, then use
        spiral similarity to show that $ AL'\parallel MT $.  
    \end{solution}
\end{minipage}\hfill% 
\begin{minipage}{.4\textwidth}
    \figdf{.7}{Balkan_MO_2017_P3}{}
    \figdf{.99}{USAMO_2014_P5}{}
    \figdf{.9}{SATST2015proposed_by_bura/derakynay1134-1}{} 
\end{minipage}	

\newpage

\begin{minipage}{.5\textwidth} 
    \prob{}{Bewarish 1}{E}{
        Let $ DEF $ be the
        orthic triangle, and let $ EF\cap BC = P $. Let the tangent at $ A $ to $
        \odot ABC $ meet $ BC $ at $ Q $. Let $ T $ be the reflection of $ Q $ over $
        P $. Let $ K $ be the orthogonal projection of $ H $ on $ AM $. Prove that $
        \angle OKT = 90 $.
    }

    \solu{
        Spiral similarity from $O$ to get rid of $Q$ and $T$. Then spiral
        similarity again from $P$ to get a trivial circle.
    } 
\end{minipage}\hfill%
\begin{minipage}{.48\textwidth} 
    \figdf{.99}{Bewarish_1}{} 
\end{minipage}

\begin{minipage}{.6\linewidth}
    \prob{https://artofproblemsolving.com/community/c74453h1225408_some_geometric_problems}
    {Saudi Arab 2015}{E}{
        $P$ lies on $(O)$. The line passes through $P$ and parallel to $BC$
        cuts $CA$ at $E$. $K$ is circumcenter of triangle $PCE$ and $L$ is
        nine point center of triangle $PBC$. Prove that the line passes
        through $L$ and parallel to $PK,$ always passes through a fixed point
        when $P$ moves.
    }

    \begin{solution}[Construction] 
        Notice that if we reflect $ P $ over $ L $
        to get $ P' $, then $ OP=AH $ and $ OP\perp BC $ where $ O $ is the
        circumcenter of $ \odot ABC $. Which trivially implies that the line
        throught $ L $ passes throught the midpoint of $ P'D $ where $ D $ is
        the reflection of $ H $ over $ BC $.  
    \end{solution} 

    \prob{https://artofproblemsolving.com/community/c74453h1225408_some_geometric_problems}
    {Saudi Arab 2015}{E}{
        Altitude $AH$, $H$ lies on $BC$. $P$ is a point that
        lies on bisector $\angle BAC$ and $P$ is inside triangle $ABC$. Circle
        diameter $AP$ cuts $(O)$ again at $G$. $L$ is projection of $P$ on $AH$.
        Assume that $GL$ bisects $HP$. Prove that $P$ is incenter of $ABC$.
    }

    \begin{solution}[Angle Chase] Since $ \angle APL = \angle ABD = \angle AGD
        $, $ G, L, M $ are collinear. Let $ E\in BC $ and $ PE\perp BC $. Then
        $ E $ also lies on $ DG $. 

        Again we have, $ \triangle DPE\sim \triangle DGP $. Which implies $
        DP=DB=DC $.  
    \end{solution} 
\end{minipage}\hfill%
\begin{minipage}{.35\linewidth}
    \figdf{}{SATST2015proposed_by_bura/derakynay1134-2}{} 
    \figdf{}{SATST2015proposed_by_bura/derakynay1134-4}{} 
\end{minipage}


\newpage
\begin{minipage}{.6\linewidth}
    \prob{https://artofproblemsolving.com/community/c74453h1225408_some_geometric_problems}
    {Saudi Arab 2015}{E}{
        $M$ lies on small arc $\overline{BC}$ . $P$
        lies on $AM$. Circle diameter $MP$ cuts $(O)$ again at $N$. $MO$ cuts circle
        diameter $MP$ again at $Q$. $AN$ cuts circle diameter $MP$ again at $R$. Prove
        that $\angle PRA = \angle PQA$.
    }

    \begin{solution}[Angle Chase] Let $ MO \cap \odot ABC = D $. Becase $
        NP\perp MN $, we have $ N, P, D $ collinear, and $ APQD $ cyclic.
        So, $ \triangle APQ\sim \triangle ANM \sim \triangle APR $.
    \end{solution} 

    \vspace{1em}

    \prob{https://artofproblemsolving.com/community/c74453h1225408_some_geometric_problems}
    {Saudi Arab 2015}{E}{
        Let $ABC$ be right triangle with hypotenuse $BC$, bisector $BE$, $E$ lies
        on $CA$. Assume that circumcircle of triangle $BCE$ cuts segment $AB$
        again at $F$. $K$ is projection of $A$ on $BC$. $L$ lies on segment $AB$
        such that $BL = BK$. Prove that $\dfrac{AL}{AF} = \sqrt{\dfrac{BK}{BC}}$.
    }

    \vspace{1em}

    \prob{https://artofproblemsolving.com/community/c74453h1225408_some_geometric_problems}
    {Saudi Arab 2015}{E}{
        $AD$ is diameter of $(O)$. $M, N$ lie on $BC$ such
        that $OM \parallel AB$, $ON \parallel AC$. $DM, DN$ cut $(O)$ again at $P, Q$.
        Prove that $BC = DP = DQ$.
    }

    \begin{solution}
        We will prove that $MB = MD$. Since $OD = OA$, we have $EM = MD$. And
        since $\triangle EBD$ is a right triangle, $MB = ME = MD$. And so the
        arcs $PB$ and $DC$ are equal.
    \end{solution}
\end{minipage}\hfill%
\begin{minipage}{.35\linewidth}
    \figdf{}{SATST2015proposed_by_bura/derakynay1134-5}{} 
    \figdf{}{SATST2015proposed_by_bura/derakynay1134-6}{}
    \figdf{}{SATST2015proposed_by_bura/derakynay1134-7}{}
\end{minipage}




\newpage
\prob{http://igo-official.ir/wp-content/uploads/2017/09/4th_IGO_Problems-and-Solutions.pdf}
{IGO 2017 A3}{E}{
    Let $O$ be the circumcenter of $\triangle ABC$. Line $CO$ intersects the
    altitude through $A$ at point $K$. Let $P, M$ be the midpoints of $AK, AC$
    respectively. If $PO$ intersects $BC$ at $Y$ , and the circumcircle of
    $\triangle BCM$ meets $AB$ at $X$, prove that $BXOY$ is cyclic
}
\begin{minipage}{.5\linewidth}
\solu{
    We want to prove that $\angle POX = \angle AMX$. But we also notice by
    Reim's theorem that $\angle BOY = \angle BMC$, which leads to $\angle XPO
    = \angle XAM$. \\

    Now, we want to show that 
    \[\frac{XA}{AP} = \frac{XM}{OM}\]
    Which is simple length chase.
}
\end{minipage}\hfill%
\begin{minipage}{.49\linewidth}
    \figdf{}{IGO_2017_Advanced_P3}{}
\end{minipage}



\begin{minipage}{.5\linewidth}
    \prob{}
    {}{EM}{
        Let $EF\parallel BC$ be two points on the circumcircle. Let $D$ be the center of
        $HE$, and let $K$ be the point on $AB$ for which $OK\parallel AF$. Prove that
        $DK\perp DC$.
    }

    \begin{solution}
        The main part of the solution is to get rid of $OK$ in some way. We take
        $T$ to be the point such that $T\in OK$ and $\angle OTC = 90$. And let $S
        = CT\cap \odot ABC$. Then we have, $ES\perp AB$, and if we can show that
        $D$ lies on $\odot PTC$, we are done. But then that is straightforward as
        $DT\parallel FS\parallel CH$ and $DP\parallel EQ$.
    \end{solution}
\end{minipage}\hfill%
\begin{minipage}{.48\linewidth}
    \figdf{}{bdmo_forum_geo_1}{}
\end{minipage}


\newpage
\prob{https://artofproblemsolving.com/community/c6h1615884p10101617}{Turkey
    TST 2018 P4}{E}{
    In a non-isosceles acute triangle $ABC$, $D$ is the midpoint
    of $BC$. The points $E$ and $F$ lie on $AC$ and $AB$, respectively, and the
    circumcircles of $CDE$ and $AEF$ intersect in $P$ on $AD$. The angle bisector
    from $P$ in triangle $EFP$ intersects $EF$ in $Q$. Prove that the tangent line
    to the circumcircle of $AQP$ at $A$ is perpendicular to $BC$.
}

\begin{minipage}{.5\linewidth}
    \begin{solution}[angle chase]
        Note that $AQ$ is the angle bisector of $\angle BAC$. Using this fact,
        we can easily prove that $\angle HAQ = \angle APQ$.
    \end{solution}    
    \solu{[inversion]
        Inverting around $A$ with radius $\sqrt{AP\cdot AD}$ sends $EF$ to
        the circumcircle of $ABC$ and $P$ to $D$. Since $AQ$ bisects $\angle BAC$,
        we have $DQ'\perp BC$. 
    }
\end{minipage}\hfill%
\begin{minipage}{.47\linewidth}
    \figdf{}{Turkey_TST_2018_p4}{}
\end{minipage}


\prob{https://artofproblemsolving.com/community/q1h1970138p13654481}{USA
    Winter TST 2020 P2}{E}{
    Two circles $\Gamma_1$ and $\Gamma_2$ have common
    external tangents $\ell_1$ and $\ell_2$ meeting at $T$. Suppose $\ell_1$
    touches $\Gamma_1$ at $A$ and $\ell_2$ touches $\Gamma_2$ at $B$. A circle
    $\Omega$ through $A$ and $B$ intersects $\Gamma_1$ again at $C$ and
    $\Gamma_2$ again at $D$, such that quadrilateral $ABCD$ is convex.

    Suppose lines $AC$ and $BD$ meet at point $X$, while lines $AD$ and $BC$ meet
    at point $Y$. Show that $T$, $X$, $Y$ are collinear.
}

\begin{solution}[Radical Axis] It is easy to see that $ X $ lies on the
    radical axis of $ \Gamma_1 $ and $ \Gamma_2 $. Let $ B' = l_1\cap \Gamma_2
    $ and $ A' = l_2\cap \Gamma_1 $. Let $ C'=A'X\cap \Gamma_1 $ and $
    D'=B'X\cap \Gamma_2 $. Let $ A'C\cap AC'=Z $.\\

    We have $ AD'CB' $ and $ A'DC'B $ cyclic. Also $ T, D, C' $ and $ T, D', C
    $ are collinear. Which implies $ A'D'CB $ and $ ADC'B' $ are cyclic too.\\

    Applying pascal on $ AAC'CA'A' $, we have $ T, X, Z $ are collinear.\\ 

    Now, it is easy to see that $ Z, Y, T $ lie on the radical axis of $
    A'D'CB $ and $ ADC'B' $. So we have $ T, X, Y, Z $ collinear.

\end{solution}

\figdf{.8}{USA_Winter_TST_2020_P2}{} 

\vspace{1em}
\begin{solution}[mOvInG pOiNtS, by shawnee03] Fix $\Gamma_1$ and $\Gamma_2$
    (and hence $\ell, T, A, B$) and animate $X$ linearly on $\ell$. Then

    \begin{itemize} 
        \item $C$ moves projectively on $\Gamma_1$ (it is the
            image of the perspectivity through $A$ from $\ell$ to $\Gamma_1$) and
            thus has degree $2$, and similarly for $D$.  
        \item $\overline{AD}$ has
            degree at most $0+2=2$, and similarly for $\overline{BC}$.  
        \item $Y=\overline{AD}\cap\overline{BC}$ has degree at most $2+2=4.$ 
        \item The collinearity of $T,X,Y$ has degree at most $0+1+4=5.$ 
    \end{itemize}

    Thus it suffices to verify the problem for six different choices of $X$.
    We choose:

    \begin{itemize} 
        \item $\ell\cap \ell_1$: here $Y$ approaches $A$ as $X$
            approaches $\ell\cap \ell_1$.  
        \item $\ell\cap\ell_2$: here $Y$
            approaches $B$ as $X$ approaches $\ell\cap \ell_2$.  
        \item $\ell\cap \overline{AB}$: here $Y$ approaches $\ell\cap \overline{AB}$ as $X$
            approaches $\ell\cap \overline{AB}$.  
        \item the point at infinity along $\ell$: here $Y=T$.  
        \item the two intersections of $\Gamma_1$ and $\Gamma_2$: here $Y=X$.  
    \end{itemize}

    (The final two cases may be chosen because we know that there exists a
    choice of $A,B,C,D$ for which $ABCD$ is convex; this forces $\Gamma_1$ and
    $\Gamma_2$ to intersect.) 
\end{solution}


\gene{https://artofproblemsolving.com/community/c6h1970138p13654501}
{USA Winter TST 2020 P2}{
    Let $ABCD$ be a cyclic quadrilateral, $X=AC\cap BD$, and
    $Y=AB\cap CD$. Let $T$ be a point on line $XY$, $\Gamma_1$ be the circle
    through $A$ and $C$ tangent to $TA$, and $\Gamma_2$ be the circle through $B$
    and $D$ tangent to $TD$. Then $\Gamma_1$ and $\Gamma_2$ are viewed at equal
    angles from $T$.
}

\begin{minipage}{.5\linewidth}
    \begin{solution}[Length Chase, by a1267ab]
        If the radiuses of $ \Gamma_1 $ and $ \Gamma_2 $ are $ r_1, r_2 $, then we
        have to show,
        \[\frac{TA}{r_1}=\frac{TD}{r_2}\]
        We have, 
        \[r_1= \frac{AB}{2\sin\angle TAB},\ r_2= \frac{CD}{2\sin\angle TDC}\]
        To get the sine ratios, we compare the areas of $ \triangle TAB $ and $
        \triangle TDC $. We have, 
        \[\begin{aligned}
            \frac{TA\cdot AB\ \sin\angle TAB}{TD\cdot CD\ \sin\angle TDC} &=
            \frac{[TAB]}{[TCD]}\\
            &= \frac{[XAB]}{[XCD]} = \frac{AB^2}{CD^2}\\[.7em]
            \implies \frac{r_1}{TA}&=\frac{r_2}{TD}
        \end{aligned}\]
    \end{solution}
\end{minipage}\hfill%
\begin{minipage}{.5\linewidth}
    \figdf{.9}{USA_Winter_TST_2020_P2_generalization}{} 
\end{minipage}

\begin{minipage}{.5\linewidth}
    \prob{https://artofproblemsolving.com/community/c5h1629606p10226149}
    {USAJMO 2018 P3}{E}{
        Let $ABCD$ be a quadrilateral inscribed in circle $\omega$ with
        $\overline{AC} \perp \overline{BD}$. Let $E$ and $F$ be the
        reflections of $D$ over lines $BA$ and $BC$, respectively, and let $P$
        be the intersection of lines $BD$ and $EF$. Suppose that the
        circumcircle of $\triangle EPD$ meets $\omega$ at $D$ and $Q$, and the
        circumcircle of $\triangle FPD$ meets $\omega$ at $D$ and $R$. Show
        that $EQ = FR$.
    } 
\end{minipage}\hfill%
\begin{minipage}{.45\linewidth}
    \figdf{1}{USAJMO_2018_p3}{}
\end{minipage}


\prob{https://artofproblemsolving.com/community/c6h1289432p6815893}
{IRAN 3rd Round 2016 P1}{E}{
    Let $ABC$ be an arbitrary triangle, $P$ is the intersection
    point of the altitude from $C$ and the tangent line from $A$ to the
    circumcircle. The bisector of angle $A$ intersects $BC$ at $D$. $PD$
    intersects $AB$ at $K$, if $H$ is the orthocenter then prove : $HK\perp AD$
}

\solu{Finding a set of Collinear points.}


\prob{}
{}{E}{
    Let $\triangle ABC$ be a triangle. $F, G$ be arbitrary points on $AB, AC$.
    Take $D, E$ midpoint of $BF, CG$. Show that the nine-point centers of
    $\triangle ABC,\ \triangle ADE,\ \triangle AFG$ are collinear.
}\label{eriq_lemma_3}


\prob{http://igo-official.ir/wp-content/uploads/2017/09/4th_IGO_Problems-and-Solutions.pdf}
{IGO 2017 A4}{}{
    Three circles $W_1 , W_2$ and $W_3$  touches a line $l$ at
    $A ,B ,C$  respectively ($B$ lies between $A$ and $C$). $W_2$ touches  $W_1$
    and $W_3$. Let $l_2$ be the other common external tangent of  $W_1$ and $W_3$.
    $l_2$ cuts $W_2$ at $X ,Y$. Perpendicular to $l$ at $B$ intersects $W_2$ again
    at $K$. Prove that $KX$ and $KY$ are tangent to the circle with diameter
    $AC$.
}

\figdf{}{IGO_2017_A4}{}

\begin{solution}
    Using the names of the vertices in the diagram, we let $UV$ be a segment
    parallel to $O_1, O_2$. Step by step we prove that
    \begin{enumerate}
        \item $O$ is the center of $SHB$, then $O, O_1, O_2$ are collinear.
        \item $ZM$ bisects $\angle XZY$. 
        \item $BUZVM$ is cyclic.
        \item $ZU^2 = ZH.ZB$.
    \end{enumerate}
\end{solution}



\prob{https://artofproblemsolving.com/community/c6h1513396p8996816}
{IGO 2017 A2}{E}{
    We have six pairwise non-intersecting circles that the radius of each is
    at least one (no circle lies in the interior of any other circle). Prove
    that the radius of any circle intersecting all the six circles, is at least one.
}

\begin{solution}
    We first expand the circles so that they touch each other in a ring like
    shape. Then we take the largest diameter of the convex hexagon with the
    centers. We show that any circle that intersects those two cirlces must
    have radius at least $1$.
\end{solution}

\prob{https://artofproblemsolving.com/community/c6h1709992p11022223}
{IGO 2017 A4}{M}{
    Quadrilateral $ABCD$ is circumscribed around a circle. Diagonals $AC,BD$
    are not perpendicular to each other. The angle bisectors of angles between
    these diagonals, intersect the segments $AB,BC,CD$ and $DA$ at points
    $K,L,M$ and $N$. Given that $KLMN$ is cyclic, prove that so is $ABCD$.
}

\begin{minipage}{.5\linewidth}
    \begin{solution}
        If we let $K', L', M', N'$ be the points where the incenter touches the
        sides, then we wish to prove that $K= K'$ and so on. To prove this, we
        first prove that $KL, MN, AC$ are concurrent. \\

        Then we prove that $K'L'$ and $M'N'$ also passes through the same
        point. This lets us use the lemmas of complete cyclic quadrilaterals.
    \end{solution}
\end{minipage}\hfill%
\begin{minipage}{.5\linewidth}
    \figdf{.9}{IGO_2018_A4}{}
\end{minipage}



\prob{https://artofproblemsolving.com/community/c6h1632766p10256362}
{ARO 2018 P10.2}{E}{
    Let $\triangle ABC$ be an acute-angled triangle with $AB<AC$. Let
    $M$ and $N$ be the midpoints of $AB$ and $AC$, respectively; let $AD$ be an
    altitude in this triangle. A point $K$ is chosen on the segment $MN$ so that
    $BK=CK$. The ray $KD$ meets the circumcircle $\Omega$ of $ABC$ at $Q$. Prove
    that $C, N, K, Q$ are concyclic.
}



\prob{https://artofproblemsolving.com/community/c6h587992p3480807}
{ARO 2014 P9.4}{E}{
    Let $M$ be the midpoint of the side $AC$ of acute-angled triangle
    $ABC$ with $AB>BC$. Let $\Omega $ be the circumcircle of $ ABC$. The tangents
    to $ \Omega $ at the points $A$ and $C$ meet at $P$, and $BP$ and $AC$
    intersect at $S$. Let $AD$ be the altitude of the triangle $ABP$ and $\omega$
    the circumcircle of the triangle $CSD$. Suppose $ \omega$ and $ \Omega $
    intersect at $K\not= C$. Prove that $ \angle CKM=90^\circ $.
}



\prob{https://artofproblemsolving.com/community/c6h79788p456609}
{APMO 1999 P3}{E}{
    Let $\Gamma_1$ and $\Gamma_2$ be two circles intersecting at $P$ and
    $Q$. The common tangent, closer to $P$, of $\Gamma_1$ and $\Gamma_2$ touches
    $\Gamma_1$ at $A$ and $\Gamma_2$ at $B$. The tangent of $\Gamma_1$ at $P$
    meets $\Gamma_2$ at $C$, which is different from $P$, and the extension of
    $AP$ meets $BC$ at $R$.	Prove that the circumcircle of triangle $PQR$ is
    tangent to $BP$ and $BR$.
}



\prob{}
{Simurgh 2019 P2}{E}{
    Let $ ABC $ be an isosceles triangle, $ AB=AC $.
    Suppoe that $ Q $ is a point such that $ AQ=AB,\ AQ||BC $. Let $ P $ be the
    foot of perpendicular line from $ Q $ to $ BC $. Prove that the circle with
    diameter $ PQ $ is tangent to the circumcircle of $ ABC $.
}



\prob{http://emc.mnm.hr/wp-content/uploads/2018/12/EMC_2018_Seniors_ENG_Solutions-2.pdf}
{European Mathematics Cup 2018 P2}{E}{
    Let $ABC$ be a triangle with $|AB|<|AC|$. Let $k$ be the circumcircle
    of $\triangle ABC$ and let $O$ be the center of $k$. Point $M$ is the
    midpoint of the arc $\widehat{BC}$ of $k$ not containing $A$. Let $D$ be
    the second intersection of the perpendicular line from $M$ to $AB$ with
    $k$ and $E$ be the second intersection of the perpendicular line from $M$
    to $AC$ with $k$. 

    Points $X$ and $Y$ are the intersections of $CD$ and
    $BE$ with $OM$ respectively. Denote by $k_{b}$ and $k_{c}$ circumcircles
    of triangles $BDX$ and $CEY$ respectively. Let $G$ and $H$ be the
    second intersections of $k_{b}$ and $k_{c}$ with $AB$ and $AC$
    respectively. Denote by $k_{a}$ the circumcircle of triangle $AGH$.

    Prove that $O$ is the circumcenter of $\triangle O_{a}O_{b}O_{c},$ where
    $O_{a}, O_{b}, O_{c}$ are the centers of $k_{a}, k_{b}, k_{c}$
    respectively.
}


\prob{https://artofproblemsolving.com/community/c6h1789909p11836144}{RMM 2019
    P2}{E}{
    Let $ABCD$ be an isosceles trapezoid with $AB\parallel CD$. Let $E$ be
    the midpoint of $AC$. Denote by $\omega$ and $\Omega$ the circumcircles of the
    triangles $ABE$ and $CDE$, respectively. Let $P$ be the crossing point of the
    tangent to $\omega$ at $A$ with the tangent to $\Omega$ at $D$. Prove that
    $PE$ is tangent to $\Omega$.
}


\prob{https://artofproblemsolving.com/community/c6h1709995p11022258}{IGO 2018
    A5}{E}{
    $ABCD$ is a cyclic quadrilateral. A circle passing through $A,B$ is
    tangent to segment $CD$ at point $E$. Another circle passing through $C,D$ is
    tangent to $AB$ at point $F$. Point $G$ is the intersection point of $AE,DF$,
    and point $H$ is the intersection point of $BE,CF$. Prove that the incenters
    of triangles $AGF,BHF,CHE,DGE$ lie on a circle.
}

\begin{solution}[juckter]
    The cases where two opposite sides of $ABCD$ are parallel are easily dealt
    with. Let $X = AB \cap CD$. Then $XE^2 = XA \cdot XB = XC \cdot XD =
    XF^2$, so $XE = XF$. Reflect $E$ through $X$ onto $E'$, and notice that
    $XE^2 = XC \cdot XD$ implies $(C, D; E, E') = -1$. Because $\angle EFE' =
    90^{\circ}$ (which follows from $XE = XF = XE'$) it follows that $FE$
    bisects $\angle CFD$ and analogously $EF$ bisects $\angle AEB$. It then
    follows easily that $G$ and $H$ are symmetric about $EF$.\\

    \figdf{.95}{IGO2018A5}{\autoref{problem:IGO 2018 A5} IGO 2018 A5}

    Now let $I_1, I_2, I_3$ and $I_4$ be the incenters of $AGF, DGE, CHE, BHF$
    respectively. Then $I_1I_2$ and $I_3I_4$ are the external bisectors of
    angles $EGF$ and $EHF$ respectively, and by symmetry about $EF$ these
    lines intersect at a (possibly ideal) point $X \in EF$.\\

    Finally, we may angle chase to find that $E, I_1, I_2, F$ and $E, I_3,
    I_4, F$ are quadruples of concyclic points. If $I_1I_2$ is parallel to
    $I_3I_4$ then we may easily conclude by symmetry about the perpendicular
    bisector of $EF$. Otherwise by Power of a Point from $X$ we have $XI_1
    \cdot XI_2 = XE \cdot XF = XI_3 \cdot XI_4$, so $I_1, I_2, I_3, I_4$ are
    concyclic, as desired.
\end{solution}







\prob{https://artofproblemsolving.com/community/c6h418983p2365045}
{ISL 2011 G8}{EM}{
    Let $ABC$ be an acute triangle with circumcircle $\Gamma$. Let $\ell$
    be a tangent line to $\Gamma$, and let $\ell_a, \ell_b$ and $\ell_c$ be the
    lines obtained by reflecting $\ell$ in the lines $BC$, $CA$ and $AB$,
    respectively. Show that the circumcircle of the triangle determined by the
    lines $\ell_a, \ell_b$ and $\ell_c$ is tangent to the circle $\Gamma$.
}
\figdf{.8}{ISL_2011_G8}{}

\solu{
    The main problem here are the reflected lines. We need to somehow know
    more about them. So we come up with some ways to construct the three lines
    without drawing the tangent $l$, which leads us to the reflection of $D$
    over the three sides idea. 

    And after doing some angle chasing to find out the angles of the triangle,
    we begin to see relationships between the reflection points and the
    vertices of the triangle.
}




\prob{https://artofproblemsolving.com/community/c6h1818716p12141505}{ELMO 2019
    P3}{EM}{
    Let $ABC$ be a triangle such that $\angle CAB > \angle ABC$, and let
    $I$ be its incentre. Let $D$ be the point on segment $BC$ such that $\angle
    CAD = \angle ABC$. Let $\omega$ be the circle tangent to $AC$ at $A$ and
    passing through $I$. Let $X$ be the second point of intersection of $\omega$
    and the circumcircle of $ABC$. Prove that the angle bisectors of $\angle DAB$
    and $\angle CXB$ intersect at a point on line $BC$.
}


\begin{solution}[Angle Chase] 
    Suppose the bisector of $ \angle BAD $ meet $ BC $ at $ G' $. Then we have, 
    \begin{align*} \angle BG'A &= \frac{\angle
        A-\angle B}{2}\\ \therefore \angle CG'A &= \angle B + \angle BG'A\\
        &=\frac{A+B}{2}\\[1em]
        \implies CG'&=CA\\ \therefore
        \angle G'ID &= \angle B 
    \end{align*}

    Now, let $ M $ be the midpoint of the minor arc $ BC $. Let $ G=XM\cap BC
    $. So we have \[\triangle MGI \sim \triangle MIX \implies \angle MIG =
    \angle MXI\]

    Let $ XI\cap \odot ABC = N \neq X $. Since $ AC $ is tangent to $ \odot
    AXI $, $ NC\parallel AM $. Which means $$  \angle MXI = \angle B = \angle
    MIG  $$ Which completes our proof by implying that $ G'\equiv G $. 

    \figdf{.4}{elmo_2019_P3}{} 
\end{solution}




\prob{https://artofproblemsolving.com/community/c6h596927p3542092}{ISL 2014
    G5}{M}{
    Convex quadrilateral $ABCD$ has $\angle ABC = \angle CDA = 90^{\circ}$.
    Point $H$ is the foot of the perpendicular from $A$ to $BD$. Points $S$ and
    $T$ lie on sides $AB$ and $AD$, respectively, such that $H$ lies inside
    triangle $SCT$ and \[ \angle CHS - \angle CSB = 90^{\circ}, \quad \angle THC -
    \angle DTC = 90^{\circ}.\] Prove that line $BD$ is tangent to the circumcircle
    of triangle $TSH$.
}

\solu{
    First construct using nice circles, then prove the center is on $ AH $
    using angle bisector theorem.  
    \figdf{.6}{ISL2014G5}{Construction}
    \figdf{.3}{ISL2014G5_lemma1}{Lemma}
}



\prob{https://artofproblemsolving.com/community/c6h1113194p5083564}{ISL 2014
    G7}{M}{
    Let $ABC$ be a triangle with circumcircle $\Omega$ and incentre $I$.
    Let the line passing through $I$ and perpendicular to $CI$ intersect the
    segment $BC$ and the arc $BC$ (not containing $A$) of $\Omega$ at points $U$
    and $V$ , respectively. Let the line passing through $U$ and parallel to $AI$
    intersect $AV$ at $X$, and let the line passing through $V$ and parallel to
    $AI$ intersect $AB$ at $Y$ . Let $W$ and $Z$ be the midpoints of $AX$ and
    $BC$, respectively. Prove that if the points $I, X,$ and $Y$ are collinear,
    then the points $I, W ,$ and $Z$ are also collinear.
}

\solu{Draw a nice diagram, and use the parallel property to find circles.
\figdf{.7}{ISL2014G7}{ISL 2014 G7}}



\prob{https://artofproblemsolving.com/community/c6h1112748p5079655}{ISL 2015
    G6}{E}{
    Let $ABC$ be an acute triangle with $AB > AC$. Let $\Gamma $ be its
    cirumcircle, $H$ its orthocenter, and $F$ the foot of the altitude from
    $A$. Let $M$ be the midpoint of $BC$. Let $Q$ be the point on $\Gamma$
    such that $\angle HQA = 90^{\circ}$ and let $K$ be the point on $\Gamma$
    such that $\angle HKQ = 90^{\circ}$. Assume that the points $A$, $B$, $C$,
    $K$ and $Q$ are all different and lie on $\Gamma$ in this order.\\

    Prove that the circumcircles of triangles $KQH$ and $FKM$ are tangent to each
    other.
}

\solu{
    Draw the tangent line, and find angles.  
    \figdf{.7}{ISL2015G6}{ISL 2015 G6}
}


\begin{minipage}{.5\linewidth}
    \prob{https://artofproblemsolving.com/community/c6h1268908p6622796}{ISL 2015
        G5}{M}{
        Let $ABC$ be a triangle with $CA \neq CB$. Let $D$, $F$, and $G$ be the
        midpoints of the sides $AB$, $AC$, and $BC$ respectively. A circle $\Gamma$
        passing through $C$ and tangent to $AB$ at $D$ meets the segments $AF$ and
        $BG$ at $H$ and $I$, respectively. The points $H'$ and $I'$ are symmetric to
        $H$ and $I$ about $F$ and $G$, respectively. The line $H'I'$ meets $CD$ and
        $FG$ at $Q$ and $M$, respectively. The line $CM$ meets $\Gamma$ again at $P$.
        Prove that $CQ = QP$.
    }
\end{minipage}\hfill%
\begin{minipage}{.46\linewidth}
    \figdf{}{ISL2015G5}{ISL 2015 G5}
\end{minipage}


\prob{https://artofproblemsolving.com/community/c6h418635p2361976}{ISL 2010
    G5}{E}{
    Let $ABCDE$ be a convex pentagon such that $BC \parallel AE,$ $AB = BC
    + AE,$ and $\angle ABC = \angle CDE.$ Let $M$ be the midpoint of $CE,$ and let
    $O$ be the circumcenter of triangle $BCD.$ Given that $\angle DMO =
    90^{\circ},$ prove that $2 \angle BDA = \angle CDE.$
}

\solu{First try to construct the point. Do this the long way, then find a
    easier way that includes $ B, C $, not $ B, A $ to do that. Then try to
    translate what $ 90 $ degree condition into angles, and take midpoints, since
    we have midpoints involved.  
    \figdf{.7}{ISL2010G5}{ISL 2010 G5}
}



\prob{https://artofproblemsolving.com/community/c6h1918849p13155415}{IGO 2019
    A5}{MH}{
    Let points $A, B$ and $C$ lie on the parabola $\Delta$ such that the
    point $H$, orthocenter of triangle $ABC$, coincides with the focus of parabola
    $\Delta$. Prove that by changing the position of points $A, B$ and $C$ on
    $\Delta$ so that the orthocenter remain at $H$, inradius of triangle $ABC$
    remains unchanged.
}	

\solu{ 
    \figdf{.7}{IGO2019G5}{IGO 2019 A5} 
    I think the idea for inversion should have been pretty natural after
    finding that the incircle is fixed.  
}

\prob{https://artofproblemsolving.com/community/c6h1140464p5353018}{Iran 3rd
    Round 2015 P5}{M}{
    Let $ABC$ be a triangle with orthocenter $H$ and
    circumcenter $O$. Let $R$ be the radius of circumcircle of $\triangle ABC$.
    Let $A',B',C'$ be the points on
    $\overrightarrow{AH},\overrightarrow{BH},\overrightarrow{CH}$ respectively
    such that $AH.AA'=R^2,BH.BB'=R^2,CH.CC'=R^2$. Prove that $O$ is incenter of
    $\triangle A'B'C'$.
}

\solu{The condition easily leads to a nice construction of the points. It
    should be trivial to figure that the construction is really important. Also,
noticing a similarity among the triangles is really important.}







\newpage
\subsection{The line parallel to BC}


\den{Important Points}{
    Let the line parallel to $ BC $ through $ O $ meet $ AB, AC $ at $ D, E $. 
    Let $ K $ be the midpoint of $ AH $, $ M $ be the midpoint of $ BC $. $ F $
    be the feet of $ A $-altitude on $ BC $ and let $ H' $ be the reflection of $
    H $ on $ F $. Let $ O' $ be the circumcenter of $ KBC $.
}


\lem{}{
    $ \angle DKC = \angle EKB = 90^{\circ} $
}\label{lemma:linethroughO_lemma1}

\fig{1}{linethroughO_1}{}


\lem{}{$
    CD, BE, OH', AM, KO' $ are concurrent. (by
    \autoref{isogonality_lemma})
}\label{lemma:linethroughO_lemma2}

\fig{1}{linethroughO_2}{}


\prob{https://artofproblemsolving.com/community/c6h1412598_circumcircles_intersect_on_ao}{InfinityDots
    MO Problem 3}{EM}{
    Let $\triangle ABC$ be an acute triangle with circumcenter
    $O$ and orthocenter $H$. The line through $O$ parallel to $BC$ intersect $AB$
    at $D$ and $AC$ at $E$. $X$ is the midpoint of $AH$. Prove that the
    circumcircles of $\triangle BDX$ and $\triangle CEX$ intersect again at a
    point on line $AO$.
}


\solu{Just using \autoref{lemma:linethroughO_lemma1} to get another pair of
circle where we can apply radical axis arguments.}

\solu{Noticing that the resulting point is the isogonal conjugate of a well
defined point,}



\lem{}{
    Let $ P, Q $ be on $ AB, AC $ resp. such that $ PQ\parallel BC $. And
    let $ A' $ be such that $ A'\in \odot ABC, AA'\parallel BC $. Let $ CP\cap BQ
    = X $, and let the perpendicular bisector of $ BC $ meet $ PQ $ at $ Y $.
    Prove that $ A', X, Y $ are
    collinear.
}\label{lemma:concurrency_in_prallel_lines_with_BC}

\solu{No angles... Do Lengths...}

\fig{1}{concurrency_in_prallel_lines_with_BC}{}


\prob{https://artofproblemsolving.com/community/c6h1634961p10278557}{ARO 2018
    P11.4}{M}{
    $ P \in AB,\ Q \in AC,\ PQ\parallel BC,\ BQ\cap CP = X $. $ A' $ is
    the reflection of $ A $ wrt $ BC $. $ A'X\cap \odot APQ = Y $. Prove that $
    \odot BYC $ is tangent to $ \odot APQ $.
}

\solu{Of co it can be solved using angle chase,
    \autoref{lemma:concurrency_in_prallel_lines_with_BC} makes it almost
trivial.}


\prob{https://artofproblemsolving.com/community/c6h520197}{buratinogigle}{EM}{
    Let $(O)$ be a circle and $E,F$ are two points inside $(O)$. $(K),(L)$ are
    two circles passing though $E,F$ and tangent internally to $(O)$ at $A,D$,
    respectively. $AE,AF$ cut $(O)$ again at $B,C$, respectively. $BF$ cuts
    $CE$ at $G$. Prove that reflection of $A$ though $EF$ lies on line $DG$.\\

    Rephrasing the problem as such: In the setup of
    \autoref{lemma:concurrency_in_prallel_lines_with_BC}, let $ A'X\cap
    \odot ABC = Z $, then $ \odot PQZ $ is tangent to $ \odot ABC $.
}


\solu{Simple angle chase. }

\solu{Another solution to this is by taking $ D $ as a phantom point.}

\solu{Another solution is with cross ratios}







\newpage
\subsection{Simson Line and Stuffs}


\lem{Simson Line Parallel}{
    Let $P$ be a point on the circumcircle, let $P'$ be the reflection of $P$
    on $BC$ and let $PP'\cap\Omega = Q$, and let $l_p$ be the Simson line of
    $P$. Prove that $l_p\parallel AD\parallel HP'$.
}

\lem{Simson Line Angle}{
    Given triangle $ABC$ and its circumcircle $(O)$.  Let $E, F$ be two
    arbitrary points on $(O)$. Then the angle between the Simson lines of
    two points $E$ and $F$ is half the measure of the arc $EF$.
}



\begin{minipage}{.5\linewidth}
    \figdf{.7}{SimsonLineLemma1}{\autoref{lemma:Simson Line Parallel}}
\end{minipage}\hfill%
\begin{minipage}{.5\linewidth}
    \figdf{.8}{SimsonLineLemma2}{\autoref{lemma:Simson Line Angle}}
\end{minipage}





\newpage\subsection{Euler Line}



\theo{}{Perspectivity Line with Orthic triangle is perpendicular to Euler
    line}{Let $ DEF $ be the orthic triangle. Then $ BC\cap EF, CA\cap FD, AB\cap
    ED $ are collinear, and the line is perpendicular to the Euler line. In fact
this line is the radical axis of the Circumcircle and the NinePoint circle}




\lem{}{$
    DEF $ is orthic triangle of $ ABC $, $ XYZ $ is the orthic triangle
    of $ DEF $. Prove that the perspective point of $ ABC $ and $ XYZ $ lies on
    the Euler line of $ ABC $
}

\begin{prooof}
    Thinking the stuff wrt to the incircle and using cross ratio.
\end{prooof}


\begin{minipage}{.5\linewidth}
    \lem{Perpendicular on Euler Line}{
        Let $E, F$ be the feet of altitudes from $B, C$, and let $M, N$ be the
        midpoints of $AC, AB$. $D = MN\cap AB$. Prove that $AF$ is perpendicular
        to the euler line.
    }
    \begin{prooof}
        We know that $MNEF$ lie on a circle. So $D$ is the radical center of
        $\odot MNEF, \odot AMN, \odot AEF$.
    \end{prooof}
\end{minipage}\hfill%
\begin{minipage}{.45\linewidth}
    \figdf{.9}{perp_to_euler_line}{}
\end{minipage}



\begin{minipage}{.54\linewidth}
    \prob{https://artofproblemsolving.com/community/c6h624551p3740031}
    {CHKMO 2014 P4}{EM}{
        Let $\triangle ABC$ be a scalene triangle, and let $D$ and $E$ be points
        on sides $AB$ and $AC$ respectively such that the circumcircles of
        triangles $\triangle ACD$ and $\triangle ABE$ are tangent to $BC$. Let $F$
        be the intersection point of $BC$ and $DE$. Prove that $AF$ is
        perpendicular to the Euler line of $\triangle ABC$.
    }
    \begin{solution}
        Let $B', C'$ be the points on $AC, AB$ such that $BB' = BA$, $CA = CC'$.
        By simple angle chasing, we can show that $BE, CD$ are tangent to $\odot
        BB'CC'$.\\

        Now by Pascal's theorem on $BBB'CCC'$, we can show that $D, E, F$ are
        collinear where $F = BC\cap B'C'$. Then by \autoref{lemma:Perpendicular on
        Euler Line}, we have that $AF\perp HO$.
    \end{solution}

    \lem{Extension of CHKMO}{
        $AS$, $BB'$, $CC'$ are concurrent.
    }
\end{minipage}\hfill%
\begin{minipage}{.45\linewidth}
    \figdf{.9}{CHKMO_2014_p4}{}
\end{minipage}



\newpage
\subsection{Assorted Diagrams}

\figdf{.5}{random_circle_pic_1}{$ H $ lies on the line, circles vary}

