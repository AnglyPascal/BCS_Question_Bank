\graphicspath{{Pics/}}

\newpage
\section{Conjugates}


\subsection{Isogonal Conjugate}

\begin{minipage}{.5\linewidth}
    \theo{https://artofproblemsolving.com/community/c2771h1181729}
    {Isogonal Line Lemma}{
        Let $AP,AQ$ are isogonal lines with respect to $\angle BAC$. Let $BP \cap
        CQ = F$ and $BQ \cap CP = E$. Then $AE,AF$ are isogonal lines with respect
        to $\angle BAC$.
    }

    \proof{
        \[\begin{aligned}
            A(B, F; P, X) &= (B, F; P, X) = C(B, Q; E, X) \\
                          &= (B, Q; E, X) = (X, E; Q, B) 
        \end{aligned}\]
        So if we define a projective transformation that swaps isogonal lines wrt $
        \angle BAC $, we see $ AE, AF $ are conjugates of each other.
    }
\end{minipage}\hfill%
\begin{minipage}{.47\linewidth}
    \figdf{}{Isogonal_Lemma}{}
\end{minipage}





\prob{}
{India Postals 2015 Set 2}{E}{
    Let $ABCD$ be a convex quadrilateral. In the triangle $ABC$ let $I$ and
    $J$ be the incenter and the excenter opposite the vertex $A$,
    respectively. In the triangle $ACD$ let $K$ and $L$ be the incenter and
    the excenter opposite the vertex $A$, respectively. Show that the lines
    $IL$ and $JK$, and the bisector of the angle $BCD$ are concurrent.
}

\solu{
    Using \autoref{theorem:Isogonal Line Lemma}
}


\lem{}{ 
    Let $ \w_1, \w_2 $ be two circles such that $ \w_1 $ passes through $
    A, B $ and is tangent to $ AC $ at $ A $. $ \w_2 $ is defined similarly by
    swapping $ B $ with $ C $. $ \w_1\cap\w_2 = X $.

    Let $ \gamma_1, \gamma_2 $ be two circles such that $ \gamma_1 $ passes
    through $ A, B $ and is tangent to $ BC $ at $ B $. $ \gamma_2 $ is
    defined similarly by swapping $ B $ with $ C $. $ \gamma_1\cap\gamma_2 = Y$.

    Then $ X, Y $ are isogonal conjugates wrt $ \triangle ABC $.
}	



\begin{minipage}{.5\linewidth}
    \lem{Isogonality in quadrilateral}{
        For a point $ X $, its isogonal conjugate wrt a quadrilateral $ ABCD $
        exists iff \[ \measuredangle BXA + \measuredangle DXC = 180^\circ\]
    }
    \solu{
        Draw the cirles, look for similarity.
    }
\end{minipage}\hfill%
\begin{minipage}{.49\linewidth}
\figdf{1}{isogonal_in_quad}{Isogonality in quadrilateral}
\end{minipage}







\begin{minipage}{.5\linewidth}
    \lem{Ratio}{
        Given a $ \triangle ABC $ with isogonal conjugate $ P, $ $ Q. $ Let $ AP,
        $ $ AQ $ cut the circumcircle of $ \triangle ABC $ again at $ U, $ $ V, $
        respectively and let $ D $ $ \equiv $ $ AP $ $ \cap $ $ BC. $ Then
        \[\frac{AQ}{QV} = \frac{PD}{DU}\]
    }
    \begin{prooof}
        By using cross ratio:
        \[\begin{aligned}
            (A, F; Q, V) &= C(A, F; Q, V)\\
                         &=C(D, A; P, V*) =(D, A; P, V*)\\
                         &=(A, D; V*, P)
        \end{aligned}\]
    \end{prooof}
\end{minipage}\hfill%
\begin{minipage}{.45\linewidth}
    \figdf{.9}{AoPS_c284651h2018537_cute_problem_on_radical_axis_1}{}
\end{minipage}





\newpage\subsubsection{Symmedians}




\begin{minipage}{.5\linewidth}
    \den{Symmedians}{
        In $ \triangle ABC $, let $ T_a, T_b, T_c $ be the meet points of the
        tangents at $ A, B, C $. Let $ \triangle N_aN_bN_c $ be the cevian
        triangle of $ AT_a, BT_b, CT_c $. Let $ S $ be the symmedian point of $
        \triangle ABC $. Let $ M_a, M_b, M_c $ be the midpoints of $ BC, CA, AB $.
    }

    \lem{Most Important Symmedian Property}{
        Let the circles tangent to $ AC, AB $ at $ A $ and passes through $ B, C $
        respectively meet at $ T' $ for the second time. Let $ AT_a\cap \odot ABC
        = A' $. Let the tangents to $ \odot ABC $ at $ A, A' $ meet $ BC $ at $ T
        $. Prove that, $ A, T', T_a, \text{ and } T, T', O $ are collinear.
    }

\end{minipage}\hfill%
\begin{minipage}{.47\linewidth}
    \figdf{}{Symmedian_Lemma_1}{$ T' $ is quite special!}
\end{minipage}

\vspace{2em}


\begin{minipage}{.5\linewidth}
    \prob{https://artofproblemsolving.com/community/c5h202907p1116181}
    {USAMO 2008 P2}{E}{
        Let $ ABC$ be an acute, scalene triangle, and let $ M$, $ N$, and $ P$
        be the midpoints of $ {BC}$, $ {CA}$, and $ {AB}$, respectively. Let
        the perpendicular bisectors of $ {AB}$ and $ {AC}$ intersect ray $ AM$
        in points $ D$ and $ E$ respectively, and let lines $ BD$ and $ CE$
        intersect in point $ F$, inside of triangle $ ABC$. Prove that points
        $ A$, $ N$, $ F$, and $ P$ all lie on one circle.
    }

    \begin{solution}[Phantom Point]
        First assume $ F\in BD, \text{ and } F=T' $ (Where $ T' $ comes from
        \autoref{lemma:Most Important Symmedian Property}, and prove that $
        F\in CE $.)
    \end{solution}
\end{minipage}\hfill%
\begin{minipage}{.47\linewidth}
    \figdf{}{USAMO_2008_P2}{USAMO 2008 P2}
\end{minipage}

\vspace{1em}

\begin{solution}[Isogonal Conjugate]
    Construct the isogonal conjugate of $ F $, which is the intersection of the circles touching $ BC $ and passing through $ A, B $ and $ A, C $.
\end{solution}
\solu{Using \autoref{theorem:Isogonal Line Lemma} by taking the reflections of $ B , C $ over $ D, F $}


\prob{https://artofproblemsolving.com/community/c6h1095220p4902494}
{IRAN TST 2015 Day 3, P3}{M}{
    $AH$ is the altitude of triangle $ABC$ and $H^\prime$ is the reflection of $H$
    trough the midpoint of $BC$. If the tangent lines to the circumcircle of $ABC$
    at $B$ and $C$, intersect each other at $X$ and the perpendicular line to
    $XH^\prime$ at $H^\prime$, intersects $AB$ and $AC$ at $Y$ and $Z$
    respectively, prove that $\angle ZXC=\angle YXB$.
}


\begin{minipage}{.5\linewidth}
    \prob{https://web.facebook.com/photo.php?fbid=539157073197109}
    {Two Symmedian Points}{E}{
        Let $E, F$ be the feet of $B, C$-altitudes. Let $K, K_A$ be the symmedian
        points of $\triangle ABC, \triangle AEF$. Prove that $KK_A\perp BC,
        KK_A\cap BC = P$ and $KK_A=KP$
    }
\end{minipage}\hfill%
\begin{minipage}{.45\linewidth}
    \figdf{.8}{Two_Symmedian_Points}{$KK_A\perp BC$}
\end{minipage}









\newpage\subsection{Isotonic Conjugate}

\theo{}{Isotonic Lemma}{Let $ M $ be the midpoint of $ BC $, and $ PQ $ such that $ Q $ is the reflection of $ P $ on $ M $. Two points $ Q, R $ on $ AP, AQ $, $ BQ\cap CR = X,\ BR\cap CQ = Y $. Then $ AX, AY $ are isotonic wrt $ BC $.
    \fig{1}{Isotonic_Lemma}{}
}\label{isotonic_lemma}



\prob{https://artofproblemsolving.com/community/c6h626330p3756450}{IGO 2014 S5}{M}{Two points $P$ and $Q$ lying on side $BC$ of triangle $ABC$ and their distance from the midpoint of $BC$ are equal.The perpendiculars from $P$ and $Q$ to $BC$ intersect $AC$ and $AB$ at $E$ and $F$,respectively.$M$ is point of intersection $PF$ and $EQ$.If $H_1$ and $H_2$ be the orthocenters of triangles $BFP$ and $CEQ$, respectively, prove that $ AM\perp H_1H_2 $.}

\solu{We first show that the slope of $ H_1H_2 $ is fixed, and then show that $ AM $ is fixed where we use \hrf{isotonic_lemma}{isotonic lemma}, and finally show that these two lines are perpendicular.}



\newpage\subsection{Reflection}

\lem{Homothety and Reflection}{Let two oppositely oriented congruent triangles be $ \triangle ABC, \triangle DEF $. Prove that the midpoints of $ AD, BE, CF $ are collinear. 
    \fig{.7}{homothety+reflection}{Oppositely oriented congruent triangles}
}

\prob{}{Autumn Tournament, 2012}{E}{Let two oppositely oriented equilateral triangles be $ \triangle ABC, \triangle DEF $. What is the least possible value of $ \max{\left (AD, BE, CF\right )} $?}
