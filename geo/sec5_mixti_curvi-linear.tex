\graphicspath{{Pics/}}

\newpage\section{Circles that are really touching}



\den{Mixtilinear Circle}{
    Let $\triangle ABC$ be an ordinary triangle, $I$ is its incenter, $D$ is
    the touch points of the incenter with $BC$. Let $\omega$ be the
    mixtilinear incircle. Let it touch $CA, AB$ at $E, F$. Furthermore, let
    $\omega\cap\odot ABC\equiv T$. Let $M_a, M_b, M_c$ be the midpoints of the
    smaller arcs $BC, CA, AB$, and $M_A, M_B, M_C$ be the midpoints of the
    major arcs $BC, CA, AB$.

    \figdf{.9}{Mixtilinear}{Mixtilinear Incircle: more than meets the eye}
}

\begin{prooof} 
    $E, I, F$ are collinear. Consider the circle $ TI'EC $ and do so $T, I,
    M_A$ are collinear. Consider the circle $TIEC$ and apply reim's theorem
    and angle chasing. Also, proving that $AC$ touches $\odot TCD$ proves that
    $T, D, A'$ collinear.
\end{prooof}

\lem{}{
    The bundle $ (A, T; M_b, M_c) $ is harmonic. And $ TA $ is a symmedian of
    $\triangle TM_cM_b$.  \[\frac{TM_c}{M_cA}=\frac{TM_b}{M_bA}\] 
}\label{lemma:mixtilinear_lemma_harmonic_with_arc_midpoints}
\newpage

\begin{minipage}{.4\linewidth}
    \lem{ISL 1999 G8}{
        Let $X$ be a variable point on the arc $AB$, and let $O_{1}$ and $O_{2}$
        be the incenters of the triangles $CAX$ and $CBX$. Then $X, O_{1}, O_{2}$
        and $T$ lie on a circle.
    }\label{lemma:two_incenters_of_a_cyclic_quad_cyclic_with_mixtilinear_touchpoint}

    \solu{
        Using similarity and
        \autoref{lemma:mixtilinear_lemma_harmonic_with_arc_midpoints}.
    }
\end{minipage}\hfill%
\begin{minipage}{.59\linewidth}
    \figdf{1}{incenters_cyclic_with_mixtilinear_touchpoint}{The two incenters are
    cyclic with $ T, X $}
\end{minipage}

\prob{https://artofproblemsolving.com/community/c6t48f6h1508805_nice_but_not_hard}
{AoPS1}{H}{
    Let $ABCD$ be a quadrilateral inscribed in a circle, such that the
    inradius of $\triangle ABC$ and $ACD$ are the same. Let $T$ be the
    touchpoint of $A$-mixtilinear incircle of the triangle $ABD$ with $\odot
    ABCD$. Let $I_1,I_2$ be the incenters of the triangles $ABC, ACD$
    respectively. Show that $I_1I_2$ and the tangents of $A,T$ wrt $\odot
    ABCD$ are concurrent.
}

\solu{
    We know from
    \autoref{lemma:two_incenters_of_a_cyclic_quad_cyclic_with_mixtilinear_touchpoint},
    that $TI_1I_2C$ is cylic. Moreover from the given condition, we know that
    $AC$ bijects $I_1, I_2$. Which gives us $I_1I_2\parallel TC$. Using that
    we can show that if $S= I_1I_2\cap M_cM_b$, then $TS$ is tangent to $\odot
    ABC$, which also gives us $SA$ is tangent to $\odot ABC$.

    \figdf{.6}{c6t48f6h1508805}{}
}


\begin{minipage}{.5\linewidth}
    % \prob{https://artofproblemsolving.com/community/c6h1570826p9686418}
    \den{Generalization of Mixtilinear Incirlce}{
        \href{https://artofproblemsolving.com/community/c6h1570826p9686418}{Consider}
        $\triangle ABC$ and let $M_c, M_b$ are midpoints of arcs $AB, AC$. Let $E,
        F$ on $AB, AC$ such that $EF\parallel M_bM_c$. Let $EM_b, FM_c$ meet $(ABC)$ second
        time at $P, Q$. Consider two intersection points $E', F'$ of $(EFPQ)$ with
        $AB, AC$ different from $E, F$. Then $EF'\cap E'F$ is the incenter of $ABC$.\\

        In other words, if $P$ lies on the arc $BC$, and $M_bP\cap AC = E$, 
        $M_cP\cap AB = F'$, then $E, F', I$ are collinear.
    }
\end{minipage}\hfill%
\begin{minipage}{.49\linewidth}
    \figdf{.75}{Mixtilinear_Gene}{}
\end{minipage}


\begin{minipage}{.5\linewidth}
    \prob{}
    {Archer - EChen M1P3}{E}{
        Let the incenter  touch $ BC $ at $ D $. Let $ AI\cap BC = E,\ AI\cap
        \odot ABC = F$. \hl{Prove that $\odot DEF \cap \odot ABC = T$}, the
        mixtilinear touchpoint. Now let, $\odot DEF \cap \odot (I_a) = S_1,
        S_2 $. \hl{Prove that $ AT $ goes through either $S_1$ or $S_2$}.
    }

    \begin{solution}
        We already know that $T, D, E, F$ are cyclic. Then, let $S = AT\cap
        \odot DEF$, then we have, 
        \[\angle FSE = \angle FDE = \angle FD'E\] 
        Which means $S'$ is the reflection of $D'$ over $AI$, which definitely
        lies on $\left(I_a\right)$.
    \end{solution}
\end{minipage}\hfill%
\begin{minipage}{.49\linewidth}
    \figdf{.85}{Archer_EChen_M1P3}{\autoref{problem:Archer - EChen M1P3}}
\end{minipage}


\begin{minipage}{.5\linewidth}
    \prob{}{}{}{
        Let the $ B $-mixtilinear and $ C $-mixtilinear circles touch $ BC $ at $ X_B,
        X_C$ respectively. Then $X_B, X_C, T, M_a$ lie on a circle.
    }

    \begin{solution}
        We use mixtilinear related lemma to show that, $ST \cdot SM_a = SX_b \cdot
        SX_c$. We can do that by angle chase to show that $SI^2 = SX_b\cdot SX_c$.
    \end{solution}
\end{minipage}\hfill%
\begin{minipage}{.49\linewidth}
    \figdf{.9}{mixti_touch_with_BC}{}
\end{minipage}


\theo{}
{Root BC Inversion in Mixtilinear Circles}{
    If we invert around $A$ with the radius $\sqrt{bc}$, the mixtilinear
    incircle goes to the excicle.
}

\begin{prooof}
    We use \autoref{problem:Archer - EChen M1P3} to show the results.
\end{prooof}

\figdf{.6}{mixtilinear_root_bc}{}

\prob{}{Taiwan TST 2014 T3P3}{EH}{
    Let $M$ be any point on the circumcircle of $\triangle ABC$. Suppose the
    tangents from $M$ to the incircle meet BC at two points $X_1$ and $X_2$.
    Prove that $T, M, X_1, X_2$ lie on a circle.
}

\den{Curvilinear Incircle}{
    Let $ABCD$ be a cyclic quadrilateral. $AC$ meets $BD$ at $X$. We call the
    circle that touches $AX, BX$ and the circumcircle from the inside a
    curvilinear incircle. The curvilinear incircle touch $AX, BX$ at $P, Q$
    resp. And let the incircle of $\triangle ABD$ be $I$. Then $P, Q, I$ are
    collinear.

    \figdf{.7}{curvi_lem_1}{}
}


\theo{http://forumgeom.fau.edu/FG2003volume3/FG200325.pdf}
{Sawayama and Thebault's theorem}{
    Let $O_1, O_2$ be the centers of the two curvilinear incircles on $BD$ and
    $AC$. Then $O_1, I, O_2$ are collinear.
}

\solu{
    Notice that this is similar to the circles $ TIEC $ and $ TIFB $ in the
    mixtilinear circle figures.
}





