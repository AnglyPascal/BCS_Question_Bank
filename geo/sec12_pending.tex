\newpage\section{Pending Problems}
	
	



		\prob{}{}{}{In $\triangle ABC$, $I$ is the incenter, $D$ is the touch point of the incenter with $BC$. $AD\cap\odot ABC\equiv X$. The tangents line from $X$ to $\odot I$ meet $\odot ABC$ at $Y,Z$. Prove that $YZ,BC$ and the tangent at $A$ to $\odot ABC$ concur.}
	
	
		


		\prob{https://artofproblemsolving.com/community/c6h1423007p8003905}{IRAN TST 2017 Day 1, P3}{M}{	In triangle $ABC$ let $I_a$ be the $A$-excenter. Let $\omega$ be an arbitrary circle that passes through $A, I_a$ and intersects the extensions of sides $AB, AC$ (extended from $B,C$) at $X,Y$ respectively. Let $S,T$ be points on segments $I_aB, I_aC$ respectively such that $\angle AXI_a=\angle BTI_a$ and $\angle AYI_a=\angle CSI_a$. Lines $BT, CS$ intersect at $K$. Lines $KI_a, TS$ intersect at $Z$.
		Prove that $X, Y, Z$ are collinear.}
	
	
		


		
	
	
		


		\prob{https://artofproblemsolving.com/community/c182907_2015_iran_team_selection_test}{IRAN TST 2015 Day 3, P2}{}{In triangle $ABC$(with incenter $I$) let the line parallel to $BC$ from $A$ intersect circumcircle of $\triangle ABC$ at $A_1$ let $AI\cap BC=D$ and $E$ is tangency point of incircle with $BC$ let $ EA_1\cap \odot (\triangle ADE)=T$ prove that $AI=TI$.}
	
			\fig{1}{ITST2015D3P2}{IRAN TST 2015 Day 3, P2}
	
	
	
		


		\prob{http://artofproblemsolving.com/community/c6h1423742p8012916}{Generalization of Iran TST 2017 P5}{}{Let $ABC$ be triangle and the points $P, Q$ lies on the side $BC$ s.t $B, C, P, Q$ are all different. The circumcircles of triangles $ABP$ and $ACQ$ intersect again at $G$. $AG$ intersects $BC$ at $M$. The circumcircle of triangle $APQ$ intersects $AB, AC$ again at $E, F$ , respectively. $EP$ and $FQ$ intersect at $T$. The lines through $M$ and parallel to $AB, AC$, intersect $EP, FQ$ at $X, Y$, respectively. Prove that the circumcircles of triangle $TXY$ and $APQ$ are tangent to each other.}
	
	
		


		\prob{https://artofproblemsolving.com/community/c6h535325p3070851}{ARMO 2013 Grade 11 Day 2 P4}{}{Let  $ \omega $ be the incircle of the triangle $ABC$ and with center $I$. Let $\Gamma $ be the circumcircle of the triangle $AIB$. Circles $ \omega $ and $ \Gamma $ intersect at the point $X$ and $Y$. Let $Z$ be the intersection of the common tangents of the circles $\omega$ and $\Gamma$. Show that the circumcircle of the triangle $XYZ$ is tangent to the circumcircle of the triangle $ABC$.}
	
		


		\prob{https://artofproblemsolving.com/community/c6h1490457p8788075}{AoPS}{}{Let $ABC$ be a triangle with incircle $(I)$ and $A$-excircle $(I_a)$. $(I), (I_a)$ are tangent to $BC$ at $D,P$, respectively. Let $(I_1), (I_2)$ be the incircle of triangles $APC,APB$, respectively, $(J_1), (J_2)$ be the reflections of $(I_1), (I_2)$ wrt midpoints of $AC,AB$. Prove that $AD$ is the radical axis of $(J_1)$ and $(J_2)$.}
	
	
		


		\prob{https://artofproblemsolving.com/community/c6h1491448p8781804}{AoPS}{}{Let $ABC$ be a $A-$right-angled triangle and $MNPQ$ a square inscribed into it, with $M,N$ onto $BC$ in order $B-M-N-C$, and $P,Q$ onto $CA, AB$ respectively. Let $R=BP\cap QM, S=CQ\cap PN$. Prove that $AR=AS$ and $RS$ is perpendicular to the $A-$inner angle bisector of $\triangle ABC$.}
	
	
		


		\prob{https://artofproblemsolving.com/community/c6h282732}{AoPS}{}{$P$ is an arbitrary point on the plane of $\triangle ABC$ and let $\triangle A'B'C'$ be the cevian triangle of $P$ WRT $\triangle ABC.$ The circles $\odot (ABB')$ and $\odot (ACC')$ meet at $A,X.$ Similarly, define the points $Y$ and $Z$ WRT $B$ and $C.$ Prove that the lines $ AX,BY,CZ$ concur at the isogonal conjugate of the complement of $P$ WRT $ \triangle ABC.$}
	
	
		


		\prob{https://artofproblemsolving.com/community/c6h108111p8724592}{AoPS}{}{Given are $\triangle ABC, L$ is Lemoine point,$L_{a},L_{b},L_{c}$ are three Lemoine point of triangles $LBC,LCA,LAB$ prove that $AL_{a},BL_{b},CL_{a}$ are concurrent! 
		
		A question: What is the locus of point $P$ such that $AL_{a},BL_{b},CL_{a}$ are concurrent with $L_{a},L_{b},L_{c}$ are three `Lemoine points' of triangles $PBC, PCA, PAB$?}
	
	
		


		\prob{https://artofproblemsolving.com/community/c6h1432386p8088082}{AoPS}{}{Let $ABC$ be a triangle inscribed circle $(O)$. Let $(O')$ be the circle wich is tangent to the circle $(O)$ and the sides $CA, AB$ at $D$ and $E, F$, respectively. The line $BC$ intersects the tangent line at $A$ of $(O)$, $EF$ and $AO'$ at $T, S$ and $L$, respectively. The circle $(O)$ intersects $AS$ again at $K$. Prove that the circumcenter of triangle $AKL$ lies on the circumcircle of triangle $ADT$.}
	
	
		



		\prob{}{}{}{Let $P$ and $Q$ be isogonal conjugates of each other. Let $\triangle XYZ, \triangle KLM$ be the pedal triangles of $P$ and $Q$ wrt $\triangle ABC$. ($X,K$ lie on $BC$; $Y,L$ lie on $CA$; $Z,M$ lie on $AB$) Prove that $YM,ZL,PQ$ are concurrent.}
	
	
		


		\prob{}{2nd Olympiad of Metropolises}{}{Let $ABCDEF$ be a convex hexagon which has an inscribed circle and a circumscribed circle. Denote by $\omega_A,\ \omega_B,\ \omega_C,\  \omega_D,\ \omega_E$, and $\omega_F$ the inscribed circles of the triangles $FAB,ABC,BCD,CDE,DEF$,and $EFA$,respectively. Let $l_{AB}$ be the external common tangent of $\omega_A$ and $\omega_B$ other than the line $AB$; lines $l_{BC},l_{CD},l_{DE},l_{EF}$, and $l_{FA}$ are analogously defined. Let $A_1$ be the intersection point of the lines $l_{FA}$ and $l_{AB}$; $B_1$ be the intersection point of the lines $l_{AB}$ and $l_{BC}$; points $C_1,D_1,E_1$, and $F1$ are analogously defined. Suppose that $A_1 B_1 C_1 D_1 E_1 F_1$ is a convex hexagon. Show that its diagonals $A_1 D_1, B_1 E_1$, and $C_1 F_1$ meet at a single point.}




		\prob{https://artofproblemsolving.com/community/c6h1480719p8639316}{ISL 2016 G6}{}{Let $ABCD$ be a convex quadrilateral with $\angle ABC = \angle ADC < 90^{\circ}$. The internal angle bisectors of $\angle ABC$ and $\angle ADC$ meet $AC$ at $E$ and $F$ respectively, and meet each other at point $P$. Let $M$ be the midpoint of $AC$ and let $\omega$ be the circumcircle of triangle $BPD$. Segments $BM$ and $DM$ intersect $\omega$ again at $X$ and $Y$ respectively. Denote by $Q$ the intersection point of lines $XE$ and $YF$. Prove that $PQ \perp AC$.}


