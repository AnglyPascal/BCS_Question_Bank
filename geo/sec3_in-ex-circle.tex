\graphicspath{{Pics/}}


\newpage\section{Centers of inside and outside}


\impden{Incenter and Co.}{ 
    Let $\triangle ABC$ be an ordinary triangle, $I$ is
    its incenter, $D, E, F$ are the touch points of the incenter with $BC, CA,
    AB$ and $ D', E', F' $ are the reflections of $D, E, F$ wrt $ I $. 
    Let the $ I_a, I_b, I_c $ excircles touch $BC, CA, AB$ at $D_1, E_1,
    F_1$.\\

    Let $M_a, M_b, M_c$ be the midpoints of the smaller arcs $BC, CA, AB$, and
    $M_A, M_B, M_C$ be the midpoints of the major arcs $BC, CA, AB$. $ M $ are
    the midpoint of $ BC $.  Let $ A' $ be the antipode of $ A $ wrt $ \odot
    ABC $.\\

    Let $(I_a)$ touch $BC, CA, AB$ at $D_A, E_A, F_A$. So, $D_A \equiv D_1$.\\

    Call $ EF $, `$ A $-tangent line', and $ DE, DF $ similarly. And call $
    E_AF_A $ `$ A_A $-tangent line.\\

    \figdf{.8}{incenter_main}{All the primary points related to the incenter
    and the excenters}
}

\newpage

\begin{minipage}{.59\linewidth} 
    \lem{Antipode and Incenter}{
        $ A'I, \odot ABC, \odot AEIF $ are concurrent at $ Y_A $. And $ Y_A,
        D, M_a $ are collinear.
    }

    \lem{}{
        $ DD_H \perp EF $, then $ D_H, I, A' $ are collinear.
    }

    \lem{}{
        \[\frac{FD_H}{D_HE} = \frac{BD}{DC}\] 
    }

    \vspace{2em}

    \lem{Arc Midpoint as Centers}{
        \[M_AE_1 = M_AF_1\quad M_BF_1 = M_BD_1\quad M_CD_1 = M_CE_1\]
    Moreover, $I, O$ are the orthocenter and the circumcenter of $\triangle
    I_aI_bI_c$
    }

    \vspace{4em}

    \lem{Incircle Touchpoint and Cevian}{
        Let a cevian be $ AX $ and let $ I_1, I_2 $ be the incirlces of $
        \triangle ABX, \triangle ACX $. Then $ D, I_1, I_2, X $ are concyclic.
        And the other common tangent of $ \odot I_1 $ and $ \odot I_2 $ goes
        through $ D $.
    } 

\end{minipage}\hfill% 
\begin{minipage}{.39\linewidth}
    \figdf{.9}{antipode_incenter}{\autoref{lemma:Antipode and Incenter}}

    \figdf{.9}{excenter_touchpoint_bigarc-midpoints}{\autoref{lemma:Arc Midpoint
    as Centers}}
    \figdf{.9}{incircle_touchpoint_and_cevian}{\autoref{lemma:Incircle Touchpoint
    and Cevian}} 
\end{minipage}

\newpage

\begin{minipage}{.5\linewidth} 
    \lem{Apollonius Circle and Incenter, ISL 2002 G7}{
        Let $ \omega_a $ be the circle that goes through $ B, C $ and is
        tangent to $ (I) $ at $ X $. \hl{Then $ XD', EF, BC $ are concurrent and $
        X, D, I_a $ are collinear.} The same properties is held if the roles of
        incenter and excenter are swapped.  
        \begin{itemize}[left=0pt, itemsep=0pt]
            \item The circle $ BXC $ is tangent to $ (I) $ 
            \item $ X $ lies on the Apollonius Circle of $ (B, C; D, G) $.
            \item $ XD $ bisects $ \angle BXC $.  
        \end{itemize} 
    } 

    \lem{Line parallel to BC through I}{
        Let $E, F$ be the intersection of the $B, C$ angle bisectors with $AC,
        AB$. Then the tangent to $\odot ABC$ at $A$, $EF$ and the line through
        $I$ parallel to $BC$ are concurrent.
    }
    \lem{Midline Concurrency with Incircle Touchpoints}{
        $AI$, $ B, B_A $-tangent lines and $ C $-mid-line are concurrent. And,
        if the concurrency point is $ X $, then $ CS\perp AI $
    }
    \theo{}{Paul Yui Theorem}{
        $ B $-tangent line, $ C_A $-tangent line, and $ AH $ are concurrent.
    }

    \lem{Concurrent Lines in Incenter}{
        Let $ AD \cap (I) = G, AD' \cap (I) = H $. Let the line through $ D' $
        parallel to $ BC $ meet $ AB, AC $ at $ B', C' $. Then $ AM,\ EF,\
        GH,\ DD',\ BC',\ CB' $ are concurrent.
    }\label{lemma:concurrent_lines_in_incenter}
\end{minipage}\hfill% 
\begin{minipage}{.48\linewidth}
    \figdf{.9}{InExLemma3}{\autoref{lemma:Apollonius Circle and Incenter, ISL 2002 G7}}

    \figdf{.9}{BC_parallel_through_I}{\autoref{lemma:Line parallel to BC
    through I}}

    \figdf{.8}{InExLemma4}{
        \autoref{lemma:Midline Concurrency with Incircle Touchpoints} \&
        \autoref{theorem:Paul Yui Theorem}
    } 
\end{minipage}

\figdf{.5}{concurrent_lines_in_incenter}{
    \autoref{lemma:Concurrent Lines in Incenter} 
    The lines are concurrent.
} 


\begin{minipage}{.5\linewidth}
    \lem{Insimilicenter}{
        The \emph{insimilicenter} is the positive homothety center of the
        circumcircle and the incircle. It is also the \emph{isogonal conjugate of
        the Nagel Point} wrt $\triangle ABC$.
    }

    \begin{solution}
        Let $T$ be the $A$-mixtilinear touchpoint. If $AT\cap \left(I\right)
        =A'$, then if we can show that $A'B'$ arc has angle $\angle C$, where
        $A'B'\parallel AB$, we are done.
    \end{solution}
\end{minipage}\hfill%
\begin{minipage}{.49\linewidth}
    \figdf{.9}{insimilicenter}{}
\end{minipage}



\prob{}
{Application of Aollonius Circle and Incenter Lemma}{}{
    Let triangle $ABC$, incircle $(I)$, the $A$-excircle $(I_a)$ touches
    $BC$ at $M$. $IM$ intersects $(I_a)$ at the second point $X$. Similarly,
    we get $Y$, $Z$. Prove that $AX$, $BY$, $CZ$ are concurrent.\\

    \href{http://artofproblemsolving.com/community/c6h1595900p9909100}{Extension,
    by buratinogigle}: Triangle $ABC$ and $XYZ$ are homothetic with center $I$ is
    incenter of $ABC.$ Excircles touches $BC,$ $CA,$ $AB$ at $D,$ $E,$ $F.$ $XD,$
    $YE,$ $FZ$ meets excircles again at $U,$ $V,$ $W.$ Prove that $AU,$ $BV,$ $CW$
    are concurrent.
}


\newpage

\den{Isodynamic Points}{
    Let $ ABC $ be a triangle, and let the angle bisectors
    of $ \angle A $ meet $ BC $ at $ X, Y $. Call $ \omega_a $ the
    circumcircle of $ \triangle AXY $. Define $ \omega_b, \omega_c $
    similarly. The first and second isodynamic points are the points where the
    three circles $ \omega_a, \omega_b, \omega_c $ meet. I.e. these two points
    are the intersections of the three Apollonius circles. These two points
    satisfy the following relations: 
    \begin{enumerate} 
        \item $PA\sin A = PB\sin B = PC\sin C$ 
        \item They are the isogonal congugates of the
            Fermat Points, and they lie on the `Brocard Axis' 
    \end{enumerate}
    \figdf{.6}{Isodynamic_Points}{} 
}

\theo{https://en.wikipedia.org/wiki/Isodynamic_point}{Pedal Triangles of
    Isodynamic Points}{
    Prove that the pedal triangles of the isodynamic points are
    equilateral triangles. Also, Inverting around the Isodynamic Points trasnform
    $\triangle ABC $ into an equilateral triangle.
}

\prob{https://artofproblemsolving.com/community/c6h1568534p9617561}{China TST
    2018 T1P3}{EM}{
    Circle $\omega$ is tangent to sides $AB$,$AC$ of triangle $ABC$
    at $D$,$E$ respectively, such that $D\neq B$, $E\neq C$ and $BD+CE<BC$.
    $F$,$G$ lies on $BC$ such that $BF=BD$, $CG=CE$. Let $DG$ and $EF$ meet at
    $K$. $L$ lies on minor arc $DE$ of $\omega$, such that the tangent of $L$ to
    $\omega$ is parallel to $BC$. Prove that the incenter of $\triangle ABC$ lies
    on $KL$.
}

\solu{
    Using \autoref{lemma:Collinearity with antipode and center}, in the
    touch triangle of $\omega$.
}


\prob{}
{}{E}{
    Given a triangle $A B C$ with circumcircle $\Gamma$. Points $E$ and $F$
    are the foot of angle bisectors of $B$ and $C, I$ is incenter and $K$ is
    the intersection of $A I$ and $E F$. Suppose that $N$ be the midpoint of
    arc $B A C$. Circle $\Gamma$ intersects the $A$ -median and circumcircle
    of $A E F$ for the second time at $X$ and $S$. Let $S^{\prime}$ be the
    reflection of $S$ across $A I$ and $J'$ be the second intersection of
    circumcircle of $A S^{\prime} K$ and $A X$. Prove that quadrilateral $T J'
    I X$ is cyclic.
}

\begin{solution}[Reim and lemmas]
    Since $NI\cap \odot ABC=T$, the mixtilinear touchpoint, if we can show
    that $AT||IJ'$, we will be done.  Instead of working with $S'$ and $J'$,
    we reflect them back and work with $S, J$. Then we need to prove that
    $IJ||AD'$ where $D'$ is the reflection of $D$ over $I$.

    \figdf{.7}{mixti_symmedian}{}

    From \autoref{lemma:Line parallel to BC through I} we know that the $A$
    symmedian, $EF$, $IM$ are concurrent at a point $P$. We prove that $P
    \equiv J$. For that we need to show that $P$ lies on $\odot AKS$.\\

    If $\odot AKP\cap AB, AC = U, V$, it is sufficient to prove that 
    \[\frac{UF}{FB} = \frac{VE}{EC}\] 
    
    We have:
    \begin{align*}
        \frac{UF}{KU} = \frac{\sin FAP}{\sin KFA} &\quad\frac{VE}{KV} =
        \frac{\sin EAP}{\sin KEA}\\[1em]
        \therefore \frac{UF}{VE} &=\frac{\sin FAP}{\sin EAP} \frac{\sin KEA}{\sin KFA}
        =\frac{\sin CAM}{\sin BAM}\frac{AF}{AE}\\[1em]
        &=\frac{BA}{CA}\frac{AF}{AE}=\frac{BA}{AE}\frac{AF}{CA}
        =\frac{BC}{EC}\frac{CF}{BC}\\[1em]
        &=\frac{BF}{EC}
    \end{align*}
\end{solution}

\begin{minipage}{.55\linewidth}
    \prob{https://artofproblemsolving.com/community/c6t45786f6h1618685}{Vietnamese
        TST 2018 P6.a}{M}{
        Triangle $ABC$ circumscribed $(O)$ has $A$-excircle
        $(I_a)$ that touches $AB,\ BC,\ AC$ at $F,\ D,\ E$, resp. $M$ is the
        midpoint of $BC$. Circle with diameter $MI_a$ cuts $DE,\ DF$ at $K,\ H$.
        Prove that $(BDK),\ (CDH)$ have an intersecting point on $(I_a)$.
    }

    \vspace{2em}

    \prob{}
    {After Inverting Around D}{}{
        $MD$ is a line, $ I_a $ is an arbitrary point such that $ DI_a\perp
        MD$. $l$ is the perpendicular bisector of $ DI_a $. $ F, E $ are
        arbitrary points on $ l $. $ B=I_aF\cap MD, C=I_aE\cap MD\, H=FD\cap
        MI_a, K=DE\cap MI_a $. Then $ BK, CH, l $ are concurrent.
    }

    \begin{solution}
        It is straightforward using Puppus's Theorem on lines $ BDC $
        and $ HI_aK $.
    \end{solution}
\end{minipage}\hfill%
\begin{minipage}{.4\linewidth}
    \figdf{}{Vietnamese_TST_2018_P6_a_problem}{} 
    \figdf{}{Vietnamese_TST_2018_P6_a_inv}{After inverting around $ D $}
\end{minipage}



\begin{solution}[Synthetic: Length Chase] 
    \sollem{
        Let $ G, H, B', C' $ be
        defined the same way in Lemma 3.2. Prove that $ F $ lies on the radical
        axis of $ \odot D'GI, D'C'H $. By extension prove that $ B $ lies on the
        radical axis of $ \odot D'B'I, D'C'H $
    }\label{problem:vietTST2018P6.a}

    \figdf{.7}{Vietnamese_TST_2018_P6_a_modified}{\hrf{problem:vietTST2018P6.a}{Vietnamese
    TST 2018 P6.a}}

    We prove the first part, and the second part follows using spiral
    similarity.

    Suppose $ K\in FD\cap \odot KDI $. Due to spiral similarity on $ \odot
    KDI, \odot (I) $, we have $ \triangle GFK \sim \triangle GD'I $. Which
    implies: \[\frac{FK}{GF}=\frac{ID}{GD'} \implies FK = ID\frac{GF}{GD'}\]
    Now, if $ KDCE $ is to be cyclic, we need to have $ \triangle HFK \sim
    \triangle HDC $. So we need, 

    \[\frac{FK}{HF}=\frac{DC}{HD}\implies FK=DC\frac{HF}{HD}\] 
    Combining two equations: 
    \[\frac{GF}{GD'}\cdot \frac{ID}{DC}=\frac{HF}{HD}\]

    Now, using Ptolemy's theorem in $ \square FDEH $, we have, 

    \begin{align*}
        FD\cdot EH + DE\cdot FH &= DH\cdot EF\\ 
        EH \cdot \frac{FD}{FH} + DE &= EF \cdot\frac{DH}{FH}\\ 
        2\ \frac{DE}{EF} &= \frac{DH}{FH} 
    \end{align*} 

    Similarly from $ \square FGED' $ we get, \[2\ \frac{D'E}{EF} =
    \frac{GD'}{FG}\] Combining these two equations gives us the desired result.  
\end{solution}


\begin{minipage}{.45\linewidth}
    \gene{https://artofproblemsolving.com/community/c374081h1619335}{Vietnamese
        TST 2018 P6.a Generalization}{
        Let $ABC$ be a triangle. The points $D,$
        $E,$ $F$ are on the lines $BC,$ $CA,$ $AB$ respectively. The circles $(AEF),$
        $(CFD),$ $(CDE)$ have a common point $P.$ A circle $(K)$ passes through $P,$
        $D$ meet $DE,$ $DF$ again at $Q,$ $R$ respectively. Prove that the circles
        $(DBQ),$ $(DCR)$ and $(DEF)$ are coaxial.
    } 

    \begin{solution}[Inversion] 
        Invert around $ D $, and use Pappu's Theorem as in\\
        \autoref{problem:Vietnamese TST 2018 P6.a}.  
    \end{solution} 
\end{minipage}\hfill%
\begin{minipage}{.52\linewidth}
    \figdf{}{Vietnamese_TST_2018_P6_a_gene}{\hrf{VNTST2018P6a_Gene}{Vietnamese TST
    2018 P6.a Generalization}} 
\end{minipage}

\rem{
    The synthetic solution of \autoref{problem:Vietnamese TST 2018 P6.a}
    can't be reproduced here maybe because here we don't have $ A, P, D $
    collinear, and we can't have harmonic quadrilaterals either.
}




\theo{https://en.wikipedia.org/wiki/Poncelet's_closure_theorem}{Poncelet's
    Porism}{
    Poncelet's porism (sometimes referred to as Poncelet's closure
    theorem) states that whenever a polygon is inscribed in one conic section and
    circumscribes another one, the polygon must be part of an infinite family of
    polygons that are all inscribed in and circumscribe the same two conics.
}







\prob{https://artofproblemsolving.com/community/c6h1181536p5720184}{IMO 2013
    P3}{M}{
    Let the excircle of triangle $ABC$ opposite the vertex $A$ be tangent
    to the side $BC$ at the point $A_1$. Define the points $B_1$ on $CA$ and $C_1$
    on $AB$ analogously, using the excircles opposite $B$ and $C$, respectively.
    Suppose that the circumcentre of triangle $A_1B_1C_1$ lies on the circumcircle
    of triangle $ABC$. Prove that triangle $ABC$ is right-angled.
}

\solu{
    Straightforward use of \autoref{lemma:excenter_touchpoint_bigarc-midpoints}
}






\prob{https://artofproblemsolving.com/community/c74453h1225408_some_geometric_problems}{buratinogigle's
    proposed probs for Arab Saudi team 2015}{E}{
    Let $ABC$ be acute triangle with
    $AB < AC$ inscribed circle $(O)$. Bisector of $\angle BAC$ cuts $(O)$ again at
    $D$. $E$ is reflection of $B$ through $AD$. $DE$ cuts $BC$ at $F$. Let $(K)$
    be circumcircle of triangle $BEF$. $BD, EA$ cut $(K)$ again at $M, N$, reps.
    Prove that $\angle BMN = \angle KFM$.	
}

\fig{.5}{SATST2015proposed_by_bura/derakynay1134-8}{}





\prob{https://artofproblemsolving.com/community/c6h54506p340041}{USAMO 1999
    P6}{E}{
    Let $ABCD$ be an isosceles trapezoid with $AB \parallel CD$. The
    inscribed circle $\omega$ of triangle $BCD$ meets $CD$ at $E$. Let $F$ be a
    point on the (internal) angle bisector of $\angle DAC$ such that $EF \perp
    CD$. Let the circumscribed circle of triangle $ACF$ meet line $CD$ at $C$ and
    $G$. Prove that the triangle $AFG$ is isosceles.
}




\prob{https://artofproblemsolving.com/community/c6h1619730p10134424}{Serbia
    2018 P1}{E}{
    Let $\triangle ABC$ be a triangle with incenter $I$. Points $P$
    and $Q$ are chosen on segments $BI$ and $CI$ such that $2\angle  PAQ=\angle
    BAC$. If $D$ is the touch point of incircle and side $BC$ prove that $\angle
    PDQ=90$.
}

\solu{Straightforward Trig application.}




\prob{https://artofproblemsolving.com/community/c6h1628676p10217476}{Iran TST
    T2P5}{E}{
    Let $\omega$ be the circumcircle of isosceles triangle $ABC$
    ($AB=AC$). Points $P$ and $Q$ lie on $\omega$ and $BC$ respectively such that
    $AP=AQ$ .$AP$ and $BC$ intersect at $R$. Prove that the tangents from $B$ and
    $C$ to the incircle of $\triangle AQR$ (different from $BC$) are concurrent on
    $\omega$.
}




\prob{}{}{M}{
    Let a point $ P $ inside of $ \triangle ABC $ be such that the
    following condition is satisfied \[\frac{AP+BP}{AB} = \frac{BP+CP}{BC} =
    \frac{CP+AP}{CA}\]	Lines $ AP, BP, CP $ intesect the circumcirle again at $
    A', B', C' $. Prove that $ ABC $ and $ A', B', C' $ have the same incircle.
}

\solu{
    After finiding the point $ P $, we get a lot of ideas.
    \figdf{.8}{itti_same_incircle}{two lines are parallel}	
}






\prob{https://artofproblemsolving.com/community/c6h1623012p10163453}{Iran TST
    2018 P3}{EM}{
    In triangle $ABC$ let $M$ be the midpoint of $BC$. Let $\omega$
    be a circle inside of $ABC$ and is tangent to $AB,AC$ at $E,F$, respectively.
    The tangents from $M$ to $\omega$ meet $\omega$ at $P,Q$ such that $P$ and $B$
    lie on the same side of $AM$. Let $X \equiv PM \cap BF $ and $Y \equiv QM \cap
    CE $. If $2PM=BC$ prove that $XY$ is tangent to $\omega$.
}

\solu{Work backwards}


\prob{https://artofproblemsolving.com/community/c6h1623417p10167655}{Iran TST
    2018 P4}{E}{
    Let $ABC$ be a triangle ($\angle A\neq 90^\circ$). $BE,CF$ are the
    altitudes of the triangle. The bisector of $\angle A$ intersects $EF,BC$ at
    $M,N$. Let $P$ be a point such that $MP\perp EF$ and $NP\perp BC$. Prove that
    $AP$ passes through the midpoint of $BC$.
}



\prob{https://artofproblemsolving.com/community/c6h1662902p10561154}{APMO 2018
    P1}{E}{
    Let $H$ be the orthocenter of the triangle $ABC$. Let $M$ and $N$ be
    the midpoints of the sides $AB$ and $AC$, respectively. Assume that $H$ lies
    inside the quadrilateral $BMNC$ and that the circumcircles of triangles $BMH$
    and $CNH$ are tangent to each other. The line through $H$ parallel to $BC$
    intersects the circumcircles of the triangles $BMH$ and $CNH$ in the points
    $K$ and $L$, respectively. Let $F$ be the intersection point of $MK$ and $NL$
    and let $J$ be the incenter of triangle $MHN$. Prove that $F J = F A$.
}



\prob{https://artofproblemsolving.com/community/c6h155710p875026}{ISL 2006
    G6}{E}{
    Circles $ w_{1}$ and $ w_{2}$ with centres $ O_{1}$ and $ O_{2}$ are
    externally tangent at point $ D$ and internally tangent to a circle $ w$ at
    points $ E$ and $ F$ respectively. Line $ t$ is the common tangent of $ w_{1}$
    and $ w_{2}$ at $ D$. Let $ AB$ be the diameter of $ w$ perpendicular to $ t$,
    so that $ A, E, O_{1}$ are on the same side of $ t$. Prove that lines $
    AO_{1}$, $ BO_{2}$, $ EF$ and $ t$ are concurrent.
}

\solu{\hrf{lemma:concurrent_lines_in_incenter}{This}}




\lem{Tangential Quadrilateral Incenters}{
    Let $ ABCD $ be a tangential
    quatrilateral. Let $ I_1, I_2 $ be the incenters of $ \triangle ABD, \triangle
    BCD $. Then $ (I_1), (I_2) $ is tangent to $ BD $ at the same point.
    \fig{.5}{tangential_quad_incenters}{}
}



\prob{https://artofproblemsolving.com/community/c6h21758p140322}{Four
    Incenters in a Tangential Quadrilateral}{E}{
    Let $ABCD$ be a quadrilateral.
    Denote by $X$ the point of intersection of the lines $AC$ and $BD$. Let
    $I_{1}$, $I_{2}$, $I_{3}$, $I_{4}$ be the centers of the incircles of the
    triangles $XAB$, $XBC$, $XCD$, $XDA$, respectively. Prove that the
    quadrilateral $I_{1}I_{2}I_{3}I_{4}$ has a circumscribed circle if and only if
    the quadrilateral $ABCD$ has an inscribed circle.
}

\solu{
    There is a lot going on in this figure, firstly, the $ J_1, J_2 $ and $
    M $, then $ K $, then $ \angle I_4ME = \angle I_3ME $. Connecting them
    with the \hyperref[Incircle Touchpoint and Cevian]{lemma}.

    \figdf{1}{tangential_quad_four_incenters}{}
}	



\prob{}{Geodip}{E}{
    Let $ G $ be the centeroid. Dilate $ \odot I $ from $ G $
    with constant $ -2 $ to get $ I'$. Then $ I' $ is tangent to the circumcircle.
    \figdf{.5}{nice_prob_by_geodip}{} 
}



\theo{http://mathworld.wolfram.com/FuhrmannCircle.html}{Fuhrmann Circle}{
    Let $ X', Y', Z' $ be the midpoints of the arcs not containing $ A, B, C $ of $
    \odot ABC $. Let $ X, Y, Z $ be the reflections of these points on the
    sides. Then $ \odot XYZ $ is called the \textbf{Fuhrmann Circle}. The
    orthocenter $ H $ and the nagel point $N$ lies on this circle, and $ HN $
    is a diameter of this circle.

    Furthermore, $ AH, BH, CH $ cut the circle for the second time at a
    distance $ 2r $ from the vertices.

    \figdf{1}{fuhrmann_circle}{Fuhrmann Circle}
}



\prob{https://artofproblemsolving.com/community/c6h213443p1178421}{Iran TST
    2008 P12}{E}{
    In the acute-angled triangle $ ABC$, $ D$ is the intersection of
    the altitude passing through $ A$ with $ BC$ and $ I_a$ is the excenter of the
    triangle with respect to $ A$. $ K$ is a point on the extension of $ AB$ from
    $ B$, for which $ \angle AKI_a=90^\circ+\frac 34\angle C$. $ I_aK$ intersects
    the extension of $ AD$ at $ L$. Prove that $ DI_a$ bisects the angle $ \angle
    AI_aB$ iff $ AL=2R$. ($ R$ is the circumradius of $ ABC$)
}

\solu{}

\lem{Polars in Incircle}{
    In the acute angled triangle $ABC$, $I$ is the incenter and $DEF$ is the
    touch triangle. Let $EF$ meet $\odot ABC$ at $P, Q$ such that $E$ lies 
    inside $F, Q$. If $QD$ meets $\odot ABC$ for the second time at $U$, prove
    that $AU$ is the polar line of $P$ wrt $\left(I\right)$.
} 

\proof{[Projective]
    We have:
    \begin{align*}
        \left(B, C; D, EF\cap BC\right) &= Q\left(B, C; P, U\right) \\
                                        &= A(E, F; P, U)\\
                                        &= -1
    \end{align*}

    Which means $\left(P, AU\cap EF; E, F\right)$ is harmonic.
    \figdf{.5}{ISL2019G6_lem}{}
}

\prob{}{ISL 2019 G6}{HM}{
    In the acute angled triangle $ABC$, $I$ is the incenter and $DEF$ is the
    touch triangle. Let $EF$ meet $\odot ABC$ at $P, Q$ such that $E$ lies
    inside $F, Q$. Prove that \[\angle APD + \angle AQD = \angle PIQ\]
}

\begin{solution}
    Since $PI \cap AU$ at $X$ from \autoref{lemma:Polars in Incircle}, we have
    $AFPX$ is cyclic. And so
    \[\angle FAX = \angle FIX\] 
    \figdf{.5}{ISL2019G6}{}
    After some more angle chasing, we reach our goal.
\end{solution}



\newpage
\subsection{Feurbach Point}

\den{Feurbach Point}{
    The point where the nine point circle touches the incircle is called the
    \emph{Feurbach Point}.
}

\thmbox{}
{It Exists!}{
    The nine point circle touches the incircle and the excircles.
}

\begin{prooof}[Inversion]
    Let $D, D'$ be the incircle and the $A$-excircle touchpoints with $BC$.
    Let $M, N, P$ be the midpoints of $BC, CA, AB$ resp. Also let $B'C'$ be
    the reflection of $BC$ on $AI$. Now let $N', P'$ be the intersection
    points of $MN, MP$ with $B'C'$. \\

    We invert around $M$ with radius $MD = \frac{b-c}{2}$. We prove that the
    image of $\odot MNP$ after the inversion is $B'C'$. And since $(I)$ and
    $(I_a)$ are orthogonal to $(M)$, we will be done. 
    \figdf{.8}{feurbach_inversion}{}
    Wlog, assume that $b \ge c$.
    \[\begin{aligned}
        B'N &= AB' - AN = c - \frac{b}{2} &\quad NN' &= B'N \cdot\frac{AC'}{AB'}\\[.5em]
        MN' &= MN - NN' &\quad &= \frac{2c-b}{2}\cdot \frac{b}{2}\\[1em]
        MN'\cdot MN &= \frac{c}{2}\left(\frac{c}{2} -
        \frac{b}{c}\cdot\frac{2c-b}{2}\right) &= \frac{b-c}{2}^2
    \end{aligned}\]
    Which concludes the proof.
\end{prooof}

\thmbox{}
{Construction of Feurbach Point}{
    Let $D$ be the incenter touch point with $BC$. Let $M, L$ be the midpoints
    of $BC$ and $AI$. Let $D_1, D'$ be the reflections of $D$ over $I$ and
    $M$. Let $K, P$ be the refecltions of $D_1, D$ over $L$ and $AI$. Let $Q$
    be the intersection of $AD_1$ with the incircle.\\

    Then $D_1K$ and $MP$ meet at $F$ on the incircle, which is the Feurbach
    Point of $\triangle ABC$. 
}

\figdf{.7}{feurbach_construction}{}

\begin{prooof}
    It is easy to see that the tangents at $P$ and $M$ to the incircle and the
    nine point circle are parallel. So if we let $F = MP\cap \left(I\right)$,
    then we have $F$ is the Feurbach point. \\

    And since $MQ$ is tangent to $(I)$, we also have $\left(F, P; D,
    Q\right)=-1$. But notice that
    \[\begin{aligned}
        D_1(A, I; L, P) &= D_1(K, P; Q, I)\\
                        &=-1
    \end{aligned}\]
    So $D_1K$ passes through $F$.
\end{prooof}
