\newpage\section{Problems}


	
		\prob{https://artofproblemsolving.com/community/c6h1291293p6832329}{IRAN 3rd Round 2016 P2}{E}{Let $ABC$ be an arbitrary triangle. Let $E,F$ be two points on $AB,AC$ respectively such that their distance to the midpoint of $BC$ is equal. Let $P$ be the second intersection of the triangles $ABC,AEF$ circumcircles . The tangents from $E,F$ to the circumcircle of $AEF$ intersect each other at $K$. Prove that : $\angle KPA = 90$}
	
	
		


		\prob{https://artofproblemsolving.com/community/c6h1434843p8120660}{IRAN 2nd Round 2016 P6}{E}{Let $ABC$ be a triangle and $X$ be a point on its circumcircle. $Q,P$ lie on a line $BC$ such that $XQ\perp AC , XP\perp AB$. Let $Y$ be the circumcenter of $\triangle XQP$. Prove that $ABC$ is equilateral triangle if and if only $Y$ moves on a circle when $X$ varies on the circumcircle of $ABC$}
	
	
		


		\prob{https://artofproblemsolving.com/community/c6h1460735p8471885}{AoPS}{E}{Consider $ABC$ with orthic triangle $A'B'C'$, let $AA'\cap B'C' = E$ and $E'$ be reflection of $E$ wrt $BC$. Let $M$ be midpoint of $BC$ and $O$ be circumcenter of $E'B'C'$. Let $M'$ be projection of $O$ on $BC$ and $N$ be the intersection of a perpendicular to $B'C'$ through $E$ with $BC$. Prove that $MM' = 1/4MN$.}
	
	
		


		\prob{https://artofproblemsolving.com/community/c6h360730p1973873}{IRAN 3rd Round 2010 D3, P5}{M}{In a triangle $ABC$, $I$ is the incenter. $D$ is the reflection of $A$ to $I$. the incircle is tangent to $BC$ at point $E$. $DE$ cuts $IG$ at $P$ ($G$ is centroid). $M$ is the midpoint of $BC$. Prove that $AP||DM$ and $AP=2DM$.}
	
	
		


		\prob{https://artofproblemsolving.com/community/c6h429226p2428694}{IRAN 3rd Round 2011 G5}{M}{Given triangle $ABC$, $D$ is the foot of the external angle bisector of $A$, $I$ its incenter and $I_a$ its $A$-excenter. Perpendicular from $I$ to $DI_a$ intersects the circumcircle of triangle in $A'$. Define $B'$ and $C'$ similarly. Prove that $AA',BB'$ and $CC'$ are concurrent.}
	
	
		


		\prob{https://artofproblemsolving.com/community/c6t48f6h1519616_geometry}{AoPS3}{E}{$I$ is the incenter of $ABC$,  $PI,QI{\perp}BC$, $PA,QA$ intersect $BC$ at $DE$. Prove: $IADE$ is on a circle.}
	
			\fig{1}{AoPS3}{AoPS3}
	
	
		


		\prob{https://artofproblemsolving.com/community/c6t48f6h1523774_nice_property}{AoPS4}{E}{Given a triangle $ABC$, the incircle $(I)$ touch $BC,CA,AB$ at $D,E,F$ respectively. Let $AA_1,BB_1,CC_1$ be $A,B,C-altitude$ respectively. Let $N$ be the orthocenter of the triangle $AEF$. Prove that $N$ is the incenter of $AB_1C_1$}
	
			\fig{1}{AoPS4}{AoPS4}
	
	
		


		\prob{https://artofproblemsolving.com/community/c6h1087613p4817114}{IRAN TST 2015 Day 2, P3}{M}{$ABCD$ is a circumscribed and inscribed quadrilateral. $O$ is the circumcenter of the quadrilateral. $E,\ F$ and $S$ are the intersections of $AB,\ CD$; $AD,\ BC$ and $AC,\ BD$ respectively. $E'$ and $F'$ are points on $AD$ and $AB$ such that $\angle AEE'=\angle E'ED$ and $\angle AFF'=\angle F'FB$. $X$ and $Y$ are points on $OE'$ and $OF'$ such that $\frac{XA}{XD}=\frac{EA}{ED}$ and $\frac{YA}{YB}=\frac{FA}{FB}$. $M$ is the midpoint of arc $BD$ of $(O)$ which contains $A$. Prove that the circumcircles of triangles $OXY$ and $OAM$ are coaxial with the circle with diameter $OS$.}
	
			\fig{1}{ITST2015D3P3i}{Actual Prob}
			\fig{1}{ITST2015D3P3ii}{Inverted}
	
	
		


		\prob{https://artofproblemsolving.com/community/c6h1352164p7389108}{USA TST 2017 P2}{M}{Let $ABC$ be an acute scalene triangle with circumcenter $O$, and let $T$ be on line $BC$ such that $\angle TAO = 90^{\circ}$. The circle with diameter $\overline{AT}$ intersects the circumcircle of $\triangle BOC$ at two points $A_1$ and $A_2$, where $OA_1 < OA_2$. Points $B_1$, $B_2$, $C_1$, $C_2$ are defined analogously.
	
			\begin{enumerate}
				\item Prove that $\overline{AA_1}$, $\overline{BB_1}$, $\overline{CC_1}$ are concurrent.
				\item Prove that $\overline{AA_2}$, $\overline{BB_2}$, $\overline{CC_2}$ are concurrent on the Euler line of triangle $ABC$.
			\end{enumerate}}
	
	
		


		\prob{https://artofproblemsolving.com/community/c6t48f6h1522766_tangent_circles}{AoPS2}{}{Let $ABC$ be a triangle with circumcenter $O$ and altitude $AH.$ $AO$ meets $BC$ at $M$ and meets the circle $(BOC)$ again at $N.$ $P$ is the midpoint of $MN.$ $K$ is the projection of $P$ on line $AH.$ Prove that the circle $(K,KH)$ is tangent to the circle $(BOC).$}
	
			\fig{1}{AoPS2}{AoPS2}
	
			\solu{Inversion all the way...}
	
	
		


		\prob{https://artofproblemsolving.com/community/c6h46718}{AoPS5}{}{Let $ABC$ be a triangle inscribed in $(O)$ and $P$ be a point. Call $P'$ be the isogonal conjugate point of $P$. Let $A'$ be the second intersection of $AP'$ and $(O)$. Denote by $M$ the intersection of $BC$ and $A'P$. Prove that $P'M \parallel AP$.}
	
			\fig{1}{AoPS5}{AoPS5}
	
	
		


		\prob{https://artofproblemsolving.com/community/c6h283185}{AoPS}{E}{$I$ is the incenter of a non-isosceles triangle $\triangle ABC$. If the incircle touches $BC, CA, AB$ at $A_1, B_1, C_1$ respectively, prove that the circumcentres of the triangles $\triangle AIA_1$, $\triangle BIB_1$, $\triangle CIC_1$ are collinear.}
	
	
		


		\prob{https://artofproblemsolving.com/community/c6h1501095p8898504}{AoPS}{M}{Given $\triangle ABC$ and a point $P$ inside. $AP$ cuts $BC$ at $M.$ Let $M', A'$ be the reflection of $M, A$ in the perpendicular bisector of $BC.$ $A'P$ cuts the perpendicular bisector of $BC$ at $N.$ Let $Q$ be the isogonal conjugate of $P$ in triangle $ABC.$ Prove that $QM'\parallel AN$.}
	
	
		


		\prob{https://artofproblemsolving.com/community/c6h1301365p6934249}{IRAN 3rd Round 2016 G6}{E}{Given triangle $\triangle ABC$ and let $D,E,F$ be the foot of angle bisectors of $A,B,C$ ,respectively. $M,N$ lie on $EF$ such that $AM=AN$. Let $H$ be the foot of $A$-altitude on $BC$.\\
		Points $K,L$ lie on $EF$ such that triangles $\triangle AKL, \triangle HMN$ are correspondingly similar (with the given order of vertices's) such that $AK \not\parallel HM$ and $AK \not\parallel HN$. Show that: $DK=DL$.}
	
			\fig{1}{IRAN3rd2016G6}{IRAN 3rd Round 2016 G6}
	
	
		


		\prob{https://artofproblemsolving.com/community/c6h1438017p8160421}{Iran TST 2017 T3 P6}{H}{In triangle $ABC$ let $O$ and $H$ be the circumcenter and the orthocenter. The point $P$ is the reflection of $A$ with respect to $OH$. Assume that $P$ is not on the same side of $BC$ as $A$. Points $E,F$ lie on $AB,AC$ respectively such that $BE=PC \ ,  CF=PB$. Let $K$ be the intersection point of $AP,OH$. Prove that $\angle EKF = 90 ^{\circ}$.}
	
		\tr{Spiral Similarity (points on $AB, AC$ with some properties)}
	
			\figdf{1}{ITST2017T3P6}{Iran TST 2017 T3 P6}
	
	
		


		\prob{https://artofproblemsolving.com/community/c6h360732p1973876}{IRAN 3rd Round 2010 D3, P6}{M}{In a triangle $ABC$, $\angle C=45^{\circ}$. $AD$ is the altitude of the triangle. $X$ is on $AD$ such that $\angle XBC=90-\angle B$ ($X$ is inside of the triangle). $AD$ and $CX$ cut the circumcircle of $ABC$ in $M$ and $N$ respectively. Ff the tangent to $\odot ABC$ at $M$ cuts $AN$ at $P$, prove that $P, B$ and $O$ are collinear.} 
	
		\tr{Cross-Ratio}
	
	
		


		\prob{https://artofproblemsolving.com/community/c6h585433p3462669}{Iran TST 2014 T1P6}{M}{$I$ is the incenter of triangle $ABC$. perpendicular from $I$ to $AI$ meet $AB$ and $AC$ at ${B}'$ and ${C}'$ respectively. Suppose that ${B}''$ and ${C}''$ are points on half-line $BC$ and $CB$ such that $B{B}''=BA$ and $C{C}''=CA$. Suppose that the second intersection of circumcircles of $A{B}'{B}''$ and $A{C}'{C}''$ is $T$. Prove that the circumcenter of $AIT$ is on the $BC$.} \tg{projective, inversion}
	
			\solu{Too many collinearity, need to prove concurrency, what else can come into mind except projective approach.}
	
			\solu{Too many incenter related things, $\sqrt{bc}$-inversion :o}
	
	
	
		


		\prob{https://artofproblemsolving.com/community/c6h582820p3444910}{APMO 2014 P5}{M}{Circles $\omega$ and $\Omega$ meet at points $A$ and $B$. Let $M$ be the midpoint of the arc $AB$ of circle $\omega$ ($M$ lies inside $\Omega$). A chord $MP$ of circle $\omega$ intersects $\Omega$ at $Q$ ($Q$ lies inside $\omega$). Let $\ell_P$ be the tangent line to $\omega$ at $P$, and let $\ell_Q$ be the tangent line to $\Omega$ at $Q$. Prove that the circumcircle of the triangle formed by the lines $\ell_P$, $\ell_Q$ and $AB$ is tangent to $\Omega$.}
	
			\fig{1}{APMO2014P5}{APMO 2014 P5}
	
	
		



		\prob{}{}{E}{Let $ABC$ be a triangle, $D, E, F$ are the feet of the altitudes, $DF\cap BE\equiv P, DE\cap CF\equiv Q$. Prove that the perpendicular from $A$ to $PQ$ goes through the reflection of $O$ on $BC$.}  \tg{projective} 
	
			\solu{Projective approach.}
	
	
		


		\prob{https://artofproblemsolving.com/community/c6h1598147p9931685}{RMM 2018 P6}{H}{Fix a circle $\Gamma$, a line $\ell$ to tangent $\Gamma$, and another circle $\Omega$ disjoint from $\ell$ such that $\Gamma$ and $\Omega$ lie on opposite sides of $\ell$. The tangents to $\Gamma$ from a variable point $X$ on $\Omega$ meet $\ell$ at $Y$ and $Z$. Prove that, as $X$ varies over $\Omega$, the circumcircle of $XYZ$ is tangent to two fixed circles.} \tg{inversion}
	
			\solu{Too many circles, plus tangency, what else other than inversion? After the inversion the problem turns into a pretty obvious work-around problem.}
	
	
		


		\prob{https://artofproblemsolving.com/community/c6h1573301p9738624}{AoPS6}{H but Beautiful}{Let $O$ and $I$ be the circumcenter and incenter of $\Delta ABC$. Draw circle $\omega$ so that $B,C \in \omega$ and $\omega$ touches $(I)$ internally at $P$. $AI$ intersects $BC$ at $X$. Tangent at $X$ to $(I)$ which is different from $BC$, intersects tangent at $P$ to $(I)$ at $S$. $SA \cap (O)=T \neq A$. Prove that $\angle ATI=90^{\circ}$}
	
	
			\figdf{1}{AoPS6_1}{Solution 1}
			\figdf{1}{AoPS6_2}{Solution 2}
	
	
	
		


		\prob{https://artofproblemsolving.com/community/c6h1618875_cute_radical_axis}{AoPS7}{E}{Let $ABC$ be a triangle with incenter $I$ and circumcircle $\Gamma$. Let the line through $I$ perpendicular to $AI$ meet $AB$ at $E$ and $AC$ at $F$. Let the circumcircles of triangles $AIB$ and $AIC$ intersect the circumcircle of triangle $AEF$ $\omega$ again at points $M$ and $N$, and let $\omega$ intersect $\Gamma$ again at $Q$. Prove that $AQ$, $MN$, and $BC$ are concurrent.}
	
	
	
		


		\prob{https://artofproblemsolving.com/community/q2h514889p2893347}{AoPS}{E}{Given a circle $ (O) $ with center $ O $ and $ A,B $ are $ 2 $ fixed points on $ (O) $. $ E $ lies on $ AB $. $ C,D $ are on $ (O) $ and $ CD $ pass through $ E $. $ P $ lies on the ray $ DA $, $ Q $ lies on the ray $ DB $ such that $ E $ is the midpoint of $ PQ $. Prove that the circle passing through $ C $ and touch $ PQ $ at $ E $ also pass through the midpoint of $ AB $}
	
	
		


		\prob{https://artofproblemsolving.com/community/c6h514252p2889096}{WenWuGuangHua Mathematics Workshop}{E}{$ O_B, O_C $ are the $ B $ and $ C $ mixtilinear centers respectively. $ (O_B) $ touches $ BC, AB $ at $ X_B, Y_B $ respectively, and $ X_BY_B\cap O_BO_C $ at $ Z_B $. Define $ X_C, Y_C, Z_C $ similarly. Prove that if $ BZ_C\cap CZ_B = T $, then $ AT $ is the $ A $-angle bisector.}
	
	
		


		\prob{https://artofproblemsolving.com/community/c6h514371p2889823}{All Russia 1999 P9.3}{E}{A triangle $ABC$ is inscribed in a circle $S$. Let $A_0$ and $C_0$ be the midpoints of the arcs $BC$ and $AB$ on $S$, not containing the opposite vertex, respectively. The circle $S_1$ centered at $A_0$ is tangent to $BC$, and the circle $S_2$ centered at $C_0$ is tangent to $AB$. Prove that the incenter $I$ of $\triangle ABC$ lies on a common tangent to $S_1$ and $S_2$.}
	
	
		


		\prob{https://artofproblemsolving.com/community/c6h514303p2889241}{All Russia 2000 P11.7}{E}{ A quadrilateral $ABCD$ is circumscribed about a circle $\omega$. The lines $AB$ and $CD$ meet at $O$. A circle $\omega_1$ is tangent to side $BC$ at $K$ and to the extensions of sides $AB$ and $CD$, and a circle $\omega_2$ is tangent to side $AD$ at $L$ and to the extensions of sides $AB$ and $CD$. Suppose that points $O$, $K$, $L$ lie on a line. Prove that the midpoints of $BC$ and $AD$ and the center of $\omega$ also lie on a line.}
	
	
		


		\prob{https://artofproblemsolving.com/community/c6h514286p2889224}{All Russia 2000 P9.3}{E}{Let $O$ be the center of the circumcircle $\omega$ of an acute-angle triangle $ABC$. A circle $\omega_1$ with center $K$ passes through $A$, $O$, $C$ and intersects $AB$ at $M$ and $BC$ at $N$. Point $L$ is symmetric to $K$ with respect to line $NM$. Prove that $BL \perp AC$.}
	
	
		


		\prob{https://artofproblemsolving.com/community/q1h512017p2874889}{WenWuGuangHua Mathematics Workshop}{M}{
			
			\begin{enumerate}
			
				\item $ AD, BE, CF $ are concurrent cevians. Angle bisectors of $ \angle ADB $ and $ \angle AEB $ meet at $ C_0 $. Again the angle bisectors of $ \angle ADC $ and $ \angle AFC $ meet at $ B_0 $. And bisectors of $ \angle BEC $ and $ \angle BFC $ meet at $ A_0 $. Prove that $ AA_0, BB_0, CC_0 $ are concurrent.
			
				\item Angle bisectors of $ \angle AEB $ and $ \angle AFC $ meet at $ D_0 $, of $ \angle BFC $ and $ BDA $ meet at $ E_0 $, and of $ \angle CEB $ and $ \angle CDA $ meet at $ F_0 $. Prove that $ DD_0, EE_0, FF_0 $ are concurrent.
			
			\end{enumerate}}
	
			
			\solu{As this problem is purely made up with lines, we can do a projective transformation to simplify the problem. And as there are perpendicularity at $ D, E, F $, we make $ D, E, F $ the feet of the altitudes of $ \triangle ABC $. Then the angle bisector properties get replaced by simpler properties wrt $ DEF $.}
	
	
		


		\prob{https://artofproblemsolving.com/community/c6h359172}{WenWuGuangHua Mathematics Workshop}{E}{Generalization: Let $ AD, BE, CF $ be any cevians concurrent at $ T $. $ AD\cap EF=A',\ BE\cap DF=B',\ CF\cap DE=C',\ B'A'\cap AC= X,\ B'A'\cap BC= Y,\ C'X\cap EF=Z $. Prove that $ T, Y, Z $ are collinear.}
	
	
		


		\prob{https://artofproblemsolving.com/community/q1h512618p2878013}{AoPS}{E}{On circumcircle of triangle $ABC$, $T$ and $K$ are midpoints of arcs $BC$ and $BAC$ respectively . And $E$ is foot of altitude from $C$ on $AB$ . Point $P$ is on extension of $AK$ such that $PE$ is perpendicular to $ET$ . Prove that $PC=CK$.}
	
	
		


		\prob{https://artofproblemsolving.com/community/c5t256599f5h1629606_geo_3_equals_freak_out}{USJMO 2018 P3}{E}{Let $ABCD$ be a quadrilateral inscribed in circle $\omega$ with $\overline{AC} \perp \overline{BD}$. Let $E$ and $F$ be the reflections of $D$ over lines $BA$ and $BC$, respectively, and let $P$ be the intersection of lines $BD$ and $EF$. Suppose that the circumcircle of $\triangle EPD$ meets $\omega$ at $D$ and $Q$, and the circumcircle of $\triangle FPD$ meets $\omega$ at $D$ and $R$. Show that $EQ = FR$.}
	
	
		


		\prob{https://artofproblemsolving.com/community/c6h181312p996629}{All Russia 2002 P11.6}{M}{The diagonals $AC$ and $BD$ of a cyclic quadrilateral $ABCD$ meet at $O$. The circumcircles of triangles $AOB$ and $COD$ intersect again at $K$. Point $L$ is such that the triangles $BLC$ and $AKD$ are similar and equally oriented. Prove that if the quadrilateral $BLCK$ is convex, then it has an incircle.}
	
	
		


		\prob{https://artofproblemsolving.com/community/q3h511494p2873290}{WenWuGuangHua Mathematics Workshop}{M}{Let $ O_B, O_C $ be the $ B, C $ mixtilinear excircles. $ O $ meet $ CA, CB $ at $ X_C, Y_C $ and $ O_B $ meet $ BA, BC $ at $ X_B, Y_B $. Let $ I_C $ be the $ C $-excircle. $ I_CY_B $ meet $ O_BO_C $ at $ T $. Prove that $ BT\perp O_BO_C $}
	
			\solu{From what we have to prove, we find two circles, from where we get another circle. This circle suggests that we try power of point. }
	
	
		


		\prob{https://artofproblemsolving.com/community/c6h1623012p10163453}{Iran TST 2018 T1P3}{M}{In triangle $ABC$ let $M$ be the midpoint of $BC$. Let $\omega$ be a circle inside of $ABC$ and is tangent to $AB,AC$ at $E,F$, respectively. The tangents from $M$ to $\omega$ meet $\omega$ at $P,Q$ such that $P$ and $B$ lie on the same side of $AM$. Let $X \equiv PM \cap BF $ and $Y \equiv QM \cap CE $. If $2PM=BC$ prove that $XY$ is tangent to $\omega$.}
	
	
		


		\prob{https://artofproblemsolving.com/community/c6h1623417p10167655}{Iran TST 2018 T1P4}{E}{Let $ABC$ be a triangle ($\angle A\neq 90^\circ$). $BE,CF$ are the altitudes of the triangle. The bisector of $\angle A$ intersects $EF,BC$ at $M,N$. Let $P$ be a point such that $MP\perp EF$ and $NP\perp BC$. Prove that $AP$ passes through the midpoint of $BC$.}
	
			\solu{:'3 kala para na  T\_T }
	
	
		


		\prob{https://artofproblemsolving.com/community/6h1629942p10229160}{Iran TST 2018 T3P6}{H}{Consider quadrilateral $ ABCD $ inscribed in circle $ \omega $. $ AC\cap BD = P $. $ E, F $ lie on sides $ AB, CD $, respectively such that $ \angle APE=\angle DPF $. Circles $ \omega_1, \omega_2 $ are tangent to $ \omega $ at $ X, Y $ respectively and also both tangent to the circumcircle of $ PEF $ at $ P $. Prove that: \[\frac{EX}{EY}=\frac{FX}{FY}\]}
	
			\solu{fucking beautiful.}
	
	
		


		\prob{}{ISL 2006 G6}{E}{Circles $ \omega_1 $ and $ \omega_2 $ with centres $ O_1 $ and $ O_2 $ are externally tangent at point $ D $ and internally tangent to a circle $ \omega $ at points $ E $ and $ F $ respectively. Line $ t $ is the common tangent of $ \omega_1 $ and $ \omega_2 $ at $ D $. Let $ AB $ be the diameter of $ \omega $ perpendicular to $ t $, so that $ A, E, O_1 $ are on the same side of $ t $. Prove that lines $ AO_1, BO_2, EF $ and $ t $ are concurrent.}
	
	
		


		\prob{}{ISL 2006 G7}{E}{In a triangle $ ABC $, let $ M_a, M_b, M_c $ be the midpoints of the sides $ BC, CA, AB $, respectively, and $ T_a, T_b, T_c $ be the midpoints of the arcs $BC, CA, AB$ of the circumcircle of $ABC$, not containing the vertices's $A, B, C$, respectively. For $i \in {a, b, c}$, let $w_i$ be the circle with $M_iT_i$ as diameter. Let $p_i$ be the common external common tangent to the circles $w_j$ and $w_k$ (for all ${i, j, k} = {a, b, c}$) such that $w_i$ lies on the opposite side of $p_i$ than $w_j$ and $w_k$ do. 
		
		Prove that the lines $p_a, p_b, p_c$ form a triangle similar to $ABC$ and find the ratio of similitude}
	
	
		


		\prob{}{ISL 2006 G9}{H}{Points $A_1, B_1, C_1$ are chosen on the sides $BC, CA, AB$ of a triangle $ABC$, respectively. The circumcircles of triangles $AB_1C_1,\ BC_1A_1,\ CA_1B_1$ intersect the circumcircle of triangle $ABC$ again at points $A_2, B_2, C_2$, respectively ($A_2 \not= A, B_2 \not= B, C_2 \not= C$). Points $A_3, B_3, C_3$ are symmetric to $A_1, B_1, C_1$ with respect to the midpoints of the sides $BC, CA, AB$ respectively. Prove that the triangles $A_2B_2C_2$ and $A_3B_3C_3$ are similar.}
	
			\solu{In this type of ``Miquel's Point and the intersections of the circumcircles'' related problems, it is useful to think about the second intersections of the lines joining the first intersections and the Miquel's Point with the main circle.}
	
			\figdf{1}{ISL2006G9}{IMO Shortlist G9}
	
	
		


		\prob{https://artofproblemsolving.com/community/c6h1423742p8012536}{Iran TST 2017 P5}{}{In triangle $ABC$, arbitrary points $P,Q$ lie on side $BC$ such that $BP=CQ$ and $P$ lies between $B,Q$. The circumcircle of triangle $APQ$ intersects sides $AB,AC$ at $E,F$ respectively. The point $T$ is the intersection of $EP,FQ$. Two lines passing through the midpoint of $BC$ and parallel to $AB$ and $AC$, intersect $EP$ and $FQ$ at points $X,Y$ respectively. Prove that the circumcircle of triangle $TXY$ and triangle $APQ$ are tangent to each other.}





		\prob{}{}{E}{Let $ X $  be the touchpoint of the incircle with $ BC $ and let $ AX $ meet $ \cdot ABC $ at $ D $. The tangents from $ D $ to the incircle meet $ \cdot ABC $ at $ E, F $. Prove that the tangent to the circumcircle at $ A $, $ EF $ and $ BC $ are concurrent.}





		\prob{https://artofproblemsolving.com/community/c6h546185p3160596}{ISL 2012 G8}{M}{Let $ABC$ be a triangle with circumcircle $\omega$ and $\ell$ a line without common points with $\omega$. Denote by $P$ the foot of the perpendicular from the center of $\omega$ to $\ell$. The side-lines $BC,CA,AB$ intersect $\ell$ at the points $X,Y,Z$ different from $P$. Prove that the circumcircles of the triangles $AXP$, $BYP$ and $CZP$ have a common point different from $P$ or are mutually tangent at $P$.}
	
			\solu{Using Cross ratio and Desergaus's Involution Theorem.}




		\prob{}{}{E}{Suppose an involution on a line $ l $ sending $ X, Y, Z $ to $ X', Y', Z' $. Let $ l_x, l_y, l_z $ be three lines passing through $ X, Y, Z $ respectively. And let $ X_0=l_y\cap l_z,\ Y_0=l_x\cap l_z,\ Z_0=l_x\cap l_y $. Then $ X_0X', Y_0Y', Z_0Z' $ are concurrent.}





		\prob{}{USAMO 2018 P5}{E}{In convex cyclic quadrilateral $ABCD$, we know that lines $AC$ and $BD$ intersect at $E$, lines $AB$ and $CD$ intersect at $F$, and lines $BC$ and $DA$ intersect at $G$. Suppose that the circumcircle of $\triangle ABE$ intersects line $CB$ at $B$ and $P$, and the circumcircle of $\triangle ADE$ intersects line $CD$ at $D$ and $Q$, where $C,B,P,G$ and $C,Q,D,F$ are collinear in that order. Prove that if lines $FP$ and $GQ$ intersect at $M$, then $\angle MAC = 90^\circ$.}





		\prob{https://artofproblemsolving.com/community/c6h1381506p7662154}{Japan MO 2017 P3}{E}{Let $ABC$ be an acute-angled triangle with the circumcenter $O$. Let $D,E$ and $F$ be the feet of the altitudes from $A,B$ and $C$, respectively, and let $M$ be the midpoint of $BC$. $AD$ and $EF$ meet at $X$, $AO$ and $BC$ meet at $Y$, and let $Z$ be the midpoint of $XY$. Prove that $A,Z,M$ are collinear.}
	
	
		


		\prob{https://artofproblemsolving.com/community/c6h17316p118667}{ISL 2002 G1}{E}{Let $B$ be a point on a circle $S_1$, and let $A$ be a point distinct from $B$ on the tangent at $B$ to $S_1$. Let $C$ be a point not on $S_1$ such that the line segment $AC$ meets $S_1$ at two distinct points. Let $S_2$ be the circle touching $AC$ at $C$ and touching $S_1$ at a point $D$ on the opposite side of $AC$ from $B$. Prove that the circumcenter of triangle $BCD$ lies on the circumcircle of triangle $ABC$.}





		\prob{https://artofproblemsolving.com/community/c6h17317p118668}{ISL 2002 G2}{M}{Let $ABC$ be a triangle for which there exists an interior point $F$ such that $\angle AFB=\angle BFC=\angle CFA$. Let the lines $BF$ and $CF$ meet the sides $AC$ and $AB$ at $D$ and $E$ respectively. Prove that \[ AB+AC\geq4DE. \]}
	
			\solu{Pari nai.}
	
	
		


		\prob{https://artofproblemsolving.com/community/c6h17318p118672}{ISL 2002 G3}{E}{The circle $S$ has center $O$, and $BC$ is a diameter of $S$. Let $A$ be a point of $S$ such that $\angle AOB<120{{}^\circ}$. Let $D$ be the midpoint of the arc $AB$ which does not contain $C$. The line through $O$ parallel to $DA$ meets the line $AC$ at $I$. The perpendicular bisector of $OA$ meets $S$ at $E$ and at $F$. Prove that $I$ is the incenter of the triangle $CEF.$}
	
	
		


		\prob{https://artofproblemsolving.com/community/c6h17319p118673}{ISL 2002 G4}{E}{Circles $S_1$ and $S_2$ intersect at points $P$ and $Q$. Distinct points $A_1$ and $B_1$ (not at $P$ or $Q$) are selected on $S_1$. The lines $A_1P$ and $B_1P$ meet $S_2$ again at $A_2$ and $B_2$ respectively, and the lines $A_1B_1$ and $A_2B_2$ meet at $C$. Prove that, as $A_1$ and $B_1$ vary, the circumcentres of triangles $A_1A_2C$ all lie on one fixed circle.}
	
	
		


		\prob{https://artofproblemsolving.com/community/c6h17323p118682}{ISL 2002 G7}{E}{The incircle $ \Omega$ of the acute-angled triangle $ ABC$ is tangent to its side $ BC$ at a point $ K$. Let $ AD$ be an altitude of triangle $ ABC$, and let $ M$ be the midpoint of the segment $ AD$. If $ N$ is the common point of the circle $ \Omega$ and the line $ KM$ (distinct from $ K$), then prove that the incircle $ \Omega$ and the circumcircle of triangle $ BCN$ are tangent to each other at the point $ N$.}
	
	
		


		\prob{https://artofproblemsolving.com/community/c6h1381506p7662154}{Japan MO 2017 P3}{E}{Let $ABC$ be an acute-angled triangle with the circumcenter $O$. Let $D,E$ and $F$ be the feet of the altitudes from $A,B$ and $C$, respectively, and let $M$ be the midpoint of $BC$. $AD$ and $EF$ meet at $X$, $AO$ and $BC$ meet at $Y$, and let $Z$ be the midpoint of $XY$. Prove that $A,Z,M$ are collinear.}
	
	
		


		\prob{}{India TST}{E}{$ ABC $ triangle, $ D, E, F $ touchpoints, $ M $ midpoint of $ BC $, $ K $ orthocenter of $ \triangle AIC $, prove that $ MI \perp KD $}
	
	
		


		\prob{https://artofproblemsolving.com/community/c6h355790p1932935}{ISL 2009 G3}{E}{Let $ABC$ be a triangle. The incircle of $ABC$ touches the sides $AB$ and $AC$ at the points $Z$ and $Y$, respectively. Let $G$ be the point where the lines $BY$ and $CZ$ meet, and let $R$ and $S$ be points such that the two quadrilaterals $BCYR$ and $BCSZ$ are parallelogram. Prove that $GR=GS$.}
	
			\solu{Point Circle, distance same means Power same wrt point circles.}
	
	
		


		
	
	
	
	
		


		\prob{https://artofproblemsolving.com/community/c6h1634974p10278638}{ARO 2018 P11.6}{E}{Three diagonals of a regular $n$-gon prism intersect at an interior point $O$. Show that $O$ is the center of the prism.
		
		(The diagonal of the prism is a segment joining two vertices's not lying on the same face of the prism.)}
	
	
	
	
		\vspace{8mm}
	
		


	


		\prob{https://artofproblemsolving.com/community/c6h488829p2739327}{ISL 2011 G4}{EM}{Let $ABC$ be an acute triangle with circumcircle $\Omega$. Let $B_0$ be the midpoint of $AC$ and let $C_0$ be the midpoint of $AB$. Let $D$ be the foot of the altitude from $A$ and let $G$ be the centroid of the triangle $ABC$. Let $\omega$ be a circle through $B_0$ and $C_0$ that is tangent to the circle $\Omega$ at a point $X\not= A$. Prove that the points $D,G$ and $X$ are collinear.}
	
			\vspace{8mm}
	
	



		\prob{}{}{Constructing a forth circle tangent}{Given 3 circle, construct another circle that is tangent to these three circles.}
	
			\solu{A trick to remember: decreasing the radius's of some circles doesn't effect much.}
	
	
	


		\prob{}{}{H}{Let $ ABCD $ be a convex quadrilateral, let $ AD\cap BC = P $. Let $ O, O';\ H, H' $ be the circumcentres and orthocenter of $ \triangle PCD, \triangle PAB $. $ \odot DOC $ is tangent to $ \odot AD'B $, if and only if $ \odot DHC $ is tangent to $ \odot AH'B $}
		
		
		
		
		\prob{https://artofproblemsolving.com/community/c6h1493342p8776187}{Iran MO 3rd round 2017 mid-terms Geometry P3}{M}{Let $ABC$ be an acute-angle triangle. Suppose that $M$ be the midpoint of $BC$ and $H$ be the orthocenter of $ABC$. Let $F\equiv BH\cap AC$ and $E\equiv CH\cap AB$. Suppose that $X$ be a point on $EF$ such that $\angle XMH=\angle HAM$ and $A,X$ are in the distinct side of $MH$. Prove that $AH$ bisects $MX$.}
		
		
		
		\prob{https://artofproblemsolving.com/community/c6h2153166p15896528}{RMM SL 2019 G1}{M}{Let $BM$ be a median in an acute-angled triangle $ABC$. A point $K$ is chosen on the line through $C$ tangent to the circumcircle of $\vartriangle BMC$ so that $\angle KBC = 90^\circ$. The segments $AK$ and $BM$ meet at $J$. Prove that the circumcenter of $\triangle BJK$ lies on the line $AC$.}
		
		
		
		
%todo: curvilinear incircle prob