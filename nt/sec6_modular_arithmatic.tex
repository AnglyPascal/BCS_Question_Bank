\section{Modular Arithmatic}

\theo{http://services.artofproblemsolving.com/download.php?id=YXR0YWNobWVudHMvMy8yL2QzYjEzOGM0ODE3YzYwZGU4NGFmOTEwZDc0ZGNhODRjOGMyMzZlLnBkZg==&rn=dGh1ZS12NC5wZGY=}
{Thue's Lemma}{
    Let $n>1$ be an integer and a be an integer co-prime to $n$. Then there
    are integers $x, y$ with $0<|x|,|y|<\sqrt{n}$ so that
    \[ x \equiv a y \ (\mod\ n) \]
    Such a solution $(x, y)$ is called a ``small solution'' sometimes.
}

\begin{prooof}
    Let $r=\lfloor\sqrt{n}\rfloor$ i.e. $r$ is the unique integer for which
    $r^{2} \leq n<(r+1)^{2}$ The number of pairs $(x, y)$ so that $0 \leq x, y
    \leq r$ is $(r+1)^{2}$ which is greater than $n .$ Then there must be two
    different pairs $\left(x_{1}, y_{1}\right)$ and $\left(x_{2},
    y_{2}\right)$ so that
    \[\begin{aligned}
        x_{1}-a y_{1} &\equiv x_{2}-a y_{2} &(\mod\ n) \\
        x_{1}-x_{2} &\equiv a\left(y_{1}-y_{2}\right) &(\mod\ n) \\
    \end{aligned}\]

    Let $x=x_{1}-x_{2}$ and $y=y_{1}-y_{2},$ and we get $x \equiv a y\ (\mod\ n)$.
    Now, we need to show that $0<|x|,|y|<r$ and $x, y \neq 0 .$ Certainly, if
    one of $x, y$ is zero, the other is zero as well. If both $x$ and $y$ are
    zero, that would mean that two pairs $\left(x_{1}, y_{1}\right)$ and
    $\left(x_{2}, y_{2}\right)$ are actually same. That is not the case, and so
    both $x, y$ can not be $0 .$ Therefore, none of $x$ or $y$ is $0,$ and we are
    done.
\end{prooof}

\theo{}
{Generalization of Thue's Lemma}{
    Let $\alpha$ and $\beta$ are two real numbers so that $\alpha \beta \geq p$. 
    Then for an integer $x,$ there are integers $a, b$ with $0<|a|<\alpha$
    and $0<|b|<\beta$ so that
    \[a \equiv x b \quad(\mod\ p) \]
    And we can even make this lemma a two dimensional one.
}


\thmbox{}
{Fermat's 4n+1 Theorem}{
    Every prime of the form $4n+1$ can be written as the sum of squares of two
    coprime integers.
}

\begin{prooof}
    We know that there is an $x$ such that 
    \[x^2 \equiv -1 \ (\mod\ p)\] 
    And by \autoref{theorem:Thue's Lemma}, there are $a, b$ with
    $0<\left|a\right|, \left|b\right|< \sqrt{n}$ for which
    \[\begin{aligned}
        a &\equiv xb \ &(\mod\ p)\\
        a^2 &\equiv x^2b^2 \ &(\mod\ p)\\
        a^2 + b^2 &\equiv 0 \ &(\mod\ p)
    \end{aligned}\] 
    Since $a^2+ b^2 < 2p$, we are done.
\end{prooof}

\theo{}
{General Fermat's 4n+1 Theorem}{
    Let $n\in \left\{1, 2, 3\right\}$. If $-n$ is a quadratic residue modulo
    $p$, then there exists $a, b$ such that $a^2 + nb^2 = p$
}

\thmbox{}
{Factors are of the same form}{
    If $D\in \left\{1, 2, 3\right\}$ and $n = x^2 + Dy^2$ for some $x\perp y$,
    then all of the factors of $n$ are of the form $a^2 + Db^2$.
}

\begin{prooof}
    This is because the product of two numbers of such form is the same form
    as them:
    \[\begin{aligned}
        \left(a^2+Db^2\right)\left(c^2 + Dd^2\right) &=\left(ac-Dbd\right)^2 +
        D\left(ad+bc\right)^2\\
        &=\left(ac+Dbd\right)^2 + D\left(ad-bc\right)^2
    \end{aligned}\] 
    And by \autoref{theorem:General Fermat's 4n+1 Theorem} the prime factors
    of $n$ are of the same form. And so all factors of $n$ are of the same form.
\end{prooof}


\thmbox{}
{Quadratic Residue -3}{
    $-3$ is a quadratic residue of modulo $p$ iff $p$ is of the form $3k+1$.
}
\begin{prooof}
    The only if part is easy with Thue's Lemma. For the if part, we have
    \[\begin{aligned}
        \left(\frac{p}{3}\right)\left(\frac{3}{p}\right) &=
        (-1)^{\left(\frac{p-1}{2}\right) \left(\frac{3-1}{2}\right)}=
        (-1)^{\left(\frac{p-1}{2}\right) }
    \end{aligned}\] 
    Then we casework on $p\equiv 1, -1 \ (\mod\ 4)$ to show that in either
    case, \[\left(\frac{-3}{p}\right) = 1\]
\end{prooof}



\prob{}
{Thue's Lemma Note}{}{
    Let $p$ be prime number, prove that there exists $x, y$ such that
    $p=2x^2+3y^3$ iff $p\equiv 5, 11 \ (\mod\ 24)$.   
}

\begin{solution}
    We need to show that $\frac{-3}{2}$ is a quadratic residue mod $p$, and
    the rest will follow from Thue's Lemma. We do that using the quadratic
    residue rules for $2$ and $-3$.
\end{solution}
