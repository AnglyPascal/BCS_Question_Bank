\section{Tricks}
	
	
	
	``Every object in the problem's universe is important, and should be considered when approaching the problem.'' -- me
	
	
	
	
	
	\begin{itemize}
		
		
		\item permutation type problem
		\item do there exist...
		\item proving identities
		\item Dunno
		\item divisibility by primes and prime divisors stuff
		
		
	\end{itemize}
	
	
	
	\begin{enumerate}
		
		\item Add. Everything. Up.
		
		\item Infinitude of primes:
		
		\begin{enumerate}
			
			\item Eulerian infinitude trick
			\item For large enough numbers, there is a larger prime divisor
			\item Assuming contradiction, if there are any number co-prime to the product of the primes, then that must be $ 1 $. 
			
			
		\end{enumerate}
		
	\end{enumerate}
	
	
	
	
	
	
	
	\Faka\subsection{Digit Sum or Product}
	
	{When dealing with the sum of the digits or the product of them, to find the construction it is very important to consider $ 0 $ and $ 1 $'s in the number.}
	
	
	
	
	
	\Faka\subsection{Diophantine Equations}
	
	
		\begin{enumerate}
			
			\item finding some solutions
			\item trying modular cases
			\item making some variables depended on other variables
			\item putting constrains on variables which would make the problems easier
			\item if there are infinitely many solutions, can you find a construction?
			\item factorize (this is BIG)
			\item In these problems, investigation, induction, recursion, constructions etc. are essentials
			
		\end{enumerate}
	
	
	
	
	\Faka\subsection{Sequences}
	
	
	\Faka\subsection{NT Functions}
	
	
	\Faka\subsection{Construction Problems}
	
	
	\Faka\subsection{Sets satisfying certain properties}
	
	
	\Faka\subsection{Other Small Techniques to Remember}
	
	
		\begin{enumerate}
			\item $ a-b $ stays invariant upon addition, just as $ \frac{a}{b} $ stays invariant upon multiplication.
		\end{enumerate}
	
	
	
	
\newpage\section{Lemmas}
	
	
	
	\lem{Maybe not useful at all}{Let $p\geq 5$ be a prime number. Prove that if $p\mid a^2+ab+b^2$, then $$p^3\mid {a+b}^p-a^p-b^p$$} 
	
	
	\lem{Non-zero digits in base $ b $}{If $b\geq 2$ and $b^n-1\mid a$ then there exist at least $n$ non-zero digits in the representation of $a$ in base $b$}
	
	
	\theo{https://en.wikipedia.org/wiki/Coin_problem}{Frobenius Coin Problem, extended Chicken McNugget}{Given a bunch of coprime integers, one can write all integers after a certain limit as a linear combination of these integers.\\
		
	Formally, for any integers $a_1, a_2 \dots a_n$ such that $\gcd(a_1, a_2\dots a_n)=1$, there exists positive integers $m$ such that for any integer $M\geq m$, there are non-negative integers $b_1, b_2 \dots b_n$ such that \[\sum_{i=1}^{n}b_ia_i=M\]}
	
		\begin{Remark}
			If $n=2$ then $m=a_1a_2-a_1-a_2$.\\
			
			If $n\geq 2$, then there doesn't exist an explicit formula, but if $\{a_i\}$ are in arithmetic progression, ($a_i=a_1+(i-1)d$) then \[m=\floor*{\frac{a-2}{n-1}}a + (d-1)(a-1) - 1\]
		\end{Remark}
	
	
	
	
	\theobox{https://en.wikipedia.org/wiki/Beatty_sequence}{Beatty's Theorem or Rayleigh Theorem}{If $a, b$ are two irrational numbers such that $\dfrac{1}{a}+\dfrac{1}{b}=1$, then the two sets $\left\{\floor*{ia}\right\}$ and $\left\{\floor*{ib}\right\}$, where $i$ are the positive integers, form a partition of the set of natural numbers.}
	
		\rem{Express the two conditions in numeric terms assuming the contrary.}
		
		\proof{\emph{No integer belongs to both sets:} Suppose there exists some $ j, k, m $ such that $ j = \floor{ka} = \floor{mb} $. Then the inequalities will hold:
			\[j\le ka < j+1 \text{ and } j\le mb < j+1\]
		Which leads to
			\[j<ka<j+1 \text{ and } j<mb<j+1\]
		From there, it follows that
			\[\frac{j}{a}<k<\frac{j+1}{a} \text{ and } \frac{j}{b}<m<\frac{j+1}{b}\]
		Adding them gives 
			\[j<k+m<j+1\]
		Which is false.\\
		
		\emph{Every integer belongs to one set:} Suppose there is are integers $ j, k, m $ such that $ \floor{ka} <j <\floor{ka+a} $ and $ \floor{mb} <j <\floor{mb+b} $
		Which can be written as
			\[ka<j<ka+a-1 \text{ and } mb<j<mb+b-1\]
		It follows from here that,
			\[k<\frac{j}{a}<k+1-\frac{1}{a} \text{ and } m<\frac{j}{b}<m+1-\frac{1}{b}\]
		Adding them gives us
			\[k+m<j<k+m+1\]
		Which, again, is absurd. So the theorem is proved.}
	
	
	\lem{}{Let $x,y$ be co-prime. Then \[\gcd(z,xy) = \gcd(z,x)\gcd(z,y) = \gcd(z \bmod x,x)\gcd(z \bmod y,y)\] \[\implies\gcd(r(a,b),xy))=\gcd(a,x)\gcd(b,y)\] (here $r(a,b)$ denotes the smallest integer that satisfies $r(a,b) \equiv a \bmod x,\ r(a,b) \equiv b \bmod y$)}
	
	
	\lem{}{Let $0 < a_1 < a_2 < · · · < a_(mn+1)$ be $mn + 1$ integers. Prove that you can select either $m + 1$ of them no one of which divides any other, or $n + 1$ of them each dividing the following one.}
	
	
	\theo{https://math.stackexchange.com/questions/1019538/primes-dividing-a-polynomial}{Prime divisors of an integer polynomial}{If $ P(x)\in \Z[x] $, then the set of primes, $ P = \{ p: p\mid P(x) \} $ is infinite.}
	
	
	
	
	
	
	
	
	
	
	\newpage\subsection{Modular Arithmatic Theorems and Useful Results}
	
		\theo{https://en.wikipedia.org/wiki/Wolstenholme's_theorem}{Wolstenholme's Theorem}{For all prime $ p $ the following relation is true:
			\[ p^2 \ \bigg|\ 1 + \frac{1}{2} + \frac{1}{3} \dots \frac{1}{p-1} = \sum_{i=1}^{p-1} \frac{1}{i} \]}
		
			
			\coro{\[ p \ \bigg|\ 1 + \frac{1}{2^2} + \frac{1}{3^2} \dots \frac{1}{(p-1)^2} = \sum_{i=1}^{p-1} \frac{1}{i^2} \]}
			
			
			\coro{If $ p > 3 $ is a prime, then 
				\[\binom{2p}{p} \equiv 2\ (mod\ p^3)\]5
				\[\binom{2p-1}{p-1} \equiv 1\ (mod\ p^3)\]}
	
		
			\lem{}{
				\[ \frac{1}{(p-i)!} \equiv (-1)^i (i-1)!\ (mod\ p) \]
				\[ \binom{p-1}{k} \equiv (-1)^k\ (mod\ p) \]
			}
		
		
			\prob{}{}{E}{\[\binom{p^{n+1}}{p} \equiv p^n\ (mod\ p^{2n+3})\]}
			
		