\newpage\section{Divisibility}

\prob{https://artofproblemsolving.com/community/c6h17329p118695} {ISL 2002
    N6}{E}{ Find all pairs of positive integers $m,n\geq3$ for which there
    exist infinitely many positive integers $a$ such that \[
\frac{a^m+a-1}{a^n+a^2-1} \] is itself an integer.  }

\begin{solution}[Manipulation] \sollem{For $ a, m, n\in \N $, we have
        \[a^n+a^2-1\ |\ a^{m-i} + (-1)^i\left(a^{n-1} - a^{n-2} +\dots +
    (-1)^{i+1}a^{n-i}\right) + (-1)^i\left(a-1\right)\]}

    \proof{ 
        We proceed by induction on $ i $. For $ i=0 $, it is true. So
        assume for $ i<k $, it is true.

        \[\begin{aligned}
            a^n+a^2-1\ &| &&\ a^{m-i} + (-1)^i\left(a^{n-1} - a^{n-2}
            +\dots + (-1)^{i+1}a^{n-i}\right) + (-1)^i\left(a-1\right)\\[1em]
            \implies a^n+a^2-1\ &| &&\ a^{m-i} + (-1)^i\left(a^{n-1} - a^{n-2}
            +\dots + (-1)^{i+1}a^{n-i}\right)\\
            & &&+ (-1)^i\left(a-1\right) - (-1)^i(a^n+a^2-1)\\[1em] 
            & &&= a^{m-i} +  (-1)^i\left(-a^n+ a^{n-1} -
            a^{n-2}+\dots + (-1)^{i+1}a^{n-i}\right) \\
            & &&+ (-1)^{i+1}(a^2-a)\\[1em] 
            \therefore a^n+a^2-1\ &| &&\ a^{m-i-1} +
            (-1)^{i+1}\left(a^{n-1} - a^{n-2} +\dots +
            (-1)^{i+1}a^{n-i}\right) \\
            & && +(-1)^{i+1}\left(a-1\right)\\
        \end{aligned}\]
    }

    For $ i=n $, we have,

    \[a^n+a^2-1\ |\ a^{m-n}  + (-1)^n \left(\frac{a^n+1}{a+1}\right) +
    (-1)^n(a-1)\]

    Clearly, $ m\ge n $. Let $ m=nk+q $. If $ n $ is even, \begin{align*}
        a^n+a^2-1\ &|\ a^{m-n}+a-1 + \left(\frac{a^n+1}{a+1}\right)\\[.5em]
    \implies a^n+a^2+1\ &|\ a^q+a-1 + k\left(\frac{a^n+1}{a+1}\right) \end{align*}

    Which is not possible since the polynomial on the right side has degree at
    most $ n-1 $, and can't be $ 0 $ for all $ a\in \Z $.\\ 

    So, $ n $ is odd. Then we have,

    \begin{align*} a^n+a^2-1\ &|\ a^{m-n} + a-1 -
        \left(\frac{a^n+1}{a+1}\right) - 2(a-1)\\[.5em] \implies a^n+a^2-1\ &|\
    a^q + a -1 - k\left(\frac{a^n+1}{a+1}\right) -2k(a-1) = P(a) \end{align*}

    Since $ P(a) $ has degree at most $ n-1 $, $ P(a) =0 $ for all $ a\in \Z
    $.


    \begin{align*} P(a) &= a^q + a -1 - k\left(\frac{a^n+1}{a+1}\right)
        -2k(a-1)\\ \therefore a^q + a - 1 &=
    k\left(\frac{a^n+1}{a+1}\right)+2k(a-1) \end{align*}

    Comparing the coefficients and degrees on both sides, we have, $ q=n-1=2,\
    k=1 $. Which gives us the only solution $ (m, n) = (5, 3) $.  
    \end{solution}

    \prob{https://artofproblemsolving.com/community/c6h1300990_number_theory}{Iran
        3rd Round 2016 N3}{EM}{A sequence $P=\left \{ a_{n} \right \}$ is
        called a $ \text{Permutation}$ of natural numbers (positive integers)
        if for any natural number $m,$ there exists a unique natural number
        $n$ such that $a_n=m.$

        We also define $S_k(P)$ as: $S_k(P)=a_{1}+a_{2}+\cdots +a_{k}$ (the
        sum of the first $k$ elements of the sequence).

        Prove that there exists infinitely many distinct $
        \text{Permutations}$ of natural numbers like $P_1,P_2, \cdots$ such
        that$:$ $$\forall k, \forall i<j: S_k(P_i)|S_k(P_j)$$
    }

    \begin{solution} Instead of giving a construction for the sequence we
        prove that for a given permutation $P$ we can find another permutation
        $Q$ such that the partial sums of $P$ divide the corresponding partial
        sums of $Q$.\\

        As we try to build $Q$ from $P$, we have a constraint of divisibility.
        And we need to make sure every integer $i$ gets to be in $Q$. For that
        to be always possible, we need a special property of $P$ to be true.
        Finding out that property is the main task of this problem. \\

        After the property is determined, we need to add in some more details,
        that is, for the induction to work, we need to maintain that property in
        $Q$ as well.  
    \end{solution}
