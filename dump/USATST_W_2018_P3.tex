	\begin{solution}[Grid Representation]
			We put the information in a grid in the usual manner. We take the configuration for which the score (which we define to be the total amount the university has to pay) is maximum.
			
			\begin{solution_def}
				Let $ C_i $ be the price of the dish $ i $. Let $ P_j $ be the pair of dishes $ j $ ordered.
			\end{solution_def}
			
			We now ``sort'' the grid in the following way:
			\begin{itemize}
				\item Sort the columns first, from the least expensive to the most.
				\item Now we have $ 2017 $ binary strings as rows. We sort them decreasingly from top to bottom.
			\end{itemize}
		
			We end up with something like this:
			\figdf{.3}{USA_TST_2018_P3_1}{}
			
			\begin{solution_def}
				We make some sets:\[ A_i=\{x| x \text{ ordered dish } i \text{ but didn't order any of the dishes } j<i\} \]Let $ a_i = |A_i| $.
				We also call $ A_i $ block, the $ x $th rows with $ x\in A_i $. It is easy to see that the committee has to pay for dish $ i $ $ a_i $ times.
			\end{solution_def}
		
			Take a $ j $ such that $ a_j=0 $.
			 
			If $ j<k $, let \[S_j = \{x\mid \text{In the } j^{th} \text{ column, there is a } 1 \text{ in the set of the rows from } A_x \}\]
				\figdf{.3}{USA_TST_2018_P3_2}{}
			In the example of the previous diagram, $ S_4 = \{1, 2, 3\} $
			
			Take the largest element of $ S_j $, namely $ t $. We want to move the $ 1 $ in the $ A_t $ block of column $ j $ to another column to the right. Notice that this won't decrease the score, because for each of the mathematician after the block $ A_t $, the score would be non decreasing. If we could do this, we could do this inductively, moving the $ j $th column to the left each move, eventually making it disappear. 
			
			So suppose we can't do this. So there is no column on the right of $ t $, that does not have a $ 1 $ in the block $ A_t $. It is straightforward to deduce that $ N=t+a_t $. 
			
			From there, we can say:
				\[a_{t+m}\le a_t-m\text{ and }a_{t-m}\le a_t+m\]
			So,
			\begin{align*}
				2017=\sum a_i &\le (a_t+t-1) +\dots 1 \\
				\implies 2017 &\le \frac{N(N-1)}{2}\\
				\implies N&\ge 65
			\end{align*}
			
			We know that the score, $ S= \sum C_ka_k $. where $ a_i $ is strictly decreasing and $ C_i $ is non decreasing. So the maximum sum would have $ C_i=K $ for some constant (by rearrangement inequality). And to have $ K $ maximum, we need $ N $ minimum or $ N=64 $. 
			
			And if $ j>k $, the same reason holds. So $ N=65, a_1=63, a_2=62\dots a_63=1 $. But we need another $ a_i=1 $, that has to be on its own. 
		\end{solution}