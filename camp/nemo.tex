\documentclass[a4paper]{article}
\usepackage{myfile}

\title{\textbf{Non-Existential Mathematical Oxymoron}}
\author{team\_7}


\begin{document}
	
	\maketitle
	
	\begin{problem}
		$ 2N $ students take a quiz in which the possible scores are $ 0, 1\dots 10 $. It is given that each of these scores appeared at least once, and the average of their scores is $ 7.4 $. Prove that the students can be divided into two sets of $ N $ student with both sets having an average score of $ 7.4 $.
	\end{problem}

	\bigskip



	\begin{problem}
		Triangle $ABC$ circumscribed $(O)$ has $A$-excircle $(J)$ that touches $AB,\ BC,\ AC$ at $F,\ D,\ E$, resp.
		
		\begin{enumerate}
			\item $L$ is the midpoint of $BC$. Circle with diameter $LJ$ cuts $DE,\ DF$ at $K,\ H$. Prove that $(BDK),\ (CDH)$ has an intersecting point on $(J)$.
			\item Let $EF\cap BC =\{G\}$ and $GJ$ cuts $AB,\ AC$ at $M,\ N$, resp. $P\in JB$ and $Q\in JC$ such that
			$$\angle PAB=\angle QAC=90{}^\circ .$$$PM\cap QN=\{T\}$ and $S$ is the midpoint of the larger $BC$-arc of $(O)$. $(I)$ is the incircle of $ABC$. Prove that $SI\cap AT\in (O)$.
		\end{enumerate}
	\end{problem}
	
	\bigskip
	
	
	
	\begin{problem}
		Let $p_n$ be the $n^{\mbox{th}}$ prime counting from the smallest prime $2$ in increasing order. For example, $p_1=2, p_2=3, p_3 =5, \cdots$
		
		\begin{enumerate}
			\item For a given $n \ge 10$, let $r$ be the smallest integer satisfying
			\[2\le r \le n-2, \quad n-r+1 < p_r\]
			and define $N_s=(sp_1p_2\cdots p_{r-1})-1$ for $s=1,2,\ldots, p_r$. Prove that there exists $j, 1\le j \le p_r$, such that none of $p_1,p_2,\cdots, p_n$ divides $N_j$.
			
			\item Using the result of (3.1), find all positive integers $m$ for which
			\[p_{m+1}^2 < p_1p_2\cdots p_m\]
		\end{enumerate}
	\end{problem}
	
\end{document}