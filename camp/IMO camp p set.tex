\documentclass[12pt]{article}
\usepackage{amsmath,amsfonts,amssymb}
\usepackage[margin=1in]{geometry}
\title{{\Huge IMO camp 2018}}
\author{$\mathbb{Asif E Elahi}$}




\begin{document}
	\maketitle
	\begin{enumerate}
		\item $ \bigstar $ Suppose $a_1,a_2, \dots$ is an infinite strictly increasing sequence of positive integers and $p_1, p_2, \dots$ is a sequence of distinct primes such that $p_n \mid a_n$ for all $n \ge 1$. It turned out that $a_n-a_k=p_n-p_k$ for all $n,k \ge 1$. Prove that the sequence $(a_n)_n$ consists only of prime numbers.
		\begin{flushright} \textit{All Russian Olympiad 2018}\end{flushright}
		
		\item $ \bigstar $ Let $n\geq 2$ and $x_{1},x_{2},....,x_{n}$ positive real numbers.Prove that
		$\frac{1+x_{1}^2}{1+x_{1}x_{2}}+\frac{1+x_{2}^2}{1+x_{2}x_{3}}+...+\frac{1+x_{n}^2}{1+x_{n}x_{1}}\geq n$
		\begin{flushright}\textit{All Russian Olympiad 2018}\end{flushright}
		
		\item $ \bigstar $ On the sides $AB$ and $AC$ of the triangle $ABC$, the points $P$ and $Q$ are chosen, respectively, so that $PQ||BC$. Segments of $BQ$
		and $CP$ intersect at point $O$. Point $A'$ is symmetric to point $A$ relative to line$ BC$. The segment $A'O$ intersects
		circle $w$ circumcircle of the triangle $APQ$, at the point $S$.
		Prove that circumcircle of $BSC$ is tangent to the circle $w$.
		\begin{flushright}\textit{All Russian Olympiad 2018}\end{flushright}
		
		\item $ \bigstar $ Triangle $ABC$ satisfies $\angle ABC < \angle BCA < \angle CAB < 90^{\circ}$. $O$ is the circumcenter of triangle $ABC$, and $K$ is the reflection of $O$ in $BC$. $D,E$ is the foot of perpendicular line from $K$ to line $AB$, $AC$, respectively. Line $DE$ meets $BC$ at $P$, and a circle with diameter $AK$ meets the circumcircle of triangle $ABC$ at $Q(\neq A)$. If $PQ$ cuts the perpendicular bisector of $BC$ at $S$, then prove that $S$ lies on the circle with diameter $AK$.
		\begin{flushright}\textit{Korea Final Round 2018}\end{flushright}
		
		\section*{Turkey TST 2018}
		
		\item Find all $f:\mathbb{R}\to\mathbb{R}$ surjective functions such that
		$$f(xf(y)+y^2)=f((x+y)^2)-xf(x) $$for all real numbers $x,y$.
		
		\item $ \bigstar $ In a non-isosceles acute triangle $ABC$, $D$ is the midpoint of the edge $[BC]$. The points $E$ and $F$ lie on $[AC]$ and $[AB]$, respectively, and the circumcircles of $CDE$ and $AEF$ intersect in $P$ on $[AD]$. The angle bisector from $P$ in triangle $EFP$ intersects $EF$ in $Q$. Prove that the tangent line to the circumcircle of $AQP$ at $A$ is perpendicular to $BC$. 
		
		
	
		
		\section*{Seriban National Olympiad 2018}
		
		\item $ \bigstar $ Let $\triangle ABC$ be a triangle with incenter $I$. Points $P$ and $Q$ are chosen on segmets $BI$ and $CI$ such that $2\angle  PAQ=\angle BAC$. If $D$ is the touch point of incircle and side $BC$ prove that $\angle PDQ=90$.
		
		\item Let $n>1$ be an integer. Call a number beautiful if its square leaves an odd remainder upon divison by $n$. Prove that the number of consecutive beautiful numbers is less or equal to $1+\lfloor \sqrt{3n} \rfloor$.
		
		\item Let $n$ be a positive integer. There are given $n$ lines such that no two are parallel and no three meet at a single point.
		a) Prove that there exists a line such that the number of intersection points of these $n$ lines on both of its sides is at least
		$$\left \lfloor \frac{(n-1)(n-2)}{10} \right \rfloor.$$Notice that the points on the line are not counted.
		b) Find all $n$ for which there exists a configurations where the equality is achieved.
		
		\item $ \bigstar $ Prove that there exists a uniqe $P(x)$ polynomial with real coefficients such that
		$xy-x-y|(x+y)^{1000}-P(x)-P(y)$ for all real $x,y$.
		
		\item Let $a,b>1$ be odd positive integers. A board with $a$ rows and $b$ columns without fields $(2,1),(a-2,b)$ and $(a,b)$ is tiled with $2\times 2$ squares and $2\times 1$ dominoes (that can be rotated). Prove that the number of dominoes is at least $$\frac{3}{2}(a+b)-6.$$
		
		\item For each positive integer $k$, let $n_k$ be the smallest positive integer such that there exists a finite set $A$ of integers satisfy the following properties:
		
		For every $a\in A$, there exists $x,y\in A$ (not necessary distinct) that
		$$n_k\mid a-x-y$$
		There's no subset $B$ of $A$ that $|B|\leq k$ and $$n_k\mid \sum_{b\in B}{b}.$$
		
		Show that for all positive integers $k\geq 3$, we've $$n_k<\Big( \frac{13}{8}\Big)^{k+2}.$$
		
		
		
		
	
		
		\section*{China national Olympiad 2018}
		
		\item $ \bigstar $ Let $n$ and $k$ be positive integers and let
		$$T = \{ (x,y,z) \in \mathbb{N}^3 \mid 1 \leq x,y,z \leq n \}$$be the length $n$ lattice cube. Suppose that $3n^2 - 3n + 1 + k$ points of $T$ are colored red such that if $P$ and $Q$ are red points and $PQ$ is parallel to one of the coordinate axes, then the whole line segment $PQ$ consists of only red points.
		
		Prove that there exists at least $k$ unit cubes of length $1$, all of whose vertices are colored red.
		
		\item $ \bigstar $ Let $n \geq 3$ be an odd number and suppose that each square in a $n \times n$ chessboard is colored either black or white. Two squares are considered adjacent if they are of the same color and share a common vertex and two squares $a,b$ are considered connected if there exists a sequence of squares $c_1,\ldots,c_k$ with $c_1 = a, c_k = b$ such that $c_i, c_{i+1}$ are adjacent for $i=1,2,\ldots,k-1$.
		
		
		Find the maximal number $M$ such that there exists a coloring admitting $M$ pairwise disconnected squares.
		
		\section*{China TST 2018}
		
		\item Let $k, M$ be positive integers such that $k-1$ is not squarefree. Prove that there exist a positive real $\alpha$, such that $\lfloor \alpha\cdot k^n \rfloor$ and $M$ are coprime for any positive integer $n$.
		
		\item Given a positive integer $k$, call $n$ good if among $$\binom{n}{0},\binom{n}{1},\binom{n}{2},...,\binom{n}{n}$$at least $0.99n$ of them are divisible by $k$. Show that exists some positive integer $N$ such that among $1,2,...,N$, there are at least $0.99N$ good numbers.
		
		\item A number $n$ is interesting if $2018$ divides $d(n)$ (the number of positive divisors of $n$). Determine all positive integers $k$ such that there exists an infinite arithmetic progression with common difference $k$ whose terms are all interesting.
		
		\item Functions $f,g:\mathbb{Z}\to\mathbb{Z}$ satisfy $$f(g(x)+y)=g(f(y)+x)$$for any integers $x,y$. If $f$ is bounded, prove that $g$ is periodic.
		
		\item Let $p,q$ be positive reals with sum 1. Show that for any $n$-tuple of reals $(y_1,y_2,...,y_n)$, there exists an $n$-tuple of reals $(x_1,x_2,...,x_n)$ satisfying $$p\cdot \max\{x_i,x_{i+1}\} + q\cdot \min\{x_i,x_{i+1}\} = y_i$$for all $i=1,2,...,2017$, where $x_{2018}=x_1$.
		
		\item Suppose $A_1,A_2,\cdots ,A_n \subseteq \left \{ 1,2,\cdots ,2018 \right \}$ and $\left | A_i \right |=2, i=1,2,\cdots ,n$, satisfying that $$A_i + A_j, \; 1 \le i \le j \le n ,$$are distinct from each other. $A + B = \left \{ a+b|a\in A,\,b\in B \right \}$. Determine the maximal value of $n$.
		
		\item In isosceles $\triangle ABC$, $AB=AC$, points $D,E,F$ lie on segments $BC,AC,AB$ such that $DE\parallel AB$, $DF\parallel AC$. The circumcircle of $\triangle ABC$ $\omega_1$ and the circumcircle of $\triangle AEF$ $\omega_2$ intersect at $A,G$. Let $DE$ meet $\omega_2$ at $K\neq E$. Points $L,M$ lie on $\omega_1,\omega_2$ respectively such that $LG\perp KG, MG\perp CG$. Let $P,Q$ be the circumcenters of $\triangle DGL$ and $\triangle DGM$ respectively. Prove that $A,G,P,Q$ are concyclic.
		
		\item Circle $\omega$ is tangent to sides $AB$,$AC$ of triangle $ABC$ at $D$,$E$ respectively, such that $D\neq B$, $E\neq C$ and $BD+CE<BC$. $F$,$G$ lies on $BC$ such that $BF=BD$, $CG=CE$. Let $DG$ and $EF$ meet at $K$. $L$ lies on minor arc $DE$ of $\omega$, such that the tangent of $L$ to $\omega$ is parallel to $BC$. Prove that the incenter of $\triangle ABC$ lies on $KL$
		
		
		
		
		
		
		
		
		
	\end{enumerate}
\end{document}