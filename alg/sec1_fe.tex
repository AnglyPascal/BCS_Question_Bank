\newpage\section{Functional Equations}

\begin{itemize}[left=0pt, itemsep=2em]
    \item Equating terms:
        \begin{enumerate}[left=0pt,label=\textbf{\arabic*.}]
            \item \textbf{Pseudo-symmetry}: If an equation is almost but not
                completely symmetrical, what happens if you change the order of
                the variables and compare with what you started with?

            \item \textbf{Fudging}: Can you change one variable so as to alter the
                equation only slightly? If so, compare with what you started with.

            \item \textbf{Self-cancelation}: Can you make two terms in the same
                functional equation cancel each other out?

            \item These are the most mechanical ways of getting the same value to
                show up multiple times, but each problem has its own tricks. If
                you see an interesting expression pop up, always ask yourself
                whether you can get it to pop up in a slightly different way too.
        \end{enumerate}
    \item Induction, Cauchy
        \begin{enumerate}[left=0pt,label=\textbf{\arabic*.}]
            \item If you want to solve a functional equation over the integers or
                over the rationals, it often helps to inductively calculate
                something like $f(n x)$ in terms of $f(x)$

            \item A slight variation: if the domain or range of $f$ is $\mathbb{N},$
                definitely look at induction! In addition to asking what is $f(1),$ you
                can also ask when is $f(n)=1$

            \item If you want to show $f(x) \geq y,$ it suffices to show
                $f(x) \geq y-\epsilon$ for all $\epsilon>0 .$ Can you find
                progressively tighter ways of bounding $f(x)$ and then
                apply this argument? Many of the hardest functional
                equations use this kind of idea.
        \end{enumerate}

    \item Injective, Surjective, Bijective
        \begin{enumerate}[left=0pt,label=\textbf{\arabic*.}]
            \item There are many variations on injectivity and you should
                not be too fixated on the form used here. The main idea is
                this: if you can show a relationship between $f(x)$ and
                $f(y),$ what can conclude about $x$ and $y$ ?
                \begin{enumerate}
                    \item If $f$ is injective and $f(x)=f(y),$ then $x=y$
                    \item If $f$ is increasing and $f(x)>f(y),$ then $x>y$
                    \item Often it helps to start with a weaker version of
                        injectivity: if $f(x)=f(y)=0,$ then $x=y$
                    \item Even if $f$ is not injective, we can often still
                        end up with something useful. For example, if
                        $f(x)=x^{2}$ is a valid solution, we will not be
                        able to show $f(x)$ is injective, but perhaps we
                        can show that if $f(x)=f(y),$ then $x=\pm y .$
                        That is almost as good.
                    \item If you can show any kind of injectivity results,
                        it is often useful to set $x=f(z)$ for some
                        arbitrary z.
                \end{enumerate}
            \item Surjectivity is a little less common but it still comes
                in a couple flavours. The main idea is this: is there some
                nasty expression in your equation that you wish could be
                replaced by $x ?$ If so, prove that expression is
                surjective, and you are good to go.
                \begin{enumerate}
                    \item Usually the nasty expression will be $f$ itself.
                    \item It does not have to be though. For example, if
                        you could show $f$ (blah) has some nice property,
                        then a good follow-up would be to show that blah
                        is surjective.
                \end{enumerate}
        \end{enumerate}
\end{itemize}


\begin{take_note*}[title={Can't Start? Try These}]{}
    \begin{enumerate}[wide=0em, label=\arabic*, itemsep=10pt, parsep=5pt, font=\bfseries]
        \item GUESS THE POSSIBLE SOLUTIONS.

        \item SUBSTITUTION 
            \begin{enumerate}
                \item Try EVERY possible substitutions, and write them in a
                    list, dont think during this time.
                \item Now think what these results give you.
                \item Find values of $ f(0), f(1), f(2), f(-x) $ etc.
                \item Tweak the function a little bit, do substitution again.
                \item Assume some other functions according to the solutions,
                    substitute them to make the fe easier to get info out of.
            \end{enumerate}

        \item PROPERTIES OF THE FUNCTION
            \begin{enumerate}
                \item Try proving INJECTIVITY, SURJECTIVITY etc.
                \item Look for Injectivity or Surjectivity of $ f(x)-f(y) $.
            \end{enumerate}

        \item Assume for the sake of contradiction that the value of the
            function is greater or smaller than the estimated value at some
            point.

        \item Sometimes consider the difference of two values of $ f $.
    \end{enumerate}
\end{take_note*}

\begin{take_note*}{}
    \begin{enumerate}[wide=0em, label=\arabic*, itemsep=10pt, parsep=5pt, font=\bfseries]
        \item Proving that $f(x)-x$ is injective might come handy in some cases.
        \item If you're \emph{NOT} able to make one side of the equation equal
            to $0$, try to make it equal to any real or some particular real.
            (pco 169 P11)
        \item Sometimes in integer functions, divisibility of the type
            $f(1)^{k-1}\mid f(x)^k$ helps.	
        \item Durr$\dots$ I want things to cancel. 
    \end{enumerate}
\end{take_note*}


\newpage\subsection{Problems}



\prob{https://artofproblemsolving.com/community/c6h474746p2658967}{EGMO 2012 P3}{E}{Find all functions $f:\mathbb{R}\to\mathbb{R}$ such that \[f\left( {yf(x + y) + f(x)} \right) = 4x + 2yf\left(x + y\right)\] for all $x,y\in\mathbb{R}$.}





\prob{}{pco 169 P11}{M}{Find all $f:\R\rightarrow\R$ such that for all real numbers $x, y$ the following holds: \[ f(x)^2+2yf(x)+f(y)=f(y+f(x)) \]}


\prob{https://artofproblemsolving.com/community/c6h506p1611}{IMO 1994 P5}{MH}{Let $ S$ be the set of all real numbers strictly greater than $-1$. Find all functions $ f: S \to S$ satisfying the two conditions:

    \begin{enumerate}
        \item $ f(x + f(y) + xf(y)) = y + f(x) + yf(x)$ for all $ x, y$ in $ S$;
        \item $ \frac {f(x)}{x}$ is strictly increasing on each of the two intervals $ - 1 < x < 0$ and $ 0 < x$.
\end{enumerate}
}


\prob{https://artofproblemsolving.com/community/c6h219930p1219657}{ISL 1994 A4}{H}{Let $ \mathbb{R}$ denote the set of all real numbers and $ \mathbb{R}^+$ the subset of all positive ones. Let $ \alpha$ and $ \beta$ be given elements in $ \mathbb{R},$ not necessarily distinct. Find all functions $ f: \mathbb{R}^+ \mapsto \mathbb{R}$ such that: \[ f(x)f(y) = y^{\alpha} f \left( \frac{x}{2} \right) + x^{\beta} f \left( \frac{y}{2} \right) \forall x,y \in \mathbb{R}^+.\]}





\prob{https://artofproblemsolving.com/community/c6h1480146p8633190}{IMO 2017 P2}{H}{Let $\mathbb{R}$ be the set of real numbers. Determine all functions $f: \mathbb{R} \rightarrow \mathbb{R}$ such that, for any real numbers $x$ and $y$, \[ f(f(x)f(y)) + f(x+y) = f(xy). \]}





\prob{https://artofproblemsolving.com/community/c6h215430p1191683}{ISL 2008 A1}{E}{Find all functions $ f: (0, \infty) \mapsto (0, \infty)$ (so $ f$ is a function from the positive real numbers) such that
    \[ \frac {\left( f(w) \right)^2 + \left( f(x) \right)^2}{f(y^2) + f(z^2) } = \frac {w^2 + x^2}{y^2 + z^2} \]
for all positive real numbers $ w,x,y,z,$ satisfying $ wx = yz$.}





\prob{}{pco 169 P15}{EH}{Find all $a\in\R$ for which there exists a non-constant function $f:(0,1]\rightarrow\R$ such that \[a+f(x+y-xy)+f(x)f(y)\leq f(x)+f(y)\] for all $x, y \in (0,1]$} 





\prob{}{pco 168 P18}{E}{Find all functions $f:\R\rightarrow\R$ such that \[f(f(x)+y)=f(x^2-y)+4f(x)y\] for all $x, y \in\R$}





\prob{https://artofproblemsolving.com/community/c6h488535p2737643}{ISL 2011 A3}{M}{Determine all pairs $(f,g)$ of functions from the set of real numbers to itself that satisfy \[g(f(x+y)) = f(x) + (2x + y)g(y)\] for all real numbers $x$ and $y$.}






\prob{https://artofproblemsolving.com/community/c6h85071p494821}{ISL 2005 A2}{M}{We denote by $\mathbb{R}^+$ the set of all positive real numbers. Find all functions $f: \mathbb R^ + \rightarrow\mathbb R^ +$ which have the property:
    \[f(x)f(y)=2f(x+yf(x))\]
for all positive real numbers $x$ and $y$.}

\solu{Let's first substitute. If there existed some $ x $ such that $ f(x)<1 $, we could find a nice substitution. But that leads to a contradiction. So what if we could do something like this for the other cases, $ f(x)<2 $ and $ f(x)>2 $?}






\prob{https://alrtofproblemsolving.com/community/c6h78909p452035}{ISL 2005 A4}{M}{Find all functions $ f: \mathbb{R}\to\mathbb{R}$ such that $ f(x+y)+f(x)f(y)=f(xy)+2xy+1$ for all real numbers $ x$ and $ y$.}

\solu{Substitution. }





\prob{https://artofproblemsolving.com/community/c6h1627524p10206608}{Iran TST T2P1}{E}{Find all functions $f:\mathbb{R}\rightarrow \mathbb{R}$ that satisfy the following conditions:

    \begin{enumerate}
        \item  $x+f(y+f(x))=y+f(x+f(y)) \quad \forall x,y \in \mathbb{R}$

        \item  The set $I=\left\{\frac{f(x)-f(y)}{x-y}\mid x,y\in \mathbb{R},x\neq y \right\}$ is an interval.
\end{enumerate}}





\prob{}{169 P20}{E}{Let $a$ be a real number and let $f : \R \rightarrow \R$ be a function satisfying: $f(0) = \frac{1}{2}$ and $$f(x + y) = f(x)f(a - y) + f(y)f(a - x)$$ $\forall x, y \in \R$. Prove that $f$ is constant}





\prob{}{Vietnam 1991}{E}{Find all functions $ f : \R \rightarrow \R $ for which
\[\frac{1}{2}f(xy)+\frac{1}{2}f(xz)-f(x)f(yz)\geq \frac{1}{4}\]}

\solu{Just substitute.}






\prob{}{}{M}{Suppose that $f$ and $g$ are two functions defined on the set of positive integers and taking positive integer values. Suppose also that the equations $f(g(n)) = f(n) + 1$ and $g(f(n)) = g(n) + 1$ hold for all positive integer $n$. Prove that $f(n) = g(n)$ for all positive integer $n$.}

\solu{Durrr... I want things to cancel... Hint: You want to show $ f(n)-g(n) = 0 $.}



\prob{https://artofproblemsolving.com/community/c6h17330p118698}{ISL 2002 A1}{EM}{Find all functions $f$ from the reals to the reals such that\[f\left(f(x)+y\right)=2x+f\left(f(y)-x\right)\]for all real $x,y$.}

\solu{On one of our substitution, we see that there is surjectivity in the equation. So trying to show injectivity is the most intuitive move after that. Again, we have $ x $ on the outside, so we need to make $ x $, $ a $ once and $ b $ once. but we have $ f(y)-x $ which we need to eleminate, keeping $ y $ constant. We can make it either $ a $ or $ b $ since we already have $ f(a)=f(b) $. And again we can take whatever value we want for $ f(y) $.}



\prob{https://artofproblemsolving.com/community/c6h17447p119159}{ISL 2001 A1}{EM}{Let $ T$ denote the set of all ordered triples $ (p,q,r)$ of nonnegative integers. Find all functions $ f: T \rightarrow \mathbb{R}$ satisfying
    \[ f(p,q,r) = \begin{cases} 0 & \text{if} \; pqr = 0, \\ 1 + \frac{1}{6}(f(p + 1,q - 1,r) + f(p - 1,q + 1,r) & \\ + f(p - 1,q,r + 1) + f(p + 1,q,r - 1) & \\ + f(p,q + 1,r - 1) + f(p,q - 1,r + 1)) & \text{otherwise} \end{cases} \]
for all nonnegative integers $ p$, $ q$, $ r$.}

\solu{First let us guess the ans. For all points on the $ 3 $ sides, our function gives $ 0 $. We get $ f(1, 1, 1)=1 $. We get $ f(1, 1, 2) = f(1, 2, 1) = f(2, 1, 1) = \frac{3}{2} $. We get $ f(1, 1, 3)=\frac{9}{5} $. We get $ f(1, 2, 2)=\frac{12}{5} $. Now, since for $ pqr=0 $, we have $ f=0 $, we need the expression $ pqr $ on the numerator. And we kinda guess that the denominator is $ p+q+r $. From here the guess is obvious.

Now proving that this solution is the only solution. Let the solution be $ g $. Define, $ h := f-g $. Our aim is to prove that $ h=0 $ for all inputs.}




\prob{https://artofproblemsolving.com/community/c6h1790448p11841779}{RMM 2019 P5}{M}{Determine all functions $f: \mathbb{R} \to \mathbb{R}$ satisfying\[f(x + yf(x)) + f(xy) = f(x) + f(2019y),\]for all real numbers $x$ and $y$.}

\solu{After getting $f(yf(0)) = f(y2019)$, one should think of proving that either $ f $ is constant, all zero except $ 0 $, or linear. How to do this?}



\prob{https://artofproblemsolving.com/community/c6h1071764p4663900}{APMO 2015 P2}{E}{Let $S = \{2, 3, 4, \ldots\}$ denote the set of integers that are greater than or equal to $2$. Does there exist a function $f : S \to S$ such that \[f (a)f (b) = f (a^2 b^2 )\text{ for all }a, b \in S\text{ with }a \ne b?\]}

\solu{Try to break the symmetry, add another variable.}




\prob{https://artofproblemsolving.com/community/c6h1268817p6621849}{ISL 2015 A2}{E}{Determine all functions $f:\mathbb{Z}\rightarrow\mathbb{Z}$ with the property that \[f(x-f(y))=f(f(x))-f(y)-1\]holds for all $x,y\in\mathbb{Z}$.}

\solu{It just flows.}



\prob{https://artofproblemsolving.com/community/c6h1113162p5083463}{ISL 2015 A4}{M}{Let $\mathbb R$ be the set of real numbers. Determine all functions $f:\mathbb R\to\mathbb R$ that satisfy the equation\[f(x+f(x+y))+f(xy)=x+f(x+y)+yf(x)\]for all real numbers $x$ and $y$.}

\solu{When you don't know any heavy techniques, just plug in simple values into the function, and write down all of the equations in a list.}



\prob{https://artofproblemsolving.com/community/c6h546165p3160554}{ISL 2012 A5}{M}{Find all functions $f:\mathbb{R} \rightarrow \mathbb{R}$ that satisfy the conditions
    \[f(1+xy)-f(x+y)=f(x)f(y) \quad \text{for all } x,y \in \mathbb{R},\]
and $f(-1) \neq 0$.}

\solu{In FE, always look back to what you have, and what things can you make from those.}


\prob{https://artofproblemsolving.com/community/c6h488500p2737336}{ISL 2012 A1}{E}{Find all functions $f:\mathbb Z\rightarrow \mathbb Z$ such that, for all integers $a,b,c$ that satisfy $a+b+c=0$, the following equality holds:
    \[f(a)^2+f(b)^2+f(c)^2=2f(a)f(b)+2f(b)f(c)+2f(c)f(a).\]
(Here $\mathbb{Z}$ denotes the set of integers.)}

\solu{Go with the flow.}



\prob{https://artofproblemsolving.com/community/c6h356075p1935849}{ISL 2012 A1}{E}{Find all function $f:\mathbb{R}\rightarrow\mathbb{R}$ such that for all $x,y\in\mathbb{R}$ the following equality holds \[ f(\left\lfloor x\right\rfloor y)=f(x)\left\lfloor f(y)\right\rfloor \] where $\left\lfloor a\right\rfloor $ is greatest integer not greater than $a.$}

\solu{Go with the flow.}


\prob{https://artofproblemsolving.com/community/c6h287852p1555894}
{ISL 2008 A3}{E}{
    Let $ S\subseteq\mathbb{R}$ be a set of real numbers. We say that a pair $
    (f, g)$ of functions from $ S$ into $ S$ is a Spanish Couple on $ S$, if
    they satisfy the following conditions:
    \begin{enumerate}
        \item Both functions are strictly increasing, i.e. $ f(x) < f(y)$ and
            $ g(x) < g(y)$ for all $ x$, $ y\in S$ with $ x < y$;
        \item The inequality $ f\left(g\left(g\left(x\right)\right)\right) <
            g\left(f\left(x\right)\right)$ holds for all $ x\in S$.
    \end{enumerate}
    Decide whether there exists a Spanish Couple
    \begin{itemize}
        \item on the set $ S = \mathbb{N}$ of positive integers;
        \item on the set $ S = \{a - \frac {1}{b}: a, b\in\mathbb{N}\}$
    \end{itemize}
}

\begin{solution}
    Inspecting the fe immediately gives us $g(g(x))<f(x)$.
    Any attempt at constructing a pair for $\mathbb{N}$, fails because
    eventually $g(x)$ becomes larger than $f(x)$. So there might be something
    with how $g$ grows that restricts the construction.\\

    This idea motivates us to further inspect the fe. We eventually notice
    that $g^n(x) < f(x)$ for all $n\in \mathbb{N}$, and that means $S =
    \mathbb{N}$ just won't work.\\

    Now we begin to suspect that this is definitely why the second set has be
    constructed that way. Thinking about the basic construction where $f(x) =
    x+1$, we see that if we think of $g(x)$ as moving $x$ by some step to the
    right on the number line, we get a better picture of how $f(g(g(x)))$
    behaves.\\

    And that motivates our solution:
    \[\boxed{f(x) = x+1,\quad g\left(a-\frac{1}{b}\right) = a - \frac{1}{b+3^a}}\] 
\end{solution}


\prob{https://artofproblemsolving.com/community/c6h211465p1165901}
{ISL 2007 A4}{E}{
    Find all functions $ f: \mathbb{R}^{ + }\to\mathbb{R}^{ + }$ satisfying 
    \[f\left(x + f\left(y\right)\right) = f\left(x + y\right) + f\left(y\right)\] 
    for all pairs of positive reals $ x$ and $ y$. Here, $ \mathbb{R}^{ + }$
    denotes the set of all positive reals.
}

\begin{solution}[substitution]
    Set $x = f(y)$, we have
    \[\begin{aligned}
        f\left(2f(y)\right) &= f(2y) + 2f(y)\\ 
        \text{and, } f(x+kf(y)) &= f(x+ky) + kf(y)
    \end{aligned}\]
    Now setting $y = 2f(y)$, we get
    \[\begin{aligned}
        f\left(x+f(2f(y))\right) &= f\left(x+2f(y)\right) + f(2f(y))\\
                                 &=f(x+2y) + 4f(y) + f(2y)\\[1em]
        \text{Also, } f(x+f(2f(y))) &= f(x+f(2y)+2f(y))\\
        &= f(x+4y) + 2f(y) + f(2y)\\[1em]
        \implies f(x+4y) &= f(x+2y) + 2f(y) = f(x+2f(y))\\
    \end{aligned}\] 
    \begin{equation}
        \boxed{\therefore f(x+4y) = f(x+2f(y))} \ \forall y\in \mathbb{R}^+
    \end{equation}
    If $f$ is injective, then we have $\boxed{f(x) = 2x}$. If not, then
    suppose $f(a) = f(b)$. Substituting $y = a, b$ we get 
    \[\begin{aligned}
        f(x+a) = f(x+b) \ \forall x \in \mathbb{R}^+
    \end{aligned}\]
    Combining it with (1), we get that $f(x)$ is a constant function, and so
    $\boxed{f=0}$.
\end{solution}

\begin{solution}[cauchy, Raja Oktovin]
    For any positive real numbers $ z$, we have that 
    \[f(x+f(y))+z=f(x+y)+f(y)+z\] 
    \[f(f(x+f(y))+z)=f(f(x+y)+f(y)+z)\] 
    \[f(x+f(y)+z)+f(x+f(y))=f(x+y+f(y)+z)+f(x+y)\] 
    \[f(x+y+z)+f(y)+f(x+y)+f(y)=f(x+2y+z)+f(y)+f(x+y)\] 
    \[f(x+y+z)+f(y)=f(x+2y+z)\] 

    \[\boxed{f(a)+f(b)=f(a+b)}\] and by Cauchy in positive reals, then $
    f(x)=\alpha x$ for all $ x \in (0, \infty)$. Now it's easy to see that $
    \alpha=2$, then $ f(x)=2x$ for all positive real numbers $ x$.
\end{solution}


\prob{https://artofproblemsolving.com/community/c6h214702p1187155}
{ISL 2007 A2}{E}{
    Consider those functions $ f: \mathbb{N} \mapsto \mathbb{N}$ which satisfy the condition
    \[f(m + n) \geq f(m)+f(f(n))-1\]
    for all $ m,n \in \mathbb{N}.$ Find all possible values of $f(2007)$.
}

\begin{solution}
    Substituting $n=1$, we have 
    \[f(m+1) \ge f(m), \quad f(m+1) \ge f(f(m))\]
    So $f$ is non decreasing. We prove that $f(n) \le n+1$. Suppose not, so
    there exists a $n$ for which
    \[\begin{aligned}
        f(n) \ge n+2&\\
        \implies f(2^kn) \ge 2^kn + 2^k +1& \qquad \forall k
    \end{aligned}\] 
    Since $f$ grows without any bounds, there is a $t$ such that $f(2^t+1)>1$.
    Then we have,
    \[\begin{aligned}
        f(2^tn+2^t+1) &\ge f(2^t+1) + f(f(2^tn)) -1\\
                      &\ge f(2^t+1)-1 +f(2^tn+2^t+1)\\
                      &> f(2^tn+2^t+1)
    \end{aligned}\]
    A contradiction.\\

    So we get a nice bounding for $f(2007)$, that is $\left\{1, 2, \dots
    2008\right\}$. It's time to construct functions for that. For $1\le k \le
    2007$, consider the functions
    \[f(n) = \begin{cases}
        1 & \text{if } n < k\\
        n-k+1 & \text{if } n\ge k
    \end{cases}\]
    And for $2008$, consider the function
    \[f(n) = \begin{cases}
        n+1 & \text{if } 2007|n\\
        n & \text{otherwise}
    \end{cases}\]
    The constructions comes from the idea that we want $f(f(n)) =f(n)$, and we
    can use $1$'s to push our values to the right.
\end{solution}


\newpage\subsection{Weird Ones}


\prob{https://artofproblemsolving.com/community/c6h289055p1562848}{ISL 2009 A3}{E}{Determine all functions $ f$ from the set of positive integers to the set of positive integers such that, for all positive integers $ a$ and $ b$, there exists a non-degenerate triangle with sides of lengths
    \[ a, f(b) \text{ and } f(b + f(a) - 1).\]
(A triangle is non-degenerate if its vertices are not collinear.)}

\solu{$ f(1)>1 \implies\ f $ is periodic $ \implies $ repeatation $ \implies $ contradiction.\\
$ f(2)>2 \implies $ strictly increasing $ \implies $ repeatation.}



\prob{https://artofproblemsolving.com/community/c6h1558131p9513099}{USA TST 2018 P2}{H}{Find all functions $f: \mathbb{Z}^2 \to [0, 1]$ such that for any integers $x$ and $y$, \[f(x, y) = \frac{f(x - 1, y) + f(x, y - 1)}{2}.\]}

\solu{We know that the function has to be a constant function. So it is a intuitive idea considering the difference of two values of the function. Again as we wish to show that this difference is $ 0 $, we have to use either equality of limit. As equality is quite ambiguous in this problem, we approach with limits. We see that $ f(x,y) $ can be written as a term depending on the values of the $ 3 $rd quarter of the plane with $ (x, y) $ as its origin. With infinite values in our hand, we try bounding.}


\prob{https://artofproblemsolving.com/community/c6h2192982p16443789}
{IMEO 2020 P3}{M}{
    Find all functions $f:\mathbb{R^+} \to \mathbb{R^+}$ such that for all
    positive real $x, y$ holds
    \[xf(x)+yf(y)=(x+y)f\left(\frac{x^2+y^2}{x+y}\right)\]
}

\begin{solution}
    Note that \[\frac{xf(x)+yf(y)}{x+y} = f\left(\frac{x^2+y^2}{x+y}\right)\] 
    Which looks just like a linear form. Also since we know that the solution
    is probably only $f(x) = ax+b$, we pursue this idea. \\

    It can be proved that all the rational points lie on a line. Now what if a
    irrational lies outside of the line?
\end{solution}

\begin{solution}[MarkBcc168, geometric]
    Our first step is to do Vieta Jumping on the functional equation. The main
    result from this is the following. 

    \begin{claim}
        $af(a) - bf(b) = (a-b)f(a+b)$ for any $a,b\in\mathbb{R}^+$.
    \end{claim}

    \begin{prooof}
        Fix $t,k\in\mathbb{R}^+$ where $t<k$. Let $g(x)=\tfrac{x^2+t^2}{x+t}$.
        Since $\lim_{x\to\infty}g(x)=\infty$ and $g(t)=t$, there exists $a$
        such that $g(a) = \tfrac{t^2+a^2}{t+a}=k$. Moreover, by Vieta's
        Jumping, $g(k-a)=k$. Thus, by considering

        \[\begin{array}{rcrcl} 
            P(t,a) & \implies & tf(t) + af(a)&=&(a+t)f(k)\qquad\text{ and}
            \\[4pt] P(t,k-a) & \implies & tf(t) + (k-a)f(k-a)&=&(k-a+t)f(k), 
        \end{array}\]

        we get that
        \[af(a) - (k-a)f(k-a) = (2a-k)f(k)\]
        for any $a,k$ such that $k>a$. Thus, by setting $b=k-a$, we get the
        desired claim. 
    \end{prooof}

    Now let's do some geometry. Let $P_x = (x,f(x))$. Then the claim above
    implies that $P_a, P_b, P_{a+b}$ are colinear whenever
    $a,b\in\mathbb{R}^+$.\\

    Fix any pairwise distinct $a,b,c\in\mathbb{R}^+$. Set $A=P_a$, $B=P_b$,
    $C=P_c$, $D=P_{b+c}$, $E=P_{c+a}$, and $F=P_{a+b}$. By the claim, $D\in
    BC$, $E\in CA$, $F\in AB$, and $AD,BE,CF$ are concurrent at $P_{a+b+c}$.
    However, we claim that 

    \begin{claim}
        $D,E,F$ are colinear.
    \end{claim}

    \begin{prooof}
        If $A,B,C$ are colinear, then the result is trivial. Otherwise, we use
        Menelaus theorem. Observe that
        $$\frac{\overline{BD}}{\overline{DC}} = \frac{b - (b+c)}{(b+c)-c} =
        -\frac bc$$hence multiplying the result cyclically gives the result. 
    \end{prooof}

    By Ceva's theorem, both $D,E,F$ being colinear and $AD,BE,CF$ being
    concurrent can only happen when $A,B,C$ are colinear. This means that
    $(a,f(a))$, $(b,f(b))$, $(c,f(c))$ are colinear for any
    $a,b,c\in\mathbb{R}^+$. This concludes, say by varying $c$, that $f$ must
    be linear.
\end{solution}
