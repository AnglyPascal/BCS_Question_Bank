\newpage
\section{Polynomials}
	
	
	\subsection{Techniques to remember}
	
		\begin{take_note*}{}
			\begin{enumerate}[wide=0em, label=\arabic*, itemsep=10pt, parsep=5pt, font=\bfseries]
				\item A polynomial with odd degree always has at least one real root.
				\item If a polynomial with even degree has a negative value on its graph, then it has at least one real root.
				\item Roots of unity divide a polynomial in parts like congruence classes.
				\item MODULUS SIGN: use Triangle Inequality.
				\item They say, In Poly, chase ROOTS.
				\item $ x^nf(\dfrac{1}{x}) $ has the same coefficients as $ f(x) $, but in opposite order.
			\end{enumerate}
		\end{take_note*}
	
	
	
		\theo{}{Lagrange Interpolation Theorem}{Given $n$ real numbers, there exist a polynomial with at most $n-1$ degree such that the graph of the polynomial goes through all of the points.}
			
		\theo{}{Finite Differences}{This is the descrete form of derivatives. The first finite difference of a function $ f $ is defined as $ g(x) := f(x+1)-f(x) $.
		
		$ n+1 $th finite difference of a $ n $ degree polynomial: For any polynomial $P(x)$ of degree at most $n$ the following equation holds: \[\sum_{i=0}^{n+1}(-1)^i \binom{n+1}{i}P(i)=0\]}\label{finite_difference}
		
		\begin{Remark}
			This can be used 
			\begin{enumerate}
				\item to reduce the degree of a polynomial, manipulate the coefficients etc.
				\item to solve recurrences, where the recurrence equation is a bit messy and contains a lot of previous values. Like this recurrence is quite messy to solve as it is, but if we take the first finite difference here, it becomes easy:\[f(x) = \frac{1}{3} f(x+1) + \frac{2}{3} f(x-1) + 1\]
				
				It's like solving for the first derivative and then finding the original function.
			\end{enumerate}
		\end{Remark}
		
		
	
	
	\newpage
	\subsection{General Problems}
	
	
	\prob{https://artofproblemsolving.com/community/c6h587303p3476290}{USA TST 2014 P4}{E}{Let $n$ be a positive even integer, and let $c_1, c_2, \dots, c_{n-1}$ be real numbers satisfying \[ \sum_{i=1}^{n-1} \left\lvert c_i-1 \right\rvert < 1. \] Prove that \[2x^n - c_{n-1}x^{n-1} + c_{n-2}x^{n-2} - \dots - c_1x^1 + 2\] has no real roots.}
	
		\solu{A polynomial has no real root means that the polynomial completely lies in either of the two sides of the $x$-axis. So in this case, we have to prove that $P(x)>0$. So we gotta try to bound. Again, to make $\vert c_i-1 \vert$ a little bit more approachable, we assign $b_i=c_i-1$ and write $P(x)$ in terms of $b_i$. Now, how to bring the modulus sign in our polynomial? Oh, we have triangle ineq for those kinda work :0}
	
	
	
	
	
	\prob{https://artofproblemsolving.com/community/c6h54043p337849}{USAMO 2002 P3}{H}{Prove that any monic polynomial (a polynomial with leading coefficient $1$) of degree $n$ with real coefficients is the average of two monic polynomials of degree $n$ with $n$ real roots.}
	
		\solu{(i) If we have $ n+1 $ points, we have an unique polynomial through them.\\
		(ii) If we have one positive value of a polynomial and one negative value, then there exists a real root between that two values.}
	 
	
	
	
	
	\prob{https://artofproblemsolving.com/community/c6h37599p234868}{China TST 1995 P5}{M}{$ A$ and $ B$ play the following game with a polynomial of degree at least 4: \[ x^{2n} + \_\ x^{2n - 1} + \_\ x^{2n - 2} + \ldots + \_\ x + 1 = 0 \] $A$ and $ B$ take turns to fill in one of the blanks with a real number until all the blanks are filled up. If the resulting polynomial has no real roots, $ A$ wins. Otherwise, $ B$ wins. If $ A$ begins, which player has a winning strategy?}
	
		\solu{Not always (actually in very few cases) the first move decides the winning strategy. In this case, if $B$ could make the last move, he would definitely win. But as he can't, consider the final two moves. Again ``Waves''.}
	
	
	
	
	
	\prob{http://yufeizhao.com/olympiad/imo2008/zhao-polynomials.pdf}{Zhao Polynomials}{E}{A set of $n$ numbers are considered to be $k$-cool if $a_1+a_{k+1}\dots =a_2+a_{k+2}\dots =\dots =a_k+a_{2k}\dots$. Suppose a set of $50$ numbers are $3,5,7,11,13,17$-cool. Prove that every element of that set is $0$.} 
	
		\solu{Equivalence class :0 roots of unity :0 :0 :0}
	
	
	
	
	
	\prob{https://artofproblemsolving.com/community/c6h1238101p6307084}{All Russian Olympiad 2016, Day2, Grade 11, P5}{E}{Let $n$ be a positive integer and let $k_0,k_1, \dots,k_{2n}$ be nonzero integers such that $k_0+k_1 +\dots+k_{2n}\neq 0$. Is it always possible to find a permutation $(a_0,a_1,\dots,a_{2n})$ of $(k_0,k_1,\dots,k_{2n})$ so that the equation \[a_{2n}x^{2n}+a_{2n-1}x^{2n-1}+\dots+a_0=0 \] has no integer roots?}
	
	\solu{The degree is $2n$, and we have to find a zero, so proving/disproving the existence of negative value of $P(x)$ is enough. If all the values of $P(x)$ are to be positive, the leading coefficient must be very big...}
	
	
	
	
	
	\prob{https://artofproblemsolving.com/community/c6h1296837p6886800}{Zhao Poly}{E}{Let $f(x)$ be a monic polynomial with degree $n$ with distinct zeroes $x_1,x_2,...,x_n$. Let $g(x)$ be any monic polynomial of degree $n-1$. Show that $$\sum_{j=1}^{n}\dfrac{g(x_j)}{f'(x_j)} =1$$ where $f'(x_i) = \displaystyle\prod_{j\neq i} (x_i-x_j)$}
	
	\solu{Lagrange's Interpolation}
	
	
	
	
	
	
	\prob{https://artofproblemsolving.com/community/c6h1634950p10278510}{ARO 2018 P11.1}{E}{The polynomial $P (x)$ is such that the polynomials $P(P(x))$ and $P(P(P(x)))$ are strictly monotone on the whole real axis. Prove that $P (x)$ is also strictly monotone on the whole real axis.}
	
	
	
	\prob{https://artofproblemsolving.com/community/c6h1619736p10134447}{Serbia 2018 P4}{E}{Prove that there exists a uniqe $P(x)$ polynomial with real coefficients such that
		\[xy-x-y\mid (x+y)^{1000}-P(x)-P(y)\] for all real $x,y$.}
	
	
		\solu{Substitution.}
	
	
	
	\subsection{Root Hunting}
		
		\prob{https://artofproblemsolving.com/community/c7h1554572p9472758}{Putnam 2017 A2}{E}{Let $Q_0(x)=1$, $Q_1(x)=x,$ and	\[Q_n(x)=\frac{(Q_{n-1}(x))^2-1}{Q_{n-2}(x)}\]for all $n\ge 2.$ Show that, whenever $n$ is a positive integer, $Q_n(x)$ is equal to a polynomial with integer coefficients.}
		
	
	
	
	\subsection{NT Polynomials}
	
		\prob{https://artofproblemsolving.com/community/c6h276183p1494541}{Iran TST 2009 P4}{E}{Find all polynomials $f$ with integer coefficient such that, for every prime $p$ and natural numbers $u$ and $v$ with the condition:
			\[ p \mid uv - 1 \]
		we always have 
			\[p \mid f(u)f(v) - 1\]}
		
			\solu{Notice that we can disregard $ v $ by considering it $ \dfrac{1}{u} $, and the condition won't be affected, because primes allow multiplicative inverses. After this observation the problem is almost solved.}
		
		
		\prob{https://artofproblemsolving.com/community/c6h6448p22328}{Iran TST 2004 P6}{E}{$p$ is a polynomial with integer coefficients and for every natural $n$ we have $p(n)>n$. $x_k $ is a sequence that: $x_1=1, x_{i+1}=p(x_i)$ for every $N$ one of $x_i$ is divisible by $N.$ Prove that $p(x)=x+1$}
		
			\solu{Notice that $ \{x_i\} $ becomes periodic mod any prime. Now, we start by showing that $ P(1)=2 $. We have, $ P(x) $ has to be even. If it is $ >2 $ then what happens? what if we take $ N=P(1)-1 $?}
		
		
		\prob{https://artofproblemsolving.com/community/c6h101487p572821}{ISL 2006 N4}{E}{Let $P(x)$ be a polynomial of degree $n > 1$ with integer coefficients and let $k$ be a positive integer. Consider the polynomial $Q(x) = P(P(\dots P(P(x))\dots ))$, where $P$ occurs $k$ times. Prove that there are at most $n$ integers $t$ such that $Q(t)=t$.}
		
			\solu{Suppose that there are more than $n$ fixed points. So at least one of them can’t be a fixed point of $P$. Use that. Follow.}
		
		
		\prob{https://artofproblemsolving.com/community/c6h546164p3160553}{ISL 2012 A4}{H}{Let $f$ and $g$ be two nonzero polynomials with integer coefficients and $\deg f>\deg g$. Suppose that for infinitely many primes $p$ the polynomial $pf+g$ has a rational root. Prove that $f$ has a rational root.}
		
			\solu{dunno}
		
	
	
	\subsection{Fourier Transformation}
		
		\lem{Dealing with binomial terms with a common factor}{Let $ n, k $ be two integers, and let $ z $ be a $ k $th root of unity other than $ 1 $. Then, \[\binom{n}{0} + \binom{n}{k} + \binom{n}{2k} + \dots = \frac{(1+z^1)^n + (1+z^2)^n + \dots+ (1+z^k)^n }{k} \]}
			
			\proof{For $ j $ not divisible by $ k $, \[z^j\left(\sum_{i=1}^{k} z^i\right) =0\]}
			