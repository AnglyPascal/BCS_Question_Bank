\graphicspath{{Pics/}}

\newpage\section{Inequalities}


	\tcbset{
		colback=black!2!white,
		colframe=black!70!white,
		boxrule=0.3mm,
	}
	
	
	\faka\begin{myitemize}
		\item \href{https://artofproblemsolving.com/articles/files/MildorfInequalities.pdf}{Olympiad Inequalities - Thomas J. Mildorf}
		\item \href{https://artofproblemsolving.com/articles/files/KedlayaInequalities.pdf}{$ A<B $ - Keran Kedlaya}
		\item \href{http://www.math.cmu.edu/~lohp/docs/math/mop2008/convexity-soln.pdf}{Convexity - Po Shen Loh}
		\item \href{https://web.evanchen.cc/handouts/Ineq/en.pdf}{Brief Intro to Ineqs - Evan Chen}
	\end{myitemize}

	\vspace{2em}

	\den{Majorizes}{Given two sequences of real numbers $ x_1\ge x_2\ge\cdots \ge x_n $ and $ y_1\ge y_2\ge \cdots\ge y_n  $, we say $ (x_n) $ \emph{majorizes} $ (y_n) $, written $ (x_n)\succ(y_n) $ if
		\begin{align*}
		x_1+x_2\cdots +x_n &=y_1+y_2\cdots +y_n\ \text{ and,}\\
		x_1+x_2\cdots +x_k &\ge y_1+y_2\cdots +y_k \ \ \forall 1\le k \le n-1
		\end{align*}
	}

	\den{Mean Values}{Given $ n $ \emph{positive reals} $ a_2 \dots a_n $, we have 
		\begin{align*}
			\text{Arithmetic Mean} &:\quad \dfrac{\sum a_i}{n} = \dfrac{a_1 + a_2 + \dots + a_n}{n}\\[1em]
			\text{Geometric Mean} &:\quad \sqrt[\leftroot{1}\uproot{4}n]{\prod a_i} = \sqrt[\leftroot{1}\uproot{4}n]{a_1a_2\dots a_n}\\[1em]
			\text{Quadratic Mean} &:\quad \sqrt{\dfrac{\sum a_i^2}{n}} = \sqrt{\dfrac{a_1^2+a_2^2+\dots +a_n^2}{n}}\\[1em]
			\text{Harmonic Mean} &:\quad \dfrac{n}{\sum \tfrac{1}{a_i}} = \dfrac{n}{\tfrac{1}{a_1}+\tfrac{1}{a_2} + \dots + \tfrac{1}{a_n}}
		\end{align*}
	}

\subsection{Basic Inequalities}

	


	\theo{}{Muirhead's Inequality}{If $ a_1, a_2,\dots a_n $ are real positive reals, and $ (x_n)\succ(y_n) $, then we have, 
		
		\[\impeq{\sum_{\text{sym}}a_1^{x_1}a_2^{x_2}\cdots a_n^{x_n} \ge \sum_{\text{sym}}a_1^{y_1}a_2^{y_2}\cdots a_n^{y_n}}\]}

	
	\theo{}{Jensen's Inequality}{If $ f $ is convex, then 
		\[\impeq{\frac{f(a_1)+ f(a_2)\dots +f(a_n)}{n} \ge f\left(\frac{a_1+a_2\dots +a_n}{n}\right)}\]
	The reverse inequality holds if $ f $ is concave.}
	

	
	\theo{}{Karamata's Inequality}{If $ f $ is convex and $(x_n)\succ(y_n)$, then
		\[\impeq{f(x_1) + f(x_2) \dots +f(x_n) \ge f(y_1)+f(y_2)\dots + f(x_n)}\]
		The reverse inequality holds if $ f $ is concave.}
	
	\begin{BoxedTheorem}[title=Tangent Line Trick: Best-Nice bound for function]{}{}
	Even if a fucntion is not convex or concave for us to use Jensen's Inequality, we can still find a number $ a $ such that for our required interval $ I $, $ f $ stays above (or below) the tangent line to $ f $ at $ (a,f(a)) $, that is 
			\[f(x) \ge f(a) + f'(a)(x-a) \]
	\figdf{.7}{tangent_line_tricc}{Here the tangent line at $ (1, f(1)) $ is the best to bound with in the interval $ [0,2] $}
	\end{BoxedTheorem}
	
	
	\theo{}{$ n-1 $ Equal Values}{Let $ a_1, a_2, \dots a_n $ be real numbers with $ a_1+a_2\dots a_n $ fixed. Let $ f:\R\to\R $ be a function with exactly one \emph{inflection point}. If
		\[f(a_1)+f(a_2)+\dots + f(a_n)\]
	achieves a maximal or minimal value, then $ n-1 $ of the $ a_i $ are equal to each other.	
	}

	
	\lem{Power function convexity}{If $ (x_i), (y_i), (m_i) $ be three sequences of real numbers, $ x, y\in \R $ and $ p>1 $. Then for $ \a\in(0, 1) $,
		\[\impeq{
			(x+y)^p\ \le\ \a^{1-p} x^p \ + \ (1-\a)^{1-p} y^p}
		\]
		\[\impeq{
			\sum (x_i+y_i)^pm_i \ \le\ \a^{1-p}\sum x_i^pm_i \ + \ (1-\a)^{1-p}\sum y_i^pm_i}
		\]
	Equality holds iff 
		\[\frac{x}{y} = \frac{x_i}{y_i} = \frac{\a}{1-\a}\]
	}
	
	\proof{Because $ f(x) = x^p $ is a convex function,
		\[(\a a + (1-\a)b)^p < \a a^p + (1-\a)b^p\]
	So setting $ \a a = x $ and $ (1-\a) b = y $, we get the first inequality. The second one is just an extention of the first one.}
	
	
	\theo{}{Minkowski}{If $ x_1, x_2\dots x_n $, $ y_1, y_2,\dots y_n $ and $ m_1, m_2, \dots m_n $ be three sequence of real numbers and $ p>1 $, then
		\[\impeq{\left(\sum \left(x_i+y_i\right)^p m_i\right)^{1/p} \le \left(\sum x_i^pm_i\right)^{1/p} + \left(\sum y_i^pm_i\right)^{1/p}}\]
	Equality holds iff $ (x_i) $ and $ (y_i) $ are proportional.
	}

	
	
	\proof{
		Let us write,
		\begin{align*}
			A = \left(\sum x_i^pm_i\right)^{1/p},\ B = \left(\sum y_i^pm_i\right)^{1/p}
		\end{align*}	
		
		From the second equation of \autoref{lemma:Power function convexity}, for $ \a\in(0, 1) $ we have
		\begin{align*}
			\sum(x_i+y_i)^pm_i \le \a^{1-p}A^p + (1-\a)^{1-p}B^p
		\end{align*}
		Setting $ \a $ be such that $ A/B = \a/(1-\a) $, by the equality case of the first equation of the lemma we get,
		\[\sum(x_i+y_i)^pm_i \ \le\ \a^{1-p}A^p + (1-\a)^{1-p}B^p= (A+B)^p\]
	}


	\theo{}{Young's Inequality}{If $a, b> 0$ and $p, q> 0$ and $\dfrac{1}{p}+\dfrac{1}{q} = 1$, then we have \[ab \le \frac{a^p}{p} + \frac{b^q}{q} \]
	Equality occurs when $a^p = b^q $. }

		\proof{Consider the function $f(x) = e^x$, it is convex. So we have 
			\[e^{\tfrac{1}{p}x + \tfrac{1}{q}y} \le e^{\tfrac{1}{p}x} + e^{\tfrac{1}{q}y}\]
		If we let $a=e^{\tfrac{x}{p}}, b= e^{\tfrac{y}{q}} $, we are done. Equality occurs when $ x = y $.}

	
	\lem{}{}
	
	

	\theo{https://en.wikipedia.org/wiki/Generalized_mean}{Weighted Power Mean}{Let $ a_1, a_2, \dots a_n $ and $ w_1, w_2, \dots + w_n $ be positive real numbers with $w_1 + w_2\dots w_n=1$. For real number $ r $, we define,
		\[P(r) = \left\{
			\begin{array}{ll}
				\left(w_1a_1^r + w_2a_2^r \dots +w_na_n^r\right)^{1/r} & r\ne 0\\[1em]
				a_1^{w_1}a_2^{w_2}\dots a_n^{w_n} & r=0
			\end{array}
		\right.\]
	If $ r> s $, then $ P(r)\ge P(s) $. Equality occurs iff $ a_1 = a_2 =\dots = a_n $
	}

	\proof{First we show that, $ \displaystyle \lim_{r\to 0} P(r) = P(0) $. Using L'Hopital's law,
		\begin{align*}
			\lim_{r\to 0} \ln P(r) &= \lim_{r\to 0}\frac{\ln \sum w_ia_i^r}{r} = \lim_{r\to 0} \frac{\dfrac{\sum w_ia_i^r\ln a_i}{\sum w_ia_i^r}}{1}\\[.5em]
			&= \lim_{r\to 0} \frac{\sum w_ia_i^r\ln a_i}{\sum w_ia_i^r}\\[.5em]
			&= \sum\ln a_i^{w_i} \\[.2em]
			&= \ln P(0)
		\end{align*}
		
	Now we have, 
		 
	
	
	}


\subsection{Theorems}

	
	\faka\subsection*{\centering{\large{`Mean' Inequalities}}}
	
		
	
		
\faka
		
		
	\theo{http://mathworld.wolfram.com/TriangleInequality.html}{Triangle Inequality}{ For any complex numbers $ a_1, a_2 \ldots a_n $ the following holds:
		\[\left | a_1 + a_2 \dots + a_k \right | \leq |a_1| + |a_2| + \dots + |a_k|\]}
	
		
	\begin{enumerate}[label={\bfseries\textsection\arabic* }]
		\item \hrf{https://artofproblemsolving.com/wiki/index.php?title=Root-Mean_Square-Arithmetic_Mean-Geometric_Mean-Harmonic_mean_Inequality}{\bfseries\large{One Mean Ineq -- QM-AM-GM-HM}}
		
		Given $ n $ positive real numbers $ x_1, x_2, \dots x_n $, the following relation holds:
		
		\begin{align*}
		\tcboxmath{\sqrt{\dfrac{x_1^2+\cdots+x_n^2}{n}} \ge\dfrac{x_1+\cdots+x_n}{n}\ge\sqrt[n]{x_1\cdots x_n}\ge\dfrac{n}{\dfrac{1}{x_1}+\cdots+\dfrac{1}{x_n}}}
		\end{align*}
		
		
		
		with equality if and only if $x_1=x_2=\cdots=x_n$.
		
		
		\theo{https://artofproblemsolving.com/wiki/index.php?title=Arithmetic_Mean-Geometric_Mean_Inequality}{Weighted AM-GM}{If $a_1, a_2, \dotsc, a_n$ are nonnegative real numbers, and $\lambda_1, \lambda_2, \dotsc, \lambda_n$ are nonnegative real numbers (the "weights") which sum to $ 1 $, then 
			
			\begin{align*}
				\tcboxmath{\lambda_1a_1 + \lambda_2a_2 +\dots +\lambda_na_n \ge a_1^{\lambda_1}a_2^{\lambda_2}\dots a_n^{\lambda_n}}
			\end{align*}
			
		Equality holds if and only if $a_i = a_j$ for all integers $i, j$ such that $\lambda_i \neq 0$ and $\lambda_j \neq 0$. We obtain the unweighted form of AM-GM by setting \[\lambda_1 = \lambda_2 = \dotsb = \lambda_n = \dfrac1n\] }
		
	\end{enumerate}
	
	
	\theo{https://artofproblemsolving.com/wiki/index.php?title=Cauchy-Schwarz_Inequality}{Cauchy-Schward Inequality}{For any real numbers $a_1, \ldots, a_n$ and $b_1, \ldots, b_n$, 
		\begin{align*}
			\tcboxmath{\left(a_1^2 + a_2^2 \dots a_n^2 \right) \left(b_1^2 + b_2^2\dots b_n^2 \right) \ge \left( a_1b_1 + a_2b_2 \dots a_nb_n \right)^2 }
		\end{align*}	
	with equality when there exist constants $\mu, \lambda$ not both zero such that for all $1 \le i \le n$, $\mu a_i = \lambda b_i$.}
	
	The inequality sometimes appears in the following form.
	
	\theo{https://artofproblemsolving.com/wiki/index.php?title=Cauchy-Schwarz_Inequality}{Cauchy-Schwarz Inequality Complex form}{Let $a_1, \ldots, a_n$ and $b_1, \ldots, b_n$ be complex numbers. Then 
		\begin{align*}
			\tcboxmath{\left( |a_1^2| + |a_2^2| \dots |a_n^2| \right) \left( |b_1^2| + |b_2^2| \dots |b_n^2| \right) \ge \left| a_1b_1 + a_2b_2 \dots a_nb_n \right|^2 }
	\end{align*}}
	
	
	
	\theo{https://brilliant.org/wiki/titus-lemma/}{Titu's Lemma}{For positive reals $ a_1, a_2 \ldots a_n $ and $ b_1, b_2 \ldots b_n $ the following holds:
		\begin{align*}
			\tcboxmath{\dfrac{a_1^2}{b_1} + \dfrac{a_2^2}{b_2} +\dots+ \dfrac{a_n^2}{b_n} \ge \dfrac{\left(a_1 + a_2+\dots+ a_n\right)^2}{\left(b_1+b_2+\dots+b_n\right)}} 
		\end{align*}}
	
	
	
	\theo{https://artofproblemsolving.com/wiki/index.php?title=Hölder's_Inequality}{Holder's Inequality}{If $a_1, a_2, \dotsc, a_n, b_1, b_2, \dotsc, b_n, \dotsc, z_1, z_2, \dotsc, z_n$ are nonnegative real numbers and $\lambda_a, \lambda_b, \dotsc, \lambda_z$ are nonnegative reals with sum of 1, then \[ a_1^{\lambda_a}b_1^{\lambda_b} \dots z_1^{\lambda_z} + \dots + a_n^{\lambda_a} b_n^{\lambda_b} \dots z_n^{\lambda_z}  \leq (a_1 + \dots + a_n)^{\lambda_a} (b_1 + \dots + b_n)^{\lambda_b} \dots (z_1 + \dots + z_n)^{\lambda_z} \]. }
	
	
	
	
	

\newpage\subsection{Tricks}
	
	
	\begin{take_note*}[title={Some tricks to try}]{}
		\begin{enumerate}[wide=3pt, label=\arabic*, itemsep=10pt, parsep=12pt, font=\bfseries]
			\item Replace trigonometric functions by reals, and translate the problem
			\item \textbf{Smoothing}, replace two variable while keeping something invariant, to make the inequality sharper.
			\item \textbf{Convexity}, differentiate to check convexity, if the second derivative is positive on some interval, then the function is convex on that interval except probably at the endpoints, and concave otherwise. 
			
			An example is $ \ln \dfrac{1-x}{x} $. It is convex in $ (0, \dfrac12] $ and concave in $ [\dfrac12, 1) $
			\item If there is product, and if the problem is `ad-hoc'y, then apply $ AM-GM $ and $ \ln $ to see if there is something to play with.
		\end{enumerate}
	\end{take_note*}


	
	\den{Homogeneous Expression}{Expression $ F(a_1, a_2 \dots a_n) $ is said to be homogeneous of degree $ k $ if and only if there exists real $ k $ such that for every $ t>0 $ we have 
		\[t^kF(a_1, a_2\dots a_n)=F(ta_1, ta_2\dots ta_n)\]\vspace{1mm}
		
		If an expression is homogeneous, then the following can be assumed (one at a time):
		
	\begin{align}
		\sum^n_{i=1} a_i = 1\\
		\prod^n_{i=1} a_i = 1\\
		a_1 = 1 \text{ or for some } i,\  a_i=1\\
		\sum^n_{i=1} a_i^2 = 1 \\
		\sum_{Cyc}\ a_ia_{i+1} = 1 
	\end{align}}		
	
		
	\den{Substitutions}{
	
	\begin{itemize}
		\item For the condition $ abc=1 $, set  \[a=\dfrac{x}{y},\ b=\dfrac{y}{z},\ c=\dfrac{z}{x}\]\label{susbstitution:case_abc}
		\item \[ xyx=x+y+z+2 \implies \dfrac{1}{x+1} + \dfrac{1}{y+1} + \dfrac{1}{z+1} = 1 \]implies the existence of $ a, b, c $ such that \[ x=\dfrac{b+c}{a},\ y=\dfrac{a+c}{b},\ z=\dfrac{b+a}{c} \].
		\item $ 2xyz+xy+yz+zx=1 $ is just the inverse of the previouss condition.
		\item \[x^2+y^2+z^2=xyz+4  \text{ and }  |x|,\ |y|,\ |z| \geq 2\] implies the existence of $ a, b, c $ such that 
			\[abc=1 \text{ and }  x=a+\dfrac{1}{a},\ y=b+\dfrac{1}{b},\ z=c+\dfrac{1}{c}\]
		In fact even if only $ \max(|x|, |y|, |z|) > 2 $ is given, the result still holds.
	\end{itemize}}
	
	
	\lem{}{The following inequality holds for every positive integer $ n $ 
		\[2\sqrt{n+1} - 2\sqrt{n} < \sqrt{\dfrac{1}{n}} < 2\sqrt{n} - 2\sqrt{n-1} \]}
	
	
	\lem{}{Given $ 4 $ posititve real numbers $ a<b<c<d $. Call the score of a permutation $ a_1, a_2, a_3, a_4 $ of the four given reals be equal to the real 
		\[|\dfrac{a_1}{a_2} - \dfrac{a_3}{a_4}|\]
	Prove that the minimum the score can get is equal to 
		\[|\dfrac{a}{c} - \dfrac{b}{d}|\]}
	
	
	
\newpage\subsection{Problems}
	
	\prob{https://artofproblemsolving.com/community/c6h78687}{APMO 1991 P3}{E}{Let $a_1$, $a_2$, $\cdots$, $a_n$, $b_1$, $b_2$, $\cdots$, $b_n$ be positive real numbers such that $a_1 + a_2 + \cdots + a_n = b_1 + b_2 + \cdots + b_n$. Show that
		\[ \dfrac{a_1^2}{a_1 + b_1} + \dfrac{a_2^2}{a_2 + b_2} + \cdots + \dfrac{a_n^2}{a_n + b_n} \geq \dfrac{a_1 + a_2 + \cdots + a_n}{2} \]}
	
	
	\prob{https://artofproblemsolving.com/community/c6h355781p1932917}{ISL 2009 A2}{\hl{(Kori nai)}}{Let $a$, $b$, $c$ be positive real numbers such that $\dfrac{1}{a} + \dfrac{1}{b} + \dfrac{1}{c} = a+b+c$. Prove that:
		\[\dfrac{1}{(2a+b+c)^2}+\dfrac{1}{(a+2b+c)^2}+\dfrac{1}{(a+b+2c)^2}\leq \dfrac{3}{16}.\]}
	
	
	
	\prob{https://artofproblemsolving.com/community/c6h1634929p10278399}{ARO 2018 P11.2}{M}{Let $n\geq 2$ and $x_{1}, x_{2},\dots, x_{n}$ positive real numbers. Prove that
		\[\dfrac{1+x_{1}^2}{1+x_{1}x_{2}}+\dfrac{1+x_{2}^2}{1+x_{2}x_{3}}+...+\dfrac{1+x_{n}^2}{1+x_{n}x_{1}}\geq n\] }
	
	\solu{The inequality says sum is greater, so if the product is greater, then we are done by AM-GM (\ref{sharpening}). }
	
	
	
	\prob{https://artofproblemsolving.com/community/c6h1455842p8375615}{Turkey TST 2017 P5}{M}{For all positive real numbers $a,b,c$ with $a+b+c=3$, show that $$a^3b+b^3c+c^3a+9\geq 4(ab+bc+ca)$$}
	
	\solu{Always try the most simple ineq possible, AM-GM}
	
	
	
	\prob{https://artofproblemsolving.com/community/c6h488342p2736375}{IMO 2012 P2}{Varies}{Let $n\ge 3$ be an integer, and let $a_2,a_3,\ldots ,a_n$ be positive real numbers such that $a_{2}a_{3}\cdots a_{n}=1$. Prove that
		\[(1 + a_2)^2 (1 + a_3)^3 \dotsm (1 + a_n)^n > n^n.\]}\label{problem:imo_2012_p2}
	
		\solu{The main idea is to look for the ans of the ques, $ (1+a_k)^k >= ? $. We have $ k^{th} $ power. So if we can get a $ k $ term sum inside of the brackets, we can get a clean term for $ ? $ from AM-GM. And $ 1 $ seems like it's crying to be partitioned. So we write the term as $ \left(a_k + \dfrac{1}{k-1}+\dots+\dfrac{1}{k-1}\right) $}
	
		\solu{*Looks at the $ a_2a_3\dots a_n=1 $ condition*\\Hey, we have a \hrf{susbstitution:case_abc}{substitution} for this one, why not try it out...\\darn it, i still have to do the partition thing to cancel out the powers $ >( $}
	
	
	\prob{https://artofproblemsolving.com/community/c6h18486p124417}{ISL 1998 A1}{can't judge now}{Let $a_{1},a_{2},\ldots ,a_{n}$ be positive real numbers such that $a_{1}+a_{2}+\cdots +a_{n}<1$. Prove that
		
		\[ \dfrac{a_{1} a_{2} \cdots a_{n} \left[ 1 - (a_{1} + a_{2} + \cdots + a_{n}) \right] }{(a_{1} + a_{2} + \cdots + a_{n})( 1 - a_{1})(1 - a_{2}) \cdots (1 - a_{n})} \leq \dfrac{1}{ n^{n+1}}. \]}\label{problem:1998A1}
	
		\solu{Simplifying and making it symmetric, we get to the inequality \[\prod_{i=1}{n} \dfrac{1-a_i}{a_i} \ge n^{n+1}\]Now approaching similarly as \hrf{problem:imo_2012_p2}{this} problem, we get to the solution.}
	
	
	
	\prob{https://artofproblemsolving.com/community/c6h57603p354109}{ISL 2001 A1}{E}{Let $ a, b, c$ be positive real numbers so that $ abc = 1$. Prove that
		\[ \left( a - 1 + \dfrac 1b \right) \left( b - 1 + \dfrac 1c \right) \left( c - 1 + \dfrac 1a \right) \leq 1. \]}
	
		\solu{Substitute.}
		
	\prob{https://artofproblemsolving.com/community/c6h19778p131846}{ISL 1999 A1}{EM}{Let $n \geq 2$ be a fixed integer. Find the least constant $C$ such the inequality
		
	\[\sum_{i<j} x_{i}x_{j} \left(x^{2}_{i}+x^{2}_{j} \right) \leq C \left(\sum_{i}x_{i} \right)^4\]
		
	holds for any $x_{1}, \ldots ,x_{n} \geq 0$ (the sum on the left consists of $\binom{n}{2}$ summands). For this constant $C$, characterize the instances of equality.}
	
		\solu{Follow the ineq sign and remember AM-GM.}
		
	
	\prob{https://artofproblemsolving.com/community/c6h1671272p10632289}{ISL 2017 A1}{E}{Let $a_1,a_2,\ldots a_n,k$, and $M$ be positive integers such that
		\[\dfrac{1}{a_1}+\dfrac{1}{a_2}+\cdots+\dfrac{1}{a_n}=k\quad\text{and}\quad a_1a_2\cdots a_n=M\]
	
	If $M>1$, prove that the polynomial
		\[P(x)=M(x+1)^k-(x+a_1)(x+a_2)\cdots (x+a_n)\]
	has no positive roots.}
	
		\solu{The same idea used in \hrf{problem:1998A1}{this} and \hrf{problem:imo_2012_p2}{this}, spreading an expression to perform AM-GM on it.}
	
	
	\prob{https://artofproblemsolving.com/community/c6h1480690p8639254}{ISL 2016 A1}{M}{Let $a$, $b$, $c$ be positive real numbers such that $\min(ab,bc,ca) \ge 1$. Prove that \[\sqrt[3]{(a^2+1)(b^2+1)(c^2+1)} \le \left(\dfrac{a+b+c}{3}\right)^2 + 1.\]}
	
		\solu{Try the simpler version with two variables first. Now you can use this discovery with a little bit of cleverness to solve the problem. The clever part is to notice that 4 variable ineq is more solvable than a 3 variable one.}
	
	
	
	\prob{https://artofproblemsolving.com/community/c6h1480705p8639281}{ISL 2016 A2}{M}{Find the smallest constant $C > 0$ for which the following statement holds: among any five positive real numbers $a_1,a_2,a_3,a_4,a_5$ (not necessarily distinct), one can always choose distinct subscripts $i,j,k,l$ such that
		\[ \left| \dfrac{a_i}{a_j} - \dfrac {a_k}{a_l} \right| \le C. \]}
	
		\solu{Simplify the problem to get the ans first. Think about what is the smallest such value for any given $ 4 $ positive reals.}
	
	
	
	\prob{https://artofproblemsolving.com/community/c6h14091p99756}{ISL 2004 A1}{E}{Let $n \geq 3$ be an integer. Let $t_1$, $t_2$, ..., $t_n$ be positive real numbers such that \[n^2 + 1 > \left( t_1 + t_2 + \cdots + t_n \right) \left( \dfrac{1}{t_1} + \dfrac{1}{t_2} + \cdots + \dfrac{1}{t_n} \right).\] Show that $t_i$, $t_j$, $t_k$ are side lengths of a triangle for all $i$, $j$, $k$ with $1 \leq i < j < k \leq n$.}
	
		\solu{Easy solution by induction. For a more elegant solution, write the right side as sum of paired factors. Finding when the inequality breaks and relating it to the end statement.}
	
	
	
	
	\prob{https://artofproblemsolving.com/community/c6h219647p1218538}{ISL 1996 A2}{EM}{Let $ a_1 \geq a_2 \geq \ldots \geq a_n$ be real numbers such that for all integers $ k > 0,$
		
		\[ a^k_1 + a^k_2 + \ldots + a^k_n \geq 0.\]
		
		Let $ p =\max\{|a_1|, \ldots, |a_n|\}.$ Prove that $ p = a_1$ and that
		
		\[ (x - a_1) \cdot (x - a_2) \cdots (x - a_n) \leq x^n - a^n_1\] for all $ x > a_1.$}
	
		
		\solu{After the first part, apply AM-GM on the whole left side, this not gonna work, since we can't bound $ \sum a_i $ wrt $ a_1 $. So what if we divide both side by $ (x-a_1) $ and then apply AM-GM?}
	
	
	
	
\newpage\subsubsection{Smoothing And Convexity}

\bigskip

	\begin{take_note*}[title={Some usual tricks}]{}
		\begin{enumerate}
			\item Bring $ x, y $ closer, keeping $ x+y $ constant.
			\item If we need to smoothen up the value $ xy $, then take $ \ln $ on both side.
			\item Work with different variables.
		\end{enumerate}
	\end{take_note*}


	\theo{}{Convexity}{
		\begin{enumerate}[parsep=20pt]
			\item The function is convex in interval $ I $ iff for all $ a, b\in I $ and for all $ t<1 $, \[tf(a) + (1-t)f(b) \ge f\left(ta + (1-t)b\right)\]
			Which if put in words, means that the line segment joining $ (a, f(a)) $ and $ (b, f(b)) $ lies completely above the graph of the function.
			
			\item The function is convex in interval $ I $ if $ f' $ is increasing in $ I $ or $ f'' $ is positive in $ I $.
		\end{enumerate}
	}


	\theo{https://artofproblemsolving.com/wiki/index.php?title=Jensen's_Inequality}{Jensen's Inequality}{Let $x_1,\dots,x_n\in\R$ and let $\a_1,\dots, \a_n\ge 0$ satisfy $\a_1+\dots+\a_n=1$.\\
		
		If $ f $ is a Convex Function, we have:
		
		\[\a_1f(x_1)+\a_2f(x_2)\dots+a_nf(x_n) \ge f(\a_1x_1+\a_2x_2\dots+\a_nx_n)\]
		
		If $f$ is a Concave Function, we have:
		
		\[\a_1f(x_1)+\a_2f(x_2)\dots+a_nf(x_n) \le f(\a_1x_1+\a_2x_2\dots+\a_nx_n)\]
	}


	\theo{https://en.wikipedia.org/wiki/Popoviciu's_inequality}{Popoviciu's inequality}{Let $ f $ be a convex function on and interval $ I\in \R $. Then for any numbers $ x, y, z \in I $, 
		\[f(x)+f(y)+f(z)\ +3f\left(\frac{x+y+z}{3}\right) \ \ge\  2f\left(\frac{x+y}{2}\right)+2f\left(\frac{y+z}{2}\right)+2f\left(\frac{z+x}{2}\right)\]	
	}
	
	
	\prob{https://artofproblemsolving.com/community/c6h55341p343867}{USAMO 1998 P3}{H}{Let $a_0,a_1,\cdots ,a_n$ be numbers from the interval $(0,\pi/2)$ such that 
		\[ \tan (a_0-\dfrac{\pi}{4})+ \tan (a_1-\dfrac{\pi}{4})+\cdots +\tan (a_n-\dfrac{\pi}{4})\geq n-1. \] 
		Prove that 
		\[ \tan a_0\tan a_1 \cdots \tan a_n\geq n^{n+1}. \]}
	
	
		\solu{Get rid of the $ \tan $'s. AM-GM, Jensen doesn't works, so try \textbf{smoothing}. The conventional smoothing trick fails at one case, but works for all other cases. Means we have to deal with that case specially.}
		
	
	\prob{https://artofproblemsolving.com/community/c6h337923p1808247}{USAMO 1974 P2}{E}{Prove that if $ a,b,$ and $ c$ are positive real numbers, then \[ a^ab^bc^c \ge (abc)^{(a+b+c)/3}.\]}
	
		\solu{$ \ln $}
	
	
	\prob{https://artofproblemsolving.com/community/q1h61144p368290}{India 1995}{E}{Let $x_{1},x_{2},...,x_{n}>0$ be real numbers such that $x_{1}+x_{2}+x_{3}+...+x_{n}=1$. Prove the inequality
		\[\frac{x_{1}}{\sqrt{1-x_{1}}}+\frac{x_{2}}{\sqrt{1-x_{2}}}+....+\frac{x_{n}}{\sqrt{1-x_{n}}}\geq \sqrt{\frac{n}{n-1}}\]
	}

		\solu{easy smoothing}
	
	
	\prob{https://artofproblemsolving.com/community/q1h355706p1932030}{Vietnam 1998}{E}{$ x_1,x_2...x_n$ are real numbers such that \[\dfrac{1}{x_1+1998} +\dots +\dfrac{1}{x_n+1998}= \dfrac{1}{1998}\]
		Prove that \[\frac{\sqrt[n]{x_1\dots x_n}}{n-1} \ge 1998\]}
	
		\solu{Translate the given expression in a nicer way with new variables...}
	
	
	
	\prob{https://artofproblemsolving.com/community/c6h58608p358019}{IMO 1974 P2}{M}{The variables $a,b,c,d,$ traverse, independently from each other, the set of positive real values. What are the values which the expression \[ S= \dfrac{a}{a+b+d} + \dfrac{b}{a+b+c} + \dfrac{c}{b+c+d} + \dfrac{d}{a+c+d} \] takes?}
	
		\solu{$ \dfrac{x}{y+c} \le \dfrac{x}{y} \le \dfrac{x}{y-c} $}
	
	
	\prob{https://artofproblemsolving.com/community/c6h45679p289327}{Bulgaria 1995}{E}{Given $n$ real number $x_1,x_2,...,x_n\in [0,1]$ . Prove the following inequality
		\[(x_1 + x_2 +\dots+ x_n) - \left( x_1x_2 + x_2x_3 +\dots+ x_nx_1\right) \le \left\lfloor\frac{n}{2}\right\rfloor\]}
