\section{CP Algorithms}

\begin{myitemize}
\item \href{cp-algorithms.com/}{CP Algorithms}
\end{myitemize}

\subsection{Cycle Finding Algorithms}

\algorithm{}{Floyd's cycle-finding algorithm}{
    This algorithm finds a cycle by using two pointer. These pointers move over the sequence at different speeds. \textbf{In each iteration the first pointer advances to the next element, but the second pointer advances two elements.} It's not hard to see, that if there exists a cycle, the second pointer will make at least one full cycle and then meet the first pointer during the next few cycle loops. \emph{If the cycle length is $\lambda$ and the $\mu$ is the first index at which the cycle starts, then the algorithm will run in $ O(\lambda+\mu) $ time.}

    This algorithm is also known as \textbf{tortoise and the hare algorithm}, based on the tale in which a tortoise (here a slow pointer) and a hare (here a faster pointer) make a race.

    It is actually possible to determine the parameter $\lambda$
and $\mu$ using this algorithm (also in $ O(\lambda+\mu) $ time and $ O(1) $ space), but here is just the simplified version for finding the cycle at all. The algorithm and returns true as soon as it detects a cycle.}

\begin{lstlisting}[language=python]
def floyd(f, x0, n):
    tortoise = x0
    hare = x0
    while(tortoise != hare):
        tortoise = f(tortoise, n)
        hare = f(hare, n)
        hare = f(hare, n)
    return True
\end{lstlisting}

\vspace{2em}

\algorithm{}{Brent's algorithm}{Brent uses a similar algorithm as Floyd. It also uses two pointer. But instead of advancing the pointers by one and two respectably, we advance them in powers of two. As soon as $ 2^i  $is greater than $\lambda$ and $\mu$, we will find the cycle.}

\begin{lstlisting}[language=python]
def f(x, n):
    return (x*x + 3)%n

def brent(f, x0, n):
    tortoise = x0
    hare = f(x0, n)
    l = 1
    while(tortoise != hare):
        tortoise = f(tortoise, n)
        for i in range(l):
            hare = f(hare, n)
            if(hare == tortoise):
                return True
        l *= 2
    return True
\end{lstlisting}


