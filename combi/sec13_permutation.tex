\section{Permutations}

\prob{https://artofproblemsolving.com/community/c5h1434000p8108658}{USAMO 2017
    P5}{M}{
    Let $m_1, m_2, \ldots, m_n$ be a collection of $n$ positive integers, not
    necessarily distinct. For any sequence of integers $A = (a_1, \ldots,
    a_n)$ and any permutation $w = w_1, \ldots, w_n$ of $m_1, \ldots, m_n$,
    define an $A$-inversion of $w$ to be a pair of entries $w_i, w_j$ with $i
    < j$ for which one of the following conditions holds:
    \[\begin{aligned}
    a_i \ge w_i > w_j,\quad
    w_j > a_i \ge w_i,\quad
    w_i > w_j > a_i
    \end{aligned}\]
    Show that, for any two sequences of integers $A = (a_1, \ldots, a_n)$ and $B =
    (b_1, \ldots, b_n)$, and for any positive integer $k$, the number of
    permutations of $m_1, \ldots, m_n$ having exactly $k$ $A$-inversions is equal
    to the number of permutations of $m_1, \ldots, m_n$ having exactly $k$ $B$-inversions.

    \index[cat]{Permutations!USAMO 2017 P5}
    \index[strat]{Bijection!USAMO 2017 P5}
}

\begin{solution}
    Notice that if we take $B$ as a sequence with all elements greater than
    all $w_i$, then we have the $B$-inversions to be normal inversions wrt
    $M$. So we need to show that there exists a bijection between $A$-inversion and
    normal inversion. \\

    So we can either show that there for a permutation $w$ with $k$ normal
    inversions, there is a permutation $p$ with $k$ $A$-inversions. But we
    soon figure out it is pretty hard.\\

    If we try the other way, show that for every $w$ with $k$ $A$-inversions,
    there is a $p$ with the same $k$ normal inversions, and if we can show
    injectivity, we will be done. It turns out that this is much easier.
\end{solution}


\prob{https://artofproblemsolving.com/community/c6h271833p1472110}
{ISL 2008 C2}{E}{
    Let $n \in \mathbb N$ and $A_n$ set of all permutations $(a_1, \ldots,
    a_n)$ of the set $\{1, 2, \ldots , n\}$ for which
    \[k\ |\ 2(a_1 + \cdots+ a_k), \text{ for all } 1 \leq k \leq n.\]
    Find the number of elements of the set $A_n$.

    \index[cat]{Permutations!ISL 2008 C2}
    \index[strat]{Induction!ISL 2008 C2}
}

\begin{solution}
    First we try some smaller cases: $ |A_1| = 1$, $ |A_2| = 2$, $ |A_3| = 6$,
    $ |A_4| = 12$, which has a clear pattern. So we proceed with induction.\\

    With induction, we focus on $a_n$ only, it can have values eiher $n, 1$ or
    $\frac{n+1}{2}$. But the later case is impossible, and so we only have two
    options for $a_n$, which gives us our desired inductive relation.
\end{solution}
