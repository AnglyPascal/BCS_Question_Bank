\graphicspath{{Pics/combi/config/}}


\newpage\section{Exploring Configurations}

Problems where there is some kind of a configuration is given, the question
usually asks to proof or find some specific properties of the configuration.


\subsection{Problems}



\prob{https://artofproblemsolving.com/community/c6h1634977p10278658}{ARO 2018
    P11.5}{E}{
    On the table, there're $ 1000 $ cards arranged on a circle. On each
    card, a positive integer was written so that all $ 1000 $ numbers are
    distinct. 

    First, Vasya selects one of the card, remove it from the circle, and
    do the following operation: If on the last card taken out was written positive
    integer $ k $, count the $ k^{th} $ clockwise card not removed, from that
    position, then remove it and repeat the operation. This continues until only
    one card left on the table. 

    Is it possible that, initially, there's a card $A$ such that, no matter
    what other card Vasya selects as first card, the one that left is always
    card $A$?
}\label{problem:constructive_algo_7}

\solu{
    Consider the numbering, \[1, 1001!+1, 1002!+1, \dots 1998!+1,
    1999!+2\]It's easy to check that it works.
}

\rem{
    We want to find a configuration, where one card, let's call it $ a $,
    gets skipped over all the time. Now, controlling the skipping for every move
    is kinda hard. Instead of doing that, we want to control only one move that
    skips $ a $, and all other moves will go to the next card in the clockwise
    rotation. And the card before $ a $ will skip over $ a $, and land of the next
    card. That being said, the construction is rather trivial.
}


\prob{https://artofproblemsolving.com/community/c6h1446905p8271388}
{APMO 2017 P1}{M}{
    We call a $5$-tuple of integers arrangeable if its elements can be
    labeled $a,b,c,d,e$ in some order so that $a-b+c-d+e=29$. Determine all $ 2017$
    -tuples of integers $n_1, n_2, n_3\dots n_{2017}$ such that if we place
    them in a circle in clockwise order, then any $5$-tuple of numbers in
    consecutive positions on the circle is arrangeable.
}\label{problem:minus_constant_1}\label{problem:invariant_rules_of_thumb_5}

\solu{
    The annoying part is the $a-b+c-d+e = 29$ condition, as $29$ is too random.
    Can we do something to make this sum equal to a nicer interger, possibly
    $0$?
}



\prob{https://artofproblemsolving.com/community/c6h41116p258304}
{ISL 2004 C1}{E}{
    There are $10001$ students at an university. Some students join
    together to form several clubs (a student may belong to different clubs).
    Some clubs join together to form several societies (a club may belong to
    different societies). There are a total of $k$ societies. Find all
    possible values of $k$ so that the following conditions are satisfied:
    \begin{enumerate}
        \item  Each pair of students are in exactly one club.
        \item  For each student and each society, the student is in exactly
            one club of the society.
        \item  Each club has an odd number of students. In addition, a club
            with ${2m+1}$ students ($m$ is a positive integer) is in exactly $m$
            societies.  
    \end{enumerate}
}\label{problem:double_counting_7}

\solu{Just Double-Counting.}


\prob{https://artofproblemsolving.com/community/c6h17336p118710}{ISL 2002
C1}{E}{Let $ n $ be a positive integer. Each point $ (x,y) $ in the plane,
where $ x $ and $ y $ are non-negative integers with $ x+y<n $ , is coloured
red or blue, subject to the following condition: if a point $ (x,y) $ is red,
then so are all points $ (x',y') $ with $ x'\leq x $ and $ y'\leq y $. Let $ A
$ be the number of ways to choose $ n $ blue points with distinct $ x $
-coordinates, and let $ B $ be the number of ways to choose $ n $ blue points
with distinct $ y $ -coordinates. Prove that $ A=B
$.}\label{problem:induction_type1_5}\label{problem:recursive_solution_2}\label{problem:bijection_1}


\prob{https://artofproblemsolving.com/community/c5h476723p2669115}{USAMO 2012
P2}{M}{A circle is divided into $ 432 $ congruent arcs by $ 432 $ points. The
points are colored in four colors such that some $ 108 $ points are colored
Red, some $ 108 $ points are colored Green, some $ 108 $ points are colored
Blue, and the remaining $ 108 $ points are colored Yellow. Prove that one can
choose three points of each color in such a way that the four triangles formed
by the chosen points of the same color are
congruent.}\label{problem:double_counting_6}

\solu{Double counting saves the day :) The trick is to rotate ;)}



\prob{http://emc.mnm.hr/wp-content/uploads/2018/12/EMC_2018_Seniors_ENG_Solutions-2.pdf}{European
Mathematics Cup 2018 P1}{E}{Call a partition of $ n $ a set $ a_1, \dots a_k $
with $ a_1 \le a_2\dots \le a_k $ and $ a_1 + a_2 \dots + a_k =  $. A
partition of a positive integer is `even' if all of its elements are even
numbers. Similarly, a partition is `odd' if all of its elements are odd.
Determine all positive integers $ n $ such that the number of even partitions
of $ n $ is equal to the number of odd partitions of $ n $.}

\solu{Bijection.}


\prob{https://artofproblemsolving.com/community/c6h1268873p6622370}{ISL 2015 C1}{E}{In Lineland there are $n\geq1$ towns, arranged along a road running from left to right. Each town has a left bulldozer (put to the left of the town and facing left) and a right bulldozer (put to the right of the town and facing right). The sizes of the $2n$ bulldozers are distinct. Every time when a left and right bulldozer confront each other, the larger bulldozer pushes the smaller one off the road. On the other hand, bulldozers are quite unprotected at their rears; so, if a bulldozer reaches the rear-end of another one, the first one pushes the second one off the road, regardless of their sizes.

    Let $A$ and $B$ be two towns, with $B$ to the right of $A$. We say that town $A$ can sweep town $B$ away if the right bulldozer of $A$ can move over to $B$ pushing off all bulldozers it meets. Similarly town $B$ can sweep town $A$ away if the left bulldozer of $B$ can move over to $A$ pushing off all bulldozers of all towns on its way.

Prove that there is exactly one town that cannot be swept away by any other one.}

\solu{Focus on the heaviest bulldozer.}


\prob{https://artofproblemsolving.com/community/c6h195492p1073989}{APMO 2008
P2}{EM}{Students in a class form groups each of which contains exactly three
members such that any two distinct groups have at most one member in common.
Prove that, when the class size is $ 46 $ , there is a set of $ 10 $ students
in which no group is properly
contained.}\label{problem:bijection_8}\label{problem:extremal_case_whole_5}

\solu{Taking the maximum set that follows the ``in which no group is properly
contained'' rule. Now the elements that are \emph{not} in this set, we can
connect this element to only one of the pairs from the set. Now defining a
bijection, and counting the elements, we are done.}



\prob{https://artofproblemsolving.com/community/c6h364231p2000940}{IMO SL
1985}{M}{A set of $ 1985 $ points is distributed around the circumference of a
circle and each of the points is marked with $ 1 $ or $ -1 $. A point is
called ``good'' if the partial sums that can be formed by starting at that
point and proceeding around the circle for any distance in either direction
are all strictly positive. Show that if the number of points marked with $ -1
$ is less than $ 662 $ , there must be at least one good
point.}\label{problem:induction_type1_12}

\solu{First thing to notice, the number $ 3*661 + 2 = 1985 $. And these
numbers are completely random. So what if we try to replace $ 1985 $ by $ n $
? Will the condition still hold?}


\prob{https://artofproblemsolving.com/community/c6h418978p2365036}{IMO 2011
P4}{E}{Let $ n > 0 $ be an integer. We are given a balance and $ n $ weights
of weight $ 2^0, 2^1, \cdots, 2^{n-1} $. We are to place each of the $ n $
weights on the balance, one after another, in such a way that the right pan is
never heavier than the left pan. At each step we choose one of the weights
that has not yet been placed on the balance, and place it on either the left
pan or the right pan, until all of the weights have been placed. Determine the
number of ways in which this can be done.}\label{problem:recursive_solution_5}

\solu{Writing the whole process as a sum, we see that only $ 2^0 $ is the odd
term here, if we remove that we can divide by $ 2 $ to get a recursive
formula.}

\solu{Calculating wrt to the last placed weight.}

\solu{Getting recursive formula considering the position of $ 2^{n-1} $.}





\prob{https://artofproblemsolving.com/community/c5h202936p1116367}{USAMO 2008
P3}{H}{Let $ n $ be a positive integer. Denote by $ S_n $ the set of points $
(x, y) $ with integer coordinates such that \[ \left\lvert x\right\rvert +
\left\lvert y + \frac{1}{2} \right\rvert < n. \] A path is a sequence of
distinct points $ (x_1 , y_1), (x_2, y_2), \ldots, (x_\ell, y_\ell) $ in $ S_n
$ such that, for $ i = 2, \ldots, \ell $ , the distance between $ (x_i , y_i)
$ and $ (x_{i-1} , y_{i-1} ) $ is $ 1 $ (in other words, the points $ (x_i,
y_i) $ and $ (x_{i-1} , y_{i-1} ) $ are neighbors in the lattice of points
with integer coordinates). Prove that the points in $ S_n $ cannot be
partitioned into fewer than $ n $ paths (a partition of $ S_n $ into $ m $
paths is a set $ \mathcal{P} $ of $ m $ nonempty paths such that each point in
$ S_n $ appears in exactly one of the $ m $ paths in $ \mathcal{P} $
).}\label{problem:alternating_chains_2}\label{problem:coloring_2}


\solu{Graph + Partition, coloring is just natural. Again, the edges join two
neighbor lattice points, so checkerboard coloring. But checkerboard doesn't do
much good. So the next thing we try is to apply some derivations of it,
pseudo!!! Well, overkill.}


\solu{For all n, induction is very natural. The optimal partition (the most
beautiful one) and the longest path in it, say $ P $ , gives us a way to
perform induction. As always, we suppose a partition with $ n-1 $ paths. As
there are a lot of partitions, we need to choose a certain partition, say $
\mathbb{M} $. Again as our goal is to include $ P $ in $ \mathbb{M} $. So
suppose that the set with all the points in $ P $ is $ A $. And further more,
suppose that in $ \mathbb{M} $ there is a path $ Q $ with $ \vert Q\cap A\vert
$ being maximal among all other partitions of the points. Some some easy case
work shows that we must have $ P\in \mathbb{M} $.}



\prob{https://artofproblemsolving.com/community/c5h532235p3041823}{USAMO 2013
P2}{H}{For a positive integer $ n\geq 3 $ plot $ n $ equally spaced points
around a circle. Label one of them $ A $ , and place a marker at $ A $. One
may move the marker forward in a clockwise direction to either the next point
or the point after that. Hence there are a total of $ 2n $ distinct moves
available; two from each point. Let $ a_n $ count the number of ways to
advance around the circle exactly twice, beginning and ending at $ A $ ,
without repeating a move. Prove that $ a_{n-1}+a_n=2^n $ for all $ n\geq 4
$.}\label{problem:recursive_solution_4}\label{problem:bijection_7}


\solu{Problems where there are multiple possible value of a function
regardless of the current position, one of dealing with these is to assigning
labels of these possible values to each points of the function, and this will
give a combinatorial model and a way to deal it with bijection.}

\solu{First investigate the problem condition, $ a_n + a_{n-1} = 2^n $ , now,
$ 2^n $ means the number of differently coloring every point black or white,
and the left side is the number of such paths for $ n $ and $ n-1 $. Which
means we should try to color the points and see what happens.}

\proof{EChen's solution: In this problem, the main obstacle seems to be the
circle condition. And on top of that, on can land on the starting point. So
things are pretty messed up here. What we want to do is to make things a
little bit more easy to deal with. So our best option is to change the problem
so that we get the similar problem with a different explanation. So we change
the condition circle with matrix, $ 2 $ round with $ 2 $ rows. $ n $ points
with $ n $ entries in each rows. What we get now is the same problem, just a
bit easier to deal with. We call this \hl{Tweak The Problem} strategy.}



\prob{}{}{E}{ $ 10 $ persons went to a bookstore. It is known that: Every
person has bought 3 kinds on books and for every 2 persons, there is at least
one kind of books which they both have bought. Let $ m_i $ be the number of
the persons who bought the $ i^{th} $ kind of books and $ M= \max\lbrace
m_i\rbrace $ Find the smallest possible value of $ M
$.}\label{problem:double_counting_4}


\prob{https://artofproblemsolving.com/community/c6h86560p504805}{ARO 2006
    11.3}{M}{On a $49\times 69$ rectangle formed by a grid of lattice squares,
    all $50\cdot 70$ lattice points are colored blue. Two persons play the
    following game: In each step, a player colors two blue points red, and
    draws a segment between these two points. (Different segments can
    intersect in their interior.) Segments are drawn this way until all
    formerly blue points are colored red. At this moment, the first player
    directs all segments drawn - i. e., he takes every segment AB, and
    replaces it either by the vector $\overrightarrow{AB}$, or by the vector
    $\overrightarrow{BA}$. If the first player succeeds to direct all the
    segments drawn in such a way that the sum of the resulting vectors is
    $\overrightarrow{0}$, then he wins; else, the second player wins.

Which player has a winning strategy?}

\proof{The basic idea comes from wishing that the first player might be able
to copy the second player to ``nullify'' his moves. But since this isn't
always possible, because no nice symmetry exists on the board, the idea of
coloring the board with dominoes and copying moves wrt the dominoes comes.
\index[strat]{copycat!partition!aro 2006 11.3}}


\prob{https://artofproblemsolving.com/community/c6h17338p118714}{ISL 2002
C3}{EM}{Let $n$ be a positive integer. A sequence of $n$ positive integers
(not necessarily distinct) is called full if it satisfies the following
condition: for each positive integer $k\geq2$, if the number $k$ appears in
the sequence then so does the number $k-1$, and moreover the first occurrence
of $k-1$ comes before the last occurrence of $k$. For each $n$, how many full
sequences are
there?}\label{problem:bijection_13}\label{problem:graph_representation_9}

\proof{After guessing the ans, the first thing that I did was to draw a level
based graph. Suppose that a full sequence has $ k $ different entries. Then
the top level contains the positions of $ k $ in the sequence sorted from left
to right. The next level contains the positions of $ k-1 $ in the sequence
sorted so, and so on till the last level. What I noticed is that if we draw
arrows pointing from a larger integer to a smaller integer, the only arrows
(or more like relations between entries of the sequence) we need to worry
about are the arrows pointing left to right in each levels, and the arrows
from the last entry of level $ i $ to the first entry of level $ i+1 $. After
this, if we try with a smaller case, we see that this leads to a bijection
from the set of sequences of length $ n $ with $ n $ different integers to the
set of full-sequences of length $ n $.}


\solu{Another bijection approach is as followed, in a full-sequence, on first
run, go from right to left, placing integers starting with $ 1 $ onwards on
the $ 1 $'s in the sequence. on the second run continue counting and placing
integers on the $ 2 $'s and so on.}


\solu{Another idea is to prove $ a_n = n a_{n-1} $. To do this, remove the
rightmost $ 1 $ and do some casework.}



\prob{https://artofproblemsolving.com/community/c6h219938p1219679}{ISL 1994
C2}{M}{In a certain city, age is reckoned in terms of real numbers rather than
integers. Every two citizens $x$ and $x'$ either know each other or do not
know each other. Moreover, if they do not, then there exists a chain of
citizens $x = x_0, x_1, \ldots, x_n = x'$ for some integer $n \geq 2$ such
that $ x_{i-1}$ and $x_i$ know each other. In a census, all male citizens
declare their ages, and there is at least one male citizen. Each female
citizen provides only the information that her age is the average of the ages
of all the citizens she knows. Prove that this is enough to determine uniquely
the ages of all the female citizens.}

\solu{Describing the problem using matrix and vector spaces, the problem
reduces to well known theorems of linear algebra.}


\prob{https://artofproblemsolving.com/community/c6h93p261}{ISL 2003 C1}{E}{Let
$A$ be a $101$-element subset of the set $S=\{1,2,\ldots,1000000\}$. Prove
that there exist numbers $t_1$, $t_2, \ldots, t_{100}$ in $S$ such that the
sets \[ A_j=\{x+t_j\mid x\in A\},\qquad j=1,2,\ldots,100 \] are pairwise
disjoint.}

\solu{just count...}


\prob{https://artofproblemsolving.com/community/c6h1425422p8029376}{EGMO 2017
    P5}{E}{Let $n\geq2$ be an integer. An $n$-tuple $(a_1,a_2,\dots,a_n)$ of
    not necessarily different positive integers is expensive if there exists a
    positive integer $k$ such that 

    \[(a_1+a_2)(a_2+a_3)\dots(a_{n-1}+a_n)(a_n+a_1)=2^{2k-1}\]

    a) Find all integers $n\geq2$ for which there exists an expensive
    $n$-tuple.

b) Prove that for every odd positive integer $m$ there exists an integer
$n\geq2$ such that $m$ belongs to an expensive $n$-tuple.}

\rem{gutaguti solution}

\solu{All odd $ n $ works, you can prove for even $ n $ by $ +1, -1 $
addition. For the second part, start with the odd number, move on both side,
you will eventually reach $ 1 $.}



\prob{https://artofproblemsolving.com/community/c6h84550p490581}{USAMO 2006
P2}{E}{For a given positive integer $k$ find, in terms of $k$, the minimum
value of $N$ for which there is a set of $2k + 1$ distinct positive integers
that has sum greater than $N$ but every subset of size $k$ has sum at most
$\tfrac{N}{2}.$}

\solu{Compactness is the optimal decision.}



\prob{https://artofproblemsolving.com/community/c6h1076949p4710743}{MOP
Problem}{E}{Prove that for any positive integer $c$, there exists an integer
$n$ such that $n$ has more 1's in its binary expansion than $n^2+c$ does.}

\solu{For $ x= 2^a-1 $, $ x $ and $ x^2 $ have the same number of $ 1's $. So
does $ x=2^a-2^b $. But what if increase the number of $ 1's $ in this $ x $
by substracting $ 1 $? Let $ x=2^a - 2^b -1 $. This might work if we can
choose nice $ a, b $'s}



\prob{https://artofproblemsolving.com/community/c6h1078552p4732509}{EGMO 2015
P2}{EM}{A domino is a $2 \times 1$ or $1 \times 2$ tile. Determine in how many
ways exactly $n^2$ dominoes can be placed without overlapping on a $2n \times
2n$ chessboard so that every $2 \times 2$ square contains at least two
uncovered unit squares which lie in the same row or column.}

\solu{Notice how each of the four kind of dominoes needs to be in a group. So
if we separated them into blocks, inverstigation shows that there can only be
$ 4 $ blocks and each strictly attached to the sides. The reason why this is
happening is pretty obvious. Now those blocks create two paths between the two
opposite vertices of the square. This gives our desired bijection.}



\prob{https://artofproblemsolving.com/community/c6h148826p841252}{USA TST 2006
P5}{TE}{Let $n$ be a given integer with $n$ greater than $7$ , and let
$\mathcal{P}$ be a convex polygon with $n$ sides. Any set of $n-3$ diagonals
of $\mathcal{P}$ that do not intersect in the interior of the polygon
determine a triangulation of $\mathcal{P}$ into $n-2$ triangles. A triangle in
the triangulation of $\mathcal{P}$ is an interior triangle if all of its sides
are diagonals of $\mathcal{P}$. Express, in terms of $n$, the number of
triangulations of $\mathcal{P}$ with exactly two interior triangles, in closed
form.}

\solu{Just mindless calculation...}



\prob{https://artofproblemsolving.com/community/c6h418687p2362298}{ISL 2010
    C3}{E}{2500 chess kings have to be placed on a $100 \times 100$ chessboard
    so that

    \begin{enumerate} \item  no king can capture any other one (i.e. no two
    kings are placed in two squares sharing a common vertex); \item  each row
    and each column contains exactly 25 kings.  \end{enumerate}

Find the number of such arrangements. (Two arrangements differing by rotation
or symmetry are supposed to be different.)}



\solu{In a $ 2\times 2 $ box, one can place only one king. So we divide the
board in that way, and explore...}



\prob{https://artofproblemsolving.com/community/c6h1558133_a_tasty_dish_of_graph_theory}{USA
    Winter TST 2018 P3}{M}{
    At a university dinner, there are $ 2017 $
    mathematicians who each order two distinct entrées, with no two mathematicians
    ordering the same pair of entrées. The cost of each entrée is equal to the
    number of mathematicians who ordered it, and the university pays for each
    mathematician's less expensive entrée (ties broken arbitrarily). Over all
    possible sets of orders, what is the maximum total amount the university could
    have paid?
}

	\begin{solution}[Grid Representation]
			We put the information in a grid in the usual manner. We take the configuration for which the score (which we define to be the total amount the university has to pay) is maximum.
			
			\begin{solution_def}
				Let $ C_i $ be the price of the dish $ i $. Let $ P_j $ be the pair of dishes $ j $ ordered.
			\end{solution_def}
			
			We now ``sort'' the grid in the following way:
			\begin{itemize}
				\item Sort the columns first, from the least expensive to the most.
				\item Now we have $ 2017 $ binary strings as rows. We sort them decreasingly from top to bottom.
			\end{itemize}
		
			We end up with something like this:
			\figdf{.3}{USA_TST_2018_P3_1}{}
			
			\begin{solution_def}
				We make some sets:\[ A_i=\{x| x \text{ ordered dish } i \text{ but didn't order any of the dishes } j<i\} \]Let $ a_i = |A_i| $.
				We also call $ A_i $ block, the $ x $th rows with $ x\in A_i $. It is easy to see that the committee has to pay for dish $ i $ $ a_i $ times.
			\end{solution_def}
		
			Take a $ j $ such that $ a_j=0 $.
			 
			If $ j<k $, let \[S_j = \{x\mid \text{In the } j^{th} \text{ column, there is a } 1 \text{ in the set of the rows from } A_x \}\]
				\figdf{.3}{USA_TST_2018_P3_2}{}
			In the example of the previous diagram, $ S_4 = \{1, 2, 3\} $
			
			Take the largest element of $ S_j $, namely $ t $. We want to move the $ 1 $ in the $ A_t $ block of column $ j $ to another column to the right. Notice that this won't decrease the score, because for each of the mathematician after the block $ A_t $, the score would be non decreasing. If we could do this, we could do this inductively, moving the $ j $th column to the left each move, eventually making it disappear. 
			
			So suppose we can't do this. So there is no column on the right of $ t $, that does not have a $ 1 $ in the block $ A_t $. It is straightforward to deduce that $ N=t+a_t $. 
			
			From there, we can say:
				\[a_{t+m}\le a_t-m\text{ and }a_{t-m}\le a_t+m\]
			So,
			\begin{align*}
				2017=\sum a_i &\le (a_t+t-1) +\dots 1 \\
				\implies 2017 &\le \frac{N(N-1)}{2}\\
				\implies N&\ge 65
			\end{align*}
			
			We know that the score, $ S= \sum C_ka_k $. where $ a_i $ is strictly decreasing and $ C_i $ is non decreasing. So the maximum sum would have $ C_i=K $ for some constant (by rearrangement inequality). And to have $ K $ maximum, we need $ N $ minimum or $ N=64 $. 
			
			And if $ j>k $, the same reason holds. So $ N=65, a_1=63, a_2=62\dots a_63=1 $. But we need another $ a_i=1 $, that has to be on its own. 
		\end{solution}

\rem{Easy to think in grids, but it was quite difficult to formulate the
solution rigorously, I still am not 100\% convinced myself. Next time when I
feel like it, I will reconstruct the solution. Roughly it is: take the final 0
value columns, prove that there is only one such, and prove the bounds on the
solution before using that instead. Then use general bounding to find the
maximum. It's not hard, it's just I don't have time now. GAH!! HSC!!}

\rem{Didn't read other solutions too, gonna give a todo}
%todo: read other solutions.


\subsubsection{Conway's Soldiers}


\prob{https://artofproblemsolving.com/community/c6h62194p372309}{ISL 1993
C5}{MH}{On an infinite chessboard, a solitaire game is played as follows: at
the start, we have $n^2$ pieces occupying a square of side $n.$ The only
allowed move is to jump over an occupied square to an unoccupied one, and the
piece which has been jumped over is removed. For which $n$ can the game end
with only one piece remaining on the board?}

\solu{We want to find an invariant. So we need to find a weight for each of
the cells such that any two consecutive cells' values equals to the values of
the two cells on the two sides. Some mind bashing gives the idea of mod $ 3 $.
And a construction for the other $ n $'s can be easily generated after some
casework.}



\prob{https://artofproblemsolving.com/community/c6h262542p1426437}{ARO 1999
P4}{M}{A frog is placed on each cell of a $n \times n$ square inside an
infinite chessboard (so initially there are a total of $n \times n$ frogs).
Each move consists of a frog $A$ jumping over a frog $B$ adjacent to it with
$A$ landing in the next cell and $B$ disappearing (adjacent means two cells
sharing a side). Prove that at least $ \left[\frac{n^2}{3}\right]$ moves are
needed to reach a configuration where no more moves are possible.}	

\solu{In the final stage, no two neighboring cells are occupied. Could we
double count the number of frogs with this information? What about the number
of frogs in the original $ n\times n $ board? Another small information needed
for this is that we need $ 2 $ moves to ``empty'' a $ 2\times 2 $ board.}






\subsubsection{Triominos}

\prob{https://artofproblemsolving.com/community/c6h404324p2254569}{ARO 2011
P10.8}{M}{A $2010\times 2010$ board is divided into corner-shaped figures of
three cells. Prove that it is possible to mark one cell in each figure such
that each row and each column will have the same number of marked cells.}

\solu{First we will mark the corner pieces of the triominos. Then shift the
mark to either of the legs. Our objective is to show that we can always do
this. First if we only focus on the rows, we can easily show that this can be
done using some counting. To show that we can do the same for columns as well,
we create a graph from columns and rows to triominos which should be operated
on, and using Hall's Marriage we pove the result.}



\prob{}{St. Petersburg 2000}{E}{On an infinite checkerboard are placed $111$
non-overlapping corners, L-shaped figures made of $3$ unit squares. Suppose
that for any corner, the $2\times 2$ square containing it is entirely covered
by the corners. Prove that one can remove some number between $1$ and $110$ of
the corners so that the property will be preserved.}



\subsubsection{Dominos}

\prob{}{}{E}{An $m\times n$ rectangular grid is covered by dominoes. Prove
that the vertices of the grid can be coloured using three colours so that any
two vertices a distance $1$ apart are colored with different colours if and
only if their segment lies on the boundary of a domino.}

\solu{Create a graph with the midpoints of the dominos.}




\subsection{Clearly Bijection}

\prob{https://artofproblemsolving.com/community/c6h1446909p8271411}
{APMO 2017 P3}{H}{
    Let $ A(n) $ denote the number of sequences $ a_1\geq a_2\geq \dots\geq
    a_k $ of positive integers for which   $ \sum_{i=1}^k a_k =n $ and each $
    a_i+1 $ is a power of two. 

    Let $ B(n) $ denote the number of sequences $
    b_1\geq b_2\geq\dots\geq b_k $ of positive integers for which $ \sum_{i=1}^k
    b_k =n $ and each inequality $ b_j\geq 2b_{j+1} $ holds $(j=1,2\dots m-1)$. 

    Prove that $ \vert A(n)\vert =\vert B(n)\vert $ for every
    positive integer. 
}\label{problem:add_stuffs_1}\label{problem:bijection_6}

\solu{
    A sequence of the first type can be rewritten as: \[
    n=x_1+3x_2+7x_3\dots +(2^i-1)x_i+\dots (2^k-1)x_k \] Where $ x_i $ are
    non-negative integers. This motivates us to find a way to represent $ b_i $ as
    sums of $ ( 2^i-1)x_i $. Then since $ b_j\geq 2b_{j+1} $, we write: $
    b_i=2b_{i-1}+x_i $ with $ x_i $ being non-negative integers.
}




\prob{https://artofproblemsolving.com/community/c6h215429p1191679}
{ISL 2008 C4}{M}{
    Let $ n $ and $ k $ be positive integers with $ k\geq n $ and $k-n$ 
    an even number. Let $ 2n $ lamps labeled $ 1,2\dots 2n $ be given, each
    of which can be either on or off. Initially, all the lamps are off. We
    consider sequences of steps: at each step one of the lamps is switched
    (from on to off or from off to on).

    Let $ N $ be the number of such sequences consisting of $ k $ steps and
    resulting in the state where lamps $ 1 $ through $ n $ are all on, and
    lamps $ n+1 $ through $ 2n $ are all off.

    Let $ M $ be number of such sequences consisting of $ k $ steps, resulting
    in the state where lamps $ 1 $ through $ n $ are all on, and lamps $ n+1 $
    through $ 2n $ are all off, but where none of the lamps $ n+1 $ through $
    2n $ is ever switched on.

    Determine $ \dfrac{N}{M} $.
}\label{problem:bijection_5}

\solu{
    These type of problems most of the time have bijection or algo
    solutions. Think of a way to perform bijection from the set $ S\{M\}
    \rightarrow S\{N\} $. Find an algorithm to get a sequence of the first type
    from a sequence of the second type.
}


\prob{https://artofproblemsolving.com/community/c6h57380p353058}
{USAMO 1996 P4}{E}{
    An $ n $ -term sequence $ (x_1, x_2, \ldots, x_n) $ in which each term
    is either 0 or 1 is called a binary sequence of length $ n $. Let $ a_n $ be
    the number of binary sequences of length $ n $ containing no three consecutive
    terms equal to 0, 1, 0 in that order. Let $ b_n $ be the number of binary
    sequences of length $ n $ that contain no four consecutive terms equal to 0,
    0, 1, 1 or 1, 1, 0, 0 in that order. Prove that $ b_{n+1} = 2a_n $ for all
    positive integers $ n $.
}\label{problem:bijection_4}

\solu{
    These type of problems cries for a nice bijection. That is a way to get
    from $ a\rightarrow b $ and vice versa. What if there is no $ 0,0,1,1 $ ? Or
    what if there is no $ 0,1,0 $ ? What is an one way bijection?
}
