\newpage\section{Problems}

	
	\prob{https://artofproblemsolving.com/community/c6h355783p1932923}{ISL 2009 C1}{E}{Consider $ 2009 $ cards, each having one gold side and one black side, lying on parallel on a long table. Initially all cards show their gold sides. Two player, standing by the same long side of the table, play a game with alternating moves. Each move consists of choosing a block of $ 50 $ consecutive cards, the leftmost of which is showing gold, and turning them all over, so those which showed gold now show black and vice versa. The last player who can make a legal move wins.

		\begin{enumerate}
			\item  Does the game necessarily end?
			\item  Does there exist a winning strategy for the starting player?
		\end{enumerate}}\label{problem:extremal_case_whole_3}

	\solu{Simplicity is the key.}





	\prob{https://artofproblemsolving.com/community/c6h355915p1934456}{ISL 2009 C3}{H}{Let $ n $ be a positive integer. Given a sequence $ \varepsilon_1 $ , $ \dots $ , $ \varepsilon_{n - 1} $ with $ \varepsilon_i = 0 $ or $ \varepsilon_i = 1 $ for each $ i = 1 $ , $ \dots $ , $ n - 1 $ , the sequences $ a_0 $ , $ \dots $ , $ a_n $ and $ b_0 $ , $ \dots $ , $ b_n $ are constructed by the following rules:

		\[a_0 = b_0 = 1, \quad a_1 = b_1 = 7,\]

		\[\begin{array}{lll} a_{i+1} = \begin{cases} 2a_{i-1} + 3a_i, \\ 3a_{i-1} + a_i, \end{cases} & \begin{array}{l} \text{if } \varepsilon_i = 0, \\ \text{if } \varepsilon_i = 1, \end{array} & \text{for each } i = 1, \dots, n - 1,

		\\[15pt] b_{i+1}= \begin{cases} 2b_{i-1} + 3b_i, \\ 3b_{i-1} + b_i, \end{cases} & \begin{array}{l} \text{if } \varepsilon_{n-i} = 0, \\ \text{if } \varepsilon_{n-i} = 1, \end{array} & \text{for each } i = 1, \dots, n - 1. \end{array}\]

		Prove that $ a_n = b_n $.}\label{problem:bijection_3}\label{problem:recursive_solution_3}

	\solu{\hl{Got the idea, will try later.}}







	\prob{https://artofproblemsolving.com/community/c6h1634977p10278658}{ARO 2018 P11.5}{E}{On the table, there're $ 1000 $ cards arranged on a circle. On each card, a positive integer was written so that all $ 1000 $ numbers are distinct. First, Vasya selects one of the card, remove it from the circle, and do the following operation: If on the last card taken out was written positive integer $ k $ , count the $ k^{th} $ clockwise card not removed, from that position, then remove it and repeat the operation. This continues until only one card left on the table. Is it possible that, initially, there's a card $ A $ such that, no matter what other card Vasya selects as first card, the one that left is always card $ A $ ?}\label{problem:constructive_algo_7}





	\prob{https://artofproblemsolving.com/community/c6h1441121p8200413}{ARO 2017 P9.1}{E}{In country some cities are connected by oneway flights (There are no more then one flight between two cities). City $ A $ called "available" for city $ B $ , if there is flight from $ B $ to $ A $ , maybe with some transfers. It is known, that for every 2 cities $ P $ and $ Q $ exist city $ R $ , such that $ P $ and $ Q $ are available from $ R $. Prove, that exist city $ A $ , such that every city is available for $ A $.}\label{problem:induction_type1_3}





	\prob{https://artofproblemsolving.com/community/c6h1632768p10256373}{ARO 2018 10.3}{E}{A positive integer $ k $ is given. Initially, $ N $ cells are marked on an infinite checkered plane. We say that the cross of a cell $ A $ is the set of all cells lying in the same row or in the same column as $ A $. By a turn, it is allowed to mark an unmarked cell $ A $ if the cross of $ A $ contains at least $ k $ marked cells. It appears that every cell can be marked in a sequence of such turns. Determine the smallest possible value of $ N $.}\label{problem:constructive_algo_2}

	\solu{First find the construction.}





	\prob{https://artofproblemsolving.com/community/c6h1635128p10279946}{ARO 2018 P9.5}{E}{On the circle, $ 99 $ points are marked, dividing this circle into $ 99 $ equal arcs. Petya and Vasya play the game, taking turns. Petya goes first; on his first move, he paints in red or blue any marked point. Then each player can paint on his own turn, in red or blue, any uncolored marked point adjacent to the already painted one. Vasya wins, if after painting all points there is an equilateral triangle, all three vertices's of which are colored in the same color. Could Petya prevent him?}\label{problem:forget_and_focus_1}

	\solu{Think of what Petya must do to prevent immediate losing.}





	\prob{https://artofproblemsolving.com/community/c6h24077p152742}{ISL 2004 C2}{E}{Let $ {n} $ and $ k $ be positive integers. There are given $ {n} $ circles in the plane. Every two of them intersect at two distinct points, and all points of intersection they determine are pairwise distinct (i. e. no three circles have a common point). No three circles have a point in common. Each intersection point must be colored with one of $ n $ distinct colors so that each color is used at least once and exactly $ k $ distinct colors occur on each circle. Find all values of $ n\geq 2 $ and $ k $ for which such a coloring is possible.}\label{problem:induction_type1_1}





	\prob{https://artofproblemsolving.com/community/c6h40115p251391}{ISL 2004 C3}{E}{The following operation is allowed on a finite graph: Choose an arbitrary cycle of length 4 (if there is any), choose an arbitrary edge in that cycle, and delete it from the graph. For a fixed integer $ {n\ge 4} $ , find the least number of edges of a graph that can be obtained by repeated applications of this operation from the complete graph on $ n $ vertices's (where each pair of vertices's are joined by an edge).}\label{problem:bipartite_graph_2}

	\solu{Walk backwards. or the same thing with \hrf{lemma:bipartite_graph}{Bipartite Graphs}.}





	\prob{https://artofproblemsolving.com/community/c6h476661p2668792}{Iran TST 2012 P4}{E}{Consider $ m+1 $ horizontal and $ n+1 $ vertical lines ( $ m,n\ge 4 $ ) in the plane forming an $ m\times n $ table. Cosider a closed path on the segments of this table such that it does not intersect itself and also it passes through all $ (m-1)(n-1) $ interior vertices's (each vertex is an intersection point of two lines) and it doesn't pass through any of outer vertices. Suppose $ A $ is the number of vertices's such that the path passes through them straight forward, $ B $ number of the table squares that only their two opposite sides are used in the path, and $ C $ number of the table squares that none of their sides is used in the path. Prove that $ A=B-C+m+n-1 $.}\label{problem:double_counting_2}







	\prob{https://artofproblemsolving.com/community/q2h1060228p4589671}{AoPS}{E}{Given $ 2n+1 $ irrational numbers, prove that one can pick $ n $ from them s.t. no two of the choosen $ n $ sum up to a rational number.}\label{problem:bipartite_graph_1}\label{problem:graph_representation_3}

	\solu{Use a graph theory representation.}





	\prob{https://artofproblemsolving.com/community/c6h542686p3131086}{Bulgarian IMO TST 2004, Day 3, Problem 3}{H}{Prove that among any $ 2n+1 $ irrational numbers there are $ n+1 $ numbers such that the sum of any $ k $ of them is irrational, for all $ k \in \{1,2,3,\ldots, n+1 \} $.}\label{problem:add_time_5}\label{problem:constructive_algo_1}

	\solu{We first create a set $ B $ such that any linear combination of the elements in it are irrational. Then for convenience, we add $ 1 $ to it, so that now the sum equals to $ 0 $ of any linear combinations. An algorithm for building it comes into our mind, which leaves some other original elements, which we then later add to the final solution set $ A $ along with the elements in the set $ B $ except $ 1 $.}







	\prob{https://artofproblemsolving.com/community/c6h176p611}{ISL 1997 P4}{E}{An $ n\times n $ matrix whose entries come from the set $ S = \{1,2, . . . ,2n-1\} $ is called a ``silver matrix'' if, for each $ i = 1,2, . . . , n $ , the $ i $ -th row and the $ i $ -th column together contain all elements of $ S $. Show that:

		\begin{enumerate}[wide=0em, label=\arabic*, itemsep=0pt, parsep=0pt, font=\footnotesize\bfseries]

			\item there is no silver matrix for $ n = 1997 $ ;
			\item silver matrices exist for infinitely many values of $ n $.
	\end{enumerate}}\label{problem:induction_type1_2}

	\solu{Proving that for odd $ n $ 's isn't hard. Then A small try-around with $ n=2, 4 $ , we see a pattern that leads to a construction for $ 2^n $ }






	\prob{}{}{E}{A rectangle is completely partitioned into smaller rectangles such that each smaller rectangles has at least one integral side. Prove that the original rectangle also has at least one integral side.}\label{problem:extreme_object_5}

	\solu{Try a special grid system with $.5\times .5 $ boxes.}

	\solu{Consider the number of corners in the rectangle.}







	\prob{https://artofproblemsolving.com/community/c6h40197p251895}{ISL 2004 C5}{M}{ $ A $ and $ B $ play a game, given an integer $ N $, $ A $ writes down $ 1 $ first, then every player sees the last number written and if it is $ n $ then in his turn he writes $ n+1 $ or $ 2n $ , but his number cannot be bigger than $ N $. The player who writes $ N $ wins. For which values of $ N $ does $ B $ win?}\label{problem:win_lose_1}

	\solu{Trying with smaller cases, it's easy. Using most important game theory \hrf{win_lose}{trick}.}







	\prob{https://artofproblemsolving.com/community/c6h155692p874978}{ISL 2006 C1}{E}{We have $ n \geq 2 $ lamps $ L_1, L_2 \dots L_n $ in a row, each of them being either on or off. Every second we simultaneously modify the state of each lamp as follows: if the lamp $ L_i $ and its neighbors (only one neighbor for $ i = 1 $ or $ i = n $ , two neighbors for other $ i $ ) are in the same state, then $ L_i $ is switched off; otherwise, $ L_i $ is switched on. Initially all the lamps are off except the leftmost one which is on.

		\begin{enumerate}[wide=0em, label=\arabic*, itemsep=0pt, parsep=0pt, font=\footnotesize\bfseries]

			\item  Prove that there are infinitely many integers $ n $ for which all the lamps will eventually be off.
			\item  Prove that there are infinitely many integers $ n $ for which the lamps will never be all off
	\end{enumerate}}\label{problem:induction_type1_4}






	\prob{https://artofproblemsolving.com/community/c6h155696p874991}{ISL 2006 C4}{M}{A cake has the form of an $ n \times n $ square composed of $ n^2 $ unit squares. Strawberries lie on some of the unit squares so that each row or column contains exactly one strawberry; call this arrangement $ \mathbb{A} $.

		Let $ \mathbb{B} $ be another such arrangement. Suppose that every grid rectangle with one vertex at the top left corner of the cake contains no fewer strawberries of arrangement $ \mathbb{B} $ than of arrangement $ \mathbb{A} $. Prove that arrangement $ \mathbb{B} $ can be obtained from $ \mathbb{A} $ by performing a number of switches, defined as follows:

		A switch consists in selecting a grid rectangle with only two strawberries, situated at its top right corner and bottom left corner, and moving these two strawberries to the other two corners of that rectangle.}\label{problem:extreme_object_6}

	\solu{When the first approach fails, don't throw that idea yet. Stick to it, as it is most probably the closest to a correct solution. Taking the smallest rectangle with $ 0 $ 's equal to $ 1 $ 's, we see that we can 'shrink' the rectangle. Which leads to a solution instantly.}




	\prob{https://artofproblemsolving.com/community/c6h1113184p5083546}{ISL 2014 C2}{E}{We have $ 2^m $ sheets of paper, with the number $ 1 $ written on each of them. We perform the following operation. In every step we choose two distinct sheets; if the numbers on the two sheets are $ a $ and $ b $ , then we erase these numbers and write the number $ a + b $ on both sheets. Prove that after $ m2^{m -1} $ steps, the sum of the numbers on all the sheets is at least $ 4^m $.}\label{problem:invariant_rules_of_thumb_1}

	\solu{When you know that the problem can be solved using invariants, go through all of the possible invariants (from \href{invariant_rules_of_thumb}{the rules of thumb}). Don't give up on one so quickly. And product and sum are actually more close than you think. Because if you are told to prove some bound on the sum, then product can come very handy. After all there is AM-GM to connect sum and product.}






	\prob{https://artofproblemsolving.com/community/c6h1480694p8639260}{ISL 2016 C3}{E}{Let $ n $ be a positive integer relatively prime to $ 6 $. We paint the vertices's of a regular $ n $ -gon with three colours so that there is an odd number of vertices's of each colour. Show that there exists an isosceles triangle whose three vertices's are of different colours.}\label{problem:double_counting_1}

	\solu{Double Count with the number of points of each colors.}






	\prob{https://artofproblemsolving.com/community/c6h5052p15988}{Iran TST 2002 P3}{E}{A ``2-line'' is the area between two parallel lines. Length of ``2-line'' is distance of two parallel lines. We have covered unit circle with some ``2-lines''. Prove sum of lengths of ``2-lines'' is at least $ 2 $.}\label{problem:extreme_object_4}

	\solu{Consider the ``2-line'' of the largest length.}





	\prob{https://artofproblemsolving.com/community/c6h209803p1155601}{ARO 2008 P9.5}{E}{The distance between two cells of an infinite chessboard is defined as the minimum number to moves needed for a king to move from one to the other. On the board are chosen three cells on pairwise distances equal to $ 100 $. How many cells are there that are at the distance $ 50 $ from each of the three cells?}\label{problem:forget_and_focus_3}





	\prob{https://artofproblemsolving.com/community/c6h424768p2403861}{USAMO 1986 P2}{E}{During a certain lecture, each of five mathematicians fell asleep exactly twice. For each pair of mathematicians, there was some moment when both were asleep simultaneously. Prove that, at some moment, three of them were sleeping simultaneously.}\label{problem:graph_representation_1}





	\prob{https://artofproblemsolving.com/community/q2h607881p3617126}{Mexican Regional 2014 P6}{E}{Let $ A=n\times n $ be a $ \{0, 1\} $ matrix, where each row is different. Prove that you can remove a column such that the resulting $ n\times (n-1) $ matrix has $ n $ different rows.}\label{problem:induction_type2_2}\label{problem:graph_representation_2}

	\solu{Try to represent the sets in a nicer way, with graph. or. Induction on the number of columns deleted and the number or different rows being there.}





	\prob{https://artofproblemsolving.com/community/c6h1480685p8639240}{IMO 2017 P5}{M (H)}{An integer $ N \ge 2 $ is given. A collection of $ N(N + 1) $ soccer players, no two of whom are of the same height, stand in a row. Show that Sir Alex can always remove $ N(N - 1) $ players from this row leaving a new row of $ 2N $ players in which the following $ N $ conditions hold:

		( $ 1 $ ) no one stands between the two tallest players,

		( $ 2 $ ) no one stands between the third and fourth tallest players,

		$ \;\;\vdots $

		( $ N $ ) no one stands between the two shortest players.} \label{problem:matrix_creation_1}

	\solu{ $ N(N+1) $ , rows, removing $ \dots $ these things just begs for to be arranged in a \hrf{matrix_creation}{systematic} order. As arranging thing in a matrix is the simplest way, we arrange the bad-bois in a $ N \cdot (N+1) $ matrix. Now finding the algorithm is not very hard.}





	\prob{https://artofproblemsolving.com/community/c6h60737p366461}{ISL 1990 P3}{E}{Let $ n \geq 3 $ and consider a set $ E $ of $ 2n - 1 $ distinct points on a circle. Suppose that exactly $ k $ of these points are to be colored black. Such a coloring is good if there is at least one pair of black points such that the interior of one of the arcs between them contains exactly $ n $ points from $ E $. Find the smallest value of $ k $ so that every such coloring of $ k $ points of $ E $ is good.}\label{problem:alternating_chains_1}

	\solu{Creating a graph and using \hrf{alternating_chains}{Alternating Chains Technique}}





	\prob{https://artofproblemsolving.com/community/c6h54501p340035}{USAMO 1999 P1}{E}{Some checkers placed on an $ n \times n $ checkerboard satisfy the following conditions:

		\begin{enumerate}[wide=0em, label=\arabic*, itemsep=0pt, parsep=0pt, font=\footnotesize\bfseries]

			\item  every square that does not contain a checker shares a side with one that does;

			\item  given any pair of squares that contain checkers, there is a sequence of squares containing checkers, starting and ending with the given squares, such that every two consecutive squares of the sequence share a side.
		\end{enumerate}

		Prove that at least $ (n^{2}-2)/3 $ checkers have been placed on the board.}\label{problem:add_time_4}


	\solu{As the problem simply seems to exist, we can't count how much contribution a checker cntaining square contributes to the whole board. So we place \hrf{add_time}{one at a time} and see the changes.}


	\gene{www.hehe.com}{USAMO 1999 P1 generalization}{Find the smallest positive integer $ m $ such that if $ m $ squares of an $ n\times n $ board are colored, then there will exist $ 3 $ colored squares whose centers form a right triangle with sides parallel to the edges of the board.}\label{problem:induction_type1_22}







	\prob{https://artofproblemsolving.com/community/c6h597127p3543383}{ISL 2013 C1}{E}{Let $ n $ be an positive integer. Find the smallest integer $ k $ with the following property; Given any real numbers $ a_1 , \cdots , a_d $ such that $ a_1 + a_2 + \cdots + a_d = n $ and $ 0 \le a_i \le 1 $ for $ i=1,2,\cdots ,d $ , it is possible to partition these numbers into $ k $ groups (some of which may be empty) such that the sum of the numbers in each group is at most $ 1 $.}\label{problem:extremal_case_whole_1}

	\solu{Think about the worst case where $ d $ is the minimum and the ans is $ d $ , it would only be possible if each $ a_i> \frac{1}{2} $ but this can't be true, so, the ans is $ 2n-1 $. Now the ques should become obvious.}





	\prob{https://artofproblemsolving.com/community/c6h587946p3480604}{Brazilian Olympic Revenge 2014}{M}{Let $ n $ a positive integer. In a $ 2n\times 2n $ board, $ 1\times n $ and $ n\times 1 $ pieces are arranged without overlap. Call an arrangement maximal if it is impossible to put a new piece in the board without overlapping the previous ones. Find the least $ k $ such that there is a maximal arrangement that uses $ k $ pieces.}\label{problem:add_time_2}\label{problem:extremal_case_whole_2}

	\solu{Intuition gives that there is at least one $ n $ -mino in each row. But we can easily guess that there is no maximal arrangement with $ 2n $ minos. Suppose in a maximal arrangement, there are no vertical $ n $ -mino, that means there are more than $ 2n+1 $ n-minos. So suppose that there is at least one vertical suppose that it lies in a column $ i $ between $ 1 $ and $ n $. Then we have that there is at least one $ n $ -mino in each column in between $ 1 $ and $ i $. If there is one in between $ 1 $ and $ 2n $ , say $ j $ , then there is one in each of the columns on the right side of it. Then we count horizontal $ n $ -minos, we show that $ 2n+1 $ is the answer.}





	\prob{https://artofproblemsolving.com/community/c6h287859p1555905}{ISL 2008 C1}{E}{In the plane we consider rectangles whose sides are parallel to the coordinate axes and have positive length. Such a rectangle will be called a box. Two boxes intersect if they have a common point in their interior or on their boundary. Find the largest $ n $ for which there exist $ n $ boxes $ B_1, B_2\dots B_n $ such that $ B_i $ and $ B_j $ intersect if and only if $ i \not\equiv j\pm 1\ (\bmod\ n) $.}\label{problem:add_time_3}\label{problem:extreme_object_2}

	\solu{Instead of focusing on building the boxes from only one side (i.e. starting with $ 1, 2\dots $ , we should include $ n $ in our investigation, and follow from both direction, (i.e. $ 1, 2\dots $ and $ \dots , n-1, n $ ).}







	\prob{https://artofproblemsolving.com/community/c5h202905p1116177}{USAMO 2008 P4}{E}{Let $ \mathcal{P} $ be a convex polygon with $ n $ sides, $ n\ge3 $. Any set of $ n - 3 $ diagonals of $ \mathcal{P} $ that do not intersect in the interior of the polygon determine a triangulation of $ \mathcal{P} $ into $ n - 2 $ triangles. If $ \mathcal{P} $ is regular and there is a triangulation of $ \mathcal{P} $ consisting of only isosceles triangles, find all the possible values of $ n $.}

	\solu{It’s not hard after getting the ans.}\label{problem:extreme_object_1}





	\prob{https://artofproblemsolving.com/community/c6h1238107p6307111}{ARO 2016 P3}{M}{We have a sheet of paper, divided into $ 100\times 100 $ unit squares. In some squares, we put right-angled isosceles triangles with $ leg=1 $ (Every triangle lies in one unit square and is half of this square). Every unit grid segment (boundary too) is under one $ leg $ of a triangle. Find maximal number of unit squares, that don't contains any triangles.}

	\solu{What is the minimum number of triangles you can use in a row? Create a good row \hrf{add_time}{one at a time}}\label{problem:add_time_1}





	\prob{https://artofproblemsolving.com/community/c6h546367p3162058}{India TST 2013 Test 3, P1}{E}{For a positive integer $ n $ , a \textit{Sum-Friendly Odd Partition} of $ n $ is a sequence $ \left( a_1, a_2\dots a_k\right) $ of odd positive integers with $ a_1\leq a_2\leq\dots\leq a_k $ and $ a_1+a_2+\dots +a_k = n $ such that for all positive integers $ m\leq n $ , $ m $ can be uniquely written as a subsum $ m = a_{i_1}+a_{i_2}+\dots +a_{i_r} $. (Two subsums $ a_{i_1}+a_{i_2}+\dots +a_{i_r} $ and $ a_{j_1}+a_{j_2}+\dots +a_{j_s} $ with $ i_1< i_2<\dots < i_r $ and $ j_1< j_2 <\dots < j_s $ are considered the same if $ r = s $ and $ a_{i_l}=a_{j_l} $ for $ 1\leq l\leq r $.) For example, $ \left( 1,1,3,3\right) $ is a \textit{sum-friendly odd partition} of $ 8 $. Find the number of sum-friendly odd partitions of $ 9999 $.}\label{problem:recursive_solution_1}

	\solu{Firstly we explore one SFOP \hrf{recursive_solution}{at a time}. Which gives us a way to tell what $ a_{i+1} $ is going to be by looking at $ a_1\dots a_i $.}






	\prob{https://artofproblemsolving.com/community/c6h418796p2363537}{IMO 2011 P2}{H}{Let $ \mathcal{S} $ be a finite set of at least two points in the plane. Assume that no three points of $ \mathcal S $ are collinear. A windmill is a process that starts with a line $ \ell $ going through a single point $ P \in \mathcal S $. The line rotates clockwise about the pivot $ P $ until the first time that the line meets some other point belonging to $ \mathcal S $. This point, $ Q $ , takes over as the new pivot, and the line now rotates clockwise about $ Q $ , until it next meets a point of $ \mathcal S $. This process continues indefinitely.

		Show that we can choose a point $ P $ in $ \mathcal S $ and a line $ \ell $ going through $ P $ such that the resulting windmill uses each point of $ \mathcal S $ as a pivot infinitely many times.}\label{problem:extreme_object_3}

	\solu{Some workaround gives us the idea that the starting line has to be kinda ``\hrf{extreme_object}{in between}'' the points. Formal words could be: the line should divide the set of points in two sets so that the two sets have equal number of points. Once we take a such line, we see that after every move we get a new line which has similar properties of the first line.}

	\solu{So moral of the story is that if you get some vague idea that something has to satisfy something-ish, remove the -ish part, and try with a formal assumption.}






	\prob{http://ioinformatics.org/locations/ioi16/contest/day2/messy.pdf}{IOI 2016 P5}{M}{A computer bug has a permutation $ P $ of length $ 2^k = N $ that changes any string added to a DS according to the permutation, i.e. it makes $ S[i]=S[P[i]] $. Your task it to find the permutation in the following ways:

		\begin{enumerate}[wide=0em, label=\arabic*, itemsep=0pt, parsep=0pt, font=\footnotesize\bfseries]

			\item You can add at most $ n\log_{2}n $ $ N $ bit binary strings to the DS.
			\item You can ask at most $ n\log_{2}n $, in the form of $ N $ bit binary strings. The answer will be ``true'' if the string exists in the DS after the Bug had changed the strings and ``no'' otherwise.
	\end{enumerate}}\label{problem:divide_and_conquer_5}

	\solu{Typical Divide and Conquer approach. You want to do the same thing for $ N = \frac{N}{2} $, and to do so you need to tell exactly what the first $ \frac{N}{2} $ terms of the permutation are. To do this, you can use at most $ N $ questions. This is easy, you first add strings with only one bit present in the first $ \frac{N}{2} $ positions, and then ask $ N $ questions with only one bit in every $ N $ positions. This maps the first $ \frac{N}{2} $ numbers of the permutation to a set of $ \frac{N}{2} $ integers. And we can proceed by induction now.}




	\prob{https://artofproblemsolving.com/community/c6h17458p119184}{ISL 2001 C6}{M}{For a positive integer $n$ define a sequence of zeros and ones to be balanced if it contains $n$ zeros and $n$ ones. Two balanced sequences $a$ and $b$ are neighbors if you can move one of the $2n$ symbols of $a$ to another position to form $b$. For instance, when $n = 4$, the balanced sequences $01101001$ and $00110101$ are neighbors because the third (or fourth) zero in the first sequence can be moved to the first or second position to form the second sequence. Prove that there is a set $S$ of at most $\frac{1}{n+1} \binom{2n}{n}$ balanced sequences such that every balanced sequence is equal to or is a neighbor of at least one sequence in $S$.}\label{problem:forget_and_focus_4}



	\prob{https://artofproblemsolving.com/community/c6h18502p124456}{ISL 1998 C4}{M}{Let $U=\{1,2,\ldots ,n\}$, where $n\geq 3$. A subset $S$ of $U$ is said to be split by an arrangement of the elements of $U$ if an element not in $S$ occurs in the arrangement somewhere between two elements of $S$. For example, 13542 splits $\{1,2,3\}$ but not $\{3,4,5\}$. Prove that for any $n-2$ subsets of $U$, each containing at least 2 and at most $n-1$ elements, there is an arrangement of the elements of $U$ which splits all of them.}\label{problem:induction_type1_21}\label{problem:extreme_object_11}

	\solu{If we try to apply induction, we see that the sets with $ 2 $ and $ n-1 $ elements create problems, so we handle them first.}




	\prob{https://artofproblemsolving.com/community/c6h289581p1566044}{USA TST 2009 P1}{M}{Let $m$ and $n$ be positive integers. Mr. Fat has a set $S$ containing every rectangular tile with integer side lengths and area of a power of $2$. Mr. Fat also has a rectangle $R$ with dimensions $2^m \times 2^n$ and a $1 \times 1$ square removed from one of the corners. Mr. Fat wants to choose $m + n$ rectangles from $S$, with respective areas $2^0, 2^1, \ldots, 2^{m + n - 1}$, and then tile $R$ with the chosen rectangles. Prove that this can be done in at most $(m + n)!$ ways.}\label{problem:bijection_10}\label{problem:extreme_object_12}

	\solu{The fact that this can be done in $ (m+n)! $ asks for a bijective proof. Now an intuition gives us that we have to sort the tiles wrt the missing square in some way. Now since the numbers }



	\prob{https://artofproblemsolving.com/community/c6h1238105p6307095}{ARO 2016 P1}{E}{There are $ 30 $ teams in \textbf{NBA} and every team play $ 82 $ games in the year. Bosses of \textbf{NBA} want to divide all teams on Western and Eastern Conferences (not necessarily equally), such that the number of games between teams from different conferences is half of the number of all games. Can they do it?}

	\solu{You want to divide something. Check the parity.}\label{problem:invariant_rules_of_thumb_7}




	\prob{https://artofproblemsolving.com/community/c6h1288343p6805414}{AoPS}{M}{Each edge of a polyhedron is oriented with an arrow such that at each vertex, there is at least one arrow leaving the vertex and at least one arrow entering the vertex. Prove that there exists a face on the polyhedron such that the edges on its boundary form a directed cycle.}\label{problem:extreme_object_13}

	\solu{The trick which is used to prove Euler's Polyhedron theorem.}




	\prob{https://artofproblemsolving.com/community/c6h596929p3542094}{ISL 2014 C3}{M}{Let $ n \ge 2 $ be an integer. Consider an $ n \times n $ chessboard consisting of $ n^2 $ unit squares. A configuration of $ n $ rooks on this board is peaceful if every row and every column contains exactly one rook. Find the greatest positive integer $ k $ such that, for each peaceful configuration of $ n $ rooks, there is a $ k \times k $ square which does not contain a rook on any of its $ k^2 $ unit squares.}\label{problem:extremal_case_whole_7}

	\solu{Guessing the "Correct" ans is the challenge, think of the worst case you can produce.}




	\prob{https://artofproblemsolving.com/community/c6h472958p2648127}{APMO 2012 P2}{E}{Into each box of a $ n \times n $ square grid, a real number greater than or equal to $ 0 $ and less than or equal to $ 1 $ is inserted. Consider splitting the grid into $2$ non-empty rectangles consisting of boxes of the grid by drawing a line parallel either to the horizontal or the vertical side of the grid. Suppose that for at least one of the resulting rectangles the sum of the numbers in the boxes within the rectangle is less than or equal to $ 1 $, no matter how the grid is split into $2$ such rectangles. Determine the maximum possible value for the sum of all the $ n \times n $ numbers inserted into the boxes. Find the ans for $ k $-dimension grids too.}\label{problem:extreme_object_14}


	\solu{As the maximal rectangle defines other smaller rectangles in it, we take that.}




	\prob{www.hehe.com}{Indian Postal Coaching 2011}{M}{Consider $ 2011^2 $ points arranged in the form of a $ 2011 \times 2011 $ grid. What is the maximum number of points that can be chosen among them so that no four of them form the vertices's of either an isosceles trapezium or a rectangle whose parallel sides are parallel to the grid lines?}\label{problem:forget_and_focus_6}

	\solu{Since we need to maintain the relation of perpendicular bisectors, we focus on perp bisectors and the points on one line only and then count.}



	\prob{https://artofproblemsolving.com/community/c6h418686p2362296}{ISL 2010 C2}{M}{On some planet, there are $2^N$ countries $(N \geq 4).$ Each country has a flag $N$ units wide and one unit high composed of $N$ fields of size $1 \times 1,$ each field being either yellow or blue. No two countries have the same flag. We say that a set of $N$ flags is diverse if these flags can be arranged into an $N \times N$ square so that all $N$ fields on its main diagonal will have the same color. Determine the smallest positive integer $M$ such that among any $M$ distinct flags, there exist $N$ flags forming a diverse set.}\label{problem:induction_type1_6}


	\solu{Using induction we see that if we have found the value of $ M $ for $ N-1 $, then possibly the value for $ M_N $ is twice as large than $ M_{N-1} $. With some further calculation, we see that if we have $ 2*M_{N-1}-1 = M_N $, then we can pick half of them and apply induction and still be left with a `lot' of flags to choose the $ N $th element of the diverse set.\\\\ After that the only work left is to proof for $ N=4 $. Which is easy casework.}

	\solu{Another way to prove the ans, is to prove the bound for any non-diverse set. In this case, we use hall's marriage to prove the contradiction.}



	\prob{https://artofproblemsolving.com/community/c6h147501p834073}{Iran TST 2007 P2}{E}{Let $A$ be the largest subset of $\{1,\dots,n\}$ such that for each $x\in A$, $x$ divides at most one other element in $A$. Prove that \[\frac{2n}3\leq |A|\leq \left\lceil \frac{3n}4\right\rceil. \]}\label{problem:divide_and_conquer_6}

	\solu{Partition the set optimally.}






	\prob{https://artofproblemsolving.com/community/c6h1485051p8702069}{India IMO Camp 2017}{H}{Find all positive integers $ n $ s.t. the set $ \{1, 2, \dots, 3n\} $ can be partitioned into $ n $ triplets $ (a_i, b_i, c_i) $ such that $ a_i+b_i=c_i $ for all $ 1 \le i \le n $.}\label{problem:constructive_algo_11}




	\prob{https://artofproblemsolving.com/community/c6h546169p3160560}{ISL 2012 C2}{TE}{Let $ n \geq 1 $ be an integer. What is the maximum number of disjoint pairs of elements of the set $ \{ 1,2,\ldots , n \} $ such that the sums of the different pairs are different integers not exceeding $ n $ ?}\label{problem:constructive_algo_12}

	\solu{As Usual, first find the ans. Using double counting is quite natural. Working with small cases easily gives a construction.}




	\prob{http://codeforces.com/problemset/problem/989/C}{CodeForces 989C}{E}{}\label{problem:constructive_algo_13}




	\prob{http://codeforces.com/problemset/problem/989/B}{CodeForces 989B}{E}{}\label{problem:constructive_algo_14}





	\prob{https://artofproblemsolving.com/community/c6h488537p2737645}{ISL 2011 A5}{MH}{Prove that for every positive integer $ n $ , the set $ \{2, 3, \ldots, 3n+1\} $ can be partitioned into $ n $ triples in such a way that the numbers from each triple are the lengths of the sides of some obtuse triangle.}\label{problem:constructive_algo_15}

	\solu{What is the best way to choose the side lengths of an obtuse triangle? Obviously by maintaining some strict rules to get the third side from the first two sides and making the rules invariant. One way of doing this is to take $ (a,b,a+b-1) $.

		After that, some (literally this is the hardest part of the problem) experiment to find a construction. First, we try to partition the set into tuples of our desired form, but we soon realize that that can’t be done so easily. So we try a little bit of different approach and make one tuple different from the others. Luckily this approach gives us a nice construction.}





	\prob{https://artofproblemsolving.com/community/c6h1423008p8003911}{Iran TST 2017 D1P1}{TE}{In the country of Sugarland, there are $ 13 $ students in the IMO team selection camp. $ 6 $ team selection tests were taken and the results have came out. Assume that no students have the same score on the same test. To select the IMO team, the national committee of math Olympiad have decided to choose a permutation of these $ 6 $ tests and starting from the first test, the person with the highest score between the remaining students will become a member of the team. The committee is having a session to choose the permutation.

		Is it possible that all $ 13 $ students have a chance of being a team member?}\label{problem:constructive_algo_16}

	\solu{If a student is in $ x^{th} $ place in a test $ t_y $ , and he has a chance to get into the team iff the $ 1^th, 2^th\dots {x-1}^th $ persons in test $ t_y $ are already in the team. So $ x\leq 5 $. Make a $ 6\cdot 6 $ grid with place $ \cdot $ test. WHY?? Because it makes the best sense among other possible choices of the grid. A little bit of work produces a configuration where every student has a chance to get into the team.}





	\prob{https://artofproblemsolving.com/community/c6h355784p1932924}{ISL 2009 C2}{M}{For any integer $ n\geq 2 $ , let $ N(n) $ be the maximum number of triples $ (a_i, b_i, c_i) $ , $ i=1, 2 \ldots, N(n) $ , consisting of nonnegative integers $ a_i $ , $ b_i $ and $ c_i $ such that the following two conditions are satisfied:

		\begin{enumerate}[wide=0em, label=\arabic*, itemsep=0pt, parsep=0pt, font=\footnotesize\bfseries]


			\item $ a_i+b_i+c_i=n $ for all $ i=1, \ldots, N(n) $ ,
			\item If $ i\neq j $ then $ a_i\neq a_j $ , $ b_i\neq b_j $ and $ c_i\neq c_j $

		\end{enumerate}

		Determine $ N(n) $ for all $ n\geq 2 $.}\label{problem:constructive_algo_17}

	\solu{Find an upper bound. It’s easy. Then with some experiment, we see that this upper bound is achievable. So our next task is to find a construction. As it is related to $ 3 $ , we first try with $ n=3k $. Some experiment and experience gives us a construction.}





	\prob{}{}{M}{Let $ n $ be an integer. What is the maximum number of disjoint pairs of elements of the set $ \{ 1,2,\ldots , n \} $ such that the sums of the different pairs are different integers not exceeding $ n $ ?} \label{problem:constructive_algo_18}




	\prob{https://artofproblemsolving.com/community/c6h17342p118721}{ISL 2002 C6}{H}{Let n be an even positive integer. Show that there is a permutation $ (x_1,x_2 \dots x_n) $ of $ (1,2\dots n) $ such that for every $ 1\leq i\leq n $ , the number $ x_{i+1} $ is one of the numbers $ 2x_i,2x_i-1,2x_i-n,2x_i-n-1 $. Hereby, we use the cyclic subscript convention, so that $ x_{n+1} $ means $ x_1 $.}\label{problem:graph_representation_5}

	\medskip Some experiments show that our graph has more than $ 2 $ incoming and outgoing degree in all vertexes expect the first and last vertexes. So our lemma won’t work yet. To make use of our lemma we take a graph with half of the vertexes of our original graph and make each vertex $ v_{2k} $ represent two integers: $ (2k-1, 2k) $. Simple argument shows that this graph has an Euler Circuit, and surprisingly this itself is sufficient, as we can follow this circuit to get every integers in the interval $ [1,n] $.





	\prob{https://artofproblemsolving.com/community/c6h1352165p7389115}{USA TST 2017 P1}{E}{In a sports league, each team uses a set of at most $ t $ signature colors. A set $ S $ of teams is color-identifiable if one can assign each team in $ S $ one of their signature colors, such that no team in $ S $ is assigned any signature color of a different team in $ S $.\\

		For all positive integers $ n $ and $ t $, determine the maximum integer $ g(n, t) $ such that: In any sports league with exactly $ n $ distinct colors present over all teams, one can always find a color-identifiable set of size at least $ g(n, t) $.}\label{problem:extremal_case_whole_8}

	\solu{First, guess the answer, then try taking the minimal set.}





	\prob{https://artofproblemsolving.com/community/c7h1554574p9472772}{Putnam 2017 A4}{E}{$ 2N $ students take a quiz in which the possible scores are $ 0, 1\dots 10 $. It is given that each of these scores appeared at least once, and the average of their scores is $ 7.4 $. Prove that the students can be divided into two sets of $ N $ student with both sets having an average score of $ 7.4 $.}\label{problem:constructive_algo_19}

	\solu{We take a set $ S_1=\{0, 1\dots 10\} $. Basically we have to partition the set of $ 2N $ into two equal sets with equal sum. So we pair $ S $ , and other leftovers and see what happens.}



	\prob{https://artofproblemsolving.com/community/c6h126193p715430}{ISL 2005 C3}{MH}{Consider a $m\times n$ rectangular board consisting of $mn$ unit squares. Two of its unit squares are called adjacent if they have a common edge, and a path is a sequence of unit squares in which any two consecutive squares are adjacent. Two paths are called non-intersecting if they don't share any common squares.\\

		Each unit square of the rectangular board can be colored black or white. We speak of a coloring of the board if all its $mn$ unit squares are colored.\\

		Let $N$ be the number of colorings of the board such that there exists at least one black path from the left edge of the board to its right edge. Let $M$ be the number of colorings of the board for which there exist at least two non-intersecting black paths from the left edge of the board to its right edge.\\

		Prove that $N^{2}\geq M\times 2^{mn}$.}\label{problem:bijection_11}

	\solu{Bijective relation problem, the condition has $ \times $, means we find a combinatorial model for the R.H.S. which is a pair of boards satisfying conditions. We want to show a surjection from this model to the model on the L.H.S.}



	\prob{www.hehe.com}{Result by Erdos}{MH}{Given two \emph{different} sequence of integers $ (a_1, a_2\dots a_n), (b_1, b_2, \dots b_n) $ such that two $ \frac{n(n-1)}{2} $-tuples \[ a_1+a_2, a_1+a_3\dots a_{n-1}a_n\ \text{ and }\ b_1+b_2, b_1+b_3\dots b_{n-1}b_n \] are equal upto permutation. Prove that $ n=2^k $ for some $ k $.}\label{problem:generating_function_1}\label{algebraic_manipulation}


	\prob{www.hehe.com}{A reformulation of Catalan's Numbers}{MH}{Let $ n\geq 3 $ students all have different heights. In how many ways can they be arranged such that the heights of any three of them are not from left to right in the order: medium, tall, short?}\label{problem:catalan_recursion_1}\label{problem:generating_function_3}

	\solu{The proof uses derivatives to construct a polynomial similar to a \textbf{Maclaurin Series}.}


	\prob{}{}{E}{There are $n$ cubic polynomials with three distinct real roots each. Call them $P_1(x), P_2(x),\dots, P_n(x)$. Furthermore for any two polynomials $P_i, P_j$, $P_i(x)P_j(x)=0$ has exactly $5$ distinct real roots. Let $S$ be the set of roots of the equation \[P_1(x)P_2(x)\dots P_n(x)=0\]. Prove that

		\begin{enumerate}[wide=0em, label=\arabic*, itemsep=0pt, parsep=0pt, font=\footnotesize\bfseries]


			\item If for each $a, b$ there is exactly one $i \in \{1, \dots n\}$ such that $P_i(a)=P_i(b)=0$, then $n=7$.
			\item If $n>7$, $|S| = 2n+1$.

	\end{enumerate}}\label{problem:extremal_case_whole_9}



	\prob{https://artofproblemsolving.com/community/c6h1450192p8312110}{Serbia TST 2017 P2}{E}{Initially a pair $(x, y)$ is written on the board, such that exactly one of it's coordinates is odd. On such a pair we perform an operation to get pair $(\frac x 2, y+\frac x 2)$ if $2|x$ and $(x+\frac y 2, \frac y 2)$ if $2|y$. Prove that for every odd $n>1$ there is a even positive integer $b<n$ such that starting from the pair $(n, b)$ we will get the pair $(b, n)$ after finitely many operations.}\label{problem:invariant_rules_of_thumb_8}

	\solu{Finding a construction through investigation and realizing that the infos and operations on $ x $ only defines the changes are enough for this problem.}



	\prob{https://artofproblemsolving.com/community/c6h1450624p8316894}{Serbia TST 2017 P4}{E}{We have an $n \times n$ square divided into unit squares. Each side of unit square is called unit segment. Some isosceles right triangles of hypotenuse $2$ are put on the square so all their vertices's are also vertices's of unit squares. For which $n$ it is possible that every unit segment belongs to exactly one triangle (unit segment belongs to a triangle even if it's on the border of the triangle)?}\label{problem:constructive_algo_20}

	\solu{Finding $ n $ is even, seeing $ 4 $ fails...}




	\prob{https://artofproblemsolving.com/community/c6h1545675p9374373}{China MO 2018 P2}{M}{Let $n$ and $k$ be positive integers and let
		\[T = \{ (x,y,z) \in \mathbb{N}^3 \mid 1 \leq x,y,z \leq n \}\]
		be the length $n$ lattice cube. Suppose that $3n^2 - 3n + 1 + k$ points of $T$ are colored red such that if $P$ and $Q$ are red points and $PQ$ is parallel to one of the coordinate axes, then the whole line segment $PQ$ consists of only red points.\\

		Prove that there exists at least $k$ unit cubes of length $1$, all of whose vertices's are colored red.}\label{problem:double_counting_8}


	\solu{The inductive solution is tedious, and since we have to count the number of ``good'' boxes, we can try double counting. Explicitly counting all the ``good'' boxes.}



	\prob{https://artofproblemsolving.com/community/c6h1546234p9380562}{China MO 2018 P5}{MH}{Let $n \geq 3$ be an odd number and suppose that each square in a $n \times n$ chessboard is colored either black or white. Two squares are considered adjacent if they are of the same color and share a common vertex and two squares $a,b$ are considered connected if there exists a sequence of squares $c_1,\ldots,c_k$ with $c_1 = a, c_k = b$ such that $c_i, c_{i+1}$ are adjacent for $i=1,2,\ldots,k-1$.\\

		Find the maximal number $M$ such that there exists a coloring admitting $M$ pairwise disconnected squares.}

	\solu{It's not hard to get the ans, now that the answer is guesses, and we have tried to prove with induction and couldn't find anything good, we try double counting. We notice that all the connected components in the $ n\times n $ are planar graphs. Now we use Euler's \hrf{theorem:planar_graph_theorem}{theorem} on Planar Graphs to find a value of $ M $ wrt to other values, and we double count the other values.}




	\prob{https://artofproblemsolving.com/community/c6h84550p490581}{USAMO 2006 P2}{E}{For a given positive integer $k$ find, in terms of $k$, the minimum value of $N$ for which there is a set of $2k + 1$ distinct positive integers that has sum greater than $N$ but every subset of size $k$ has sum at most $\tfrac{N}{2}.$}\label{problem:extremal_case_whole_10}

	\solu{The best or simple looking set is the set of consecutive integers. So if there are some `holes', we can fill them up to some extent, this opens two sub-cases.}



	\prob{https://artofproblemsolving.com/community/c6h34314p213007}{USAMO 2005 P1}{E}{Determine all composite positive integers $n$ for which it is possible to arrange all divisors of $n$ that are greater than $ 1 $ in a circle so that no two adjacent divisors are relatively prime.}\label{problem:induction_type1_23}



	\prob{https://artofproblemsolving.com/community/c6h84558p490682}{USAMO 2005 P5}{E}{A mathematical frog jumps along the number line. The frog starts at $1$, and jumps according to the following rule: if the frog is at integer $n$, then it can jump either to $n+1$ or to $n + 2^{m_n+1}$ where $2^{m_n}$ is the largest power of $2$ that is a factor of $n$. Show that if $k \geq 2$ is a positive integer and $i$ is a nonnegative integer, then the minimum number of jumps needed to reach $2^ik$ is greater than the minimum number of jumps needed to reach $2^i.$}\label{problem:induction_type1_24}

	\solu{The main idea is to notice that the operation only uses powers of $ 2 $. And it depends on only the power of $ 2 $ in the integers, and in the sequence of $ 2 $-powers, the operation is very nice.}



	\prob{https://artofproblemsolving.com/community/c6h60727p366446}{ISL 1991 P10}{E}{Suppose $ \,G\,$ is a connected graph with $ \,k\,$ edges. Prove that it is possible to label the edges $ 1,2,\ldots ,k\,$ in such a way that at each vertex which belongs to two or more edges, the greatest common divisor of the integers labeling those edges is equal to $ 1 $.}\label{problem:induction_type1_27}



	\prob{}{}{E}{A robot has $ n $ modes, and programmed as such: in mode $ i $ the robot will go at a speed of $ i \text{ms}^{-1} $ for $ i $ seconds. At the beginning of its journey, you have to give it a permutation of $ \{1, 2, \dots n \} $. What is the maximum distance you can make the robot go?}\label{problem:swapping_5}



	\prob{}{}{E}{A slight variation of the previous problem, in this case, the problem goes at a speed of $ (n-1) \text{ms}^{-1} $ for $ i $ seconds in mode $ i $.}\label{problem:swapping_6}




	\prob{}{}{E}{$ m $ people each ordered $ n $ books but because Ittihad was the mailman, he messed up. Everyone got $ n $ books but not necessarily the one they wanted you need to fix this. To go to a house from another house it takes one hour. You can carry one book with you during any trip (at most one). You know who has which books and all books are different (i,e, $ n * m $ different books). Prove that you can always finish the job in $ m*(n+\frac{1}{2}) $ hours}\label{problem:graph_representation_6}

	\solu{Thinking about the penultimate step, when we have to go to a house empty handed. Thinking in this way gives us a way to pair the houses up, and since pairing...}

	\solu{Another way to do this is to convert it to a multi-graph. Now go to a house and return with a book means removing two edges from that vertex. We play around with it for sometime}




	\prob{}{}{E}{There are $ n $ campers in a camp and they will try to solve a IMO P6 but everyone has a confidence threshold (they will solve the problem by group solving). For example Laxem has threshold $ 5 $. I.e. if he's in the group, the group needs to contain at least $ 5 $ people (him included). A group is `confident' when everyone of the team is confident. Now MM wants to make a list of possible ``perfect confident'' groups. I.e. groups that are confident but adding anyone else will destroy the confidence. How long can his list be?}\label{problem:extreme_object_15}



	\prob{http://acm.timus.ru/problem.aspx?space=1&num=1862}{timus 1862}{ME}{}\label{problem:binary_heap_1}\label{problem:graph_representation_7}




	\prob{https://artofproblemsolving.com/community/c6h589935p3493451}{ARO 2014 P9.7}{E}{In a country, mathematicians chose an $\alpha> 2$ and issued coins in denominations of 1 ruble, as well as $\alpha ^k$ rubles for each positive integer k. $\alpha$ was chosen so that the value of each coins, except the smallest, was irrational. Is it possible that any natural number of rubles can be formed with at most 6 of each denomination of coins?}\label{problem:recursive_solution_6}



	\prob{www.hehe.com}{Saint Petersburg 2001}{MH}{The number $n$ is written on a board. $A$ and $B$ take turns, each turn consisting of replacing the number $n$ on the board with $n - 1$ or $\floor{\frac{n+1}{2}}$. The player who writes the number $1$ wins. Who has the winning strategy?}\label{problem:recursive_solution_7}

	\solu{Recursively building the losing positions.}



	\prob{https://artofproblemsolving.com/community/c6h405391p2262420}{ARO 2011 P11.6}{E}{There are more than $n^2$ stones on the table. Peter and Vasya play a game, Peter starts. Each turn, a player can take any prime number less than $n$ stones, or any multiple of $n$ stones, or $1$ stone. Prove that Peter always can take the last stone (regardless of Vasya's strategy).}\label{problem:pairing_and_copying_1}



	\prob{https://artofproblemsolving.com/community/c6h147159p832730}{ARO 2007 P9.7}{E}{Two players by turns draw diagonals in a regular $(2n+1)$-gon ($n>1$). It is forbidden to draw a diagonal, which was already drawn, or intersects an odd number of already drawn diagonals. The player, who has no legal move, loses. Who has a winning strategy?}\label{problem:graph_representation_8}

	\solu{Turning the diagonals as vertices, and connection being intersections, we get a graph to play the game on. We then count the degrees.}



	\prob{}{}{E}{After tiling a $ 6\times 6 $ box with dominoes, prove that a line parallel to the sides of the box can be drawn that this line doesn't cut any dominoes.}\label{problem:double_counting_9}

	\solu{Double count how many lines ``cut'' a domino, and domino number.}



	\prob{}{}{E}{There are $ 100 $ points on the plane. You have to cover them with discs, so that any two disks are at a distance of $ 1 $. Prove that you can do this in such a way that the total diameter of the disks is $ < 100 $.}\label{problem:induction_type1_28}

	\solu{As the number $ 100 $ is very random, we suspect that is true for all values. So we can use induction}



	\prob{https://artofproblemsolving.com/community/c6h589938p3493455}{ARO 2014 P10.8}{M}{Given are $n$ pairwise intersecting convex $k$-gons on the plane. Any of them can be transferred to any other by a homothety with a positive coefficient. Prove that there is a point in a plane belonging to at least $1 +\frac{n-1}{2k}$ of these $k$-gons.}\label{problem:changing_term_1}

	\solu{The most natural such point should be a vertex of a polygon. And these kinda problems use PHP more often, so we will have to divide by $k$ somewhere. Again to find the polygon to use the PHP we will have to divide by $n$ also. So we want to have $nk$ in the denominator. We change the term to achieve this and Ta-Da! we get a fine term to work with.}



	\prob{https://ioi2018.jp/wp-content/tasks/contest1/combo.pdf}{IOI 2018 P1}{E}{}\label{problem:induction_type1_29}



	\prob{https://artofproblemsolving.com/community/c6h5976p20088}{German TST 2004 E7P3}{M}{We consider graphs with vertices colored black or white. "Switching" a vertex means: coloring it black if it was formerly white, and coloring it white if it was formerly black.\\

		Consider a finite graph with all vertices colored white. Now, we can do the following operation: Switch a vertex and simultaneously switch all of its neighbours (i. e. all vertices connected to this vertex by an edge). Can we, just by performing this operation several times, obtain a graph with all vertices colored black?}\label{problem:induction_type1_30}



	\prob{https://artofproblemsolving.com/community/c6h17455p119177}{ISL 2001 C3}{E}{Define a $ k $-clique to be a set of $ k $ people such that every pair of them are acquainted with each other. At a certain party, every pair of $ 3 $-cliques has at least one person in common, and there are no $ 5 $-cliques. Prove that there are two or fewer people at the party whose departure leaves no $ 3 $-clique remaining.}\label{problem:extreme_object_16}

	\solu{Casework with the point where most of the triangles are joined.}



	--------------------




	



	\prob{https://artofproblemsolving.com/community/c6h34314p213007}{USAMO 2005 P1}{E}{Determine all composite positive integers $n$ for which it is possible to arrange all divisors of $n$ that are greater than 1 in a circle so that no two adjacent divisors are relatively prime.}


	\prob{https://artofproblemsolving.com/community/c6h34317p213012}{USAMO 2005 P4}{E}{Legs $L_1, L_2, L_3, L_4$ of a square table each have length $n$, where $n$ is a positive integer. For how many ordered 4-tuples $(k_1, k_2, k_3, k_4)$ of nonnegative integers can we cut a piece of length $k_i$ from the end of leg $L_i \; (i=1,2,3,4)$ and still have a stable table?

		(The table is stable if it can be placed so that all four of the leg ends touch the floor. Note that a cut leg of length 0 is permitted.)}


	\prob{https://artofproblemsolving.com/community/c6h84550p490581}{USAMO 2006 P2}{M}{For a given positive integer $k$ find, in terms of $k$, the minimum value of $N$ for which there is a set of $2k + 1$ distinct positive integers that has sum greater than $N$ but every subset of size $k$ has sum at most $\tfrac{N}{2}.$}


	\prob{https://artofproblemsolving.com/community/c6h84558p490682}{USAMO 2006 P5}{M}{A mathematical frog jumps along the number line. The frog starts at $1$, and jumps according to the following rule: if the frog is at integer $n$, then it can jump either to $n+1$ or to $n + 2^{m_n+1}$ where $2^{m_n}$ is the largest power of $2$ that is a factor of $n.$ Show that if $k \geq 2$ is a positive integer and $i$ is a nonnegative integer, then the minimum number of jumps needed to reach $2^ik$ is greater than the minimum number of jumps needed to reach $2^i$.}	


	\prob{https://artofproblemsolving.com/community/c5h274370p1485139}{USAMO 2009 P2}{EM}{Let $n$ be a positive integer. Determine the size of the largest subset of $\{ -n, -n+1, \dots, n-1, n\}$ which does not contain three elements $a$, $b$, $c$ (not necessarily distinct) satisfying $a+b+c=0$.}


	\prob{https://artofproblemsolving.com/community/c6h514375p2889828}{ARO 1999 P9.8}{M}{There are $2000$ components in a circuit, every two of which were initially joined by a wire. The hooligans Vasya and Petya cut the wires one after another. Vasya, who starts, cuts one wire on his turn, while Petya cuts one or three. The hooligan who cuts the last wire from some component loses. Who has the winning strategy?}
\newpage\section{Tricks}


	\Faka\Faka\subsection{Bijection}

	{

	Ideas for the bijection function:

	\begin{itemize}

		\item Induction
		\item Forming sets that are not already formed
		\item Building combinatorial models from the investigation of the problem conditions.
		\item Trying to define the later set by the former set.

	\end{itemize}


	}\faka



	\begin{enumerate}[wide=0em, label=\arabic*, itemsep=0pt, parsep=0pt, font=\footnotesize\bfseries]

		\iref{problem:bijection_1}{ISL 2002 C1,}{Red-Blue Under $ x+y < n $ and Bijection}
		\iref{problem:bijection_2}{OC Chap2 P2,}{Magic trick of hiding two digits}
		\iref{problem:bijection_3}{ISL 2009 C3}{}
		\iref{problem:bijection_4}{USAMO 1996 P4,}{Binary Strings NOT containing certain Combinations}
		\iref{problem:bijection_5}{ISL 2008 C4,}{Lamp States and Probability}
		\iref{problem:bijection_6}{APMO 2017 P3,}{Bijection Problem}
		\iref{problem:bijection_7}{USAMO 2013 P2,}{Around the circle on points with move or $ 1 $ or $ 2 $}
		\iref{problem:bijection_8}{APMO 2008 P2}{}
		\iref{problem:bijection_9}{ISL 2006 C2}{}
		\iref{problem:bijection_10}{USA TST 2009 P1}{}
		\iref{problem:bijection_11}{ISL 2005 C3}{}
		\iref{problem:bijection_12}{ISL 2002 C2}{Cover all black squares with L-tromino}
		\iref{problem:bijection_13}{ISL 2002 C3}{Full-Sequences}
	\end{enumerate}








	\Faka\subsubsection{Hall's Marriage Lemma\label{hall_marriage}}

	\begin{enumerate}[wide=0em, label=\arabic*, itemsep=0pt, parsep=0pt, font=\footnotesize\bfseries]

		\iref{problem:hall_marriage_1}{OC Chap2 P2}{}
		\iref{problem:hall_marriage_2}{ARO 2005 P9.4}{}
	\end{enumerate}





	\newpage\subsection{Extremal Principal}




	\Faka\subsubsection{Whole Extremal Cases\label{extremal_case_whole}}{Exploring the extreme case as a whole}

	\begin{multicols}{2}
		\begin{enumerate}[wide=0em, label=\arabic*, itemsep=0pt, parsep=0pt, font=\footnotesize\bfseries]

			\iref{problem:extremal_case_whole_1}{ISL 2013 C1}{}

			\iref{problem:extremal_case_whole_2}{Brazilian Olympic Revenge 2014}{}
			\iref{problem:extremal_case_whole_3}{ISL 2009 C1}{}
			\iref{problem:extremal_case_whole_4}{EGMO 2017 P2}{}
			\iref{problem:extremal_case_whole_5}{APMO 2008 P2}{}
			\iref{problem:extremal_case_whole_6}{Belarus 2001}{}
			\iref{problem:extremal_case_whole_7}{ISL 2014 C3}{}
			\iref{problem:extremal_case_whole_8}{USA TST 2017 P1}{}
			\iref{problem:extremal_case_whole_9}{Polynomials and Roots problem}{}
			\iref{problem:extremal_case_whole_10}{USAMO 2006 P2}{}
			\iref{problem:extremal_case_whole_11}{ISL 2014 N3,}{Cape Town Coin problem}
		\end{enumerate}
	\end{multicols}



	\Faka\subsubsection{Forget and Focus}{Explore only one part of the problem at a time, choose the most crucial part of the problem and focus only on that.}\label{forget_and_focus}

	\begin{multicols}{2}
		\begin{enumerate}[wide=0em, label=\arabic*, itemsep=0pt, parsep=0pt, font=\footnotesize\bfseries]

			\iref{problem:forget_and_focus_1}{ARO 2018 P9.5}{}
			\iref{problem:forget_and_focus_2}{Polish OI}{}
			\iref{problem:forget_and_focus_3}{ARO 2008 P9.5}{}
			\iref{problem:forget_and_focus_4}{ISL 2001 C6}{}
			\iref{problem:forget_and_focus_5}{Swell Coloring}{}
			\iref{problem:forget_and_focus_6}{Indian Postal Coaching 2011}{}
			\iref{problem:forget_and_focus_7}{Romanian TST 2016 D1P2,}{associating $ x_i $ with $ S_i $}

		\end{enumerate}
	\end{multicols}



		\Faka\paragraph{Swapping / Forget and Focus (2)}{Focusing on two neighboring elements in the extremal case.}\label{swapping}

		\begin{multicols}{2}
			\begin{enumerate}[wide=0em, label=\arabic*, itemsep=0pt, parsep=0pt, font=\footnotesize\bfseries]

				\iref{problem:swapping_1}{Polish OI}{}
				\iref{problem:swapping_2}{IOI 2007 P3}{}
				\iref{problem:swapping_3}{Problem}{}
				\iref{problem:swapping_4}{ARO 2014 P9.8}{}
				\iref{problem:swapping_5}{Problem,}{Robot goes at a speed of $ i $ for $ i $ seconds in mode $ i $}
				\iref{problem:swapping_6}{Problem,}{The same with speed $ n-i $}
			\end{enumerate}
		\end{multicols}


		\Faka\paragraph{Game Positions}{Considering Winning/Losing positions and describing the game with these definitions is an important game theory tactic.}\label{win_lose}

		\begin{multicols}{3}
			\begin{enumerate}[wide=0em, label=\arabic*, itemsep=0pt, parsep=0pt, font=\footnotesize\bfseries]

				\iref{problem:win_lose_1}{ISL 2004 C5}{}
			\end{enumerate}
		\end{multicols}





	\Faka\subsubsection{Extreme Objects}{Concentrating on the Extreme object only}\label{extreme_object}

	\begin{enumerate}[wide=0em, label=\arabic*, itemsep=0pt, parsep=0pt, font=\footnotesize\bfseries]

		\iref{problem:extreme_object_1}{USAMO 2008 P4}{}
		\iref{problem:extreme_object_2}{ISL 2008 C1}{}
		\iref{problem:extreme_object_3}{IMO 2011 P2}{}
		\iref{problem:extreme_object_4}{Iran TST 2002 P3}{}
		\iref{problem:extreme_object_5}{Problem}{}
		\iref{problem:extreme_object_6}{ISL 2006 C4}{}
		\iref{problem:extreme_object_7}{ISL 2014 C1}{}
		\iref{problem:extreme_object_8}{ARO 1993 P10.4}{}
		\iref{problem:extreme_object_10}{Sunflower Lemma}{}
		\iref{problem:extreme_object_11}{ISL 1998 C4}{}
		\iref{problem:extreme_object_12}{USA TST 2009 P1}{}
		\iref{problem:extreme_object_13}{AoPS}{}
		\iref{problem:extreme_object_14}{APMO 2012 P2}{}
		\iref{problem:extreme_object_15}{Problem,}{Confidence in solving a P6}
		\iref{problem:extreme_object_16}{ISL 2001 C3,}{$ 3 $-cliques with common points}
		\iref{problem:extreme_object_17}{ISL 2007 C2,}{dissecting a rectangle into $ n $ smaller rectangles, there exists a rectangle inside}


	\end{enumerate}




	\newpage\subsection{Coloring}


		\begin{itemize}

			\item If the nodes are connected in lattice point manner, then \textbf{Checkerboard} coloring is the most natural coloring technique. But if this coloring does not do any good, then there may be other alternatives and derivatives of checkerboard, like \textbf{Pseudo Checkerboard} or \textbf{Double Checkerboard}. The Pseudo Checkerboard's each row (or column) starts and end with the same color (If there are odd nodes in each row). In a Double Checkerboard, two consecutive nodes are of the same color. (You get the picture, don't you?)

			\item Checkerboard with $ \frac{1}{2}\times \frac{1}{2} $ sized cells. Proof of the rectangle with integer side problem.

			\item Color with ``Roots of Unity''.

			\item A knight's move always changes the color of the cell.

		\end{itemize}



	\begin{enumerate}[wide=0em, label=\arabic*, itemsep=0pt, parsep=0pt, font=\footnotesize\bfseries]

		\iref{problem:coloring_1}{USAMO 2014 P1}{}
		\iref{problem:coloring_2}{USAMO 2008 P3}{}
		\iref{problem:coloring_3}{IMO 2018 P4}{}
		\iref{problem:coloring_4}{Codeforces 101954/G}{}

	\end{enumerate}


	\Faka\subsubsection{Plane divided by lines}{In problems regarding the plane being divided by straight lines, color the plane with chessboard colors.}\label{plane_coloring}

	\begin{multicols}{3}
		\begin{enumerate}[wide=0em, label=\arabic*, itemsep=0pt, parsep=0pt, font=\footnotesize\bfseries]

			\iref{problem:plane_coloring_1}{EGMO 2017 P3}{{}}
		\end{enumerate}
	\end{multicols}





	\newpage\subsection{Divide and Conquer\label{divide_and_conquer}}

	\vspace{10mm}


	Divide the problem/grid/graph into smaller pieces and work through them separately and finally join them together. The main difference between this and induction/recursion is that we have to actually work in the smaller cases instead of assuming that they are true.

	\begin{enumerate}[wide=0em, label=\arabic*, itemsep=0pt, parsep=0pt, font=\footnotesize\bfseries]

		\iref{problem:divide_and_conquer_1}{USA TST 2011 P2,}{Capacity $ 1, 2 $ roads}
		\iref{problem:divide_and_conquer_2}{CodeForces 744B,}{Finding the minimum number in the rows}
		\iref{problem:divide_and_conquer_3}{Problem,}{Double binary search}
		\iref{problem:divide_and_conquer_4}{ISL 2005 C1,}{Lamps in rooms}
		\iref{problem:divide_and_conquer_5}{IOI 2016 P5,}{Bug changes the strings}
		\iref{problem:divide_and_conquer_6}{Iran TST 2007 P2,}{$x$ divides at most one other element in $A$}
	\end{enumerate}




	\Faka\subsubsection{Induction}{\textbf{Cauchy Induction}: $ n \rightarrow 2n, n \rightarrow n-1 $ }\label{induction}


		Can be used in almost any kind of problems, often called `\textit{goriber bondhu}'.

		\begin{itemize}

			\item In MO probs $ 2-3-5-6 $ or SL $ 3+ $ (often $ 1, 2 $ as well) you can be sure that applying only induction isn't going to do any good. You'll need extra tools, and you might need to apply induction more than once.

			\begin{itemize}

				\item Sometimes, in graph probs, apply indution on more than one node gives better results.
				\item Often you can set up your induction in more than one way, and finding the right way makes the problem much simpler.
				\item Sometimes trying to prove more by adding a stronger induction hypothesis makes it easier to carry out the induction.

			\end{itemize}

		\end{itemize}


	\faka\textbf{Type 1}: $ n-1 \rightarrow n $

	\begin{enumerate}[wide=0em, label=\arabic*, itemsep=0pt, parsep=0pt, font=\footnotesize\bfseries]

		\iref{problem:induction_type1_1}{ISL 2004 C2,}{$ n $ circles intersect, colors}
		\iref{problem:induction_type1_2}{ISL 1997 P4,}{Silver matrix}
		\iref{problem:induction_type1_3}{ARO 2018 P11.5,}{an easy graph}
		\iref{problem:induction_type1_4}{ISL 2006 C1,}{lamps will eventually be off}
		\iref{problem:induction_type1_5}{ISL 2002 C1,}{bijection in $ x+y=n $ and red-blue colors}
		\iref{problem:induction_type1_6}{ISL 2012 C2}{}
		\iref{problem:induction_type1_7}{ARO 2013 P9.4}{}
		\iref{problem:induction_type1_8}{ISL 2005 C2}{}
		\iref{problem:induction_type1_9}{ISL 2013 C3}{}
		\iref{problem:induction_type1_10}{ISL 2005 C1}{}
		\iref{problem:induction_type1_11}{ISL 2016 C6,}{the ferry problem}
		\iref{problem:induction_type1_12}{IMO SL 1985}{}
		\iref{problem:induction_type1_13}{ELMO 2017 P5}{}
		\iref{problem:induction_type1_14}{ISL 1990}{}
		\iref{problem:induction_type1_15}{USAMO 2017 P4}{}
		\iref{problem:induction_type1_16}{Jacob Tsimerman Induction}{}
		\iref{problem:induction_type1_17}{All Russia 2017 9.1}{}
		\iref{problem:induction_type1_18}{Iran TST 2008 D3P1}{}
		\iref{problem:induction_type1_19}{USA TST 2011 D3P2}{}
		\iref{problem:induction_type1_20}{Sunflower Lemma}{}
		\iref{problem:induction_type1_21}{ISL 1998 C4}{}
		\iref{problem:induction_type1_22}{Generalization of USAMO 1999 P1}{}
		\iref{problem:induction_type1_23}{USAMO 2005 P1,}{arranging divisors on a circle with no co-prime neighbors}
		\iref{problem:induction_type1_24}{USAMO 2006 P5,}{a frog jumps jumps of $ 2 $-powers}
		\iref{problem:induction_type1_25}{Romanian TST 2016 D1P2,}{associating $ x_i $ with $ S_i $}
		\iref{problem:induction_type1_26}{American Mathematical Monthly,}{$ n $ subsets from $ S={1\dots n-1} $ and a weird relation}
		\iref{problem:induction_type1_27}{ISL 1991 P10,}{Color the graph by numbers such that any vertex is gcd $ 1 $}
		\iref{problem:induction_type1_28}{Problem,}{Circles $ 1 $ unit apart, gotta cover them up.}
		\iref{problem:induction_type1_29}{IOI 2018 P1,}{Prefixes of a string}
		\iref{problem:induction_type1_30}{German TST 2004 E7P3,}{A white graph to a black graph}
		\iref{problem:induction_type1_31}{ISL 2002 C5,}{An finite family of sets of size $ r $ has a intersecting set of size $ r-1 $}
		\iref{problem:induction_type1_32}{US Dec TST 2016, P1,}{$ k $ bijections, and cycles in those}
	\end{enumerate}



	\faka\textbf{Type 2}: $ k (k<n-1) \rightarrow n $

	\begin{enumerate}[wide=0em, label=\arabic*, itemsep=0pt, parsep=0pt, font=\footnotesize\bfseries]

		\iref{problem:induction_type2_1}{ARO 2014 P9.3}{}
		\iref{problem:induction_type2_2}{Mexican Regional 2014 P6}{}
		\iref{problem:induction_type2_3}{USA TST 2011 P2}{}
		\iref{problem:induction_type2_4}{ISL 2006 C2}{}
		\iref{problem:induction_type2_5}{APMO 1999 P2}{$a_{i+j} \leq a_i+a_j$}
	\end{enumerate}




	\Faka\subsubsection{Inductive/Recursive Relations}{Building other solutions depending on already or easily tweakable solutions.}\label{recursive_solution}


		\begin{enumerate}[wide=0em, label=\arabic*, itemsep=0pt, parsep=0pt, font=\footnotesize\bfseries]

			\iref{problem:recursive_solution_1}{India TST 2013 Test 3, P1}{}
			\iref{problem:recursive_solution_2}{ISL 2002 C1}{}
			\iref{problem:recursive_solution_3}{ISL 2009 C3}{}
			\iref{problem:recursive_solution_4}{USAMO 2013 P2}{}
			\iref{problem:recursive_solution_5}{IMO 2011 P4}{}
			\iref{problem:recursive_solution_6}{ARO 2014 P9.7,}{stable coin system with coins of value $ \a^k $}
			\iref{problem:recursive_solution_7}{Saint Petersburg 2001}{}
		\end{enumerate}





		\Faka\paragraph{Catalan Numbers}{$ C_n = \frac{1}{n+1}\binom{2n}{n} $, this little number is associated with a lot of combinatorial setups. And has the \hrf{lemma:catalan_recursion}{recursion}.}\label{catalan_numbers}




	\newpage\subsection{Count the Shit Up}


	\Faka\subsubsection{Double Counting\label{double_counting}}

	Explicitly count the number of things, but Twice!

	\begin{enumerate}[wide=0em, label=\arabic*, itemsep=0pt, parsep=0pt, font=\footnotesize\bfseries]

		\iref{problem:double_counting_1}{ISL 2016 C3,}{$ n $-gon colored with $ 3 $ colors, exists isosceles triangle.}
		\iref{problem:double_counting_2}{Iran TST 2012 P4,}{Path inside of a $ m\times n $.}
		\iref{problem:double_counting_3}{ISL 2014 C1,}{Dissecting a Rectangle wrt to some given points inside.}
		\iref{problem:double_counting_4}{Problem,}{$ 10 $ person bookstore problem.}
		\iref{problem:double_counting_5}{USA TST 2005 P1,}{Subsets of the set $ \{1, 2,\dots, mn\} $}
		\iref{problem:double_counting_6}{USAMO 2012 P2,}{$ 4 $ colors on the circle, rotating.}
		\iref{problem:double_counting_7}{ISL 2004 C1,}{A Cauchy type function based on a coloring of the integers.}
		\iref{problem:double_counting_8}{China MO 2018 P2,}{$ 3n^2-3n+1+k $ points are red, $ k $ good boxes exist.}
		\iref{problem:double_counting_9}{Problem}{}
		\iref{problem:double_counting_10}{ISL 2003 C3}{}

	\end{enumerate}





	\faka\subsubsection{Generating Function\label{generating_function}}

	For a sequence $ A = (a_0, a_1, a_2 \dots) $, the generating function$ E_A(x) $ for this sequence is of several types types:

	\begin{enumerate}[wide=0em, label=\arabic*, itemsep=0pt, parsep=0pt, font=\footnotesize\bfseries]

		\item $ E_A(x) = a_0x^0 + a_1x^1 \dots + a_ix^i \dots $, Useful for usual recursive sequences.

		\item $ E_A(x) = x^{a_0} + x^{a_1} \dots + x^{a_i} \dots $, Useful for sum of any two elements from \emph{any} two sequences.

		\item $ E_A(x) = \prod (1 + x^{a_i}) $, Useful for sum of multiple numbers from one sequence.

		\item $ E_A(t, x) = \prod (t + x^{a_i}) $, Useful for keeping track of how many numbers are being added to the sum.

	\end{enumerate}



	\begin{enumerate}[wide=0em, label=\arabic*, itemsep=0pt, parsep=0pt, font=\footnotesize\bfseries]

		\iref{problem:generating_function_1}{Problem,}{two sequence's pairwise sum's tuples are the same, $ n=2^k $.}
		\iref{problem:generating_function_2}{Result by Erdos,}{Partitioning the integers into arithmetic sequences.}
		\iref{problem:generating_function_3}{Problem on Catalan's Recursion}{}
	\end{enumerate}






		\faka\paragraph{Roots of Unity}{These things can be used in a lot of places, like coloring boards, coloring (dividing into modular classes) the integers, using as variables in generating functions etc.}\label{roots_of_unity}


		\begin{multicols}{2}
			\begin{enumerate}[wide=0em, label=\arabic*, itemsep=0pt, parsep=0pt, font=\footnotesize\bfseries]

				\iref{problem:roots_of_unity_1}{Result by Erdos}{}
			\end{enumerate}
		\end{multicols}





	\newpage\subsection{Different Representation}

	Represent the problem or the problem objects differently, usually by binary strings, graphs or matrices.




		\Faka\subsubsection{Binary}\label{binary}

		Associate a binary string to elements, like for handling subsets, add a string to each element, representing if it is in a certain set or not.

			\begin{enumerate}[wide=0em, label=\arabic*, itemsep=0pt, parsep=0pt, font=\footnotesize\bfseries]

				\iref{problem:binary_1}{IOI Practice 2017}{}
				\iref{problem:binary_2}{ISL 1988 P10}{}
			\end{enumerate}




		\Faka\subsubsection{Binary Query}\label{binary_query}

		Asking questions of the kind: if in the binary expansion of $ k $ , if the $ i $ th bit is $ 0 $ , add $ k $ to one kind of query and if it is $ 1 $ , then add it to another.

			\begin{enumerate}[wide=0em, label=\arabic*, itemsep=0pt, parsep=0pt, font=\footnotesize\bfseries]

				\iref{problem:binary_query_1}{CodeForces 744B}{}
				\iref{problem:binary_query_2}{Problem}{}
			\end{enumerate}




		\Faka\subsubsection{Matrix Creation}\label{matrix_creation}

		When there is some sort of $ a\times b $ always try to create a matrix.

			\begin{enumerate}[wide=0em, label=\arabic*, itemsep=0pt, parsep=0pt, font=\footnotesize\bfseries]

				\iref{problem:matrix_creation_1}{IMO 2017 P5}{}
			\end{enumerate}




		\Faka\subsubsection{Graph}\label{graph_representation}


		Problems concerning sets and their relations, consider representing using graph, with some fixed mapping rules. Some times changing grid cells into vertices's also helps.


			\begin{enumerate}[wide=0em, label=\arabic*, itemsep=0pt, parsep=0pt, font=\footnotesize\bfseries]

				\iref{problem:graph_representation_1}{USAMO 1986 P2}{}
				\iref{problem:graph_representation_2}{Mexican Regional 2014 P6}{}
				\iref{problem:graph_representation_3}{AoPS}{}
				\iref{problem:graph_representation_4}{ARO 2013 P9.5}{}
				\iref{problem:graph_representation_5}{ISL 2002 C6}{}
				\iref{problem:graph_representation_6}{Problem,}{Mailman messes up}
				\iref{problem:graph_representation_7}{timus 1862,}{Sum of operations}
				\iref{problem:graph_representation_8}{ARO 2007 P9.7,}{Adding diagrams that cut even number of already drawn ones}
				\iref{problem:graph_representation_9}{ISL 2002 C3}{Full-Sequences}
			\end{enumerate}


		\Faka\subsubsection{Changing the Target Term}

		Changing the term you have to achieve to a slightly more intuitive one, usually thinking about what values you can get more naturally from the given conditions, and to build a similar term from the original term.

			\begin{enumerate}[wide=0em, label=\arabic*, itemsep=0pt, parsep=0pt, font=\footnotesize\bfseries]

				\iref{problem:changing_term_1}{ARO 2014 P10.8,}{Mutually intersecting $ k $-gons, one point inside of a bunch of gons}{}

			\end{enumerate}








	\newpage\subsection{Algorithms}


		\Faka\subsubsection{Greedy Algorithm}\label{greedy_algorithm}


		\begin{enumerate}[wide=0em, label=\arabic*, itemsep=0pt, parsep=0pt, font=\footnotesize\bfseries]

			\iref{problem:greedy_algorithm_1}{ISL 2014 N3,}{Cape Town Coin problem}
			\iref{problem:greedy_algorithm_2}{China TST 2006}{}
			\iref{problem:greedy_algorithm_3}{Timus 1578}{}

		\end{enumerate}





		\Faka\subsubsection{Constructive Algorithm}\label{constructive_algo}


		In these kinda problems, you have to prove using a construction. In other words, proof by \emph{Existence}. The key is to add one object to the solution set one at a time depending on already added objects in the set and maintaining the problem conditions. Sometimes by adding additional constraints or prioritizing already given constraints.


		\begin{enumerate}[wide=0em, label=\arabic*, itemsep=0pt, parsep=0pt, font=\footnotesize\bfseries]

			\iref{problem:constructive_algo_1}{Bulgarian IMO TST 2004, D3P3}{}
			\iref{problem:constructive_algo_2}{ARO 2018 10.3}{}
			\iref{problem:constructive_algo_3}{CodeForces 960/C}{}
			\iref{problem:constructive_algo_4}{ARO 2005 P10.3, P11.2}{}
			\iref{problem:constructive_algo_5}{IOI 2007 P3}{}
			\iref{problem:constructive_algo_6}{Problem}{}
			\iref{problem:constructive_algo_7}{ARO 2018 P11.5}{}
			\iref{problem:constructive_algo_8}{ARO 2013 P9.4}{}
			\iref{problem:constructive_algo_9}{ARO 2014 P9.8}{}
			\iref{problem:constructive_algo_10}{ISL 2016 C1}{}
			\iref{problem:constructive_algo_11}{India IMO Camp 2017}{}
			\iref{problem:constructive_algo_12}{ISL 2012 C2}{}
			\iref{problem:constructive_algo_13}{CodeForces 989C}{}
			\iref{problem:constructive_algo_14}{CodeForces 989B}{}
			\iref{problem:constructive_algo_15}{ISL 2011 A5}{}
			\iref{problem:constructive_algo_16}{Iran TST 2017 D1P1}{}
			\iref{problem:constructive_algo_17}{ISL 2009 C2}{}
			\iref{problem:constructive_algo_18}{Problem}{}
			\iref{problem:constructive_algo_19}{Putnam 2017 A4}{}
			\iref{problem:constructive_algo_20}{Serbia TST 2017 P4}{}
			\iref{problem:constructive_algo_21}{ISL 2014 A1}{}
			\iref{problem:constructive_algo_22}{ISL 2005 N2,}{sequence that contains all of the integers}
			\iref{problem:constructive_algo_23}{Problem,}{switch states of a row and column}
			\iref{problem:constructive_algo_24}{USAMO 2015 P4,}{piles of stone on cells, mone on the corners of a rectangle}
			\iref{problem:constructive_algo_25}{ISL 2003 C4}{}

		\end{enumerate}



		\Faka\subsubsection{Element : Time}\label{add_time}

		Adding an element of Time to give the static problem a dynamic view. In less formal words, if a problem environments seems to just *exist*, add a dynamic way to slowly visualize the environment to exist. One kind of constructive algorithm, but this algo doesn't build the answer or solution, instead it builds up the whole environment step by step.


		\begin{enumerate}[wide=0em, label=\arabic*, itemsep=0pt, parsep=0pt, font=\footnotesize\bfseries]

			\iref{problem:add_time_1}{ARO 2016 P3}{}
			\iref{problem:add_time_2}{Brazilian Olympic Revenge 2014}{}
			\iref{problem:add_time_3}{ISL 2008 C1}{}
			\iref{problem:add_time_4}{USAMO 1999 P1}{}
			\iref{problem:add_time_5}{AoPS}{}

		\end{enumerate}



		\newpage\subsubsection{Gaming Tricks}


			\paragraph{Pairing and Copying}

			{Who said you can't cheat in a combinatorial game? Just follow your opponents movements, and copy them cleverly.}


			\begin{enumerate}[wide=0em, label=\arabic*, itemsep=0pt, parsep=0pt, font=\footnotesize\bfseries]

				\iref{problem:pairing_and_copying_1}{ARO 2011 P11.6,}{Take a number of stones off the heap of size $ n^2 $}

			\end{enumerate}


			\paragraph{Nim Equivalence}





	\newpage\subsection{Invariance Rules of Thumb}\label{invariant_rules_of_thumb}

		\begin{enumerate}[wide=0em, label=\arabic*, itemsep=0pt, parsep=0pt, font=\footnotesize\bfseries]

			\item Natural Sum
			\item Alternating Sum
			\item Sum of Squares
			\item Product

			\item Giving weight to each of the elements, in problems where usually no trivial invariants exists. Like weights of $ 2^i, \frac{1}{i}, \text{roots of unity} $ etc. depending on the problem's nature.
		\end{enumerate}

	\begin{itemize}

		\iref{problem:invariant_rules_of_thumb_1}{ISL 2014 C2,}{$ 1 $ written on $ 2^m $ papers and and an addition operation}
		\iref{problem:invariant_rules_of_thumb_2}{ISL 2012 C1,}{Operation almost alike to swapping and sorting}
		\iref{problem:invariant_rules_of_thumb_3}{Indian TST 2004,}{Pebble makes a clone and moves up and right}
		\iref{problem:invariant_rules_of_thumb_4}{ISL 1998 C7,}{One lamp on each cell, switching one lamp switches neighbors}
		\iref{problem:invariant_rules_of_thumb_5}{APMO 2017 P1,}{$ a-b+c-d+e=29 $}
		\iref{problem:invariant_rules_of_thumb_6}{AoPS}{}
		\iref{problem:invariant_rules_of_thumb_7}{ARO 2016 P1}{}
		\iref{problem:invariant_rules_of_thumb_8}{Serbia TST 2017 P2,}{$ (x+y) \rightarrow (\frac{x}{2}, y+\frac{x}{2}) $ or $ (x+\frac{y}{2}, \frac{y}{2}) $}
		\iref{problem:invariant_rules_of_thumb_9}{ISL 1994 C3,}{$ 3 $ bank accounts}
		\iref{problem:invariant_rules_of_thumb_10}{Codeforces 987E}{}
		\iref{problem:invariant_rules_of_thumb_11}{ISL 2007 C4,}{Dividing a sequence with almost equal sum, and getting another sequence from it.}
		\iref{problem:invariant_rules_of_thumb_12}{USAMO 2015 P4,}{piles of stone on cells, mone on the corners of a rectangle}

	\end{itemize}



	\Faka\subsubsection{Monotonicity with strict constraints\label{monotonicity_with_constraints}}

	If regular monotonicity doesn't apply, then some monotonicity with special properties might work. Like keeping the sum \emph{even}, \emph{odd}, \emph{square} etc.


	\begin{enumerate}[wide=0em, label=\arabic*, itemsep=0pt, parsep=0pt, font=\footnotesize\bfseries]

		\iref{problem:monotonicity_with_constraints_1}{USAMO 2013 P6,}{Replace $ x $ with the difference of the two neighboring numbers.}
		\iref{problem:monotonicity_with_constraints_2}{MEMO 2008, Team, P6,}{$ a, b \rightarrow a+b, a+b $}

	\end{enumerate}








	\newpage\subsection{Pigeonhole Principal}\label{php}



	\Faka\subsubsection{Alternating Chains Technique}{In a Cyclic graph with $ n $ nodes, if your task is to color some of the nodes so that no two neighboring nodes will be colored, you can color at most $ \floor{\frac{n}{2}} $. And in a path, this value is $ \ceil{\frac{n}{2}} $ }\label{alternating_chains}

	\begin{multicols}{3}
		\begin{enumerate}[wide=0em, label=\arabic*, itemsep=0pt, parsep=0pt, font=\footnotesize\bfseries]

			\iref{problem:alternating_chains_1}{ISL 1990 P3}{}
			\iref{problem:alternating_chains_2}{USAMO 2008 P3}{}
		\end{enumerate}
	\end{multicols}











	\newpage\subsection{Other Useful Techniques and Philosophies}


	\Faka\subsubsection{Include potentially important players in the game}{If the problem condition is completely or partially but crucially depended on some problem object, but the proof condition doesn't directly depend on that object, think of a way to include that object in the proof condition.}\label{add_stuffs}


	\begin{multicols}{3}
		\begin{enumerate}[wide=0em, label=\arabic*, itemsep=0pt, parsep=0pt, font=\footnotesize\bfseries]

			\iref{problem:add_stuffs_1}{APMO 2017 P3}{}
			\iref{problem:add_stuffs_2}{USAMO 2008 P5}{}
		\end{enumerate}
	\end{multicols}



	\Faka\subsubsection{Finding the Tough Nut}{Solving an easier version of the problem with some sort of constraints lose, to find out exactly what makes the problem so tough. This way we get valuable information about on what our main focus should be.}\label{finding_the_tough_nut}








	\Faka\subsubsection{eChen trick}{Subtract a constant from all the numbers to make the sum $ 0 $. This makes the numbers easier to handle.}\label{minus_constant}

	\begin{multicols}{3}
		\begin{enumerate}[wide=0em, label=\arabic*, itemsep=0pt, parsep=0pt, font=\footnotesize\bfseries]

			\iref{problem:minus_constant_1}{APMO 2017 P1}{}
			\iref{problem:minus_constant_2}{ARO 2013 P9.5}{}
		\end{enumerate}
	\end{multicols}








	\Faka\subsubsection{Send objects to the infinity}{If there are too many arbitrary objects in the problem, try making some of them vanish by sending them to the infinity.}








	\Faka\subsubsection{n+1 = (n-i) + (i+1)}{ $ n+1 = (n-i) + (i+1) $ might prove to be useful when applying induction to $ \binom{n}{k} $ }






	\Faka\subsubsection{Convex Hulls, and Sandwiching two points}{You know what convex hull is. AND, One can draw two lines to separate two points - draw two parallel lines very close to the two points.}\label{sandwiching_points}


	\begin{enumerate}[wide=0em, label=\arabic*, itemsep=0pt, parsep=0pt, font=\footnotesize\bfseries]

		\iref{problem:sandwiching_points_1}{ISL 2013 C2}{}
		\iref{problem:convex_hull_1}{Putnam 1979,}{$ n $ red $ n $ blue, pair them}
		\iref{problem:convex_hull_3}{USAMO 2005 P5,}{$ n $ red $ n $ blue, at least two segments dividing them}
		\iref{problem:convex_hull_2}{ILL 1985}{}
	\end{enumerate}





\newpage\section{Binomial Identities}


		\theo{https://en.wikipedia.org/wiki/Vandermonde's_identity}{Vandermonde's identity}{\[ \binom{m+n}{k} = \sum_{i=0}^{k} \binom{m}{i}\binom{n}{k-i} \]}


		\theo{}{}{\[\binom{2n}{n} = \binom{p}{0}^2 \binom{p}{1}^2 \dots + \binom{p}{p}^2 \]}
\newpage\section{Sets}

	\subsection{Lemmas}


		\lem{Let $ S $ be a set with $ n $ elements, and let $ F $ be a family of subsets of S such that for any pair $ A, B $ in $ F $, $ A \cap B \not= \varnothing $. Then $ |F| \leq 2^{n-1} $ .}\label{lemma:sets_lemma_1}

		\begin{enumerate}[wide=0em, label=\arabic*, itemsep=0pt, parsep=0pt, font=\footnotesize\bfseries]

			\iref{lemma:sets_lemma_1_1}{Iran TST 2008 D3P1}{}
		\end{enumerate}




		\lem{Let $ S $ be a set with $ n $ elements, and let $ F $ be a family of subsets of $ S $ such that for any pair $ A, B $ in $ F $, $ S $ is not contained by $ A \cup B $. Then $ |F| \leq 2^{n-1} $.}\label{lemma:sets_lemma_2}



		\lem{\hl{Kleitman’s lemma} A set family $F$ is said to be downwards closed if the following holds: if $X$ is a set in $F$, then all subsets of $X$ are also sets in $F$. Similarly, $F$ is said to be upwards closed if whenever $X$ is a set in $F$, all sets containing $X$ are also sets in $F$. Let $F_1$ and $F_2 $be downwards closed families of subsets of $S = \{1, 2, ..., n\}$, and let $F_3$ be an upwards closed family of subsets of $S$. Then we have

			\begin{enumerate}[wide=0em, label=\arabic*, itemsep=0pt, parsep=0pt, font=\footnotesize\bfseries]

				\item \[ |F_1 \cap F_2| \geq \frac{|F_1| \cdot |F_2|}{2^n} \]
				\item \[ |F_1 \cap F_3| \leq \frac{|F_1| \cdot |F_3|}{2^n} \]
		\end{enumerate}}\label{lemma:sets_lemma_3_Kleitman}




		\lem{Let $ S $ be a set with $ n $ elements, and let $ F $ be a family of subsets of $ S $ such that for any pair $ A, B $ in $ F $, $ A \cap B \not= \varnothing $ and $ A \cap B \not= S $. Then $ |F| \leq 2^{n-2} $.}\label{lemma:sets_lemma_4}

		\solu{Using the sets in \hrf{lemma:sets_lemma_1}{lemma 1} and \hrf{lemma:sets_lemma_1}{lemma 2}, defining upwards and downwards sets like in \hrf{lemma:sets_lemma_3_Kleitman}{Kleitman's Lemma}.}



		\lem{\href{https://en.wikipedia.org/wiki/Sunflower_(mathematics)}{\textbf{The Sunflower Lemma:}} A sunflower with $ k $ petals and a core $ X $ is a family of sets $ S_1, S_2,\dots, S_k $ such that $ S_i\cap S_j = X $ for each $ i \neq j $. (The reason for the name is that the Venn diagram representation for such a family resembles a sunflower.) The sets $ S_i \setminus X $ are known as petals and must be nonempty, though $ X $ can be empty. Show that if $ F $ is a family of sets of cardinality $ s $, and $ |F| > s!(k-1)^s $, then $ F $ contains a sunflower with $ k $ petals.}\label{problem:induction_type1_20}\label{problem:extreme_object_10}


		\solu{Applying induction and considering the best case where $ |X|=0 $}


	\newpage\subsection{Extremal Set Theory}

		\href{http://math.mit.edu/~cb_lee/18.318/lecture8.pdf}{MIT 18.314 Lecture-8}


		\theo{}{Mirsky Theorem}{A set $ S $ with a chain of height $ h $ can’t be partitioned into $ t $ anti-chains if $ t < h $. In other words, the minimum number of sets in any anti-chain partition of $ S $ is equal to the maximum height of the chains in $ S $. (And Vice Versa)}\label{theorem:mirsky_theorem}


		\theo{}{}{In any poset, the largest cardinality of an antichain is at most the smallest cardinality of a chain-decomposition of that poset.}


		\theo{}{Dilworth's Theorem}{Let $ P $ be a poset. Then there exist an antichain $ A $ and a chain decomposition $ \mathcal{C} $ of $ P $ such that $ |A| = |\mathcal{C}| $}



		\theo{http://mathworld.wolfram.com/Erdos-SzekeresTheorem.html}{Erdos-Szekeres Theorem}{Any sequence of $ ab+1 $ real numbers contains either a monotonically decreasing subsequence of length $ a+1 $ or a monotonically increasing subsequence of length $ b+1 $. The more useful case is when $ a=b=n $. }







	\newpage\subsection{Problems}



		\prob{https://artofproblemsolving.com/community/c6h148835p841269}{USA TST 2005 P1}{E}{Let $ n $ be an integer greater than $ 1 $. For a positive integer $ m $ , let $ S_{m}= \{ 1,2,\ldots, mn\} $. Suppose that there exists a $ 2n $ -element set $ T $ such that

			\begin{enumerate}[wide=0em, label=\arabic*, itemsep=0pt, parsep=0pt, font=\footnotesize\bfseries]


				\item each element of $ T $ is an $ m $ -element subset of $ S_{m} $
				\item each pair of elements of $ T $ shares at most one common element
				\item each element of $ S_{m} $ is contained in exactly two elements of $ T $

			\end{enumerate}

			Determine the maximum possible value of $ m $ in terms of $ n $.}\label{problem:double_counting_5}

		\solu{We use double counting to find the ans, after that the rest is easy.}





		\prob{https://artofproblemsolving.com/community/c6h206650p1136980}{Iran TST 2008 D3P1}{E}{Let $S$ be a set with $n$ elements, and $F$ be a family of subsets of $S$ with $ 2^{n-1}$ elements, such that for each $A,B,C\in F$, $A\cap B\cap C$ is not empty. Prove that the intersection of all of the elements of $F$ is not empty.}\label{lemma:sets_lemma_1_1}\label{problem:induction_type1_18}

		\solu{Using Induction with \hrf{lemma:sets_lemma_1}{this} lemma.}




		\prob{https://artofproblemsolving.com/community/c6t309f6h1538018}{Romanian TST 2016 D1P2}{EM}{Let $n$ be a positive integer, and let $S_1, S_2,\dots S_n$ be a collection of finite non-empty sets such that $$\sum_{1\leq i<j\leq n}{\frac{|S_i \cap S_j|}{|S_i||S_j|}} <1.$$Prove that there exist pairwise distinct elements $x_1, x_2\dots x_n$ such that $x_i$ is a member of $S_i$ for each index $i$.}\label{problem:induction_type1_25}\label{problem:forget_and_focus_7}


		\solu{The Inductive proof reduces the problem to \hrf{problem:induction_type1_26}{this} problem.}

		\solu{The other approach is to focus on the given weird condition, and interpolate it to something nice, like probabilistic condition.}



		\prob{https://artofproblemsolving.com/community/c6h225275p1252232}{American Mathematical Monthly problem E2309}{EM}{If $ A_1$, $ A_2$,\dots $ A_n$ are $ n$ nonempty subsets of the set $ \left\{1,2,...,n - 1\right\}$, then prove that

			$ \sum_{1\leq i < j\leq n}\frac {\left|A_i\cap A_j\right|}{\left|A_i\right|\cdot\left|A_j\right|}\geq 1$.}\label{problem:induction_type1_26}



		\prob{https://artofproblemsolving.com/community/c6h362426p1986928}{CGMO 2010 P1}{E}{Let $n$ be an integer greater than two, and let $A_1,A_2, \cdots , A_{2n}$ be pairwise distinct subsets of $\{1, 2, ,n\}$. Determine the maximum value of
			\[\sum_{i=1}^{2n} \dfrac{|A_i \cap A_{i+1}|}{|A_i| \cdot |A_{i+1}|}\]
			Where $A_{2n+1}=A_1$ and $|X|$ denote the number of elements in $X.$}



		\prob{https://artofproblemsolving.com/community/c6h17340p119108}{ISL 2002 C5}{M}{Let $r\geq2$ be a fixed positive integer, and let $F$ be an infinite family of sets, each of size $r$, no two of which are disjoint. Prove that there exists a set of size $r-1$ that meets each set in $F$.}\label{problem:induction_type1_31}


		\gene{https://artofproblemsolving.com/community/c6h17340p7934669}{HMMT 2016 Team Round}{Fix positive integers $r>s$, and let $\mathcal F$ be an infinite family of sets, each of size $r$, no two of which share fewer than $s$ elements. Prove that there exists a set of size $r-1$ that shares at least $s$ elements with each set in $F$.}


			\solu{First idea, if we take an arbitrary set, we can say that there exists infinitely many sets $ \in \mathbb{F} $ which includes a fixed element from our test set. If we do this argument for $ r-1 $ times, we get a set $ X $ of $ r-1 $ elements, and an infinte family of sets that contains $ X $ completely. At this point the problem is trivial.}


			\solu{Since it's tricky to work with one family, why not introduce another family, like the second monk. \hrf{http://artofproblemsolving.com/community/c6h17340p3251745}{This} solution generalizes the problem as such.}

		
		\prob{https://artofproblemsolving.com/community/c6h57290p352698}{ISL 1988 P10}{M}{Let $ N = \{1,2 \ldots, n\}, n \geq 2. $ A collection $ F = \{A_1, \ldots, A_t\} $ of subsets $ A_i \subseteq N, $  $ i = 1, \ldots, t, $ is said to be separating, if for every pair $ \{x,y\} \subseteq N, $ there is a set $ A_i \in F $ so that $ A_i \cap \{x,y\} $ contains just one element. $ F $ is said to be covering, if every element of $ N $ is contained in at least one set $ A_i \in F. $ What is the smallest value $ f(n) $ of $ t, $ so there is a set $ F = \{A_1, \ldots, A_t\} $ which is simultaneously separating and covering} \label{problem:binary_2}
		
		
			\solu{Using \hrf{binary}{Binary} Representations for the elements as in or not in, we get an easy bijection.}
		
		



\newpage\section{Algorithmic}
\newpage\section{Graph Theory}



	\subsection{Lemmas}


		\lem{\hl{\textbf{Average of Degrees:}} In a graph $ G $ with $ n $ vertexes, let $ E $ be the set of all edges. Assign an integer $ f_i $ to every vertex $ v_i $ such that $ f_i $ equals to the everage degree of the neighbors of $ v_i $. We have, \[ \sum_{i=1}^{n} f_i \geq 2\vert E\vert \] }\label{lemma:graph_lemma_1}



		\lem{\hl{\textbf{Bipartite Graph Criteria:}} Any graph having only even cycles are BIPARTITE.}\label{lemma:bipartite_graph}


		\begin{multicols}{3}
			\begin{enumerate}[wide=0em, label=\arabic*, itemsep=0pt, parsep=0pt, font=\footnotesize\bfseries]

				\iref{problem:bipartite_graph_1}{AoPS}{}
				\iref{problem:bipartite_graph_2}{ISL 2004 C3}{}
				\iref{problem:bipartite_graph_3}{Problem}{}
				\iref{problem:bipartite_graph_4}{Problem}{}
			\end{enumerate}
		\end{multicols}


		\lem{\hl{\textbf{Euler's Theorem on Planar Graphs:\label{theorem:planar_graph_theorem}}} In a planar graph with $ V $ vertices, $ E $ edges and $ C $ cycles, the following condition is always satisfied: \[V-E+C=1\]}


		\theo{http://www.ams.org/samplings/feature-column/fcarc-eulers-formula}{Euler's Polyhedron Formula\label{lemma:planar_graph_polyhedron}}{For any polyhedron with $ E, V, F $ edges, vertices's and faces resp. the following relation holds \[V+F=E+2\]}





		\theo{https://en.wikipedia.org/wiki/Prüfer_sequence}{Prüfer sequence}{Consider a labeled tree $ T $ with vertices's $ \{1, 2, ..., n\} $. At step $ i $, remove the leaf with the smallest label and set the $ i $th element of the \textit{Prüfer sequence} to be the label of this leaf's neighbour. Prove that a Prüfer sequence of length $ n-2 $ defines a Tree with length $ n $. In other words, proof Prüfer bjection, and give an algorithm to build the tree from the Prüfer sequence.

			\fig{.5}{prufer_code_example}{A labeled tree with Prüfer sequence $ {4,4,4,5} $.}
		}


		\den{\texttt{Cut:} A cut is a partition of the vertices of a graph into two disjoint subsets. Any cut determines a cut-set, the set of edges that have one endpoint in each subset of the partition. These edges are said to cross the cut. In a connected graph, each cut-set determines a unique cut, and in some cases cuts are identified with their cut-sets rather than with their vertex partitions.}\label{definition:cut_graph_theory}





	\newpage\subsection{Problems}




		\prob{https://artofproblemsolving.com/community/c6h589936p3493453}{ARO 2014 P9.8\label{tickets}}{H}{In a country of $ n $ cities, an express train runs both ways between any two cities. For any train, ticket prices either direction are equal, but for any different routes these prices are different. Prove that the traveler can select the starting city, leave it and go on, successively, $ n-1 $ trains, such that each fare is smaller than that of the previous fare. (A traveler can enter the same city several times.)}



		\prob{}{}{E}{Given a bipartite graph, prove that the minimum number of colors required to color the edges of the graph such that no node is adjacent to $ 2 $ edges of same color is the maximum degree of the graph.}\label{problem:bipartite_graph_3}


		\prob{}{}{E}{For every bipartite graph prove that it's edges can be bicolored so that each node is adjacent to atmost $ \ceil{\frac{deg}{2}} $ edges of  any color.}\label{problem:bipartite_graph_4}


			\solu{Using the main property of a bipartite graph.}


			\solu{After finding the cycle solution, to optimize it, we recall that we can find a Eulerian Path (if it exists) in $ O(V+E) $. Now we want to make the graph have a Eulerian path, so we add a vertice to both sides of the graph, and join them with odd vertices from the other side.}



		\prob{https://artofproblemsolving.com/community/c6h364267}{Generalization}{\hrf{tickets}{H}}{Let $ A $ be a set of $ n $ points in the space. From the family of all segments with endpoints in $ A $ , $ q $ segments have been selected and colored yellow. Suppose that all yellow segments are of different length. Prove that there exists a polygonal line composed of $ m $ yellow segments, where $ m\geq\frac{2q}{n} $ , arranged in order of increasing length.}\label{problem:constructive_algo_9}\label{problem:swapping_4}

			\solu{Make one person go to every node. Then let the two people on the two sides of the most expensive edge swap their position. This ensures that every edge was used exactly 2 times. Using PHP, we have the desired result. Another solution is by \hrf{theorem:mirsky_theorem}{this}}





		\prob{https://artofproblemsolving.com/community/c6h100733p568964}{ISL 2005 C1}{E}{A house has an even number of lamps distributed among its rooms in such a way that there are at least three lamps in every room. Each lamp shares a switch with exactly one other lamp, not necessarily from the same room. Each change in the switch shared by two lamps changes their states simultaneously. Prove that for every initial state of the lamps there exists a sequence of changes in some of the switches at the end of which each room contains lamps which are on as well as lamps which are off.}\label{problem:divide_and_conquer_4}\label{problem:induction_type1_10}




		\prob{https://artofproblemsolving.com/community/c6h597130p3543398}{ISL 2013 C3\label{problem:induction_type1_9}}{M}{A crazy physicist discovered a new kind of particle which he called an $ i $ -mon, after some of them mysteriously appeared in his lab. Some pairs of $ i $ -mons in the lab can be entangled, and each $ i $ -mon can participate in many entanglement relations. The physicist has found a way to perform the following two kinds of operations with these particles, one operation at a time.

		\begin{enumerate}


			\item If some $ i $ -mon is entangled with an odd number of other $ i $ -mons in the lab, then the physicist can destroy it.

			\item At any moment, he may double the whole family of $ i $ -mons in the lab by creating a copy $ I' $ of each $ i $ -mon $ I $. During this procedure, the two copies $ I' $ and $ J' $ become entangled if and only if the original $ i $ -mons $ I $ and $ J $ are entangled, and each copy $ I' $ becomes entangled with its original $ i $ -mon $ I $ ; no other entanglements occur or disappear at this moment.

		\end{enumerate}}


			\solu{Prove that the physicist may apply a sequence of much operations resulting in a family of $ i $ -mons, no two of which are entangled.

			As there are an integer number of $ i $ -mons, it is quite natural to use induction. We try to find an algorithm to reduce the number of particles.

			Another way to do this is to consider the chromatic number of the graph. If we can show that this number reduces after some move, then we are done by induction.}






		\prob{https://artofproblemsolving.com/community/c6h104152p586762}{ISL 2005 C2\label{problem:induction_type1_8}}{E}{A forest consists of rooted (i. e. oriented) trees. Each vertex of the forest is either a leaf or has two successors. A vertex $ v $ is called an extended successor of a vertex $ u $ if there is a chain of vertices's $ u_{0}=u , u_{1}, u_{2} \dots u_{t-1} , u_{t}=v $ with $ t>0 $ such that the vertex $ u_{i+1} $ is a successor of the vertex $ u_{i} $ for every integer $ i $ with $ 0\leq i\leq t-1 $.\\

		Let $ k $ be a nonnegative integer. A vertex is called dynastic if it has two successors and each of these successors has at least $ k $ extended successors.\\

		Prove that if the forest has $ n $ vertices, then there are at most $ \frac{n}{k+2} $ dynastic vertices.}

			\solu{Trying to apply induction, we realize the bound is very loosy. That's why when we try to add in the inductive step, the value becomes larger the the bound. To negate that overflow, we tighten the bound.}

			\solu{The second and dummy approach is to first doing some smaller cases, finding small infos, taking the root, seeing that the bound doesnt work, but it would work if one of the successors of the root would have exactly or less than $ 2k+3 $ successors. As we can't always guarantee that, we look for such a vertex with $ 2k+3 $ successors. We do some work with it and by induction its done.}



		\prob{https://artofproblemsolving.com/community/c6h1441121p8200413}{All Russia 2017 9.1\label{problem:induction_type1_17}}{E}{In a country some cities are connected by oneway flights (There are no more then one flight between two cities). City $ A $ called "available" for city $ B $ , if there is flight from $ B $ to $ A $ , maybe with some transfers. It is known, that for every 2 cities $ P $ and $ Q $ exist city $ R $ , such that $ P $ and $ Q $ are available from $ R $. Prove, that exist city $ A $ , such that every city is available for $ A $.}



		\prob{www.google.com}{Jacob Tsimerman Induction}{E}{There are $ 2010 $ ninjas in the village of Konoha (what? Ninjas are cool.) Certain ninjas are friends, but it is known that there do not exist $ 3 $ ninjas such that they are all pairwise friends. Find the maximum possible number of pairs of friends.(If ninja $ A $ is friends with ninja $ B $ , then ninja $ B $ is also friends with ninja $ A $.)}\label{problem:induction_type1_16}



		\prob{https://artofproblemsolving.com/community/c5h1434584p8117190}{USAMO 2017 P4}{M}{Let $ P_1 $ , $ P_2 $ , $ \dots $ , $ P_{2n} $ be $ 2n $ distinct points on the unit circle $ x^2+y^2=1 $ , other than $ (1,0) $. Each point is colored either red or blue, with exactly $ n $ red points and $ n $ blue points. Let $ R_1 $ , $ R_2 $ , $ \dots $ , $ R_n $ be any ordering of the red points. Let $ B_1 $ be the nearest blue point to $ R_1 $ traveling counterclockwise around the circle starting from $ R_1 $. Then let $ B_2 $ be the nearest of the remaining blue points to $ R_2 $ travelling counterclockwise around the circle from $ R_2 $ , and so on, until we have labeled all of the blue points $ B_1, \dots, B_n $. Show that the number of counterclockwise arcs of the form $ R_i \to B_i $ that contain the point $ (1,0) $ is independent of the way we chose the ordering $ R_1, \dots, R_n $ of the red points.}\label{problem:induction_type1_15}

			\solu{As the ques is saying that the condition is true for every positive integer $ n $ , can't we try induction? Which part of the condition makes the problem challenging? Obviously the circle condition and the $ (1, 0) $ point. So Lets remove a ``problematic'' pair $ R_t, B_t $. We encounter some problems, but we can pull it out of there.}




		\prob{www.hehe.com}{Swell coloring}{E}{Let $ K_n $ denote the complete graph on $ n $ vertices, that is, the graph with $ n $ vertices's such that every pair of vertices's is connected by an edge. A swell coloring of $ K_n $ is an assignment of a color to each of the edges such that the edges of any triangle are either all of distinct colors or all the same color. Further, more than one color must be used in total (otherwise trivially if all edges are the same color we would have a swell coloring). Show that if $ K_n $ can be swell colored with $ k $ colors, then $ k \geq \sqrt{n} + 1 $.}\label{problem:forget_and_focus_5}

			\solu{Concentrate on only one vertex.}



		\prob{www.hehe.com}{Belarus 2001}{MH}{Given $ n $ people, any two are either friends or enemies, and friendship and enmity are mutual. I want to distribute hats to	them, in such a way that any two friends possess a hat of the same color but no two enemies possess a hat of the same color. Each person can receive multiple hats. What is the minimum number of colors required to always guarantee that I can do this?}\label{problem:extremal_case_whole_6}

			\solu{In this problem, finding the worst case is a big help, because once the answer is guessed, the things become really clear.}



		\prob{https://artofproblemsolving.com/community/c6h420430p2374818}{USA TST 2011 D3P2}{M}{Let $n \geq 1$ be an integer, and let $S$ be a set of integer pairs $(a,b)$ with $1 \leq a < b \leq 2^n$. Assume $|S| > n \cdot 2^{n+1}$. Prove that there exists four integers $a < b < c < d$ such that $S$ contains all three pairs $(a,c)$, $(b,d)$ and $(a,d)$.}\label{problem:induction_type1_19}

			\solu{Using Induction to the first and last half of the set $ S $ shows us the \hrf{finding_the_tough_nut}{hardest part} of the problem. Then ordering the left and right elements with some sort of hierarchy is all the work left to do.}





		\prob{https://artofproblemsolving.com/community/c6h1480703p8639274}{ISL 2016 C6}{H}{There are $ n \geq 3 $ islands in a city. Initially, the ferry company offers some routes between some pairs of islands so that it is impossible to divide the islands into two groups such that no two islands in different groups are connected by a ferry route.\\

		After each year, the ferry company will close a ferry route between some two islands $ X $ and $ Y $. At the same time, in order to maintain its service, the company will open new routes according to the following rule: for any island which is connected to a ferry route to exactly one of $ X $ and $ Y $ , a new route between this island and the other of $ X $ and $ Y $ is added.\\

		Suppose at any moment, if we partition all islands into two nonempty groups in any way, then it is known that the ferry company will close a certain route connecting two islands from the two groups after some years. Prove that after some years there will be an island which is connected to all other islands by ferry routes.}\label{problem:induction_type1_11}

			\solu{It is only natural to use induction on this kinda problems. After some trying, we see that if we remove $ 1 $ node, We get to nowhere, but if we remove $ 2 $ nodes, we get something interesting. So now focus on those two nodes and the rest of the nodes separately. Its not hard from there.}

			\solu{As it seems, the separation of the graph was the main observation. We can call this trick \hl{Bringing Order in the Chaos}.}




		\prob{https://artofproblemsolving.com/community/c6h1468154p8509521}{ELMO 2017 P5}{H}{The edges of $ K_{2017} $ are each labeled with $ 1, 2 $ or $ 3 $ such that any triangle has sum of labels at least $ 5. $ Determine the minimum possible average of all labels. (Here $ K_{2017} $ is defined as the complete graph on 2017 vertices's, with an edge between every pair of vertices's.)}\label{problem:induction_type1_13}

			\solu{First solution is by Induction. Firstly we see that keeping all three kind of edges is sooo much work to do. What we can do is to simplify this to graph having only $ \lbrace 1, 2\rbrace $ or $ \lbrace 1, 3\rbrace $ types of edges. We actually see this works. Now we can Pick one $ 2 $ (or $ 3 $ ) edge and delete it. We get a nice recursive relation. Bam.}


			\solu{This solution is for elegancy which uses a nyc lemma:\\


			Using \hrf{lemma:graph_lemma_1}{this lemma} in our problem and considering the graph with all vertexes and all edges labeled $ 1 $ first and then considering the $ 3 $ label degree of each vertex we deduce that there must be at most $ 1008 $  $ 1 $ labeled edges more than $ 3 $ labeled edges in $ K_{2017} $.\\

			We get the intuition of this by noticing that in a vertex $ v $ , if the number of $ 3 $ labeled edges connected to it is $ d_3 $ , the maximum $ 1 $ degree of the vertexes that are connected to $ v $ with a $ 1 $ labeled edge edge is $ m_1 $ , then $ d_3\geq m_1-1 $. (This is pure intuition)\\

			As our maximum is achieved when we take the minimum number of $ 3 $ labeled edges. So we take $ E_1 $  $ 1 $ labeled edges, $ E_1  – 1008 $  $ 3 $ labeled edges and the rest $ 2 $ labeled edges. We get our desired minimum average.}



		\prob{https://artofproblemsolving.com/community/c6h535003p3067563}{ARO 2013 P9.5}{M}{ $ 2n $ real numbers with a positive sum are aligned in a circle. For each of the numbers, we can see there are two sets of $ n $ numbers such that this number is on the end. Prove that at least one of the numbers has a positive sum for both of these two sets.}\label{problem:graph_representation_4}\label{problem:minus_constant_2}

			\solu{Then make a set with the sum of all $ n $ consecutive blocks of numbers. Then its only natural to use graph representation. And $ Ta-Da $.}


		\prob{https://artofproblemsolving.com/community/c6h420424p2374799}{USA TST 2011 P2}{H}{In the nation of Onewaynia, certain pairs of cities are connected by roads. Every road connects exactly two cities (roads are allowed to cross each other, e.g., via bridges). Some roads have a traffic capacity of 1 unit and other roads have a traffic capacity of $ 2 $ units. However, on every road, traffic is only allowed to travel in one direction. It is known that for every city, the sum of the capacities of the roads connected to it is always odd. The transportation minister needs to assign a direction to every road. Prove that he can do it in such a way that for every city, the difference between the sum of the capacities of roads entering the city and the sum of the capacities of roads leaving the city is always exactly one.}\label{problem:divide_and_conquer_1}\label{problem:induction_type2_3}
		
			\solu{As there are two types of subgraph, $ 1 $ -type and $ 2 $ -type. By some work-arounds, we see that we have to work distinctly in both types of graphs. Firstly, if we work in type- $ 1 $ , we see after making a path from node $ x, y $ , the degrees of $ x, y $ will be $ \{1, -1\} $ and the degrees of other nodes on the path will be the same. After that, we make every nodes have degree either $ \{1, -1\} $. So after this operation we remove the $ 1 $ -edges. Now, when dealing with the type- $ 2 $ sub-graph. Start over from zero, we see that when making a path between nodes $ x, y $ the degree of those two changes parity, and other nodes on the path stays the same. So select two odd nodes.... }
			
			\solu{Dealing with two different kind of edges simultaneously is messy, so we work with graph $ 1 $ and graph $ 2 $ differently. Now on both graphs, we can remove cycles. And in graph $ 2 $ , we see that we can remove any big paths if there is a edge $ 1 $ joining the two endpoints. Since if the new graph works then the previous graph works too. [Several cases to show here] And if there is no edge joining the two endpoints, replace the path by joining the two endpoints by a edge $ 2 $.\\
				
			Now there are only edge $ 1 $ s, and lone edge $ 2 $ s. Now dividing the graph $ 1 $ into paths of edge $ 1 $ , and dealing with several small cases, we are done.}
		

		\prob{https://artofproblemsolving.com/community/c6h276187p1494557}{Iran TST 2009 P6}{E-M}{We have a closed path that goes from one vertex to another neighboring vertex, on the vertices of a $ n\times n$ square which pass throgugh each vertex exactly once. Prove that we have two adjacent vertices such that if we cut the path at these two points then the length of each open paths is at least $ \dfrac{n^2}{4} $.}
		
			\solu{Draw a path, isn't it a unit square tiled path? Now can we relate the area of the tiled path with its perimeter? If we could do that, we would be able to replace two neighboring vertices by an edge inside the path, which seems to make the problem simpler.}
			
		
		\prob{https://math.stackexchange.com/questions/1439430/algorithm-to-uniquely-determine-a-number-using-two-adjacent-digits}{OC Chap2 P2}{M}{Arutyun and Amayak perform a magic trick as follows. A spectator writes down on a board a sequence of $ N $ (decimal) digits. Amayak covers two adjacent digits by a black disc. Then Arutyun comes and says both closed digits (and their order). For which minimal $ N $ can this trick always work? NOTE: Arutyun and Amayak have a strategy determined beforehand.}\label{problem:bijection_2}\label{problem:hall_marriage_1}
		
			\solu{We have to actually find a bijection between all of the combinations the spectator can create, and all of the combinations that Arutyun might see when he comes back. Which tells us to use ``Perfect Matching" tricks.}
			
			\solu{Existential proof: for this trick to always work, they have to make a bijection from a set of $ N $ digits with two covered, to an unique set of $ N $ digits. Consider a bijection from the set of $ 0-9 $ strings with length $ N $ to the set of $ 0-9 $ strings with length $ N $ with $ 2 $ adjacent digits unknown. There exist a bijection iff the two sets satisfy Hall's Marriage Theorem. By double counting we get the value of $ N $ from here.}
		
		
		\prob{https://artofproblemsolving.com/community/c6h35320p220234}{ARO 2005 P9.4}{M}{ $ 100 $ people from $ 50 $ countries, two from each countries, stay on a circle. Prove that one may partition them onto $ 2 $ groups in such way that neither no two countrymen, nor three consecutive people on a circle, are in the same group.\\
			
			\textbf{Variant:} There are $ 100 $ people from $ 25 $ countries sitting around a circular table. Prove that they can be separated into four classes, so that no two countrymen are in the same class, nor any two people sitting adjacent in the circle.}\label{problem:hall_marriage_2}
		
		
		\prob{https://artofproblemsolving.com/community/c6h514376p2889829}{ARO 1999 P10.1}{E}{There are three empty jugs on a table. Winnie the pooh, Rabbit, and Piglet put walnuts in the jugs one by one. They play successively, with the initial determined by a draw. Thereby Winnie the pooh plays either in the first or second jug, Rabbit in the second or third, and Piglet in the first or third. The player after whose move there are exactly 1999 walnuts loses the games. Show that Winnie the pooh and Piglet can cooperate so as to make Rabbit lose.}
		
		
		\prob{https://artofproblemsolving.com/community/c6h5393p17438}{USAMO 2004, P4}{E}{Alice and Bob play a game on a $ 6 $ by $ 6 $ grid. On his or her turn, a player chooses a rational number not yet appearing in the grid and writes it in an empty square of the grid. Alice goes first and then the players alternate. When all squares have numbers written in them, in each row, the square with the greatest number in that row is colored black. Alice wins if she can then draw a line from the top of the grid to the bottom of the grid that stays in black squares, and Bob wins if she can't. (If two squares share a vertex, Alice can draw a line from one to the other that stays in those two squares.) Find, with proof, a winning strategy for one of the players.}
			
\newpage\section{Game Theory}

\subsection{Games}


	\den{\href{file:///home/ahsan/pDB/main/combi/Game Theory/3.htm}{Nimbers:} Nimbers are simply `Nim values' which are assigned to a game configuration - these values are written as $ 0, *1, *2, *3 \dots $ We shall first describe how to obtain the Nim values for the game Squaring the Number. First, the Nim value of $ n=0 $ is assigned $ 0 $, since it is a state in which neither player has a valid move. We then recursively adopt the following rule for each $ n $ : \texttt{find all the possible moves from $ n $ and pick the smallest Nim value which does not occur among all these possible moves}.}


	\theo{file:///home/ahsan/pDB/main/combi/Game Theory/3.htm}{Sprague-Grund Theorem}{The \emph{Sprague–Grundy theorem} states that every impartial game under the normal play convention is equivalent to a nimber.}



	\khela{https://en.wikipedia.org/wiki/Chip-firing_game}{Chip Firing Game}{Let $ G=(V, E) $ be a graph without any loops or multiedges. Let a number of $ s_i $ chips be stacked on vertex $ i $. The game follows with the player choosing a vertex $ i $, taking $ d_i $ chips from it ($ s_i-d_i > 0 $), and sending one chip to each of the neighbors of the vertex where $ d_i $ is the degree of $ i $. The Problem of this game is to determine when the game will be infinte.\\

		\begin{itemize}
			\item If $ N $ is the total number of edges in $ G $, and $ S $ is the total number of chips, then
	\end{itemize}}



	\khela{}{Cutting a stack in half}{Given a number of stacks, at his/her move, a player can choose a stack with even number stones, and divide it in two stacks with the same number of stones.}


	\khela{}{Cutting a stack in several}{Given a number of stacks, at his/her move, a player can choose a stack, and divide it in several stacks with the same number of stones.}






\newpage\subsection{Problems}



	\prob{https://artofproblemsolving.com/community/c5h202910p1116189}{USAMO 2008 P5}{M}{Three non-negative real numbers $ r_1 $ , $ r_2 $ , $ r_3 $ are written on a blackboard. These numbers have the property that there exist integers $ a_1 $ , $ a_2 $ , $ a_3 $ , not all zero, satisfying $ a_1r_1 + a_2r_2 + a_3r_3 = 0 $. We are permitted to perform the following operation: find two numbers $ x $ , $ y $ on the blackboard with $ x \le y $ , then erase $ y $ and write $ y - x $ in its place. Prove that after a finite number of such operations, we can end up with at least one $ 0 $ on the blackboard.}\label{problem:add_stuffs_2}

		\solu{When can't get info out of the reals, try the integers. Observe the integers, and check if they have any invariant. Rule of thumb of finding an invariant.}



	\prob{www.hehe.com}{USAMO 2014 P1}{E}{Let $ k $ be a positive integer. Two players $ A $ and $ B $ play a game on an infinite grid of regular hexagons. Initially all the grid cells are empty. Then the players alternately take turns with $ A $ moving first. In his move, $ A $ may choose two adjacent hexagons in the grid which are empty and place a counter in both of them. In his move, $ B $ may choose any counter on the board and remove it. If at any time there are $ k $ consecutive grid cells in a line all of which contain a counter, $ A $ wins. Find the minimum value of $ k $ for which $ A $ cannot win in a finite number of moves, or prove that no such minimum value exists.}\label{problem:coloring_1}

		\solu{Trying to block $ A $. We see that if we could alternately color the points black and white, we could've found some strategy for $ B $. But the triangle grid doesn’t seem very friendly. How can we color the triangles? And don't forget the details idiot.}




	\prob{www.hehe.com}{Indian TST 2004}{M}{The game of pebbles is played as followed: Initially there is one pebble at $ (0, 0) $. In a move one can remove the pebble at $ (i, j) $ and put one pebble each at $ (i+1, j) $ and $ (i, j+1) $ , given that both $ (i+1, j) $ and $ (i, j+1) $ were empty. Prove that at any point in the game, there will be a pebble at some lattice point $ (a, b) $ with $ a+b\leq 3 $.}\label{problem:invariant_rules_of_thumb_3}

		\solu{Two from one, means if the weight is reduced by half in the second level, then the sum would be the same.}





	\prob{https://artofproblemsolving.com/community/c6h18505p124463}{ISL 1998 C7}{H}{A solitaire game is played on an $ m\times n $ rectangular board, using $ mn $ markers which are white on one side and black on the other. Initially, each square of the board contains a marker with its white side up, except for one corner square, which contains a marker with its black side up. In each move, one may take away one marker with its black side up, but must then turn over all markers which are in squares having an edge in common with the square of the removed marker. Determine all pairs $ (m,n) $ of positive integers such that all markers can be removed from the board.}\label{problem:invariant_rules_of_thumb_4}

		\solu{If we remove one marker, then this cell becomes useless. So the neighbors to this cell will act like they are not connected to this cell. Now if a cell is connected to $ w $ white cells, and $ b $ black cells, then the resulting board state will have $ b-w $ more cells. Now only this info doesn't build up an invariant. Notice that as we are doing moves, we are reducing neighborhood relations as well, in other words, neighborhood relations decrease by $ b+w $. So if we consider the sum $ W+E $ where $ W $ is the number of all white cells, and $ E $ is the number of all neighborhood relations, we get an invariant on this value.}


	\prob{https://artofproblemsolving.com/community/c6h514376p2889829}{ARO 1999 P10.1}{E}{There are three empty jugs on a table. Winnie the pooh, Rabbit, and Piglet put walnuts in the jugs one by one. They play successively, with the initial determined by a draw. Thereby Winnie the pooh plays either in the first or second jug, Rabbit in the second or third, and Piglet in the first or third. The player after whose move there are exactly 1999 walnuts loses the games. Show that Winnie the pooh and Piglet can cooperate so as to make Rabbit lose.}

\newpage\section{Combinatorial Geometry}


Some Notes:

\begin{enumerate}
	\item \href{https://blogm4e.files.wordpress.com/2016/08/combinatorial-geometry-maria-monks-mop-2010.pdf}{Combinatorial Geometry - Maria Monk (MOP 2010)}
\end{enumerate}


\subsection{Lemma}


\newpage\subsection{Problems}


	\prob{https://artofproblemsolving.com/community/c6h535002p3067558}{ARO 2013 P9.4}{E}{$ N $ lines lie on a plane, no two of which are parallel and no three of which are concurrent. Prove that there exists a non-self-intersecting broken line $ A_1A_2A_3\dots A_N $ with $ N $ parts, such that on each of the $ N $ lines lies exactly one of the $ N $ segments of the line.}\label{problem:constructive_algo_8}\label{problem:induction_type1_7}



	\prob{https://artofproblemsolving.com/community/c6h1424941p8024557}{EGMO 2017 P3}{M}{There are $ 2017 $ lines in the plane such that no three of them go through the same point. Turbo the snail sits on a point on exactly one of the lines and starts sliding along the lines in the following fashion: she moves on a given line until she reaches an intersection of two lines. At the intersection, she follows her journey on the other line turning left or right, alternating her choice at each intersection point she reaches. She can only change direction at an intersection point. Can there exist a line segment through which she passes in both directions during her journey?}\label{problem:plane_coloring_1}

	\solu{The condition that tells us to go either right or left, seems very non-rigorous. So to rigorize this condition, instead of using right or left condition in the direction, we consider what’s on our right and left. (INTUITION) After some experiment we see (not all of us) that if we color the plane with two colors in a way where every neighboring regions have different colors, we find some interesting stuff. (CREATIVITY) With this we are done. \hrf{plane_coloring}{Color the Plane}}





	\prob{https://artofproblemsolving.com/community/c6h101306p571973}{ISL 2006 C2}{TE}{Let $ P $ be a regular $ 2006 $ -gon. A diagonal is called good if its endpoints divide the boundary of $ P $ into two parts, each composed of an odd number of sides of $ P $. The sides of $ P $ are also called good.

		Suppose $ P $ has been dissected into triangles by $ 2003 $ diagonals, no two of which have a common point in the interior of $ P $. Find the maximum number of isosceles triangles having two good sides that could appear in such a configuration.}\label{problem:induction_type2_4}\label{problem:bijection_9}


	\solu{The straight way, induction.}

	\solu{The intuitive way, bijection. There are at most $ n $ good triangles, there are $ 2n $ edges, so a mapping that takes a two edges to a single good triangle must exist. Finding it is not that hard.}



	\prob{https://artofproblemsolving.com/community/c6h589865p3493114}{ARO 2014 P9.3}{E}{In a convex $ n $ -gon, several diagonals are drawn. Among these diagonals, a diagonal is called good if it intersects exactly one other diagonal drawn (in the interior of the $ n $ -gon). Find the maximum number of good diagonals.}\label{problem:induction_type2_1}

	\solu{There can be two cases, two good diagonals intersecting each other, and no two good diagonals intersecting each other. In the first case, we just use induction, and in the later, all of the good diagonals create a ``triangulation'' of the polygon, which gives us the numbers.}


	\prob{https://artofproblemsolving.com/community/c6h1181527p5720110}{ISL 2013 C2, IMO 2013 P2}{E}{A configuration of $ 4027 $ points in the plane is called Colombian if it consists of $ 2013 $ red points and $ 2014 $ blue points, and no three of the points of the configuration are collinear. By drawing some lines, the plane is divided into several regions. An arrangement of lines is good for a Colombian configuration if the following two conditions are satisfied:

		\begin{enumerate}

			\item No line passes through any point of the configuration.
			\item No region contains points of both colors.

		\end{enumerate}

		Find the least value of $ k $ such that for any Colombian configuration of $ 4027 $ points, there is a good arrangement of $ k $ lines.}\label{problem:sandwiching_points_1}

	\solu{Obviously a n00b would think about induction. The only problem occurs when the convex hull completely consists of red points. In this case, after some investigation, we should get the sandwiching two points idea.}

	\solu{Another way of inductive approach is like this, as the problem condition says that no region contains points of both colors, which means if we connect any two red and blue points, some line must bisect this segment. Now \hrf{problem:convex_hull_1}{it is known} that there is non intersecting partition of the points in to red-blue segments. So suppose in such a partition, we draw bisectors of each segments. Now there will be some holes in this proof. We see that to fill this holes, we have to focus on two red points with their respective blue partners, and draw the two bisectors in a way that separates the two red points form the blue points. So to remove further holes, we get the sandwiching idea.}





	\prob{www.hehe.com}{Putnam 1979\label{Putnam_1979}}{E}{Let $ A $ be a set of $ 2n $ points in the plane, no three of which are collinear, $ n $ of them are colored red and the other blue. Prove that there are $ n $ line segments, no two with a point in common, such that the endpoints of each segment are points of $ A $ having different colors.}\label{problem:convex_hull_1}

	\solu{Strong induction, and a way to divide the points into two sets with the same number of red and blue points. Travel through the points around a certain point and keep track of the number of red and blue points.}




	\prob{https://artofproblemsolving.com/community/c6h34319p213018}{USAMO 2005 P5}{E}{Let $n$ be an integer greater than 1. Suppose $2n$ points are given in the plane, no three of which are collinear. Suppose $n$ of the given $2n$ points are colored blue and the other $n$ colored red. A line in the plane is called a balancing line if it passes through one blue and one red point and, for each side of the line, the number of blue points on that side is equal to the number of red points on the same side. Prove that there exist at least two balancing lines.}\label{problem:convex_hull_3}

	\solu{Using the same idea as in \hrf{Putnam_1979}{this} problem. }




	\prob{https://artofproblemsolving.com/community/c6h366745p2018324}{ILL 1985}{E}{Let $A$ and $B$ be two finite disjoint sets of points in the plane such that no three distinct points in $A \cup B$ are collinear. Assume that at least one of the sets $A, B$ contains at least five points. Show that there exists a triangle all of whose vertices's are contained in $A$ or in $B$ that does not contain in its interior any point from the other set.}

	\solu{Concentrating on one of the sets five points such that there is no other points of the same set inside the hull of those five points.}\label{problem:convex_hull_2}



	\prob{https://artofproblemsolving.com/community/c6h79789p456611}{APMO 1999 P5}{M}{Let $S$ be a set of $2n+1$ points in the plane such that no three are collinear and no four concyclic. A circle will be called ``Good'' if it has $ 3 $ points of $S$ on its circumference, $n-1$ points in its interior and $n-1$ points in its exterior. Prove that the number of good circles has the same parity as $n$.}


		\solu{When thinking about induction, got a feeling that double counting with the number of good circles going through pairs of points might be useful, because a good circle will be counted three times, if we can show that every pair has odd number of good circles, we are done. So, take a pair. Now we need to `sort' the points somehow. See that, we can't sort the points in a trivial way with numbers, so moving to angles. Now setting conditions for a point inside of a circle in terms of angles, we see amazing patter, and an easy way to calculate the number of good circle of that pair of points.}



	\prob{https://artofproblemsolving.com/community/c6h1113183p5083543}{ISL 2014 C1}{E}{Let $ n $ points be given inside a rectangle $ R $ such that no two of them lie on a line parallel to one of the sides of $ R $. The rectangle $ R $ is to be dissected into smaller rectangles with sides parallel to the sides of $ R $ in such a way that none of these rectangles contains any of the given points in its interior. Prove that we have to dissect $ R $ into at least $ n + 1 $ smaller rectangles.}\label{problem:extreme_object_7}\label{problem:double_counting_3}

	\solu{Work with the largest continuous segments, and their endpoints.}



	\prob{https://artofproblemsolving.com/community/c6h195050p1071295}{ISL 2007 C2}{EM}{A rectangle $ D$ is partitioned in several ($ \ge2$) rectangles with sides parallel to those of $ D$. Given that any line parallel to one of the sides of $ D$, and having common points with the interior of $ D$, also has common interior points with the interior of at least one rectangle of the partition; prove that there is at least one rectangle of the partition having no common points with $ D$'s boundary.}\label{problem:extreme_object_17}

		\solu{There existing such a rectangle means that there is a rectangular region inside of the original rectangle. So what if we walked along the segments, and cut a smaller rectangle from the inside of the rectangle? Like the way in the game.}

		\solu{Starting from one corner, and taking the opposite corner of the rectangle containing that corner, we use infinite decent to reach a contradiction.}

		\solu{Using \hrf{problem:extreme_object_7}{ISL 2014 C1} as a lemma.}


	
	\prob{https://artofproblemsolving.com/community/c6h5753p18977}{ISL 2003 C2}{E}{Let $D_1$, $D_2$, ..., $D_n$ be closed discs in the plane. (A closed disc is the region limited by a circle, taken jointly with this circle.) Suppose that every point in the plane is contained in at most $2003$ discs $D_i$. Prove that there exists a disc $D_k$ which intersects at most $7\cdot 2003 - 1 = 14020$ other discs $D_i$.}
	
		\solu{Just go with the natural idea.}
	
	\prob{https://artofproblemsolving.com/community/c6h5785p19086}{ISL 2003 C3}{E}{Let $n \geq 5$ be a given integer. Determine the greatest integer $k$ for which there exists a polygon with $n$ vertices (convex or not, with non-selfintersecting boundary) having $k$ internal right angles.}\label{problem:double_counting_10}
	
		\solu{double count}
	
	
	
	\prob{https://artofproblemsolving.com/community/c6h1389042p7736716}{Tournament of Towns 2015S S4}{A convex$N-$gon with equal sides is located inside a circle. Each side is extended in both directions up to the intersection with the circle so that it contains two new segments outside the polygon. Prove that one can paint some of these new $2N$ segments in red and the rest in blue so that the sum of lengths of all the red segments would be the same as for the blue ones.}
	
		\solu{Just use what's the most natural, POP, on one vertex point.}


	\prob{https://artofproblemsolving.com/community/c6h145844p825495}{USAMO 2007 P2}{E}{A square grid on the Euclidean plane consists of all points $(m,n)$, where $m$ and $n$ are integers. Is it possible to cover all grid points by an infinite family of discs with non-overlapping interiors if each disc in the family has radius at least $5$?}
	
		
	\prob{https://artofproblemsolving.com/community/c6h1135648p5301617}{MEMO 2015 T4}{EM}{Let $N$ be a positive integer. In each of the $N^2$ unit squares of an $N\times N$ board, one of the two diagonals is drawn. The drawn diagonals divide the $N\times N$ board into $K$ regions. For each $N$, determine the smallest and the largest possible values of $K$.
		\fig{.6}{MEMO2015T4}{}
	}
	
		\solu{An Algorithmic Approach: Consider each diagonal as $ 0 $ or $ 1 $, prove that the maximum configuration is the one with alternating $ 0, 1 $s and the minimum one is the one with all $ 0 $s.}
		
		\solu{A Counting Approach: Just count and bound with the minimum areas of the regions.}
\newpage\section{Sequences}


	\Faka\subsection{Lemmas}


		\theo{https://en.wikipedia.org/wiki/Van_der_Waerden's_theorem}{Van der Waerden's Theorem}{For any given positive integers$  r $ and $ k $, there is some number $ N $ such that if the integers $ \{1, 2, ..., N\} $ are colored, each with one of $ r $ different colors, then there are at least $ k $ integers in arithmetic progression all of the same color.}



	\Faka\subsection{Generating Function Lemmas}


		\lem{The infinite series defined as following: \[ a_0 = a_1 = 1,\ a_n = \prod_{i=0}^{n} a_ia_{n-i+1} = a_0a_{n-1} + a_1a_{n-2}\dots + a_{n-1}a_0 \] has the general term $ a_n = C_n = \frac{1}{n+1} \binom{2n}{n} $}\label{lemma:catalan_recursion}


		\begin{multicols}{2}
			\begin{enumerate}[wide=0em, label=\arabic*, itemsep=0pt, parsep=0pt, font=\footnotesize\bfseries]

				\iref{problem:catalan_recursion_1}{Problem to do with generating function}{}
			\end{enumerate}
		\end{multicols}



	\Faka\subsection{Sequence Problems}
	
		\href{http://alexanderrem.weebly.com/uploads/7/2/5/6/72566533/sequences.pdf}{Sequences - Alexander Remorov}



		\prob{https://artofproblemsolving.com/community/c6h68945p404543}{ISL 1990}{E}{Assume that the set of all positive integers is decomposed into $ r $ (disjoint) subsets $ A_1 \cup A_2 \cup \dots \cup A_r = \mathbb{N}. $ Prove that one of them, say $ A_i, $ has the following property: There exists a positive $ m $ such that for any $ k $ one can find numbers $ a_1, a_2, \ldots, a_k $ in $ A_i $ with $ 0 < a_{j+1} - a_j \leq m, $  $ (1 \leq j \leq k-1) $.}\label{problem:induction_type1_14}



		\prob{https://mathoverflow.net/questions/25313/finitely-many-arithmetic-progressions}{Result by Erdos, Dividing the integers into arithmetic progressions}{E}{Let $ d_1, d_2,\dots, d_k $ be differences of $ k $ arithmetic progressions that partition $ \N $. Show that $ d_i=d_j $ for some $ i,j $.}\label{problem:generating_function_2}\label{problem:roots_of_unity_1}



		\prob{https://artofproblemsolving.com/community/c6h79784p456601}{APMO 1999 P1}{E}{Find the smallest positive integer $n$ with the following property: there does not exist an arithmetic progression of $1999$ real numbers containing exactly $n$ integers.}



		\prob{https://artofproblemsolving.com/community/c6h79787p456607}{APMO 1999 P2}{E}{Let $a_1, a_2, \dots$ be a sequence of real numbers satisfying $a_{i+j} \leq a_i+a_j$ for all $i,j=1,2,\dots$. Prove that
			\[ a_1 + \frac{a_2}{2} + \frac{a_3}{3} + \cdots + \frac{a_n}{n} \geq a_n \]
			for each positive integer $n$.}\label{problem:induction_type2_5}



		\prob{https://artofproblemsolving.com/community/c6h125791p713182}{ISL 1994 A1}{E}{Let $ a_{0} = 1994$ and $ a_{n + 1} = \frac {a_{n}^{2}}{a_{n} + 1}$ for each nonnegative integer $ n$. Prove that $ 1994 - n$ is the greatest integer less than or equal to $ a_{n}$, $ 0 \leq n \leq 998$}
		
			\solu{Take the differences.}


		
		\prob{https://artofproblemsolving.com/community/c6h214669p1186971}{ISL 2007 C4}{M}{Let $ A_0 = (a_1,\dots,a_n)$ be a finite sequence of real numbers. For each $ k\geq 0$, from the sequence $ A_k = (x_1,\dots,x_k)$ we construct a new sequence $ A_{k + 1}$ in the following way.
		
		\begin{enumerate}
			\item We choose a partition $ \{1,\dots,n\} = I\cup J$, where $ I$ and $ J$ are two disjoint sets, such that the expression \[ \left|\sum_{i\in I}x_i - \sum_{j\in J}x_j\right| \] attains the smallest value. (We allow $ I$ or $ J$ to be empty; in this case the corresponding sum is 0.) If there are several such partitions, one is chosen arbitrarily.
			\item We set $ A_{k + 1} = (y_1,\dots,y_n)$ where $ y_i = x_i + 1$ if $ i\in I$, and $ y_i = x_i - 1$ if $ i\in J$.
		\end{enumerate}
		
		Prove that for some $ k$, the sequence $ A_k$ contains an element $ x$ such that $ |x|\geq\frac n2$.}\label{problem:invariant_rules_of_thumb_11}
	
			\solu{Suppose the contrary. Now, since $ A_i $ can only attain finite values, So $ A_i = A_j $ for some $ i, j $. Now, we are taking about changes here, so we need to think of some invariants. Firstly the sum, it's not much of an help, because it doesn't give us much control. So kinda sum-ish invariant with a bit more control is the sum of squares. We combine these two ideas.}
		
		
		
		\prob{https://artofproblemsolving.com/community/c6h288840p1561573}{ISL 2009 A6}{EM}{Suppose that $ s_1,s_2,s_3, \ldots$ is a strictly increasing sequence of positive integers such that the sub-sequences \[s_{s_1},\, s_{s_2},\, s_{s_3},\, \ldots\qquad\text{and}\qquad s_{s_1+1},\, s_{s_2+1},\, s_{s_3+1},\, \ldots\] are both arithmetic progressions. Prove that the sequence $ s_1, s_2, s_3, \ldots$ is itself an arithmetic progression.}
		
			\solu{First notice that the two arithmetic sequences has the same common difference. Then notice that the diffrences of the original sequence is bounded. Another advice, give everything names. After naming the smallest difference and the largest differnce, we get two different inequalities, from where we deduce that the difference is constant.}
			
		
		
		
		

\newpage\section{Exploring Configurations}

	Problems where there is some kind of a configuration is given, the question usually asks to proof or find some specific properties of the configuration.


	\subsection{Problems}



		\prob{https://artofproblemsolving.com/community/c6h1446909p8271411}{APMO 2017 P3}{H}{Let $ A(n) $ denote the number of sequences $ a_1\geq a_2\geq \dots\geq a_k $ of positive integers for which   $ \sum_{i=1}^k a_k =n $ and each $ a_i+1 $ is a power of two. Let $ B(n) $ denote the number of sequences $ b_1\geq b_2\geq\dots\geq b_k $ of positive integers for which $ \sum_{i=1}^k b_k =n $ and each inequality $ b_j\geq 2b_{j+1} $ holds $ \left( j=1,2\dots m-1\right) $. Prove that $ \vert A(n)\vert =\vert B(n)\vert $ for every positive integer. }\label{problem:add_stuffs_1}\label{problem:bijection_6}

			\solu{A sequence of the first type can be rewritten as: \[ n=x_1+3x_2+7x_3\dots +(2^i-1)x_i+\dots (2^k-1)x_k \] Where $ x_i $ are non-negative integers. This motivates us to find a way to represent $ b_i $ as sums of $ ( 2^i-1)x_i $. Then since $ b_j\geq 2b_{j+1} $, we write: $ b_i=2b_{i-1}+x_i $ with $ x_i $ being non-negative integers.}




		\prob{https://artofproblemsolving.com/community/c6h215429p1191679}{ISL 2008 C4}{M}{Let $ n $ and $ k $ be positive integers with $ k\geq n $ and $ k-n $ an even number. Let $ 2n $ lamps labeled $ 1,2\dots 2n $ be given, each of which can be either on or off. Initially, all the lamps are off. We consider sequences of steps: at each step one of the lamps is switched (from on to off or from off to on).\\

		Let $ N $ be the number of such sequences consisting of $ k $ steps and resulting in the state where lamps $ 1 $ through $ n $ are all on, and lamps $ n+1 $ through $ 2n $ are all off.\\

		Let $ M $ be number of such sequences consisting of $ k $ steps, resulting in the state where lamps $ 1 $ through $ n $ are all on, and lamps $ n+1 $ through $ 2n $ are all off, but where none of the lamps $ n+1 $ through $ 2n $ is ever switched on.\\

		Determine $ \frac{N}{M} $.}\label{problem:bijection_5}

			\solu{These type of problems most of the time have bijection or algo solutions. Think of a way to perform bijection from the set $ S\{M\} \rightarrow S\{N\} $. Find an algorithm to get a sequence of the first type from a sequence of the second type.}




		\prob{https://artofproblemsolving.com/community/c6h57380p353058}{USAMO 1996 P4}{E}{An $ n $ -term sequence $ (x_1, x_2, \ldots, x_n) $ in which each term is either 0 or 1 is called a binary sequence of length $ n $. Let $ a_n $ be the number of binary sequences of length $ n $ containing no three consecutive terms equal to 0, 1, 0 in that order. Let $ b_n $ be the number of binary sequences of length $ n $ that contain no four consecutive terms equal to 0, 0, 1, 1 or 1, 1, 0, 0 in that order. Prove that $ b_{n+1} = 2a_n $ for all positive integers $ n $.}\label{problem:bijection_4}

		\solu{These type of problems cries for a nice bijection. That is a way to get from $ a\rightarrow b $ and vice versa. What if there is no $ 0,0,1,1 $ ? Or what if there is no $ 0,1,0 $ ? What is an one way bijection?}




		\prob{https://artofproblemsolving.com/community/c6h1446905p8271388}{APMO 2017 P1}{M}{We call a $ 5 $ -tuple of integers arrangeable if its elements can be labeled $ a,b,c,d,e $ in some order so that $ a-b+c-d+e=29 $. Determine all $ 2017 $ -tuples of integers $ n_1,n_2,n_3\dots n_{2017} $ such that if we place them in a circle in clockwise order, then any $ 5 $ -tuple of numbers in consecutive positions on the circle is arrangeable. }\label{problem:minus_constant_1}\label{problem:invariant_rules_of_thumb_5}

		\solu{\hrf{minus_constant}{EChen trick}.}



		\prob{https://artofproblemsolving.com/community/c6h41116p258304}{ISL 2004 C1}{E}{There are $ 10001 $ students at an university. Some students join together to form several clubs (a student may belong to different clubs). Some clubs join together to form several societies (a club may belong to different societies). There are a total of $ k $ societies. Find all possible values of $ k $ so that the following conditions are satisfied:

			\begin{enumerate}[wide=0em, label=\arabic*, itemsep=0pt, parsep=0pt, font=\footnotesize\bfseries]

				\item  Each pair of students are in exactly one club.

				\item  For each student and each society, the student is in exactly one club of the society.

				\item  Each club has an odd number of students. In addition, a club with $ {2m+1} $ students ( $ m $ is a positive integer) is in exactly $ m $ societies.
		\end{enumerate}}\label{problem:double_counting_7}

		\solu{Just Double-Counting.}

		
		\prob{https://artofproblemsolving.com/community/c6h17336p118710}{ISL 2002 C1}{E}{Let $ n $ be a positive integer. Each point $ (x,y) $ in the plane, where $ x $ and $ y $ are non-negative integers with $ x+y<n $ , is coloured red or blue, subject to the following condition: if a point $ (x,y) $ is red, then so are all points $ (x',y') $ with $ x'\leq x $ and $ y'\leq y $. Let $ A $ be the number of ways to choose $ n $ blue points with distinct $ x $ -coordinates, and let $ B $ be the number of ways to choose $ n $ blue points with distinct $ y $ -coordinates. Prove that $ A=B $.}\label{problem:induction_type1_5}\label{problem:recursive_solution_2}\label{problem:bijection_1}


		\prob{https://artofproblemsolving.com/community/c5h476723p2669115}{USAMO 2012 P2}{M}{A circle is divided into $ 432 $ congruent arcs by $ 432 $ points. The points are colored in four colors such that some $ 108 $ points are colored Red, some $ 108 $ points are colored Green, some $ 108 $ points are colored Blue, and the remaining $ 108 $ points are colored Yellow. Prove that one can choose three points of each color in such a way that the four triangles formed by the chosen points of the same color are congruent.}\label{problem:double_counting_6}

		\solu{Double counting saves the day :) The trick is to rotate ;)}




		\prob{https://artofproblemsolving.com/community/c6h195492p1073989}{APMO 2008 P2}{EM}{Students in a class form groups each of which contains exactly three members such that any two distinct groups have at most one member in common. Prove that, when the class size is $ 46 $ , there is a set of $ 10 $ students in which no group is properly contained.}\label{problem:bijection_8}\label{problem:extremal_case_whole_5}

		\solu{Taking the maximum set that follows the ``in which no group is properly contained'' rule. Now the elements that are \emph{not} in this set, we can connect this element to only one of the pairs from the set. Now defining a bijection, and counting the elements, we are done.}



		\prob{https://artofproblemsolving.com/community/c6h364231p2000940}{IMO SL 1985}{M}{A set of $ 1985 $ points is distributed around the circumference of a circle and each of the points is marked with $ 1 $ or $ -1 $. A point is called ``good'' if the partial sums that can be formed by starting at that point and proceeding around the circle for any distance in either direction are all strictly positive. Show that if the number of points marked with $ -1 $ is less than $ 662 $ , there must be at least one good point.}\label{problem:induction_type1_12}

		\solu{First thing to notice, the number $ 3*661 + 2 = 1985 $. And these numbers are completely random. So what if we try to replace $ 1985 $ by $ n $ ? Will the condition still hold?}


		\prob{https://artofproblemsolving.com/community/c6h418978p2365036}{IMO 2011 P4}{E}{Let $ n > 0 $ be an integer. We are given a balance and $ n $ weights of weight $ 2^0, 2^1, \cdots, 2^{n-1} $. We are to place each of the $ n $ weights on the balance, one after another, in such a way that the right pan is never heavier than the left pan. At each step we choose one of the weights that has not yet been placed on the balance, and place it on either the left pan or the right pan, until all of the weights have been placed. Determine the number of ways in which this can be done.}\label{problem:recursive_solution_5}

		\solu{Writing the whole process as a sum, we see that only $ 2^0 $ is the odd term here, if we remove that we can divide by $ 2 $ to get a recursive formula.}

		\solu{Calculating wrt to the last placed weight.}

		\solu{Getting recursive formula considering the position of $ 2^{n-1} $.}





		\prob{https://artofproblemsolving.com/community/c5h202936p1116367}{USAMO 2008 P3}{H}{Let $ n $ be a positive integer. Denote by $ S_n $ the set of points $ (x, y) $ with integer coordinates such that \[ \left\lvert x\right\rvert + \left\lvert y + \frac{1}{2} \right\rvert < n. \] A path is a sequence of distinct points $ (x_1 , y_1), (x_2, y_2), \ldots, (x_\ell, y_\ell) $ in $ S_n $ such that, for $ i = 2, \ldots, \ell $ , the distance between $ (x_i , y_i) $ and $ (x_{i-1} , y_{i-1} ) $ is $ 1 $ (in other words, the points $ (x_i, y_i) $ and $ (x_{i-1} , y_{i-1} ) $ are neighbors in the lattice of points with integer coordinates). Prove that the points in $ S_n $ cannot be partitioned into fewer than $ n $ paths (a partition of $ S_n $ into $ m $ paths is a set $ \mathcal{P} $ of $ m $ nonempty paths such that each point in $ S_n $ appears in exactly one of the $ m $ paths in $ \mathcal{P} $ ).}\label{problem:alternating_chains_2}\label{problem:coloring_2}


		\solu{Graph + Partition, coloring is just natural. Again, the edges join two neighbor lattice points, so checkerboard coloring. But checkerboard doesn't do much good. So the next thing we try is to apply some derivations of it, pseudo!!! Well, overkill.}


		\solu{For all n, induction is very natural. The optimal partition (the most beautiful one) and the longest path in it, say $ P $ , gives us a way to perform induction. As always, we suppose a partition with $ n-1 $ paths. As there are a lot of partitions, we need to choose a certain partition, say $ \mathbb{M} $. Again as our goal is to include $ P $ in $ \mathbb{M} $. So suppose that the set with all the points in $ P $ is $ A $. And further more, suppose that in $ \mathbb{M} $ there is a path $ Q $ with $ \vert Q\cap A\vert $ being maximal among all other partitions of the points. Some some easy case work shows that we must have $ P\in \mathbb{M} $.}



		\prob{https://artofproblemsolving.com/community/c5h532235p3041823}{USAMO 2013 P2}{H}{For a positive integer $ n\geq 3 $ plot $ n $ equally spaced points around a circle. Label one of them $ A $ , and place a marker at $ A $. One may move the marker forward in a clockwise direction to either the next point or the point after that. Hence there are a total of $ 2n $ distinct moves available; two from each point. Let $ a_n $ count the number of ways to advance around the circle exactly twice, beginning and ending at $ A $ , without repeating a move. Prove that $ a_{n-1}+a_n=2^n $ for all $ n\geq 4 $.}\label{problem:recursive_solution_4}\label{problem:bijection_7}


		\solu{Problems where there are multiple possible value of a function regardless of the current position, one of dealing with these is to assigning labels of these possible values to each points of the function, and this will give a combinatorial model and a way to deal it with bijection.}

		\solu{First investigate the problem condition, $ a_n + a_{n-1} = 2^n $ , now, $ 2^n $ means the number of differently coloring every point black or white, and the left side is the number of such paths for $ n $ and $ n-1 $. Which means we should try to color the points and see what happens.}

		\proof{EChen's solution: In this problem, the main obstacle seems to be the circle condition. And on top of that, on can land on the starting point. So things are pretty messed up here. What we want to do is to make things a little bit more easy to deal with. So our best option is to change the problem so that we get the similar problem with a different explanation. So we change the condition circle with matrix, $ 2 $ round with $ 2 $ rows. $ n $ points with $ n $ entries in each rows. What we get now is the same problem, just a bit easier to deal with. We call this \hl{Tweak The Problem} strategy.}



		\prob{}{}{E}{ $ 10 $ persons went to a bookstore. It is known that: Every person has bought 3 kinds on books and for every 2 persons, there is at least one kind of books which they both have bought. Let $ m_i $ be the number of the persons who bought the $ i^{th} $ kind of books and $ M= \max\lbrace m_i\rbrace $ Find the smallest possible value of $ M $.}\label{problem:double_counting_4}



		\prob{https://artofproblemsolving.com/community/c6h17338p118714}{ISL 2002 C3}{EM}{Let $n$ be a positive integer. A sequence of $n$ positive integers (not necessarily distinct) is called full if it satisfies the following condition: for each positive integer $k\geq2$, if the number $k$ appears in the sequence then so does the number $k-1$, and moreover the first occurrence of $k-1$ comes before the last occurrence of $k$. For each $n$, how many full sequences are there?}\label{problem:bijection_13}\label{problem:graph_representation_9}

			\proof{After guessing the ans, the first thing that I did was to draw a level based graph. Suppose that a full sequence has $ k $ different entries. Then the top level contains the positions of $ k $ in the sequence sorted from left to right. The next level contains the positions of $ k-1 $ in the sequence sorted so, and so on till the last level. What I noticed is that if we draw arrows pointing from a larger integer to a smaller integer, the only arrows (or more like relations between entries of the sequence) we need to worry about are the arrows pointing left to right in each levels, and the arrows from the last entry of level $ i $ to the first entry of level $ i+1 $. After this, if we try with a smaller case, we see that this leads to a bijection from the set of sequences of length $ n $ with $ n $ different integers to the set of full-sequences of length $ n $.}


			\solu{Another bijection approach is as followed, in a full-sequence, on first run, go from right to left, placing integers starting with $ 1 $ onwards on the $ 1 $'s in the sequence. on the second run continue counting and placing integers on the $ 2 $'s and so on.}


			\solu{Another idea is to prove $ a_n = n a_{n-1} $. To do this, remove the rightmost $ 1 $ and do some casework.}



		\prob{https://artofproblemsolving.com/community/c6h219938p1219679}{ISL 1994 C2}{M}{In a certain city, age is reckoned in terms of real numbers rather than integers. Every two citizens $x$ and $x'$ either know each other or do not know each other. Moreover, if they do not, then there exists a chain of citizens $x = x_0, x_1, \ldots, x_n = x'$ for some integer $n \geq 2$ such that $ x_{i-1}$ and $x_i$ know each other. In a census, all male citizens declare their ages, and there is at least one male citizen. Each female citizen provides only the information that her age is the average of the ages of all the citizens she knows. Prove that this is enough to determine uniquely the ages of all the female citizens.}

			\solu{Describing the problem using matrix and vector spaces, the problem reduces to well known theorems of linear algebra.}


		\prob{https://artofproblemsolving.com/community/c6h93p261}{ISL 2003 C1}{E}{Let $A$ be a $101$-element subset of the set $S=\{1,2,\ldots,1000000\}$. Prove that there exist numbers $t_1$, $t_2, \ldots, t_{100}$ in $S$ such that the sets \[ A_j=\{x+t_j\mid x\in A\},\qquad j=1,2,\ldots,100 \] are pairwise disjoint.}
		
			\solu{just count...}




	\newpage\subsection{Coloring Problems}


		\lem{What is the maximum number of knights that can be placed on a chessboard such that no two knights attack each other?}\label{lemma:maximum_knight_problem}

			\solu{A knight's move always changes the color of the cell.}



		\prob{https://artofproblemsolving.com/community/c6h1424942p8024575}{EGMO 2017 P2}{M}{Find the smallest positive integer $ k $ for which there exists a colouring of the positive integers $ \mathbb{Z}_{>0} $ with $ k $ colours and a function $ f:\mathbb{Z}_{>0}\to \mathbb{Z}_{>0} $ with the following two properties:

		\begin{enumerate}

			\item For all positive integers $ m,n $ of the same colour, $ f(m+n)=f(m)+f(n). $
			\item There are positive integers $ m,n $ such that $ f(m+n)\ne f(m)+f(n). $

		\end{enumerate}

		In a colouring of $ \mathbb{Z}_{>0} $ with $ k $ colours, every integer is coloured in exactly one of the $ k $ colours. In both $ (i) $ and $ (ii) $ the positive integers $ m,n $ are not necessarily distinct.}\label{problem:extremal_case_whole_4}

			\solu{Firstly a modular coloring shows that $ 1<k\leq 2 $. For $ k=2 $ we do some trivial case works.}



		\prob{https://artofproblemsolving.com/community/c6h17337p118712}{ISL 2002 C2}{E}{For $n$ an odd positive integer, the unit squares of an $n\times n$ chessboard are coloured alternately black and white, with the four corners coloured black. A it tromino is an $L$-shape formed by three connected unit squares. For which values of $n$ is it possible to cover all the black squares with non-overlapping trominos? When it is possible, what is the minimum number of trominos needed?}\label{problem:bijection_12}

			\solu{First find the first ans and a configuration that works. Then guess the second ans, and see from where that might come from, usually these anses come from some special set of problems, where bijection is applicable.}



		\prob{http://codeforces.com/gym/101954/problem/G}{Codeforces 101954/G}{E/H}{Two Knights are given on a chessboard, one black one white. Which player has a winning possibility?}\label{problem:coloring_4}

			\solu{A knight's move always changes the color of the cell.}


		\prob{https://artofproblemsolving.com/community/c6h1671290p10632348P}{IMO 2018 P4}{E/H}{A site is any point $(x, y)$ in the plane such that $x$ and $y$ are both positive integers less than or equal to 20.\\

			Initially, each of the 400 sites is unoccupied. Amy and Ben take turns placing stones with Amy going first. On her turn, Amy places a new red stone on an unoccupied site such that the distance between any two sites occupied by red stones is not equal to $\sqrt{5}$. On his turn, Ben places a new blue stone on any unoccupied site. (A site occupied by a blue stone is allowed to be at any distance from any other occupied site.) They stop as soon as a player cannot place a stone.\\

			Find the greatest $K$ such that Amy can ensure that she places at least $K$ red stones, no matter how Ben places his blue stones.}\label{problem:coloring_3}


			\solu{Using the \hrf{lemma:maximum_knight_problem}{maximum knight problem} as a lemma.}


		\prob{https://artofproblemsolving.com/community/c6h417987p2356844}{ARO 1993 P10.4}{M}{Thirty people sit at a round table. Each of them is either smart or dumb. Each of them is asked: "Is your neighbor to the right smart or dumb?" A smart person always answers correctly, while a dumb person can answer both correctly and incorrectly. It is known that the number of dumb people does not exceed $ F $. What is the largest possible value of $ F $ such that knowing what the answers of the people are, you can point at at least one person, knowing he is smart?}\label{problem:extreme_object_8}
		
			\solu{We see that the strings of truth only exist either when all people are dumb or the last one is the truthful one. Now we take the longest such string, and this sting has to be of the second kind. To prove this, we use bounding with the given constraint.}


		\prob{https://artofproblemsolving.com/community/c6h1389041p7736715}{Tournament of Towns 2015S S6}{E}{An Emperor invited $2015$ wizards to a festival. Each of the wizards knows who of them is good and who is evil, however the Emperor doesn’t know this. A good wizard always tells the truth, while an evil wizard can tell the truth or lie at any moment. The Emperor gives each wizard a card with a single question, maybe different for different wizards, and after that listens to the answers of all wizards which are either “yes” or “no”. Having listened to all the answers, the Emperor expels a single wizard through a magic door which shows if this wizard is good or evil. Then the Emperor makes new cards with questions and repeats the procedure with the remaining wizards, and so on. The Emperor may stop at any moment, and after this the Emperor may expel or not expel a wizard. Prove that the Emperor can expel all the evil wizards having expelled at most one good wizard.}
		
			\solu{There is only one problem with the cyclic arrangement, that is what if all the answers are `yes'? We get rid of this problem by trying small case with $ n=3 $ and trying the most simple way to connect this strategy to any $ n $. Simplicity is the key.}
		
		


