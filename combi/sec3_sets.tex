\newpage\section{Sets}


\subsection{Lemmas}


\lem{}{\label{lemma:sets_lemma_1}
    Let $ S $ be a set with $ n $ elements, and let $ F $ be a family of
    subsets of S such that for any pair $ A, B $ in $ F $, $ A \cap B \not=
    \varnothing $. Then $ |F| \leq 2^{n-1} $ .
}

% \begin{enumerate}
%     \iref{lemma:sets_lemma_1_1}{Iran TST 2008 D3P1}{}
% \end{enumerate}


\theo{https://en.wikipedia.org/wiki/Erdos-Ko-Rado_theorem}{Erdos Ko Rado theorem}{
    Suppose that $ A $ is a family of distinct subsets of $\{ 1 , 2 \dots n
    \}$ such that each subset is of size $ r $ and each pair of subsets has a
    nonempty intersection, and suppose that $ n \geq 2r $. Then the number of
    sets in $ A $ is less than or equal to the binomial coefficient \[\binom
    {n-1}{r-1}\]
}


\lem{}{Let $ S $ be a set with $ n $ elements, and let $ F $ be a family of subsets of $ S $ such that for any pair $ A, B $ in $ F $, $ S $ is not contained by $ A \cup B $. Then $ |F| \leq 2^{n-1} $.}\label{lemma:sets_lemma_2}



\lem{Kleitman lemma}{A set family $F$ is said to be downwards closed if the following holds: if $X$ is a set in $F$, then all subsets of $X$ are also sets in $F$. Similarly, $F$ is said to be upwards closed if whenever $X$ is a set in $F$, all sets containing $X$ are also sets in $F$. Let $F_1$ and $F_2$ be downwards closed families of subsets of $S = \{1, 2, ..., n\}$, and let $F_3$ be an upwards closed family of subsets of $S$. Then we have

    \begin{align}
        |F_1 \cap F_2| &\geq \frac{|F_1| \cdot |F_2|}{2^n}\\
        |F_1 \cap F_3| &\leq \frac{|F_1| \cdot |F_3|}{2^n} 	 
\end{align}}\label{lemma:sets_lemma_3_Kleitman}




\lem{}{Let $ S $ be a set with $ n $ elements, and let $ F $ be a family of subsets of $ S $ such that for any pair $ A, B $ in $ F $, $ A \cap B \not= \varnothing $ and $ A \cap B \not= S $. Then $ |F| \leq 2^{n-2} $.}\label{lemma:sets_lemma_4}

\solu{Using the sets in \hrf{lemma:sets_lemma_1}{lemma 1} and \hrf{lemma:sets_lemma_1}{lemma 2}, defining upwards and downwards sets like in \hrf{lemma:sets_lemma_3_Kleitman}{Kleitman's Lemma}.}



\lem{The Sunflower Lemma}{A sunflower with $ k $ petals and a core $ X $ is a family of sets $ S_1, S_2,\dots, S_k $ such that $ S_i\cap S_j = X $ for each $ i \neq j $. (The reason for the name is that the Venn diagram representation for such a family resembles a sunflower.) The sets $ S_i \setminus X $ are known as petals and must be nonempty, though $ X $ can be empty. Show that if $ F $ is a family of sets of cardinality $ s $, and $ |F| > s!(k-1)^s $, then $ F $ contains a sunflower with $ k $ petals.}\label{problem:induction_type1_20}\label{problem:extreme_object_10}


\solu{Applying induction and considering the best case where $ |X|=0 $}


\newpage
\subsection{Extremal Set Theory}

\href{http://math.mit.edu/~cb_lee/18.318/lecture8.pdf}{MIT 18.314 Lecture-8}


\theo{}{Mirsky Theorem}{A set $ S $ with a chain of height $ h $ can’t be partitioned into $ t $ anti-chains if $ t < h $. In other words, the minimum number of sets in any anti-chain partition of $ S $ is equal to the maximum height of the chains in $ S $. (And Vice Versa)}\label{theorem:mirsky_theorem}


\theo{}{}{In any poset, the largest cardinality of an antichain is at most the smallest cardinality of a chain-decomposition of that poset.}


\theo{}{Dilworth's Theorem}{Let $ P $ be a poset. Then there exist an antichain $ A $ and a chain decomposition $ \mathcal{C} $ of $ P $ such that $ |A| = |\mathcal{C}| $}



\theo{http://mathworld.wolfram.com/Erdos-SzekeresTheorem.html}{Erdos-Szekeres Theorem}{Any sequence of $ ab+1 $ real numbers contains either a monotonically decreasing subsequence of length $ a+1 $ or a monotonically increasing subsequence of length $ b+1 $. The more useful case is when $ a=b=n $. }



\prob{}{}{E}{Let $ n\ge 1 $ be an integer and let $ X $ be a set of $ n^2+1 $ positive integers such that in any subset of $ X $ with $ n+1 $ elements there exist two elements $ x\neq y $ such that $ x|y $. Prove that there exists a subset $ \{x_1, x_2\dots x_{n+1} \in X $ such that $ x_i|x_{i+1} $ for all $ i=1, 2, \dots n $.}




\newpage
\subsection{Problems}

\prob{https://artofproblemsolving.com/community/c6h148835p841269}{USA TST 2005 P1}{E}{
    Let $ n $ be an integer greater than $ 1 $. For a positive integer
    $ m $ , let $ S_{m}= \{ 1,2,\ldots, mn\} $. Suppose that there exists a $
    2n $ -element set $ T $ such that

    \begin{enumerate}
        \item each element of $ T $ is an $ m $ -element subset of $ S_{m} $
        \item each pair of elements of $ T $ shares at most one common element
        \item each element of $ S_{m} $ is contained in exactly two elements of $ T $
    \end{enumerate}

Determine the maximum possible value of $ m $ in terms of $ n$.
\index[cat]{Sets!Max-set with property!USA TST 2005 P1}
\index[strat]{Double Counting!USA TST 2005 P1}
}
\label{problem:double_counting_5}

\solu{We use double counting to find the ans, after that the rest is easy.}





\prob{https://artofproblemsolving.com/community/c6h206650p1136980}{Iran TST 2008 D3P1}{E}{
    Let $S$ be a set with $n$ elements, and $F$ be a family of subsets of $S$ with
    $ 2^{n-1}$ elements, such that for each $A,B,C\in F$, $A\cap B\cap C$ is not
    empty. Prove that the intersection of all of the elements of $F$ is not empty.

    \index[cat]{Sets!Local!Iran TST 2008 D3P1}
    \index[strat]{Induction!Iran TST 2008 D3P1}
}\label{lemma:sets_lemma_1_1}\label{problem:induction_type1_18}

\solu{Using Induction with \hrf{lemma:sets_lemma_1}{this} lemma.}




\prob{https://artofproblemsolving.com/community/c6t309f6h1538018}{Romanian TST
2016 D1P2}{EM}{
Let $n$ be a positive integer, and let $S_1, S_2,\dots S_n$ be
a collection of finite non-empty sets such that 
\[\sum_{1\leq i<j\leq n}{\frac{|S_i \cap S_j|}{|S_i||S_j|}} <1\]

Prove that there exist pairwise distinct elements $x_1, x_2\dots x_n$ such that $x_i$ is a member of $S_i$ for each index $i$.

\index[cat]{Sets!Local!Romanian TST 2016 D1P2}
\index[strat]{Focus!Romanian TST 2016 D1P2}
\index[strat]{Induction!Romanian TST 2016 D1P2}
}
\label{problem:induction_type1_25}\label{problem:forget_and_focus_7}

\solu{The Inductive proof reduces the problem to
    \hrf{problem:induction_type1_26}{American Mathematical Monthly problem
E2309}}

\solu{The other approach is to focus on the given weird condition, and interpolate it to something nice, like probabilistic condition.}



\prob{https://artofproblemsolving.com/community/c6h225275p1252232}{American Mathematical Monthly E2309}{EM}{If $ A_1$, $ A_2$,\dots $ A_n$ are $ n$ nonempty subsets of the set $ \left\{1,2,...,n - 1\right\}$, then prove that

\[\sum_{1\leq i < j\leq n}\frac {\left|A_i\cap A_j\right|}{\left|A_i\right|\cdot\left|A_j\right|}\geq 1\]

\index[cat]{Sets!Subsets of $[N]$!American Mathematical Monthly E2309}
}
\label{problem:induction_type1_26}



\prob{https://artofproblemsolving.com/community/c6h362426p1986928}{CGMO 2010
    P1}{E}{Let $n$ be an integer greater than two, and let $A_1,A_2, \cdots ,
    A_{2n}$ be pairwise distinct subsets of $\{1, 2, \dots n\}$. Determine the maximum
    value of
    \[\sum_{i=1}^{2n} \dfrac{|A_i \cap A_{i+1}|}{|A_i| \cdot |A_{i+1}|}\]
    Where $A_{2n+1}=A_1$ and $|X|$ denote the number of elements in $X.$

    \index[cat]{Sets!Subsets of $[N]$!CGMO 2010 P1}
}



\prob{https://artofproblemsolving.com/community/c6h17340p119108}{ISL 2002 C5}{M}{
    Let $r\geq2$ be a fixed positive integer, and let $F$ be an infinite
    family of sets, each of size $r$, no two of which are disjoint. Prove that
    there exists a set of size $r-1$ that meets each set in $F$.

\href{https://artofproblemsolving.com/community/c6h17340p7934669}{HMMT 2016
Team Round}:
    Fix positive integers $r>s$, and let $\mathcal F$ be an infinite
    family of sets, each of size $r$, no two of which share fewer than $s$
    elements. Prove that there exists a set of size $r-1$ that shares at least $s$
    elements with each set in $F$.

\index[cat]{Sets!Local!ISL 2002 C5}
\index[strat]{Induction!ISL 2002 C5}
\index[strat]{Focus!ISL 2002 C5}
\index[strat]{Adding new stuffs!ISL 2002 C5}

}\label{problem:induction_type1_31}




\solu{[Focus on a set] If we take an arbitrary set, we can say that there
exists infinitely many sets $ \in \mathbb{F} $ which includes a fixed element
from our test set. If we do this argument for $ r-1 $ times, we get a set $ X
$ of $ r-1 $ elements, and an infinte family of sets that contains $ X $
completely. At this point the problem is trivial.}


\solu{[Adding Elements]Since it's tricky to work with one family, why not introduce another family, like the second monk. \hrf{http://artofproblemsolving.com/community/c6h17340p3251745}{This} solution generalizes the problem as such.}


\prob{https://artofproblemsolving.com/community/c6h57290p352698}{ISL 1988 P10}{M}{
    Let $ N = \{1,2 \ldots, n\}, n \geq 2. $ 

    A collection $ F = \{A_1, \ldots,A_t\} $ of subsets 
    $ A_i \subseteq N, $  $ i = 1, \ldots, t, $ is said to be
    \textbf{separating}, if for every pair $ \{x,y\} \subseteq N, $ there is a set $ A_i
    \in F $ so that $ A_i \cap \{x,y\} $ contains just one element. 

    $ F $ is said to be \textbf{covering}, if every element of $ N $ is contained 
    in at least one set $ A_i \in F. $ 

    What is the smallest value $ f(n) $ of $ t, $ so there is a set $
    F = \{A_1, \ldots, A_t\} $ which is simultaneously separating and covering.

    \index[cat]{Sets!Subsets of $[N]$!ISL 1988 P10}
    \index[strat]{Bijection!Binary Representation!ISL 1988 P10}
} \label{problem:binary_2}


\solu{[Binary Representation]Using \hrf{binary}{Binary} Representations for the elements as in or not in, we get an easy bijection.}



\prob{https://artofproblemsolving.com/community/c6h530095p3024272}{Iran TST 2013 D1P2}{E}{
    Find the maximum number of subsets from $\left \{ 1,...,n \right \}$ such
    that for any two of them like $A,B$ if $A\subset B$ then $\left | B-A
    \right |\geq 3$. (Here $\left | X \right |$ is the number of elements of
    the set $X$.)
    \index[cat]{Sets!Max-set with property!Iran TST 2013 D1P2}
    \index[strat]{Roots of unity filtering!Iran TST 2013 D1P2}
    \index[strat]{Induction!$n$ is or isn't!Iran TST 2013 D1P2}
}


\solu{By partitioning the maximum set of subsets into groups which contain the number $n$ and which don't and \href{induction}{Induction} on $n$ we can show that the maximum number of subset is
\[\frac{2^{n}-(-1)^{n}}{3}\]. }

\prob{https://artofproblemsolving.com/community/c7h64445p383300}{Putnam 2005 B4}{E}{For positive integers $ m$ and $ n$, let $f\left(m,n\right)$ denote the number of $ n$-tuples $\left(x_1,x_2,\dots,x_n\right)$ of integers such that $\left|x_1\right| + \left|x_2\right| + \cdots + \left|x_n\right|\le m$. Show that $ f\left(m,n\right) = f\left(n,m\right)$.}



\solu{Try to show \href{bijection}{Bijection} between the result and choosing $m$ or $n$ objects from $m+n$ objects or show that the result is $\binom{m+n}{n}$.}
