\newpage\section{Sequences}


\subsection{Lemmas}

\theo{https://en.wikipedia.org/wiki/Van_der_Waerden's_theorem}
{Van der Waerden's Theorem}{
    For any given positive integers $r$ and $k$, there is some number $ N $ such
    that if the integers $ \{1, 2, ..., N\} $ are colored, each with one of $ r $
    different colors, then there are at least $ k $ integers in arithmetic
    progression all of the same color.
}

\theo{}
{Catalan Recursion}{
    The infinite series defined as following: \[ a_0 = a_1 = 1,\ a_n =
    \prod_{i=0}^{n} a_ia_{n-i+1} = a_0a_{n-1} + a_1a_{n-2}\dots + a_{n-1}a_0 \]
    has the general term 
    \[a_n = C_n = \boxed{\frac{1}{n+1} \binom{2n}{n}}\] 
}



\subsection{Problems}


\begin{myitemize}
    \item \href{http://alexanderrem.weebly.com/uploads/7/2/5/6/72566533/sequences.pdf}{Sequences - Alexander Remorov}
\end{myitemize}

\prob{https://artofproblemsolving.com/community/c6h68945p404543}
{ISL 1990}{E}{
    Assume that the set of all positive integers is decomposed into $ r $
    (disjoint) subsets $ A_1 \cup A_2 \cup \dots \cup A_r = \mathbb{N}. $
    Prove that one of them, say $ A_i, $ has the following property: There
    exists a positive $ m $ such that for any $ k $ one can find numbers $
    a_1, a_2, \ldots, a_k $ in $ A_i $ with $ 0 < a_{j+1} - a_j \leq m, $  $
    (1 \leq j \leq k-1)$.
    \index[strat]{Induction!ISL 1990}
}


\prob{https://mathoverflow.net/questions/25313/finitely-many-arithmetic-progressions}
{Dividing the integers into arithmetic progressions, Erdos}{E}{
    Let $ d_1, d_2,\dots, d_k $ be differences of $ k $ arithmetic
    progressions that partition $ \N $. Show that $ d_i=d_j $ for some $i,j$.
    \index[strat]{Generating Function!Erdos, Arithmatic Progression}
    \index[strat]{Roots of unity filtering!Erdos, Arithmatic Progression}
}


\prob{https://artofproblemsolving.com/community/c6h79784p456601}
{APMO 1999 P1}{E}{
    Find the smallest positive integer $n$ with the following property: there
    does not exist an arithmetic progression of $1999$ real numbers containing
    exactly $n$ integers.
    \index[strat]{Name things!APMO 1999 P1}
}

\begin{solution}
    If the difference is $\frac{p}{q}$, then we can arrange the sequence in a
    way that we can get both exactly $\left\lfloor \frac{1999}{q} \right\rfloor$
    and $\left\lceil \frac{1999}{q} \right\rceil$ integers. \\

    So what we want is to find smallest integer $n$ for which there is a $k$
    such that
    \[n+1 \le \frac{1999}{k} \quad \text{ and } \quad \frac{1999}{k+1} \le n-1\] 
    So that $\frac{1999}{k}$ skips over $n$. After a bit of calculation, we
    get $n = 70, k = 28$ are the solutions we want.
\end{solution}


\prob{https://artofproblemsolving.com/community/c6h79787p456607}
{APMO 1999 P2}{E}{
    Let $a_1, a_2, \dots$ be a sequence of real numbers satisfying $a_{i+j}
    \leq a_i+a_j$ for all $i,j=1,2,\dots$. Prove that
    \[ a_1 + \frac{a_2}{2} + \frac{a_3}{3} + \cdots + \frac{a_n}{n} \geq a_n \]
    for each positive integer $n$.
    \index[strat]{Induction!APMO 1999 P2}
    \index[strat]{Sum them up!APMO 1999 P2}
}

\begin{solution}[jgnr]
    We will prove this by induction. Note that the inequality holds for $ n=1$. Assume that the inequality holds for $ n=1,2,\ldots,k$, that is,
\[ a_1\ge a_1,\quad a_1+\frac{a_2}2\ge a_2,\quad
    a_1+\frac{a_2}{2}+\frac{a_3}3\ge a_3, \quad \dots \quad
a_1+\frac{a_2}{2}+\frac{a_3}{3}+\cdots+\frac{a_k}k\ge a_k. \]
Sum them up:
\[ ka_1+(k-1)\frac{a_2}2a_2+\cdots+\frac{a_k}{k}\ge a_1+a_2+\cdots+a_k. \]Add $ a_1+\ldots+a_k$ to both sides:
\[ (k+1)\left(a_1+\frac{a_2}2+\cdots+\frac{a_k}k\right)\ge (a_1+a_k)+(a_2+a_{k-1})+\cdots+(a_k+a_1)\ge ka_{k+1}. \]
Divide both sides by $ k+1$:
\[ a_1+\frac{a_2}2+\cdots+\frac{a_k}k\ge\frac{ka_{k+1}}{k+1}, \] i.e.
\[ a_1 + \frac{a_2}{2} + \frac{a_3}{3} + \cdots + \frac{a_n}{n} \geq a_n. \]
\end{solution}


\prob{https://artofproblemsolving.com/community/c6h125791p713182}
{ISL 1994 A1}{E}{
    Let for each nonnegative integer $ n$, 
    \[ a_{0} = 1994 \quad a_{n + 1} = \frac {a_{n}^{2}}{a_{n} + 1}\] 
    Prove that $ 1994 - n$ is the greatest integer less than or equal to $
    a_{n}$, $ 0 \leq n \leq 998$
    \index[strat]{Rearrange!ISL 1994 A1}
    \index[strat]{Induction!ISL 1994 A1}
}

\solu{[t0rajir0u]
    Rewrite the condition as
    \[a_{n+1} = a_n -1 + \frac{1}{a_n + 1}\]
    Which gives us
    \[ a_{k} = 1994 - k + \frac {1}{a_{k - 1} + 1} + \frac {1}{a_{k - 2} + 1} +
    \dots + \frac {1}{1994 + 1}\]
    So we need to bound the fraction part below $1$ for $a_{998}$. By induction, it is
    atmost
    \[\frac{1}{997}+ \frac{1}{998}\dots + \frac{1}{1995}\] 
    Which is trivial to prove.
}


\prob{https://artofproblemsolving.com/community/c6h214669p1186971}
{ISL 2007 C4}{M}{
    Let $ A_0 = (a_1,\dots,a_n)$ be a finite sequence of real numbers. For
    each $ k\geq 0$, from the sequence $ A_k = (x_1,\dots,x_k)$ we construct a
    new sequence $ A_{k + 1}$ in the following way.
    \begin{enumerate}
        \item We choose a partition $ \{1,\dots,n\} = I\cup J$, where $ I$ and
            $ J$ are two disjoint sets, possibly empty, such that the expression 
            \[\left|\sum_{i\in I}x_i - \sum_{j\in J}x_j\right|\] 
            attains the smallest value. If there are several such partitions,
            one is chosen arbitrarily.
        \item We set $ A_{k + 1} = (y_1,\dots,y_n)$ where $ y_i = x_i + 1$ if
            $ i\in I$, and $ y_i = x_i - 1$ if $ i\in J$.
    \end{enumerate}
    Prove that for some $ k$, the sequence $ A_k$ contains an element $ x$
    such that $ |x|\geq\dfrac n2$.

    \index[strat]{Invariant!Sum of Squares!ISL 2007 C4}
}\label{problem:invariant_rules_of_thumb_11}

\solu{
    Suppose the contrary. Now, since $ A_i $ can only attain finite values, So $
    A_i = A_j $ for some $ i, j $. Now, we are taking about changes here, so we
    need to think of some invariants. Firstly the sum, it's not much of an help,
    because it doesn't give us much control. So kinda sum-ish invariant with a bit
    more control is the sum of squares. We combine these two ideas.
}



\prob{https://artofproblemsolving.com/community/c6h288840p1561573}
{ISL 2009 A6}{EM}{
    Suppose that $ s_1,s_2,s_3, \ldots$ is a strictly increasing sequence of
    positive integers such that the sub-sequences 
    \[s_{s_1},\, s_{s_2},\, s_{s_3},\, \ldots\quad\text{and}\quad
    s_{s_1+1},\, s_{s_2+1},\, s_{s_3+1},\, \ldots\] 
    are both arithmetic progressions. Prove that the sequence $ s_1, s_2, s_3,
    \ldots$ is itself an arithmetic progression.

    \index[strat]{Name things!ISL 2009 A6}
}

\solu{
    First notice that the two arithmetic sequences has the same common
    difference. Then notice that the diffrences of the original sequence is
    bounded.\\

    Another advice, give everything names. After naming the smallest
    difference and the largest differnce, we get two different inequalities,
    from where we deduce that the difference is constant.
}



\prob{https://artofproblemsolving.com/community/c6h1181540p5720240}
{ISL 2013 N2}{E}{
    Assume that $k$ and $n$ are two positive integers. Prove that there exist
    positive integers $m_1, \dots, m_k$ such that
    \[1+\frac{2^k-1}{n}=\left(1+\frac1{m_1}\right)\cdots \left(1+\frac1{m_k}\right).\]

    \index[strat]{Induction!ISL 2013 N2}
}

\solu{Just induct, and think wishfully.}



\prob{https://artofproblemsolving.com/community/c6t177f6h145842}
{USAMO 2007 P1}{E}{
    $n$ be a positive integer. Define a sequence by setting $a_1=n$ and, for each
    $k>1$, letting $a_k$ be the unique integer in the range $0\leq a_k \leq k-1$
    for which $a_0+a_1\dots +a_k$ is divisible by $k$. Prove that for any $n$ the
    sequence ${a_i}$ eventually becomes constant.

    \index[strat]{Invariant!Monovariant!USAMO 2007 P1}
}

\solu{Investigate and done.}


\prob{https://artofproblemsolving.com/community/c6h1071763p4663881}
{APMO 2015 P3}{E}{
    A sequence of real numbers $a_0, a_1, . . .$ is said to be good if the
    following three conditions hold.
    \begin{enumerate}[itemsep=5pt]
        \item The value of $a_0$ is a positive integer.
        \item For each non-negative integer $i$ we have $a_{i+1} = 2a_i + 1 $
            or $a_{i+1} =\dfrac{a_i}{a_i + 2} $
        \item There exists a positive integer $k$ such that $a_k = 2014$.
    \end{enumerate}
    Find the smallest positive integer $n$ such that there exists a good sequence
    $a_0, a_1, \dots$ of real numbers with the property that $a_n = 2014$.

    \index[strat]{Invariant!Monovariant!APMO 2015 P3}
    \index[strat]{Rearrange!APMO 2015 P3}
    \index[strat]{Get your hands dirty!APMO 2015 P3}
}

\solu{
    We will rename the sequence and call it $ \{s_i\} $, with $ s_0=x\in \Z $.
    Now let, \[s_i = \frac{a_ix+b_i}{c_ix+d_i}\]

    At the beginning we have $ a_0=d_0=1, b_0=c_0=0 $. It is easy to prove
    that \[ a_{i+1}+c_{i+1} = 2(a_i+c_i) \]\[ b_{i+1}+d_{i+1} = 2(b_i+d_i) \]
    So it follows that \[ a_i+c_i = 2^i = b_i+d_i \]
    Also, by induction (which is easy to prove), we have \[a_i - b_i = d_i - c_i = 1\]

    Suppose for some $ k $, $ s_k=2014 $. So, 
    \begin{align}
        &\frac{a_kx+b_k}{c_kx+d_k}=2014\\[1em]
        \implies x&=\frac{2014d_k-b_k}{a_k-2014c_k}=\frac{2015d_k-2^k}{2^k-2015c_k}\\[1em]
        &=\frac{2015(d_k-c_k)}{2^k-2015c_k}-1
    \end{align}

    But $ \gcd(2015, 2^k-2015c_k)=1 $, which implies $ 2^k-2015-c_k=1 $.
    Solving for $ k $ with CRT gives us $ 60|k $. 

    Now we have to prove that there is a sequence with $ s_{60} = 2014 $.
    Solving (1), $ s_0=2014 $, and,
    \[\begin{aligned}
        a_{60}&= \frac{2014\cdot 2^{60}+1}{2015} &b_{60} &=\frac{2014\cdot
        2^{60}-2014}{2015}\\[1em]
        c_{60}&= \frac{2^{60}-1}{2015} &d_{60} &=\frac{2^{60}+2014}{2015}
    \end{aligned}\]
    \vspace{1em}
    We show that we can make $ (a_{60}, c_{60}) $ from $ (a_0, c_0)=(1, 0) $.
    We prove it by induction, that $(a_k, c_k)$ can take any form $(2^k-i, i)$
    with $i\in\{0, 1, \dots 2^k-1\}$.
}


\prob{https://artofproblemsolving.com/community/c6h287853p1555896}
{ISL 2008 A4}{EM}{
    For an integer $ m$, denote by $ t(m)$ the unique number in $ \{1,    2,
    3\}$ such that $ m + t(m)$ is a multiple of $ 3$. A function $    f:
    \mathbb{Z}\to\mathbb{Z}$ satisfies $ f( - 1) = 0$, $ f(0) = 1$,    $ f(1)
    = - 1$ and $f\left(2^{n} + m\right) = f\left(2^n -    t(m)\right) - f(m)$
    for all integers $ m$, $ n\ge 0$ with $ 2^n >    m$. Prove that $ f(3p)\ge
    0$ holds for all integers $ p\ge 0$. 

    \index[strat]{Induction!ISL 2008 A4}
    \index[strat]{Get your hands dirty!ISL 2008 A4}
}

\begin{solution}
    We begin by listing values of $f(n)$ for $n\le 16$, and immediately it
    strikes us that:
    \begin{enumerate}
        \item  if $ -1\le x \le 2^{2m}-1$ then the maximal value is $
            f(2^{2m}-1)$ the minimal value is $ f(2^{2m}-2)$
        \item if $ -1\le x \le 2^{2m+1}$ then the maximal value is $ f(2^{2m+1}-2)$
            the minimal value is $ f(2^{2m+1}-1)$
    \end{enumerate}
    After which we are done by induction.
\end{solution}
